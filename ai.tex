
\chapter{Interpretazione astratta}

[Introduzione al capitolo da scrivere.]

\section{Analisi dei segni per WHILE}

\newcommand*{\Sign}{\mathrm{Sign}}

\newcommand*{\signtop}{\top}
\newcommand*{\signbot}{\bot}
\newcommand*{\signge}{\mathord{\geq}}
\newcommand*{\signgt}{\mathord{>}}
\newcommand*{\signle}{\mathord{\leq}}
\newcommand*{\signlt}{\mathord{<}}
\newcommand*{\signeq}{\mathord{=}}
\newcommand*{\signne}{\mathord{\neq}}

\begin{figure}
\begin{center}
\setlength{\unitlength}{1.8mm}
\begin{picture}(28, 40)
{\thicklines
\put(14, 2){\circle{4}}
\put(14, 2){\makebox(0, 0){$\signbot$}}

\put( 2, 14){\circle{4}}
\put( 2, 14){\makebox(0, 0){$\signlt$}}
\put(14, 14){\circle{4}}
\put(14, 14){\makebox(0, 0){$\signeq$}}
\put(26, 14){\circle{4}}
\put(26, 14){\makebox(0, 0){$\signgt$}}

\put( 2, 26){\circle{4}}
\put( 2, 26){\makebox(0, 0){$\signle$}}
\put(14, 26){\circle{4}}
\put(14, 26){\makebox(0, 0){$\signne$}}
\put(26, 26){\circle{4}}
\put(26, 26){\makebox(0, 0){$\signge$}}

\put(14, 38){\circle{4}}
\put(14, 38){\makebox(0, 0){$\signtop$}}

\put( 2, 24){\line(0, -1){8}}
\put(26, 24){\line(0, -1){8}}

\put(14, 36){\line(0, -1){8}}
\put(14, 12){\line(0, -1){8}}

\put(15.42, 36.58){\line(1, -1){9.16}}
\put( 3.42, 24.58){\line(1, -1){9.16}}
\put(15.42, 24.58){\line(1, -1){9.16}}
\put( 3.42, 12.58){\line(1, -1){9.16}}

\put( 3.42, 27.42){\line(1, 1){9.16}}
\put( 3.42, 15.42){\line(1, 1){9.16}}
\put(15.42, 15.42){\line(1, 1){9.16}}
\put(15.42,  3.42){\line(1, 1){9.16}}
}
\end{picture}
\end{center}
\caption{Diagramma di Hasse del dominio $\Sign$.}
\label{fig:ordering-rels-lattice}
\end{figure}

