\chapter{Semantica operazionale ``big-step''}
\marginpar{Mancino}

In questo capitolo introduciamo la semantica operazionale \emph{big-step}
(letteralmente ``a grandi passi''),
un particolare tipo di semantica operazionale che descrive formalmente
come il risultato complessivo dell'esecuzione di un'operazione è ottenuto.
Per fare questo, sfrutta schemi di assiomi e regole che vengono presentati
come
\[
\rulename{nome regola}
\prooftree
  \textrm{premessa}_1
  \quad\cdots\quad
  \textrm{premessa}_k
\justifies
  \textrm{conclusione}
\using
  (\textrm{side condition})
\endprooftree
\]
La semantica di questa scrittura è che la congiunzione logica di
`$\textrm{premessa}_1$', \dots, `$\textrm{premessa}_k$' implica
`$\textrm{conclusione}$' a patto che valga `$\textrm{side condition}$'.

\section{Un primo esempio di semantica big-step}\marginpar{Mancino}
\label{sec:primo-esempio-semantica-big-step}
Consideriamo il frammento di WHILE con la seguente sintassi:
\[
  \AExp \ni E ::= n \vbar E+E
\]
dove $n \in \Integer$.
Una possibile semantica big-step di questo frammento è data
dai seguenti schemi:

\[
\rulename{B-NUM}
\prooftree
  \nohyp
\justifies
  n \Downarrow n
\endprooftree
\qquad
\rulename{B-ADD}
\prooftree
  E_1 \Downarrow n_1 \quad E_2 \Downarrow n_2
\justifies
  (E_1 + E_2) \Downarrow n_3
\using
  n_3 = n_1 + n_2
\endprooftree
\]

L'operatore $\Downarrow$ può essere letto come ``valuta a'' e permette
in un solo colpo di compiere tutti i passi affinché un comando o un'espressione
giunga al termine della propria computazione.

Le definizioni semantiche date sono dette ``schemi'' perché vanno
opportunamente instanziate. Vediamone alcuni esempi; la seguente è
un'istanza dimostrabile della B-NUM:

\[
\rulename{B-NUM}
\prooftree
  \nohyp
\justifies
  3 \Downarrow 3
\endprooftree
\]

La seguente è invece un'istanza dimostrabile della B-ADD:

\[
\rulename{B-ADD}
\prooftree
  3 \Downarrow 3
  \quad
  2 \Downarrow 2
\justifies
  (3+2) \Downarrow 5
\using
  5 = 3+2
\endprooftree
\]

La seguente è ancora un'istanza della B-ADD, ma non dimostrabile:

\[
\rulename{B-ADD}
\prooftree
  3 \Downarrow 6
  \quad
  2 \Downarrow 4
\justifies
  (3+4) \Downarrow 10
\using
  10 = 6+4
\endprooftree
\]


Combinando le regole date, è possibile realizzare dimostrazioni ad albero
come la seguente:

\[
\prooftree
  \prooftree
   \justifies
     3 \Downarrow 3
  \endprooftree
  \quad
  \prooftree
        \prooftree
          \justifies
                2 \Downarrow 2
        \endprooftree
        \quad
        \prooftree
          \justifies
                1 \Downarrow 1
        \endprooftree
        \justifies
          (2+1) \Downarrow 3
        \using
          3 = 2+1
  \endprooftree
  \justifies
    (3+(2+1)) \Downarrow 6
        \using
          6 = 3+(2+1)
\endprooftree
\]

La dimostrazione è un albero che va dal basso verso l'alto, dove
in fondo viene posto ciò che vogliamo dimostrare e,
applicando le due regole date precedentemente,
si giunge a degli assiomi (le foglie dell'albero) che
quindi provano quanto volevamo.
In sostanza, per dimostrare la conclusione dobbiamo far
valere tutte le premesse che si ricavano
applicando le regole.

\section{Determinatezza}\marginpar{Mancino}

La proprietà di determinatezza asserisce che un qualunque oggetto
sintattico non può avere due semantiche differenti. In particolare,
possiamo dimostrare il seguente

\begin{teorema} \summary{(Determinatezza.)}
La semantica big-step del frammento di $\AExp$ data nella
sezione~\textup{\ref{sec:primo-esempio-semantica-big-step}}
gode della proprietà di determinatezza, ossia
per ogni $E \in \AExp$ e per ogni $n, n' \in \Zset$,
se $E \Downarrow n$ e $E \Downarrow n'$ allora $n = n'$.
\end{teorema}

\begin{proof}
La dimostrazione è per induzione strutturale sulla struttura di E.

Caso base: $n \in \Zset$.
In questo caso abbiamo che $n \Downarrow n_1$ e che $n \Downarrow n_2$.
Lo schema B-NUM implica che $n = n_1$ e $n = n_2$ e,
per la transitività dell'uguaglianza, $n_1 = n_2$.

Passo induttivo: $E \in \AExp$ con $E = E_1 + E_2$.
In questo caso abbiamo che $(E_1 + E_2) \Downarrow m_3$
e $(E_1 + E_2) \Downarrow p_3$.
Lo schema B-ADD implica che esistono
$m_1, m_2, p_1, p_2 \in \Zset$ tali che:
\begin{align*}
  m_1 + m_2 &= m_3, & E_1 &\Downarrow m_1, & E_2 &\Downarrow m_2, \\
  p_1 + p_2 &= p_3, & E_1 &\Downarrow p_1, & E_2 &\Downarrow p_2.
\end{align*}
Ma allora, per ipotesi induttiva:
\begin{align*}
m_1 &= p_1, & m_1+m_2 &= m_3 \\
m_2 &= p_2, & p_1 + p_2 &= p_3
\end{align*}
Da cui deduciamo che $ m_1 + m_2 = p_1 + p_2$, per cui
$m_3 = p_3$ come volevamo dimostrare.
\end{proof}

\section{Normalizzazione}\marginpar{Mancino}
La proprietà di normalizzazione asserisce che per ogni espressione $E$ esiste almeno un risultato
$n$ tale che $E \Downarrow n$.

Osserviamo che, in generale, ci sono esempi in cui non è possibile normalizzare:
\[ \textrm{ $\kw{while}$ true $\kw{do}$ $\kw{skip}$ $\Downarrow$ ?} \]
A destra dell'operatore di ``valuta a'' non posso scriverci nulla, il ciclo non termina.
Un altro esempio è il seguente:
\[ 5/0 \; \Downarrow \; ? \]
A destra dell'operatore di ``valuta a'' posso scrivere la codifica del disastro (se esiste)
altrimenti, come nel caso precedente, non possiamo scriverci nulla.

\begin{teorema}[Normalizzazione]
La semantica big-step del frammento di $\AExp$ data nella
sezione~\textup{\ref{sec:primo-esempio-semantica-big-step}}
gode della proprietà di normalizzazione, ossia per ogni
$E \in Exp$ esiste  $n \in \Zset$ tale che $E \Downarrow n$.
\end{teorema}

\begin{proof} \marginpar{Segalini}
Per induzione sulla struttura di $\AExp$:

Caso base: $n \in \Zset$.
Per lo schema d'assioma esiste sempre $n$ che è appunto lui stesso:
\[
\rulename{B-NUM}
\prooftree
  \nohyp
\justifies
  n \Downarrow n
\endprooftree
\]


Passo induttivo: $E \in \AExp$ con $E = E_1 + E_2$.
Per ipotesi induttiva abbiamo che:
\[
\exists n_1, n_2 \in \Zset \st E_1 \Downarrow n_1 \land E_2 \Downarrow n_2.
\]
Dimostriamo che esiste un $n_3 \in \Zset$ tale che $E_1 + E_2
\Downarrow n_3$. Esso esiste sempre dallo schema di regola (B-ADD) il
quale afferma che $n_3 = n_1 + n_2$:
\[
\prooftree
  \prooftree
    \nohyp
  \justifies
     E_1 \Downarrow n_1
  \endprooftree
  \quad
  \prooftree
    \nohyp
  \justifies
     E_2 \Downarrow n_2
  \endprooftree
\justifies
  (E_1 + E_2) \Downarrow n_3
\using
  n_3 = n_1 + n_2
\endprooftree
\]
\end{proof}

\section{Semantica big-step per WHILE}\marginpar{Mancino}
Invece di usare un semplice frammento di linguaggio come quello descritto nel capitolo
precedente possiamo definire tramite la big-step la semantica del linguaggio WHILE.
Precisiamo allora che una coppia (\emph{comando, stato}) viene valutata ad un
nuovo stato, ossia, dato un comando $C$ e uno stato $s$, in generale:

\[
\config{C}{s} \Downarrow s'
\]

Con questa premessa, definendo con $E$ la generica espressione, con $C$
il generico comando e con $s$ il generico stato,
è facile immaginare il comportamento di tutti i comandi per WHILE:

\begin{align*}
\prooftree
  \justifies
    \config{\kw{skip}}{s} \Downarrow s
\endprooftree
&&
\prooftree
  E \Downarrow n
  \justifies
        \config{x:=E}{s} \Downarrow \assign{s}{x}{n}
\endprooftree
\end{align*}

\begin{align*}
\prooftree
   B \Downarrow \textrm{tt}
   \quad \config{C_1}{s} \Downarrow s'
   \justifies
         \config{\textrm{$\kw{if}$ $B$ $\kw{then}$ $C_1$ $\kw{else}$ $C_2$}}{s} \Downarrow s'
\endprooftree
&&
\prooftree
   B \Downarrow \textrm{ff} 
   \quad \config{C_2}{s} \Downarrow s'
   \justifies
         \config{\textrm{$\kw{if}$ $B$ $\kw{then}$ $C_1$ $\kw{else}$ $C_2$}}{s} \Downarrow s'
\endprooftree
\end{align*}

\begin{align*}
\prooftree
  \config{C_1}{s} \Downarrow s' 
  \quad \config{C_2}{s'} \Downarrow s''
  \justifies
        \config{\textrm{$C_1$; $C_2$}}{s} \Downarrow s''
\endprooftree
\end{align*}

Infine rimane la regola del while,
per la quale abbiamo due comportamenti distinti:

\begin{gather*}
\prooftree
  B \Downarrow \textrm{tt} 
  \quad \config{C}{s} \Downarrow s'
  \quad \config{\textrm{$\kw{while}$ $B$ $\kw{do}$ $C$}}{s'} \Downarrow s''
  \justifies
        \config{\textrm{$\kw{while}$ $B$ $\kw{do}$ $C$}}{s} \Downarrow s''
\endprooftree
\\[2em]
\prooftree
  B \Downarrow \textrm{ff}
  \justifies
        \config{\textrm{$\kw{while}$ $B$ $\kw{do}$ $C$}}{s} \Downarrow s
\endprooftree
\end{gather*}

\section{Espressioni con effetti collaterali}\marginpar{Mancino}

In questo paragrafo ridefiniamo la semantica data precedentemente
per il linguaggio WHILE ammettendo espressioni con effetti collaterali.

\begin{definizione} \summary{(Espressione con effetti collaterali.)}
Un'espressione si dice \emph{avere effetti collaterali} se la sua
valutazione (oltre a restituire un valore) provoca la modifica
dello store.
\end{definizione}

Arricchiamo il linguaggio con l'espressione $\kw{do}$ $C$ $\kw{return}(x)$. 
In questo modo, possiamo ottenere espressioni con effetti collaterali
come la seguente:

\[
	\bigl( \kw{do} x := x+1; \kw{return}(x) \bigr) +
	\bigl( \kw{do} x := x*2; \kw{return}(x) \bigr)
\]

Il $\kw{return}$ assicura che venga usato il valore di $x$
aggiornato. Nell'esempio precedente diviene importante, dunque,
l'ordine di valutazione. Assumendo che sia $x=0$ prima
dell'esecuzione dell'espressione, si ha:
\begin{itemize}
  \item valutando da sinistra a destra, x=2
        dopo l'esecuzione dell'espressione;
  \item valutando da destra a sinistra, x=1
        dopo l'esecuzione dell'espressione.
\end{itemize}
Dal momento che l'espressione ha modificato lo store allora
l'espressione presenta un effetto collaterale.

Dobbiamo dunque modificare la relazione $\Downarrow$:
nel paragrafo precedente si trattava di
$\reld{\Downarrow}{(\AExp \times \Sigma)}{\Nset}$,
qui diventa
$\reld{\Downarrow}{(\AExp \times \Sigma)}{(\Nset \times \Sigma)}$.
Scriveremo quindi giudizi della forma
$\config{E}{s} \Downarrow \config{n}{s'}$.

Partendo da questa osservazione non è difficile definire la semantica:

\begin{align*}
\prooftree
  \justifies
        \config{\kw{skip}}{s} \Downarrow s
\endprooftree
&&
\prooftree
  \config{E}{s} \Downarrow \config{n}{s'}
  \justifies
        \config{x:=E}{s} \Downarrow \assign{s'}{x}{n}
\endprooftree
\end{align*}

\begin{gather*}
\prooftree
  \config{B}{s} \Downarrow \config{\textrm{tt}}{s'} 
  \quad \config{C_1}{s'} \Downarrow s''
  \justifies
        \config{\textrm{$\kw{if}$ $B$ $\kw{then}$ $C_1$ $\kw{else}$ $C_2$}}{s} \Downarrow s''
\endprooftree
\\[2em]
\prooftree
  \config{B}{s} \Downarrow \config{\textrm{tt}}{s'}
  \quad \config{C}{s'} \Downarrow s''
  \quad \config{\textrm{$\kw{while}$ $B$ $\kw{do}$ $C$}}{s'} \Downarrow s'''
  \justifies
        \config{\textrm{$\kw{while}$ $B$ $\kw{do}$ $C$}}{s} \Downarrow s'''
\endprooftree
\end{gather*}

E così via per i restanti schemi.
