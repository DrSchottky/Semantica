\documentclass[online,helvetica,a4]{chaksem}
%\documentclass[paper,helvetica]{chaksem}

\usepackage{amsmath}
\usepackage{amssymb}
\usepackage{stmaryrd}
\usepackage{url}
\usepackage{verbatim}
\usepackage{ifthen}
\usepackage{varioref}
\usepackage{hhline}
\input prooftree
\input macros

\begin{document}

\begin{slide}
  \vspace*{-.9\bigskipamount}
  \begin{center}
    \vspace*{\fill}
    \huge
    {\dblue Complete Partial Orders}
    \\[1.2em]
    \large
    {\dgreen Roberto \textsc{Bagnara}} \\
             Department of Mathematics \\
             University of Parma, Italy \\
  \end{center}%
  \vspace*{-.5\bigskipamount}
\setfooter{%
  \bgcolbox{bggray}{{\upshape Copyright \copyright\ 2011 Roberto Bagnara.
                              Distributed under the terms of the
                              GNU Free Documentation License 1.2}}}
\end{slide}


\begin{slide} \heading{POs, Chains and CPOs}

$(S, \preceq)$ is a partial ordering (or \Emph{PO}) if `$\preceq$'
is reflexive, transitive and antisymmetric over $S$.

$(S, \preceq)$ is \Emph{totally ordered} if, for each $x, y \in S$,
we have $x \preceq y$ or $y \preceq x$.

If $(S, \preceq)$ is totally ordered and denumerable, then it
is called a \Emph{chain}.

$(S, \preceq)$ is called a \Emph{complete partial ordering} or \Emph{CPO}
if it has a minimum element ($\bot$) and all its chains
have a least upper bound.

\vfill
\textbf{Exercise}: Prove that
\begin{enumerate} 
\item
The maximum element of a PO, when it exists, is also the least upper bound of the the PO.
\item
Every finite chain has a maximum element.
\item
Every subset of a chain is a chain.
\end{enumerate}
\end{slide}

\begin{slide} \heading{Monotonic Functions}

Let $(S, \preceq)$ and $(T, \sqsubseteq)$ be CPOs.

The function $\fund{f}{S}{T}$ is \Emph{monotonic} if, for each $x, y \in S$,
we have $f(x) \sqsubseteq f(y)$ whenever $x \preceq y$.

This can be read as the fact that $f$ preserves the information content:
when given more information it will not produce less information.

All the functions we are interested in for the purpose of defining the
concrete and abstract semantics of programming languages are monotonic.
(If you find yourself dealing with a non-monotonic function, start
looking for the bug: there is one somewhere.)
\end{slide}

\begin{slide} \heading{Continuous Functions}

Let $(S, \preceq)$ and $(T, \sqsubseteq)$ be CPOs.

The function $\fund{f}{S}{T}$ is \Emph{continuous} if,
for each chain $X \sseq S$, we have:
\begin{enumerate}
\item
$\lub\bigl\{\, f(x) \bigm| x \in X \,\bigr\}$ exists;
\item
$f(\lub X) = \lub\bigl\{\, f(x) \bigm| x \in X \,\bigr\}$.
\end{enumerate}

Suppose $y$ is an infinite object
(e.g., the set of \Emph{all} computation paths in a program)
that is only definable as the limit of a chain of finite parts $X$
(e.g., a set of \Emph{some} computation paths, then \Emph{some more}
and so forth): $y = \lub X$.

Since $y$ is an infinite object, we cannot compute $f(y)$ directly.
\end{slide}

\begin{slide} \heading{Continuous Functions, cont.}

However, \emph{sometimes} we can approach $y$ by computing $f(x)$ for $x \in X$
(we can do that if each $x$ is a finite object).

And, \emph{sometimes} $f(y)$ is the \Emph{limit} of such process:
$f(y) = \lub\bigl\{\, f(x) \bigm| x \in X \,\bigr\}$.

When the function $f$ is \Emph{continuous}, in the above sentences
we can replace \emph{sometimes} by \emph{always}.

All the functions we are interested in for the purpose of defining the
concrete and abstract semantics of programming languages are continuous.

\vfill
\textbf{Exercise}: Prove that a continuous function is also monotonic.

\vfill
\textbf{Exercise}: Exhibit a monotonic function that is not continuous.

\end{slide}

\begin{slide} \heading{Additive Functions}

The function $f$ is \Emph{additive} if
\[
	f(\lub X) = \lub\bigl\{\, f(x) \bigm| x \in X \,\bigr\}
\]
for all sets $X \sseq S$ (i.e., it preserves \Emph{all} lubs, not just
the lubs of chains).

With an additive function
any composite piece of information (i.e., something that is obtainable
as $\lub X$) can be processed by processing the components individually
(i.e., computing $\{\, f(x) \mid x \in X \,\}$)
and then composing the processed pieces.

Not all functions we are interested in are additive, unfortunately.

\vfill
\textbf{Exercise}: Give a reasonable definition of the division operation
over intervals.  Then prove that the operation you defined is additive
in the first argument, but it is not in the second argument.

\end{slide}

\begin{slide} \heading{Fixed Points}
Let $S$ be a set and $\fund{f}{S}{S}$ be any function.
A \Emph{fixed point} of $f$ is any value $x$ for which $f(x) = x$.

\textbf{Examples:} 

Let $S \defeq \Rset$.
The function $f(x) \defeq x+1$ has no fixed points.
With the function $f(x) \defeq x$, every element of $\Rset$ is a fixed point.

Let $S \defeq \wpf(\Nset)$.
The function 
\[
   f(x) = x \union \bigl\{ m-1 \bigm| m \in x \setdiff \{0\} \bigr\}
\]
has the chains of numbers $\{0, \ldots, m\}$ for all $m \in \Nset$
as its fixed points.
\end{slide}


\begin{slide} \heading{Least Fixed Points of Continuous Functions}
If $(S, \preceq)$ is a CPO with a least element $\bot$
and $\fund{f}{S}{S}$ is continuous,
then $f$ has a \Emph{least fixed point} denoted by \Emph{$\lfp f$} and given by
\[
	\lfp f = \lub \bigl\{\, f^n(\bot) \bigm| n \in \Nset \,\bigr\}
\]
where, for each $x \in S$,
\begin{align*}
  f^0(x)    &\defeq x, \\
  f^{n+1}(x) &\defeq f\bigl(f^n(x)\bigr). \\
\end{align*}
\end{slide}


\begin{slide} \heading{Least Fixed Points of Continuous Functions: Proof}
We first prove that $\forall n \in \Nset \itc f^n(\bot) \preceq f^{n+1}(\bot)$.
\begin{description}
\item[$n = 0$:]
$f^0(\bot) = \bot \preceq f^1(\bot)$ as $\bot$ is the least element;
\item[$n > 0$:]
$f^n(\bot) \preceq f^{n+1}(\bot)$ is equivalent to
$f\bigl(f^{n-1}(\bot)\bigr) \preceq f\bigl(f^n(\bot)\bigr)$;
the latter is true by inductive hypothesis and by monotonicity
of $f$.
\end{description}

As a consequence, $C = \bigl\{\, f^n(\bot) \bigm| n \in \Nset \,\bigr\}$
is a chain in $S$, hence, as $S$ is a CPO, $\lub C$ exists.

By continuity of $f$, $f(\lub C) = \lub f(C)$, but $\lub f(C) = \lub C$
by definition of $C$, and thus $f(\lub C) = \lub C$ so that $\lub C$
is a fixed point of $f$.

Suppose, towards a contradiction, the $m$ is a fixed point of $f$ with
$m \prec \lub C$.  There must exist $i \in \Nset$ such that
$m = f^j(\bot)$ for each $j \ge i$.  This would imply $\lub C$ is not
the least upper bound of $C$.
\end{slide}

\end{document}

\begin{slide} \heading{Inductively Defined Sets}
Let $\cU$ a \Emph{universe} of (not better specified) terms.

Let $\fund{F}{\wp(U)}{\wp(U)}$ be a continuous function with respect to the
complete lattice $\bigl(\wp(\cU), \sseq\bigr)$.

The set \Emph{inductively defined by $F$} is $\lfp F$.
For what we said before, we have
\[
	\lfp F = \bigunion_{i \in \Nset} F^i(\emptyset).
\]
\end{slide}

\begin{slide} \heading{Inductively Defined Sets, cont.}
\textbf{Example:} consider the BNF rule
\[
    V ::= 0 \ | \ 1V
\]
and the associated function $\fund{\bar{V}}{\{0,1\}^\star}{\{0,1\}^\star}$
\[
    \bar{V}(S) \defeq \{ 0 \} \union \{\, 1s \mid s \in S \,\}.
\]
Then $\bar{V}$ is continuous.

We have $\lfp \bar{V} = \{\, 1^n0 \mid n \in \Nset \,\}$.
\end{slide}

\begin{slide} \heading{Coinductively Defined Sets}
Dualizing the definition of continuous functions, we obtain
\Emph{co-continuous} functions.

We also can dualize the definition of inductively defined sets.

Let $\fund{F}{\wp(U)}{\wp(U)}$ be a co-continuous function with respect to the
complete lattice $\bigl(\wp(\cU), \sseq\bigr)$.

The set \Emph{coinductively defined by $F$} is given by
\[
	\gfp F = \biginters_{i \in \Nset} F^i(U).
\]
\end{slide}

\begin{slide} \heading{Coinductively Defined Sets, cont.}
\textbf{Example:} consider again the BNF rule
\[
    V ::= 0 \ | \ 1V
\]
and the associated co-continuous function
$\fund{\bar{V}}{\{0,1\}^\star}{\{0,1\}^\omega}$
\[
    \bar{V}(S) \defeq \{ 0 \} \union \{\, 1s \mid s \in S \,\}.
\]
We have
\(
  \gfp \bar{V}
     = \{\, 1^n0 \mid n \in \Nset \,\} \union \{ 1^\omega \}
     = \lfp \bar{V} \union \{ 1^\omega \}
\).

\textbf{Note:} when computing $\lfp \bar{V}$, $\bar{V}^i(\emptyset)$
contains the strings
that will \Emph{definitely be in} the final result.
When computing $\gfp \bar{V}$, strings that are not in
$\bar{V}^i\bigl({\{0,1\}^\omega}\bigr)$
will \Emph{never be in} the final result.
\end{slide}

\begin{slide} \heading{Rules and Rule Instances}

We have defined our big-step semantics by means of \Emph{rules}
such as
\[
\prooftree
      \langle c_0, \sigma \rangle
        \rightarrow
          \sigma''
\qquad
      \langle c_1, \sigma'' \rangle
        \rightarrow
          \sigma'
\justifies
      \langle c_0 ; c_1, \sigma \rangle
        \rightarrow
          \sigma'
\endprooftree
\]

By instantiating the meta-variables, a rule can spawn infinitely many
\Emph{rule instances}, such as
\[
\footnotesize
\prooftree
      \bigl\langle x := 0, \{ z = 2 \} \bigr\rangle
        \rightarrow
          \{ x = 0, z = 2 \}
\qquad
      \bigl\langle y := 1, \{ x = 0, z = 2 \} \bigr\rangle
        \rightarrow
          \{ x = 0, y = 1, z = 2 \}
\justifies
      \bigl\langle x := 0 ; y := 1, \{ z = 2 \} \bigr\rangle
        \rightarrow
          \{ x = 0, y = 1, z = 2 \}
\endprooftree
\]

Notice that the instantiation chosen above \emph{makes sense}, but this
is not required by the definition or rule instance.
This only requires the meta-variables to be \Emph{consistently replaced}
by objects of the right kind.
\end{slide}

\begin{slide} \heading{Building Trees of Rule Instances}

Let the universe $U$ be the set $\Nset^2$ of pairs of natural numbers.
The subset
\[
  \cF
    \defeq
      \bigl\{\,
        (n, m)
      \bigm|
        \text{$m$ is the $n$-th Fibonacci number}
      \,\bigr\}
\]
can be defined by the rules:
\[
\prooftree
\justifies
  (1, 1) \in \cF
\endprooftree
\qquad
\prooftree
\justifies
  (2, 1) \in \cF
\endprooftree
\qquad
\prooftree
  (n, m) \in \cF \quad (n+1, m_1) \in \cF
\justifies
  (n+2, m+m_1) \in \cF
\endprooftree
\]
An \Emph{perfectly legal} instance of the third rule is
\[
\prooftree
  (17, 2) \in \cF \quad (18, 2) \in \cF
\justifies
  (19, 4) \in \cF
\endprooftree
\]
\end{slide}

\begin{slide} \heading{Building Trees of Rule Instances, cont.}

Here is a tree constituting the proof that $5$ is the $5$-th Fibonacci
number:
\[
\prooftree
  \prooftree
    \prooftree
    \justifies
      (1, 1) \in \cF
    \endprooftree
    \quad
    \prooftree
    \justifies
      (2, 1) \in \cF
    \endprooftree
  \justifies
    (3, 2) \in \cF
  \endprooftree
  \quad
  \prooftree
    \prooftree
    \justifies
      (2, 1) \in \cF
    \endprooftree
  \quad
    \prooftree
      \prooftree
      \justifies
        (1, 1) \in \cF
      \endprooftree
      \quad
      \prooftree
      \justifies
        (2, 1) \in \cF
      \endprooftree
    \justifies
      (3, 2) \in \cF
    \endprooftree
  \justifies
    (4, 3) \in \cF
  \endprooftree
\justifies
  (5, 5) \in \cF
\endprooftree
\]

Of course, building the tree is only possible if the right instances
are chosen and put at the right place.
\end{slide}

\begin{slide} \heading{Inductive Definitions Again}

We do not need to build the tree explicitly, \Emph{if we only care
about the end result and not with its justification}.
Consider the rules again:
\[
\prooftree
\justifies
  (1, 1) \in \cF
\endprooftree
\qquad
\prooftree
\justifies
  (2, 1) \in \cF
\endprooftree
\qquad
\prooftree
  (n, m) \in \cF \quad (n+1, m_1) \in \cF
\justifies
  (n+2, m+m_1) \in \cF
\endprooftree
\]
The set of all their instances can be expressed as
\[
F \defeq
\left\{
\prooftree
  \emptyset
\justifies
  (1, 1)
\endprooftree
\right\}
\union
\left\{
\prooftree
  \emptyset
\justifies
  (2, 1)
\endprooftree
\right\}
\union
\Biggl\{\,
\prooftree
  \bigl\{ (n, m), (n+1, m_1) \bigr\}
\justifies
  (n+2, m+m_1)
\endprooftree
\Biggm|
  n, m, m_1 \in \Nset
\,\Biggr\}
\]
The \Emph{induced operator} $\fund{\bar{F}}{\wp(\Nset^2)}{\wp(\Nset^2)}$ is
\[
  \bar{F}(X)
    \defeq
      \bigl\{  (1, 1), (2, 1) \bigr\}
        \union
          \Bigl\{\,
            (n+2, m+m_1)
          \Bigm|
            \bigl\{ (n, m), (n+1, m_1) \bigr\} \sseq X
          \,\Bigr\}
\]
\end{slide}

\begin{slide} \heading{Inductive Definitions Again, cont.}

\textbf{Exercise:} prove the following:
\begin{enumerate}
\item
\(
  \bigl(\wp(\Nset^2), \emptyset, \Nset^2, \inters, \union\bigr)
\)
is a complete lattice;
\item
the ordering of
\(
  \bigl(\wp(\Nset^2), \emptyset, \Nset^2, \inters, \union\bigr)
\)
is `$\sseq$';
\item
$\bar{F}$ is continuous over 
\(
  \bigl(\wp(\Nset^2), \sseq\bigr)
\).
\end{enumerate}

It can then be proved that \Emph{the smallest set $|F|$ that satisfies
the rules $F$ is the least fixpoint of $\bar{F}$, which, in turn, is $\cF$}:
\[
  \cF = |F| = \lfp \bar{F}
    = \lub_{\sseq} \bigl\{\, \bar{F}^n(\emptyset) \bigm| n \in \Nset \,\bigr\}
    = \bigunion_{n \in \Nset} \bar{F}^n(\emptyset)
\]

By iterating the least fixpoint construction we can obtain as many Fibonacci
numbers as desired, without explicitly building trees.
\end{slide}

\begin{slide} \heading{Finite Big-Step Semantics Trees}

We start by considering \Emph{finite} trees only:
\begin{itemize}
\item
for the \Emph{concrete} semantics this means that we disregard
infinite computations (as these are captured by infinite trees);
\item
for the \Emph{abstract} semantics this means that we lose the ability
to assign an abstract meaning to many programs
(here included programs that express only finite computations!).
\end{itemize}

With these limitation in mind, both the concrete and the abstract
semantics will be defined \Emph{inductively}.
\end{slide}

\begin{slide} \heading{Finite Big-Step Semantics Trees, cont.}

Let $\cR$ be the set of the concrete or abstract rule instances for IMP.

The set $\cT_\mathrm{f}$ of \Emph{finite semantics trees built from $\cR$}
can be defined \Emph{inductively} as the \Emph{smallest} set such that
whenever
$\{ \theta_1, \ldots, \theta_n \} \sseq \cT_\mathrm{f}$
and
\[
  \prooftree
    \bigl\{ \troot(\theta_1), \ldots, \troot(\theta_n) \bigr\}
  \justifies
     \eta
  \endprooftree
    \in \cR,
\]
for some $n \geq 0$,
then
\[
  \prooftree
    \theta_1, \ldots, \theta_n
  \justifies
     \eta
  \endprooftree
    \in \cT_\mathrm{f}.
\]
\end{slide}

\begin{slide} \heading{Finite Big-Step Semantics Trees, cont.}

Let $\cR^\flat$ denote the set of concrete rule instances for IMP
and $\cT^\flat_\mathrm{f}$ the
set of trees built from $\cR^\flat$.

Let $\cR^\sharp$ denote the set of abstract rule instances for IMP
and $\cT^\sharp_\mathrm{f}$ the
set of trees built from $\cR^\sharp$.

We will now define a relation
$\reld{\propto}{\cT^\flat_\mathrm{f}}{\cT^\sharp_\mathrm{f}}$.
If $\tau^\flat \in \cT^\flat_\mathrm{f}$
and $\tau^\sharp \in \cT^\sharp_\mathrm{f}$,
then \Emph{$\tau^\flat \propto \tau^\sharp$} means that
then \Emph{$\tau^\flat$ is correctly approximated by $\tau^\sharp$}.
\end{slide}

\begin{slide} \heading{Safety for Finite Big-Step Semantics Trees}

Let $\tau^\flat \in \cT^\flat_\mathrm{f}$
and $\tau^\sharp \in \cT^\sharp_\mathrm{f}$.

Then $\tau^\flat \propto \tau^\sharp$ if and only if
\begin{enumerate}
\item
$\troot(\tau^\flat) \propto \troot(\tau^\sharp)$; and
\item
for each immediate subtree $\tau^\flat_i$ of $\tau^\flat$
there exists a immediate subtree $\tau^\sharp_j$ of $\tau^\sharp$
such that $\tau^\flat_i \propto \tau^\sharp_j$.
\end{enumerate}

In words, \Emph{every immediate subtree of $\tau^\flat$ is approximated by
some immediate subtree of $\tau^\sharp$}.

Notice that one immediate subtree in $\tau^\sharp$ may be related by
`$\propto$' to none, one or more than one immediate subtree of $\tau^\flat$.

We now need to define `$\propto$' for the trees' nodes.
\end{slide}

\begin{slide} \heading{Safety for Finite Big-Step Semantics Trees, cont.}

Since we are adopting a \Emph{Galois connection approach} it is natural
to define the notion of \Emph{sound approximation} in terms of
the abstraction or of the concretization function.
We choose the latter for no particular reason.

We thus define
\begin{align*}
  r^\flat \propto r^\sharp
    &\defiff
      r^\flat \in \gamma(r^\sharp),
&&\text{for each $r^\flat \in \mv{Int}^\flat$ and $r^\sharp \in \mv{Int}^\sharp$;} \\
  t^\flat \propto t^\sharp
    &\defiff
      t^\flat \in \gamma(t^\sharp),
&&\text{for each $t^\flat \in \mv{Bool}$ and $t^\sharp \in \mv{Bool}^\sharp$;} \\
  \sigma^\flat \propto \sigma^\sharp
    &\defiff
      \sigma^\flat \in \gamma(\sigma^\sharp),
&&\text{for each $\sigma^\flat \in \Stores_V$ and $\sigma^\sharp \in \Stores^\sharp_V$.}
\end{align*}
\end{slide}

\begin{slide} \heading{Safety for Finite Big-Step Semantics Trees, cont.}

Let $\sigma^\flat \in \Stores^\flat_V$, $\sigma^\sharp \in \Stores^\sharp_V$,
$r^\flat \in \mv{Int}^\flat$, and $r^\sharp \in \mv{Int}^\sharp$.
Then
\[
  \bigl(\langle a_1, \sigma^\flat  \rangle \rightarrow r^\flat\bigr)
    \propto
  \bigl(\langle a_2, \sigma^\sharp \rangle \rightsquigarrow r^\sharp\bigr)
\]
if and only if
$a_1 = a_2$,
$\sigma^\flat \propto \sigma^\sharp$, and
$r^\flat \propto r^\sharp$.

Similarly, if
$\sigma^\flat \in \Stores^\flat_V$, $\sigma^\sharp \in \Stores^\sharp_V$,
$t^\flat \in \mv{Bool}^\flat$, and $t^\sharp \in \mv{Bool}^\sharp$,
then
\[
  \bigl(\langle b_1, \sigma^\flat  \rangle \rightarrow t^\flat\bigr)
    \propto
  \bigl(\langle b_2, \sigma^\sharp \rangle \rightsquigarrow t^\sharp\bigr)
\]
if and only if
$b_1 = b_2$,
$\sigma^\flat \propto \sigma^\sharp$, and
$t \propto t^\sharp$.

Finally, if
$\sigma^\flat, \rho^\flat \in \Stores^\flat_V$, and
$\sigma^\sharp, \rho^\sharp \in \Stores^\sharp_V$,
then
\[
  \bigl(\langle c_1, \sigma^\flat  \rangle \rightarrow \rho^\flat\bigr)
    \propto
  \bigl(\langle c_2, \sigma^\sharp \rangle \rightsquigarrow \rho^\sharp\bigr)
\]
if and only if
$c_1 = c_2$,
$\sigma^\flat \propto \sigma^\sharp$, and
$\rho^\flat \propto \rho^\sharp$.
\end{slide}

\begin{slide} \heading{Safety for Finite Big-Step Semantics Trees, cont.}
%The objective is now to prove the following results:

\textbf{Theorem 1}:
\begin{quote}
  For each command $c \in \mv{Com}$,
  for each concrete store $\sigma^\flat  \in \Stores_V^\flat$,
  for each tree $\tau^\flat \in \cT^\flat_\mathrm{f}$,
  such that
  \(
    \troot(\tau^\flat)
      = \bigl(\langle c, \sigma^\flat \rangle \rightarrow \rho^\flat\bigr)
  \),
  for each abstract store $\sigma^\sharp \in \Stores^\sharp_V$
  such that $\sigma^\flat \propto \sigma^\sharp$,
  for each tree $\tau^\sharp \in \cT^\sharp_\mathrm{f}$
  such that
  \(
    \troot(\tau^\sharp)
      = \bigl(\langle c, \sigma^\sharp \rangle
                \rightsquigarrow \rho^\sharp\bigr)
  \),
  we have $\tau^\flat \propto \tau^\sharp$.
\end{quote}
\textbf{Theorem 2}:
\begin{quote}
  For each command $c \in \mv{Com}$,
  for each concrete store $\sigma^\flat  \in \Stores_V^\flat$,
  for each tree $\tau^\flat \in \cT^\flat_\mathrm{f}$
  such that
  \(
    \troot(\tau^\flat)
      = \bigl(\langle c, \sigma^\flat \rangle \rightarrow \rho^\flat\bigr)
  \),
  there exists a tree $\tau^\sharp \in \cT^\sharp_\mathrm{f}$
  such that
  $\tau^\flat \propto \tau^\sharp$.
\end{quote}

\textbf{Course Project 1}:
Prove the above theorems.

\end{slide}

\begin{slide} \heading{Safety for Finite Big-Step Semantics Trees, cont.}

To prove Theorem 2, we have to establish that we have
enough abstract rules to cover all concrete rules.
(The proof can then be completed by induction on the height of the concrete
tree.)

More formally, we should prove that, for each concrete rule instance
\[
  \prooftree
    \bigl\{ s^\flat_1, \ldots, s^\flat_h \bigr\}
  \justifies
     s^\flat
  \endprooftree
    \in \cR^\flat
\]
there exist an abstract rule instance
\[
  \prooftree
    \bigl\{ s^\sharp_1, \ldots, s^\sharp_k \bigr\}
  \justifies
     s^\sharp
  \endprooftree
    \in \cR^\sharp
\]
such that (1)
$s^\flat \propto s^\sharp$,
and, (2),
for each $i \in \{ 1, \ldots, h \}$ there exists $j \in \{ 1, \ldots, k \}$
such that $s^\flat_i \propto s^\sharp_j$.

\end{slide}

\begin{slide} \heading{Safety for Finite Big-Step Semantics Trees, cont.}

Here is an example showing that concrete rules are adequately covered.

Let $m \in \mv{Int}$.  The rule instance
\[
\prooftree
\justifies
\langle m, \sigma \rangle
        \rightarrow m
\endprooftree
\qquad\text{ is covered by }\qquad
\prooftree
\justifies
\bigl\langle m, \alpha\bigl(\{\sigma\}\bigr) \bigr\rangle
        \rightsquigarrow \alpha\bigl(\{m\}\bigr)
\endprooftree
\]
since
\(
  \bigl(\langle m, \sigma \rangle \rightarrow m\bigr)
    \propto
  \bigl(\bigl\langle m, \alpha\bigl(\{\sigma\}\bigr) \bigr\rangle
          \rightsquigarrow \alpha\bigl(\{m\}\bigr)\bigr)
\).

This is so because $\sigma \propto \alpha\bigl(\{\sigma\}\bigr)$, that is,
$\sigma \in \gamma\bigl(\alpha\bigl(\{\sigma\}\bigr)\bigr)$, which is
true because $\alpha$ and $\gamma$ constitute a Galois connection
between $\wp(\Stores^\flat)$ and $\Stores^\sharp$.

Similarly, we have $m \propto \alpha\bigl(\{m\}\bigr)$, that is,
$m \in \gamma\bigl(\alpha\bigl(\{m\}\bigr)\bigr)$, which is
true because $\alpha$ and $\gamma$ constitute a Galois connection
between $\wp(\mv{Int}^\flat)$ and $\mv{Int}^\sharp$.

\end{slide}

\begin{slide} \heading{Safety for Finite Big-Step Semantics Trees, cont.}

Another example.

Let $x \in V \sseqf \mv{Var}$ and $\sigma \neq \error_V$.
The rule instance
\[
\prooftree
\justifies
      \langle x, \sigma \rangle
        \rightarrow
          \sigma(x)
\endprooftree
\qquad\text{ is covered by }\qquad
\prooftree
\justifies
      \bigl\langle x, \alpha\bigl(\{\sigma\}\bigr) \bigr\rangle
        \rightsquigarrow
          \alpha\bigl(\{\sigma\}\bigr)(x)
\endprooftree
\]
since
\(
  \bigl(\langle x, \sigma \rangle \rightarrow \sigma(x)\bigr)
    \propto
  \bigl(\bigl\langle x, \alpha\bigl(\{\sigma\}\bigr) \bigr\rangle
          \rightsquigarrow \alpha\bigl(\{\sigma\}\bigr)(x)\bigr)
\).

This is so because $\sigma \propto \alpha\bigl(\{\sigma\}\bigr)$
as before.

Moreover, we have
$\sigma(x) \propto \alpha\bigl(\{\sigma\}\bigr)(x)$
because $\sigma \in \gamma\bigl(\alpha\bigl(\{\sigma\}\bigr)\bigr)$,
which implies
\(
  \forall y \in V
    \itc
      \sigma(y) \in \gamma\bigl(\alpha\bigl(\{\sigma\}\bigr)(y)\bigr)
\),
and thus
$\sigma(x) \in \gamma\bigl(\alpha\bigl(\{\sigma\}\bigr)(x)\bigr)$.
\end{slide}

\begin{slide} \heading{Infinite Big-Step Semantics Trees}

Let $\cR$ be the set of the concrete or abstract rule instances for IMP.

The set $\cT_\omega$ of \Emph{infinite semantics trees built from $\cR$}
can be defined \Emph{coinductively} as the \Emph{greatest} set
of \emph{finitely branching trees of at most $\omega$ depth} such that
whenever, for some $n \geq 0$,
\[
  \prooftree
    \theta_1, \ldots, \theta_n
  \justifies
     \eta
  \endprooftree
    \in \cT_\omega
\]
there exists $\{ \theta_1, \ldots, \theta_n \} \sseq \cT_\omega$
such that
\[
  \prooftree
    \bigl\{ \troot(\theta_1), \ldots, \troot(\theta_n) \bigr\}
  \justifies
     \eta
  \endprooftree
    \in \cR.
\]
\end{slide}

\begin{slide} \heading{Infinite Big-Step Abstract Semantics Trees, cont.}

For correctness, each program (fragment) with a concrete semantics tree
must posses \Emph{at least one} abstract semantics tree.

Depending on the choice of abstract domain, there are programs
possessing a finite concrete tree for which \Emph{no finite abstract
semantics tree exists}.

This means that we do need to consider inifinite abstract semantics trees.

\textbf{Example}: Consider, for each $n \in \Nset$, the program
\[
    x = n; \kw{while} \evenop(x) \kw{do} x := x \divop 2
\]
This always has a finite concrete tree (it always terminates).
\end{slide}

\begin{slide} \heading{Infinite Big-Step Abstract Semantics Trees, cont.}
In contrast, and assuming an abstract domain for polarities,
every tree rooted at
\[
  \bigl\langle \{x = e\}, \kw{while} \evenop(x) \kw{do} x := x \divop 2 \bigr\rangle
    \rightsquigarrow \{ x = r^\sharp \},
\]
where $r^\sharp \in \mv{Int}^\sharp$, is infinite.

Let us abbreviate $\evenop(x)$ with $b$, $x := x \divop 2$ with $a$,
the command $\kw{while} \evenop(x) \kw{do} x := x \divop 2$
with $c$, and the abstract stores $\{ x = e \}$, $\{ x = m^\sharp \}$
and $\{ x = \top \}$ with $x_e$, $x_{m^\sharp}$ and $x_\top$, respectively.
\[
\footnotesize
\prooftree
      \langle b, x_e \rangle
        \rightsquigarrow
          \ttv
\quad
      \langle a, x_e \rangle
        \rightsquigarrow
          x_\top
\quad
      \prooftree
            \langle b, x_\top \rangle
              \rightsquigarrow
                \top
      \quad
            \langle a, x_e \rangle
              \rightsquigarrow
                x_\top
      \quad
            \prooftree
                  \cdots
            \justifies
                  \langle c, x_\top \rangle
                    \rightsquigarrow
                      x_o \sqcup x_{m^\sharp}
            \endprooftree
      \justifies
            \langle c, x_\top \rangle
              \rightsquigarrow
                x_o \sqcup x_{m^\sharp}
      \endprooftree
\justifies
      \langle c, x_e \rangle
        \rightsquigarrow
          x_o \sqcup x_{m^\sharp}
\endprooftree
\]
\end{slide}

\begin{slide} \heading{Infinite Big-Step Abstract Semantics Trees, cont.}

Both $m^\sharp = o$ ($x$ is odd) and $m^\sharp = \top$ ($x$ is unknown)
gives a well-formed infinite tree: for precision, we want to take the
\Emph{least tree}.

We thus define a \Emph{partial ordering on infinite trees}.
Let $\cT^\sharp_\omega$ be the set of infinite abstract trees
enriched with two distinct elements $\Omega$ (the \emph{undefined} tree)
and $\Delta$ (the \emph{top} tree).

Let $\tau^\sharp, \upsilon^\sharp \in \cT^\sharp_\omega$.
Then $\tau^\sharp \preceq \upsilon^\sharp$
if and only if
$\tau^\sharp = \Omega$
or $\upsilon^\sharp = \Delta$
or
\begin{enumerate}
\item
$\troot(\tau^\sharp) \preceq \troot(\upsilon^\sharp)$; and
\item
for each immediate subtree $\tau^\sharp_i$ of $\tau^\sharp$
there exists a immediate subtree $\upsilon^\sharp_j$ of $\upsilon^\sharp$
such that $\tau^\flat_i \preceq \upsilon^\sharp_j$.
\end{enumerate}

\textbf{Exercise:}  Define `$\preceq$' for tree nodes so that
the formula $\troot(\tau^\sharp) \preceq \troot(\upsilon^\sharp)$
makes sense.

\textbf{Exercise:}  Prove that $(\cT^\sharp_\omega, \preceq)$
is a complete lattice.
\end{slide}

\begin{slide} \heading{Infinite Big-Step Abstract Semantics Trees, cont.}

The correctness proof is done as in the case of finite abstract trees
(by induction on the height of the concrete tree, which is finite).

This works because infinite paths in the abstract tree correspond
to divergent computations that are \Emph{excluded by the concrete tree}.
\end{slide}

\end{document}

