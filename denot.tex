\externaldocument{prelim.tex}
% Importo tutti i riferimenti del file prelim.tex

\chapter{Semantica denotazionale}

\section{Introduzione}
% Aggiustare intro
In questo capitolo daremo una semantica denotazionale al linguaggio WHILE. Vedremo quindi la semantica di ogni elemento sintattico sotto forma di struttura matematica, per fare ciò definiremo varie funzioni che manipolano funzioni.
Ogni comando è perciò visto come uno \emph{State Transformer}, ovvero una funzione che prende uno store come argomento e, se per tale store è definita, allora mi restituisce un altro store. È quindi una funzione parziale, possiamo farla diventare totale ampliando il codominio con un nuovo simbolo $\bot$ che rappresenta il caso indefinito di stack o di ciclo infinito. Definiamo quindi il nuovo codominio $\Sigma_\bot = \Sigma \cup \left\{\bot\right\}$.
\begin{align*}
        &\pard{\ST}{\Sigma}{\Sigma}
        &\fund{\ST}{\Sigma}{\Sigma_\bot}
\end{align*}

Questo è dunque il dominio semantico dei comandi del linguaggio WHILE, vediamo ora i domini semantici per le espressioni e per i booleani che sono definiti in modo esattamente analogo.

\begin{align*}
        &\pard{P}{\Sigma}{B}
        &\fund{P}{\Sigma}{B_\bot}
\end{align*}

\begin{align*}
        &\pard{E}{\Sigma}{\Nset}
        &\fund{E}{\Sigma}{\Nset_\bot}
\end{align*}

Vediamo quindi le funzioni semantiche per il linguaggio WHILE, queste sono funzioni di ordine superiore che prendono come argomento una funzione e restituiscono un'altra funzione. Ciò accade sempre visto che si tratta di funzioni totali poiché tutti i comandi, i booleani e le espressioni hanno una semantica (tralasciamo per semplicità quelli sintatticamente errati).

\begin{gather*}
\fund{\calC \llbracket - \rrbracket}{\Com}{\ST}
\pts
\fund{\calB \llbracket - \rrbracket}{\BExp}{P}
\pts
\fund{\calE \llbracket - \rrbracket}{\AExp}{E}
\end{gather*}

\section{Definizione delle semantica di AExp}
Abbiamo visto come è fatta la funzione semantica per le espressioni ma ora dobbiamo vedere cosa accade nei vari casi di espressione. La semantica della costante associa semplicemente alla costante $n$ il suo valore numerico.
$$ \denotE{n}{s} \defeq \llbracket n \rrbracket_{\Integer}$$
La semantica della somma invece associa alla somma di due espressioni la somma della semantica di queste due espressioni, se tali semantiche esistono:
$$ \denotE{E_1 + E_2}{s} \defeq \begin{cases}
        \denotE{E_1}{s} + \denotE{E_2}{s}, &\text{se $\denotE{E_1}{s} \neq \bot$ e $\denotE{E_2}{s} \neq \bot$;}\\
      	\bot, &\text{altrimenti.}\\
        \end{cases}
$$
Vedremo in seguito come poter eliminare questa e altre funzioni per casi definendo nuove funzioni come per esempio la somma $\conadd$ in $\Nset_\bot$.

\section{Definizione della semantica di Com}
Ora vediamo la semantica dei vari comandi presenti nel linguaggio WHILE:

\begin{align*}
&\denotC{x \weq E}{s} \defeq \begin{cases}
        \bigassign{s}{x}{\denotE{E}{s}}, &\text{se $\denotE{E}{s} \neq \bot$;}\\
        \bot, &\text{altrimenti.}
        \end{cases}
\pts
&\denotC{\kw{skip}}{s} \defeq s;
\pts
&\denotC{C_1 ; C_2}{s} \defeq \begin{cases}
        \bot, &\text{se $\denotC{C_1}{s} = \bot$;}\\
        \denotC{C_2}{\denotC{C_1}{s}}, &\text{altrimenti.}\\
        \end{cases}
\pts
&\denotC{\condif{B}{C_1}{C_2}}{s} \defeq \begin{cases}
        \denotC{C_1}{s}, &\text{se $\denotB{B}{s} = \ttv$;}\\
        \denotC{C_2}{s}, &\text{se $\denotB{B}{s} = \ffv$;}\\
        \bot, &\text{altrimenti.}
        \end{cases}
\pts
\end{align*}

Daremo in seguito una semantica più precisa del comando while, per ora possiamo solo stabilire l'uguaglianza
$$\denotC{\while{B}{C}}{s} = \denotC{\condif{B}{(C ; \while{B}{C})}{\kw{skip}}}{s}$$

\section{Eliminazione funzioni per casi}
\label{sec:definizione di F}
Vediamo ora come ridefinire alcune di queste semantiche in modo da non avere funzioni per casi, per ottenere tale risultato basta ampliare la definizione di alcuni operatori. Prendiamo per esempio la semantica della somma, in tal caso basta definire l'operatore di somma tra $\Nset_\bot$ in modo che se sommo due numeri naturali ottengo lo stesso risultato che otterrei normalmente con la somma che conosciamo, mentre se sommo $\bot$ con un naturale o con un altro $\bot$ ottengo ancora $\bot$:
$$\fund{\conadd}{\Nset_\bot \times \Nset_\bot}{\Nset_\bot} \defeq \lambda n_1, n_2 \in \Nset_\bot . \begin{cases}
        n_1 + n_2, &\text{se $n_1, n_2 \in \Nset$;}\\
        \bot, &\text{altrimenti.}
        \end{cases}$$
Ora quindi possiamo semplicemente dire che la semantica della somma è:
$$\denotE{E_1 + E_2}{s} = \denotE{E_1}{s} \conadd \denotE{E_2}{s}$$

Esattamente allo stesso modo ora definiremo delle funzioni che mappano State Transformer in State Transformer e che non hanno bisogno di trattare in maniera separata i vari casi:
\begin{gather*}
\fund{\seq}{\ST \times \ST}{\ST} \\
\seq(f,g)(s) \defeq \begin{cases}
        \bot, &\text{se $f(s) = \bot$;}\\
        g(f(s)), &\text{altrimenti.}
        \end{cases}\\
\pts
\fund{\cond}{P \times \ST \times \ST}{\ST} \\
  \cond(p,f,g)(s) \defeq \begin{cases}
        f(s), &\text{se $p(s) = \ttv$;}\\
        g(s), &\text{se $p(s) = \ffv$;}\\
        \bot, &\text{altrimenti.}
        \end{cases}\\
\end{gather*}

Perciò rivediamo le semantiche di questi due comandi:
$$ \denotC{C_1 ; C_2}{s} = \seq(\calC \llbracket C_1 \rrbracket, \calC \llbracket C_2 \rrbracket)(s)$$

$$\denotC{\condif{B}{C_1}{C_2}}{s} = \cond(\calB \llbracket B \rrbracket, \calC \llbracket C_1 \rrbracket, \calC \llbracket C_2 \rrbracket)(s)$$.

Riscrivendo anche la semantica del comando while si ottiene un risultato molto interessante:
$$ \denotC{\while{B}{C}}{s} = \denotC{\condif{B}{(C ; \while{B}{C})}{\kw{skip}}}{s}$$
Ma poniamo ora $f = \cond(\calB \llbracket B \rrbracket, \seq(\calC \llbracket C \rrbracket, f), \mathrm{id})$ dove con $\mathrm{id}$ intendiamo la semantica del comando $\kw{skip}$.
Definiamo $\fund{F}{\ST}{\ST}$ come:
\[
F(f) \defeq \cond(\calB \llbracket B \rrbracket, \seq(\calC \llbracket C \rrbracket, f), \mathrm{id})
\]
e otteniamo quindi che $f$, la semantica di $\while{B}{C}$ è un punto fisso
di $F$.
A priori, non \`e detto che un tale punto fisso esista e, se esiste,
non \`e detto che ne esista solo uno.
In realt\`a, come vedremo, siccome $\fund{F}{\ST}{\ST}$ \`e continua
sul CPO $(ST, \sqsubseteq)$, $F$ ha un minimo punto fisso e proprio
quello corrisponde alla semantica di $\while{B}{C}$.

Iniziamo col definire $(ST, \sqsubseteq)$.

\begin{definizione} \summary{($(\ST, \sqsubseteq)$.)}
La relazione binaria $\reld{\sqsubseteq}{\ST}{\ST}$ \`e tale che,
per ogni $f, g \in \ST$,  $f \sqsubseteq g$ se e solo se
\[
  \forall s \in \Sigma \itc f(s)\convrg \implies f(s) = g(s).
\]
\end{definizione}
Informalmente, $f$ precede nell'ordine o \`e uguale a $g$ se e solo se
$g$ è non meno definita di $f$, e $f$ e $g$ coincidono quando sono entrambe
definite.

\begin{proposizione}
$(\ST, \sqsubseteq)$ \`e un CPO.
\end{proposizione}
\begin{proof}
La relazione $\sqsubseteq$ \`e chiaramente di ordinamento parziale su $\ST$:
la riflessivit\`a e l'antisimmetria  valgono banalmente;
per la transitivit\`a, se $f \sqsubseteq g$ e $g \sqsubseteq h$,
allora, per ogni $s \in \Sigma$, abbiamo $f(s)\convrg \implies g(s)\convrg$
e dunque $f(s)\convrg \implies f(s) = g(s) = h(s)$ da cui
$f \sqsubseteq h$.

La funzione $\pard{\bot}{\Sigma}{\Sigma}$, tale che $\bot(s)\divrg$
per ogni $s \in \Sigma$, \`e l'elemento minimo di $(\ST, \sqsubseteq)$:
infatti, per ogni $g \in \ST$ e per ogni $s \in \Sigma$, l'implicazione
$\bot(s)\convrg \implies \bot(s) = g(s)$ vale, essendo il suo
antecedente falso.

Sia ora $I \subseteq \Nset$ e consideriamo la catena
$X = \{f_i\}_{i \in I}$ dove $f_j \sqsubseteq f_k$ se $j \leq k$.
Se $|I| < \aleph_0$, allora,
per la Proposizione~\ref{prop:catena-finita-ha-elemento-massimo},
$f_{\max I} = \max_{\sqsubseteq} X = \lub_{\sqsubseteq} X$.
Se invece $|I| = \aleph_0$, definiamo $\pard{f}{\Sigma}{\Sigma}$
come segue, per ogni $s \in \Sigma$:
\[
  f(s)
    \defeq
      \begin{cases}
        s', &\text{se $\exists i \in I, s' \in \Sigma \st f_i(s) = s'$;}  \\
      \divrg, &\text{altrimenti.}
    \end{cases}
\]
La definizione \`e ben data: per ogni $i, j \in I$ abbiamo $i \leq j$
o $j \leq i$;  se avessimo $f_i(s) = s'$ e $f_j(s) = s''$ dovrebbe
necessariamente essere $s' = s''$.
Resta da mostrare che $f = \lub_{\sqsubseteq} X$.
Sia $j \in I$: per definizione di $f$ abbiamo che
\[
  \forall s \in \Sigma \itc f_j(s)\convrg \implies f_j(s) = f(s),
\]
e dunque $f_j \sqsubseteq f$, da cui $f$ \`e un upper bound per $X$.
Sia ora $g$ un upper bound per $X$, ovvero
\begin{equation}
\label{eq:denot-cpo-g-ub}
  \forall h \in I
    \itc \forall s \in \Sigma \itc f_h(s)\convrg \implies f_h(s) = g(s).
\end{equation}
Sia $s \in \Sigma$, e supponiamo che $f(s)\convrg$;
dunque esiste $i \in I$ tale che $f_i(s)\convrg$ e $f(s) = f_i(s)$.
Per la \eqref{eq:denot-cpo-g-ub} possiamo quindi scrivere
$f(s) = f_i(s) = g(s)$ e quindi
\[
  \forall s \in \Sigma \itc f(s)\convrg \implies f(s) = g(s),
\]
ovvero $f \sqsubseteq g$ e $f = \lub_{\sqsubseteq} X$.

$(\ST, \sqsubseteq)$ \`e dunque un CPO, avendo elemento minimo
e il least upper bound di ogni catena.
\end{proof}

\begin{definizione}
Sia $\while{B}{C}$ un ciclo while qualsiasi.
La funzione $\fund{F_{B,C}}{\ST}{\ST}$ \`e definita,
per ogni $f \in \ST$, da
\[
  F_{B,C}(f)
    \defeq
      \cond\bigl(\calB \llbracket B \rrbracket,
                         \seq(\calC \llbracket C \rrbracket, f),
                         \mathrm{id}\bigr)
\]
\end{definizione}

\begin{teorema}
La funzione $\fund{F_{B,C}}{\ST}{\ST}$ è continua.
\end{teorema}

\begin{proof}
Nel seguito assumeremo che $B$ e $C$ siano definiti dal contesto e scriveremo
$F$ invece di $F_{B,C}$.

Iniziamo mostrando che $\fund{\seq}{\ST \times \ST}{\ST}$
è monotona sul secondo argomento.
Siano $f, g_1, g_2 \in \ST$ arbitrari e tali che
$g_1 \sqsubseteq g_2$;  dobbiamo mostrare che
\[
  \seq(f, g_1) \sqsubseteq \seq(f, g_2).
\]
Sia $s \in \Sigma$ arbitrario.
Se $\seq(f, g_1)(s)$ \`e indefinito, non c'\`e nulla da dimostrare.
Se invece $\seq(f, g_1)(s)$ \`e definito,
allora anche $f(s)$, $g_1\bigl(f(s)\bigr)$ e $g_2\bigl(f(s)\bigr)$
lo sono;  inoltre
\[
    \seq(f, g_1)(s) = g_1\bigl(f(s)\bigr)
                    = g_2\bigl(f(s)\bigr)
  = \seq(f, g_2)(s)
\]
come volevasi.

Mostriamo ora che $\fund{\cond}{P \times \ST \times \ST}{\ST}$
è monotona sul secondo argomento.
Siano $p \in P$ e $f_1, f_2, g \in \ST$ arbitrari e tali che
$f_1 \sqsubseteq f_2$;  dobbiamo mostrare che
\[
  \cond(p, f_1, g) \sqsubseteq \cond(p, f_2, g).
\]
Consideriamo $s \in \Sigma$ arbitrario.
Se $\cond(p, f_1, g)(s)$ \`e indefinito, non c'\`e nulla da dimostrare.
Se invece $\cond(p, f_1, g)(s)$ \`e definito,
allora ci sono due possibilità:
\begin{description}
\item[$p(s) = \false:$] dunque
\(
  \cond(p, f_1, g)(s) = g(s) = \cond(p, f_2, g)(s)
\).
\item[$p(s) = \true:$], dunque
\(
  \cond(p, f_1, g)(s) = f_1(s) = f_2(s) = \cond(p, f_2, g)(s),
\)
dove $f_1(s) = f_2(s)$ è vera per ipotesi.
\end{description}
La monotonia di $\fund{\cond}{P \times \ST \times \ST}{\ST}$
sul secondo argomento è quindi dimostrata, e cos\`\i la monotonia
di $\fund{F}{\ST}{\ST}$ in quanto composizione di $\cond$ e $\seq$.

Sia ora $X \subseteq \ST$ una catena e mostriamo che l'insieme
\[
  F(X) \defeq \bigl\{\, F(f) \bigm| f \in X \,\bigr\}
\]
\`e una catena.
Per cominciare, abbiamo che $\bigl|F(X)\bigr| \leq |X| \leq \aleph_0$.
Prendiamo poi due elementi qualsiasi $h_1, h_2 \in F(X)$:
devono esistere $f_1, f_2 \in X$ tali che $h_1 = F(f_1)$ e $h_2 = F(f_2)$.
Essendo $X$ una catena, deve essere $f_1 \sqsubseteq f_2$
o $f_2 \sqsubseteq f_1$: nel primo caso, per monotonia di $F$, abbiamo
$h_1 = F(f_1) \sqsubseteq F(f_2) = h_2$;
nel secondo caso,
$h_2 = F(f_2) \sqsubseteq F(f_1) = h_1$.
Dunque $X$ \`e totalmente ordinato ed \`e una catena.
Essendo $\ST$ \`e un CPO, abbiamo mostrato che esiste $\lub_{\sqsubseteq} F(X)$.
Più precisamente, \`e facile vedere, usando i risultati di monotonia
appena dimostrati, che $\lub_{\sqsubseteq} F(X) = h$,
dove $h \in \ST$ è definito da
\begin{align*}
  h &\defeq \cond\bigl(p, \seq(g, \lub_{\sqsubseteq} X), \id\bigr), \\
  p &\defeq \calB \llbracket B \rrbracket, \\
  g &\defeq \calC \llbracket C \rrbracket.
\end{align*}
Infatti, sia $f \in X$:
\begin{align*}
  F(f)
    &=
      \cond\bigl(p, \seq(g, f), \id\bigr) \\
    &\sqsubseteq
      \cond\bigl(p, \seq(g, \lub_{\sqsubseteq} X), \id\bigr) \\
    &= h,
\end{align*}
e quindi $h$ \`e un upper bound per $F(X)$.
Sia ora $u$ un qualsiasi upper bound per $F(X)$,
dunque $f \in X$ implica $F(f) \sqsubseteq u$:
mostreremo che $h \sqsubseteq u$.
Sia $s \in \Sigma$:
se $h(s)\divrg$ non c'\`e nulla da dimostrare;
altrimenti sappiamo che $X \neq \emptyset$ e
ci sono due casi:
\begin{description}
\item[$p(s) = \false:$]
dunque $h(s) = s$ e, se $f \in X$, allora $F(f)(s) = s$,
ma, poich\'e $F(f) \sqsubseteq u$, anche $u(s) = s$;
\item[$p(s) = \true:$]
dunque, se $f \in X$, abbiamo
\[
  h(s) = \seq(g, \lub_{\sqsubseteq} X)(s) = \seq(g, f)(s) = u(s).
\]
\end{description}
Abbiamo quindi
\[
  \lub_{\sqsubseteq} F(X) = h = F(\lub_{\sqsubseteq} X),
\]
dove la seconda uguaglianza discende immediatamente dalle definizioni
di $h$ e di $F$.
\end{proof}

Per il Teorema~\ref{teo:Tarski} (di Tarski)
$F_{B,C}$ possiede un minimo punto fisso che si ottiene come
limite delle iterate a partire da $\bot$.
Possiamo quindi definire
\[
  \calC \llbracket \kw{while} B \kw{do} C \rrbracket
    \defeq \lfp F_{B,C}
    = \lub_{\sqsubseteq} \bigl\{\, F_{B,C}^n(\bot) \bigm| n \in \Nset \,\bigr\}.
\]

E' importante osservare che la scelta del \emph{minimo} punto fisso,
e non gi\`a di un qualsiasi punto fisso, \`e fondamentale
per la semantica del while.
Si consideri, ad esempio, il ciclo $\while{i \neq 10}{i := i+1}$.
Abbiamo, per ogni $s \in \Sigma$,
\[
  \denotC{\while{i \neq 10}{i := i+1}}{s}
    \defeq
      \begin{cases}
        s[10/i], &\text{se $s(i) \leq 10$;} \\
        \divrg, &\text{altrimenti.}
      \end{cases}
\]
E' facile mostrare che la funzione $F_{B,C}$ ha infiniti altri punti fissi,
ad esempio $d \in \ST$ definito, per ogni $s \in \Sigma$, come
\begin{align}
  d(s)
    \defeq
      \begin{cases}
        s[10/i], &\text{se $s(i) \leq 10$;} \\
        s[666/i], &\text{altrimenti.}
      \end{cases}
\end{align}
D'altra parte, solo il minimo punto fisso esprime correttamente
le non terminazioni del while considerato.
I punti fissi non minimi, invece, rappresentano delle terminazioni
che in realt\`a non avvengono.

\begin{esercizio}
Si dimostri che $d$ \`e un punto fisso per $F_{B,C}$
con $B \equiv (i \neq 10)$ e $C \equiv (i := i+1)$.
\end{esercizio}
