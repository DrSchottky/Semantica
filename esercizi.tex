%% Logical quantifiers stuff
\newcommand*{\st}{\mathrel{.}}
\newcommand*{\itc}{\mathrel{:}}

\newcommand*{\Com}{\mathrm{Com}}
\newcommand*{\Exp}{\mathrm{Com}}

\newcommand*{\union}{\cup}

\newcommand*{\card}{\mathop{\#}\nolimits}

\providecommand*{\Nset}{\mathbb{N}}            % Naturals
\providecommand*{\Rset}{\mathbb{R}}            % Reals

\newcommand*{\multichoice}{\Alph{answerno}\addtocounter{answerno}{1}}
\newcommand*{\bad}{\multichoice}
\ifthenelse{\boolean{SOLUTIONS}}{%
\newcommand*{\good}{\textbf{\multichoice}}
\newcommand*{\answer}[1]{\par\noindent\textbf{Risposta:} #1\medskip}
}{
\newcommand*{\good}{\multichoice}
\newcommand*{\answer}[1]{}
}

\newcommand*{\stick}{{\,|\,}}

\begin{document}
\title{Semantica dei linguaggi di programmazione \\
       Prova in itinere}
\date{28 novembre 2014}
\maketitle

\section*{Esercizio 1}
Si estenda il linguaggio WHILE con il costrutto
\verb+REPEAT C UNTIL B+ che, lo si dia per assodato,
\`e equivalente a \verb+C; WHILE NOT B DO C+.
Per tale costrutto si definiscano:
\begin{itemize}
\item
regole/assiomi per la semantica operazionale ``big step'';
\item
regole/assiomi per la semantica operazionale ``small step'';
\item
equazione/i per la semantica denotazionale;
\item
regole/assiomi per la correttezza parziale nella logica di Floyd-Hoare;
\item
regole/assiomi per la correttezza totale nella logica di Floyd-Hoare.
\end{itemize}

\section*{Esercizio 2}

Lo xor bit a bit, $\fund{\oplus}{\Nset \times \Nset}{\Nset}$,
gode delle seguenti propriet\`a:
\begin{align*}
x \oplus (y \oplus z ) &= (x \oplus y) \oplus z,
  &\text{associativit\`a}; \\
x \oplus y &= y \oplus x,
  &\text{commutativit\`a}; \\
x \oplus 0 &= x,
  &\text{elemento neutro}; \\
x \oplus x &= 0,
  &\text{nilpotenza / ogni elemento \`e l'inverso di s\'e stesso}.
\end{align*}
Si dimostri, usando la logica di Floyd-Hoare, che la composizione
sequenziale
\verb+X := X XOR Y; Y := X XOR Y; X := X XOR Y+
realizza lo scambio dei valori tra le variabili \verb+X+ e \verb+Y+
se queste, all'inizio della computazione contengono, valori naturali.

\section*{Esercizio 3}

Si dimostri il seguente lemma sulla semantica operazionale ``small step'':
\begin{align*}
  \forall C_1, C_2 \in \Com &\itc \forall s, s' \in \Sigma
                             \itc \forall n \in \Nset \itc
    \langle C_1; C_2, s \rangle \longrightarrow^n s' \\
      &\implies \exists n_1, n_2 \in \Nset \st \exists s'' \in \Sigma \st \\
      &\qquad\qquad\qquad
        \langle C_1, s \rangle \longrightarrow^{n_1} s'' \land
        \langle C_2, s'' \rangle \longrightarrow^{n_2} s'' \land
        n = n_1 + n_2.
\end{align*}

\section*{Esercizio 4}
Si estenda il linguaggio WHILE con un costrutto
\verb+SWITCH+ dignitoso.
Per tale costrutto si definiscano:
\begin{itemize}
\item
sintassi;
\item
regole/assiomi per la semantica operazionale ``big step'';
\item
regole/assiomi per la semantica operazionale ``small step'';
\item
equazione/i per la semantica denotazionale;
\item
regole/assiomi per la correttezza parziale nella logica di Floyd-Hoare;
\item
regole/assiomi per la correttezza totale nella logica di Floyd-Hoare.
\end{itemize}
