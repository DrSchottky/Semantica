\chapter{Estensioni}

\section{Array} \marginpar{Segalini}

Aggiungere nel linguaggio while gli array.

\subsection{Sintassi}

Aggiungiamo alla sintassi del WHILE l'espressione che accede
all'$i$-esimo elemento dell'array:
\begin{align*}
  \AExp \ni E &::= \dots \vbar x(E). \\
\intertext{%
Aggiungiamo inoltre il comando d'assegnamento all'$i$-esimo elemento
dell'array:
}
  \Com \ni C &::= \dots \vbar x(E_1) := E_2.
\end{align*}

\subsection{Semantica informale}

Gli array in questo semplice linguaggio di programmazione (per scelta
di progettazione) sono dinamici: la loro dimensione è illimitata, ma
solo un numero finito di elementi è definito in ogni momento della
computazione.  Gli elementi non precedentemente assegnati dell'array
hanno il valore $0$.  Verranno mostrate due semantiche diverse: in una
l'accesso ad una variabile scalare $x$ con un'espressione della forma
$x(E)$ restituisce il valore contenuto in $x$, ignorando quindi $E$.
Nell'altra semantica, l'accesso $x(E)$ trasforma la variabile scalare
$x$ in un array con tutti gli elementi non definiti eccetto quello
nella posizione indicata da $E$, la quale assumerà il valore contenuto
in $x$ prima di tale accesso.

\subsection{Semantica formale}

\begin{definizione} \summary{(Array semantici.)}
Un \emph{array} (semantico) è una funzione da interi (indice
dell'array) a interi (contenuto della cella) in cui solo ad un insieme
finito di elementi del dominio corrisponde un numero diverso da $0$.
Formalmente:

\[
  \Lambda
    \defeq
      \Bigl\{\,
        \fund{a}{\Zset}{\Zset}
      \Bigm|
        \bigl\{\, x \in \Zset \bigm| a(x) \neq 0 \,\bigr\} \sseqf \Zset
      \,\Bigr\}.
\]
\end{definizione}

\begin{definizione} \summary{(Insieme degli stati con array.)}
Modifichiamo l'insieme degli store $\Sigma$ per includere gli array tra
le variabili del programma:

\[
  \pard{\Sigma}{\Var}{\Zset \union \Lambda}.
\]
\end{definizione}

\subsubsection{Denotazionale}
Semantica di $x(E)$ denotazionale senza conversione scalare-array:

\[
  \calE\llbracket x(E) \rrbracket(s) =
  \begin{cases}
    s(x), &\text{se $s(x) \in \Zset$; (x è uno scalare)} \\
    s(x)\bigl(\calE\llbracket E \rrbracket(s)\bigr), &\text{se $s(x)
      \in \Lambda \land \calE\llbracket E \rrbracket(s)\convrg$;} \\
    \divrg, &\text{se $s(x) \in \Lambda \land \calE\llbracket E
      \rrbracket(s)\divrg$;} \\
    \divrg, &\text{se $s(x)\divrg$.}
  \end{cases}
\]
\subsubsection{Operazionale small-step}
Semantica di $x(E)$ small step su di un array:

\begin{align*}
  &\prooftree
    \config{E}{s} \ssarrow \config{E'}{s'}
  \justifies
    \config{x(E)}{s} \ssarrow \config{x(E')}{s'}
  \endprooftree
\intertext{%
se $x$ è un array:
}
  &\prooftree
  \nohyp
  \justifies
    \config{x(n)}{s} \ssarrow \config{s(x)(n)}{s}
  \using
    s(x) \in \Lambda
  \endprooftree
\intertext{%
se $x$ è uno scalare:
}
  &\prooftree
  \nohyp
  \justifies
    \config{x(n)}{s} \ssarrow \config{0}{s'}
  \using
    s(x) \in \Zset \land s' = \substt{a}{x}
  \endprooftree
\end{align*}
\[
\text{dove } a = \lambdaop i.0.
\]
Semantica di $x(E)$ alternativa che trasforma lo scalare $x$ in un
array vuoto (composto interamente da 0), eccezione fatta per la
posizione $n$-ma la quale conterrà il valore di $x$:

\[
  \prooftree
  \nohyp
  \justifies
    \config{x(n)}{s} \ssarrow \config{s(x)}{s'}
  \using
    s(x) \in \Zset \land s' = \substt{a}{x}
  \endprooftree
\]
\begin{align*}
  \text{dove } a = \lambdaop i.
    \begin{cases}
      s(x), &\text{se $i = n$;} \\
      0,    &\text{altrimenti.}
    \end{cases}
\end{align*}
Semantica di $x(E_0) \weq E_1$ small step che permetta la conversione di uno
scalare in un array:

\begin{gather*}
  \prooftree
    \config{E_0}{s} \ssarrow \config{E_0'}{s'}
  \justifies
    \config{x(E_0) \weq E_1}{s} \ssarrow \config{x(E_0') \weq E_1}{s'}m
  \endprooftree \\[12pt]
  \prooftree
    \config{E_1}{s} \ssarrow \config{E_1'}{s'}
  \justifies
    \config{x(n) \weq E_1}{s} \ssarrow \config{x(n) \weq E_1'}{s'}
  \endprooftree
\end{gather*}
se $x$ è un array:
\[
  \prooftree
  \nohyp
  \justifies
    \config{x(n)\ \weq\ k}{s} \ssarrow \config{\kw{skip}}{s'}
  \using
    s(x) \in \Lambda \land s' = \substt{a}{x}
  \endprooftree
\]
\begin{align*}
  \text{dove } a = \lambdaop i.
  \begin{cases}
    k, &\text{se $i = n$;} \\
    s(x)(i), &\text{se $i \neq n$.}
  \end{cases}
\end{align*}
se $x$ è uno scalare (assegnamento distruttivo):
\[
  \prooftree
  \nohyp
  \justifies
    \config{x(n)\ \weq\ k}{s} \ssarrow \config{\kw{skip}}{s'}
  \using
    s(x) \in \Zset \land s' = \substt{a}{x}
  \endprooftree
\]
\begin{align*}
  \text{dove } a = \lambdaop i.
  \begin{cases}
    k, &\text{se $i = n$} \\
    0, &\text{se $i \neq n$}
  \end{cases}
\end{align*}
oppure conserviamo il valore contenuto in $x$ e lo salviamo
alla posizione~0 del nuovo array:
\[
  \prooftree
  \nohyp
  \justifies
    \config{x(n)\ \weq\ k}{s} \ssarrow \config{\kw{skip}}{s'}
  \using
    s(x) \in \Zset \land s' = \substt{a}{x}
  \endprooftree
\]
\begin{align*}
  \text{dove } a = \lambdaop i.
  \begin{cases}
    k, &\text{se $i = n$;} \\
    0, &\text{se $i \neq n$;} \\
    s(x), &\text{se $i = 0 \land h \neq 0$.}
  \end{cases}
\end{align*}

\section{Eccezioni} \marginpar{Segalini}

Aggungiamo le \emph{eccezioni} nel linguaggio WHILE. Le eccezioni in
questo linguaggio sono identificate da interi ($\kw{throw} E$).

Creiamo un nuovo insieme di possibili variabili utilizzabili in WHILE
chiamato $\Var'$. Questo insieme contiene tutte le variabili di $\Var$
ed una variabile speciale, non utilizzabile nei programmi, chiamata
$\$ecc$. Questa variabile contiene l'identificativo dell'eccezione
lanciata oppure è indefinita se la modalità di esecuzione è non
eccezionale.

\begin{align*}
  &\ecc \not \in \Var \\
  &\Var' \defeq \Var \union \{\ecc\}.
\end{align*}
l'insieme degli store è quindi definito come:

\[
  \pard{\Sigma}{\Var'}{\Zset}.
\]
\emph{N.B.:} nella sintassi delle espressioni ($E \in \AExp ::= \dots
\vbar x$) la $x$ appartiene a $\Var$ e non $\Var'$ (questo
impedisce l'uso della varibile $\ecc$ nel programma).

\subsection{Sintassi}
Aggiungiamo alla sintassi del WHILE i comandi try-catch e throw:

Siano $x \in \Var$ e $E \in \AExp$
\begin{align*}
\Com \ni C ::= \dots \vbar \kw{try} C_1 \kw{catch}(x) C_2\vbar \kw{throw} E.
\end{align*}

\subsection{Semantica formale}

\begin{align*}
  &\prooftree \config{E}{s} \ssarrow \config{E'}{s'}
  \justifies \config{\kw{throw} E}{s} \ssarrow 
    \config{\kw{throw} E'}{s'}
  \using s(\ecc) \divrg
  \endprooftree
  \pts
  &\prooftree
  \nohyp
  \justifies \config{\kw{throw} n}{s} \ssarrow 
    \config{\kw{skip}}{s \substt{n}{\ecc}}
  \using s(\ecc) \divrg
  \endprooftree
  \pts
  &\prooftree
  \nohyp
  \justifies \config{x \weq E}{s} \ssarrow \config{\kw{skip}}{s}
  \using s(\ecc) \convrg
  \endprooftree 
  \pts
  &\prooftree
  \nohyp
  \justifies \config{C_1 \weq C_2}{s} \ssarrow \config{\kw{skip}}{s}
  \using s(\ecc) \convrg
  \endprooftree
  \pts
  &\prooftree
  \nohyp
  \justifies \config{\kw{if} B \kw{then} C_1 \kw{else} C_2}{s} 
    \ssarrow \config{\kw{skip}}{s}
  \using s(\ecc) \convrg
  \endprooftree
  \pts
  &\prooftree
  \nohyp
  \justifies \config{\kw{while} B \kw{do} C}{s}
     \ssarrow \config{\kw{skip}}{s}
  \using s(\ecc) \convrg
  \endprooftree
  \pts
  &\prooftree
  \nohyp
  \justifies \config{\kw{try} C_1 \kw{catch}(x) C_2}{s}
     \ssarrow \config{\kw{skip}}{s}
  \using s(\ecc) \convrg
  \endprooftree
  \pts
  &\prooftree
  \nohyp
  \justifies \config{\kw{throw} E}{s} \ssarrow \config{\kw{skip}}{s}
  \using s(\ecc) \convrg
  \endprooftree
\end{align*}
Quando viene eseguito il blocco catch, per uscire dalla modalità
eccezionale, dobbiamo riportare la variabile $\ecc$ ad uno stato
indefinito. \'E necessario definire un operatore di
uccisione di variabile:
\[
  \fund{\unvar}{\Sigma \times \Var'}{\Sigma} 
\]
\[
  \forall s \in \Sigma \itc \forall x \in \Var' \itc 
    \forall y \in \Var' \itc (s \unvar x)(y) = \begin{cases}
        s(y),   &\text{se $x \neq y$;} \\
        \divrg, &\text{se $x = y$.}   
      \end{cases}
\]
\begin{align*}
  &\prooftree 
  \nohyp
  \justifies \config{\kw{try} \kw{skip} \kw{catch}(x) C}{s} \ssarrow
    \config{\kw{skip}}{s}
  \using s(\ecc) \divrg
  \endprooftree
  \pts
  &\prooftree \config{C_1}{s} \ssarrow \config{C_1'}{s'}
  \justifies \config{\kw{try} C_1 \kw{catch}(x) C_2}{s} \ssarrow
    \config{\kw{try} C_1' \kw{catch}(x) C_2}{s'}
  \using s(\ecc) \divrg \land\ s'(\ecc) \divrg
  \endprooftree
  \pts
  &\prooftree \config{C_1}{s} \ssarrow \config{C_1'}{s'}
  \justifies \config{\kw{try} C_1 \kw{catch}(x) C_2}{s} \ssarrow
    \config{C_2}{\bigl(s'\substt{s'(\ecc)}{x}\bigr) \unvar \ecc}
  \using s(\ecc) \divrg \land\ s'(\ecc) \convrg
  \endprooftree
\end{align*}
