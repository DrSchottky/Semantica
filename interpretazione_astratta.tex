\documentclass[a4paper, 12pt, oneside,fleqn]{book}

\usepackage{lmodern}
\usepackage[T1]{fontenc}
\usepackage[utf8]{inputenc} % Consente l'uso dei caratteri accentati italiani
\usepackage{amsmath,amssymb}
\usepackage{amsthm}
\usepackage{listings}
\usepackage{stmaryrd}
\usepackage{fancyhdr}
\pagestyle{fancy}

\renewcommand{\sectionmark}[1]{\markboth{#1}{}}
\renewcommand{\subsectionmark}[1]{\markright{\thesection\ #1}}
\fancyhf{} % rimuove l’attuale contenuto dell’intestazione
\fancyhead[L]{\bfseries\rightmark}
\fancyhead[R]{\bfseries\leftmark}
\fancyfoot[C]{\bfseries\thepage}
\renewcommand{\headrulewidth}{0.5pt}
\renewcommand{\footrulewidth}{0pt}
\addtolength{\headheight}{0.5pt} % riserva spazio per la linea
\fancypagestyle{plain}{%
\fancyhead{} % ignora, nello stile plain, le intestazioni
\renewcommand{\headrulewidth}{0pt} % e la linea
}
\frenchspacing % Forza Latex ad una spaziatura uniforme, invece di lasciare più spazio alla fine

%% alcune macro aggiunte da me (segalini)
%% in attesa di revisione e sistemazione
%% funzione non definita
\newcommand*{\convrg}{\downarrow}
%% funzione definita
\newcommand*{\divrg}{\uparrow}
%% configurazioni della semantica <skip, s>
\newcommand*{\config}[2]{\langle\ #1,\ #2\ \rangle}
%% configurazione della semantica denotazionale C[com](s)
\newcommand*{\denotC}[2]{\calC \llbracket #1 \rrbracket (#2)}
\newcommand*{\denotE}[2]{\calE \llbracket #1 \rrbracket (#2)}
\newcommand*{\denotB}[2]{\calB \llbracket #1 \rrbracket (#2)}
%% assegnamento del while :=
\newcommand*{\weq}{:=}
\newcommand*{\assign}[3]{#1[#2\mapsto#3]} % comodo per s[x -> n]
\newcommand*{\bigassign}[3]{#1\bigl[#2\mapsto#3\bigr]} % comodo per s[x -> n]
%% transizione della small step
\newcommand*{\ssarrow}{\longrightarrow}
\newcommand*{\nssarrow}{\not \longrightarrow}
%% operatore di sostituzione s[n/x]
\newcommand*{\subst}[3]{#1[\nicefrac{#2}{#3}]}
\newcommand*{\substt}[2]{[\nicefrac{#1}{#2}]}
% macro da verificare per la composizione funzionale (trombi)
\newcommand*{\compfun}[2]{\mathrm{#1}\mathop{\circ}\mathrm{#2}}
% e per if then else
\newcommand*{\condif}[3]{\kw{if}\mathrm{#1}\kw{then}\mathrm{#2}\kw{else}\mathrm{#3}}
\newcommand*{\while}[2]{\kw{while}\mathrm{#1}\kw{do}\mathrm{#2}}
% macro per $ecc
\newcommand*{\ecc}{\mathrm{\$ecc}}
% macro per operatore di anullamento var /
\newcommand{\unvar}{\mathrel{/}}

% Comando per scrivere con texttt dentro una formula matematica
\newcommand{\mactext}[1]{\text{\texttt{#1}}}

%% Proof trees.
\input prooftree
\newcommand*{\nohyp}{\phantom{x}}
\newcommand*{\rulename}[1]{\text{\footnotesize (#1)}\;}
\newcommand*{\pts}{\\[12pt]} % (marano) da controllare

%% C++.
\newcommand*{\Cplusplus}{{C\nolinebreak[4]\hspace{-.05em}\raisebox{.4ex}
{\tiny\bf ++}}}

%% BNF rules.
\newcommand*{\vbar}{\mathrel{\mid}}

%% Abstract syntax of the analyzed language.
\newcommand*{\Integer}{\mathrm{Integer}}
\newcommand*{\Bool}{\mathrm{Bool}}
\newcommand*{\Ghost}{\mathrm{Ghost}}
\newcommand*{\LE}{\mathrm{LE}}
\newcommand*{\LExp}{\mathrm{LExp}}
\newcommand*{\FV}{\mathrm{FV}}
\newcommand*{\Var}{\mathrm{Var}}
\newcommand*{\Pred}{\mathrm{Pred}}
\newcommand*{\AExp}{\mathrm{AExp}}
\newcommand*{\BExp}{\mathrm{BExp}}
\newcommand*{\Com}{\mathrm{Com}}
\newcommand*{\Exp}{\mathrm{Exp}}
\newcommand*{\ST}{\mathrm{ST}}

%% Sets of configurations
\newcommand*{\NTe}{\Gamma_\mathrm{e}}
\newcommand*{\NTb}{\Gamma_\mathrm{b}}
\newcommand*{\NTd}{\Gamma_\mathrm{d}}
\newcommand*{\NTg}{\Gamma_\mathrm{g}}
\newcommand*{\NTs}{\Gamma_\mathrm{s}}
\newcommand*{\NTk}{\Gamma_\mathrm{k}}
\newcommand*{\Te}{T_\mathrm{e}}
\newcommand*{\Tb}{T_\mathrm{b}}
\newcommand*{\Td}{T_\mathrm{d}}
\newcommand*{\Tg}{T_\mathrm{g}}
\newcommand*{\Ts}{T_\mathrm{s}}
\newcommand*{\Tk}{T_\mathrm{k}}

%% Lambda notation.
\newcommand*{\lambdaop}{\mathop{\lambda}\nolimits}

%% Sets of (no better specified) configurations.
\newcommand*{\NT}[1]{\Gamma_{#1}}
\newcommand*{\NTq}{\Gamma_q}
\newcommand*{\Tq}{T_q}

%% Denotable values.
\newcommand*{\dVal}{\mathrm{dVal}}
%% Storeable values.
\newcommand*{\sVal}{\mathrm{sVal}}
\newcommand*{\sval}{\mathrm{sval}}

%% Control modes.
\newcommand*{\CtrlMode}{\mathord{\mathrm{CtrlMode}}}
\newcommand*{\cm}{\mathrm{cm}}
%% Branch modes.
%\newcommand*{\BranchMode}{\mathord{\mathrm{BranchMode}}}
\newcommand*{\GotoMode}{\mathord{\mathrm{GotoMode}}}
\newcommand*{\SwitchMode}{\mathord{\mathrm{SwitchMode}}}
\newcommand*{\cmgoto}{\mathop{\mathrm{goto}}\nolimits}
\newcommand*{\cmswitch}{\mathop{\mathrm{switch}}\nolimits}
\newcommand*{\cmbreak}{\mathop{\mathrm{break}}\nolimits}
\newcommand*{\cmcontinue}{\mathop{\mathrm{continue}}\nolimits}
\newcommand*{\cmreturn}{\mathop{\mathrm{return}}\nolimits}
%% Exec mode.
\newcommand*{\cmexec}{\mathrm{exec}}
%% Value mode.
\newcommand*{\ValMode}{\mathord{\mathrm{ValMode}}}
\newcommand*{\cmvalue}{\mathop{\mathrm{value}}\nolimits}
%% Environment mode.
\newcommand*{\EnvMode}{\mathord{\mathrm{EnvMode}}}
\newcommand*{\cmenv}{\mathrm{env}}
%% Exception modes.
\newcommand*{\ExceptMode}{\mathord{\mathrm{ExceptMode}}}
\newcommand*{\cmexcept}{\mathrm{except}}

%% Control states.
\newcommand*{\CtrlState}{\mathord{\mathrm{CtrlState}}}
\newcommand*{\cs}{\mathord{\mathrm{cs}}}
%% Value states.
\newcommand*{\ValState}{\mathord{\mathrm{ValState}}}
\newcommand*{\valstate}{\upsilon}
%% Environment states.
%\newcommand*{\EnvState}{\mathord{\mathrm{EnvState}}}
%% Exception states.
\newcommand*{\ExceptState}{\mathord{\mathrm{ExceptState}}}
\newcommand*{\exceptstate}{\varepsilon}

%% Keywords.
\newcommand*{\kw}[1]{\mathop{\textup{\textbf{#1}}}}

\newcommand*{\bop}{\mathbin{\mathrm{bop}}}
%\newcommand*{\uop}{\mathop{\mathrm{uop}}}

%% Things that hold by definition.
\newcommand{\defrel}[1]{\mathrel{\buildrel \mathrm{def} \over {#1}}}
\newcommand{\defeq}{\defrel{=}}
\newcommand{\defiff}{\defrel{\Longleftrightarrow}}
%\newcommand{\defeq}{=}
%\newcommand{\defiff}{\Longleftrightarrow}

%% Divergence relation
\newcommand{\diverges}{\,\mathord{\buildrel \infty \over \longrightarrow}}

%% Special letters denoting sets and algebras.
\providecommand*{\Nset}{\mathbb{N}}             % Naturals
\providecommand*{\Qset}{\mathbb{Q}}             % Rationals
\providecommand*{\Zset}{\mathbb{Z}}             % Integers
\providecommand*{\Rset}{\mathbb{R}}             % Reals

%% Calligraphic alphabet.
\newcommand*{\calA}{\ensuremath{\mathcal{A}}}
\newcommand*{\calB}{\ensuremath{\mathcal{B}}}
\newcommand*{\calC}{\ensuremath{\mathcal{C}}}
\newcommand*{\calD}{\ensuremath{\mathcal{D}}}
\newcommand*{\calE}{\ensuremath{\mathcal{E}}}
\newcommand*{\calF}{\ensuremath{\mathcal{F}}}
\newcommand*{\calG}{\ensuremath{\mathcal{G}}}
\newcommand*{\calH}{\ensuremath{\mathcal{H}}}
\newcommand*{\calI}{\ensuremath{\mathcal{I}}}
\newcommand*{\calJ}{\ensuremath{\mathcal{J}}}
\newcommand*{\calK}{\ensuremath{\mathcal{K}}}
\newcommand*{\calL}{\ensuremath{\mathcal{L}}}
\newcommand*{\calM}{\ensuremath{\mathcal{M}}}
\newcommand*{\calN}{\ensuremath{\mathcal{N}}}
\newcommand*{\calO}{\ensuremath{\mathcal{O}}}
\newcommand*{\calP}{\ensuremath{\mathcal{P}}}
\newcommand*{\calQ}{\ensuremath{\mathcal{Q}}}
\newcommand*{\calR}{\ensuremath{\mathcal{R}}}
\newcommand*{\calS}{\ensuremath{\mathcal{S}}}
\newcommand*{\calT}{\ensuremath{\mathcal{T}}}
\newcommand*{\calU}{\ensuremath{\mathcal{U}}}
\newcommand*{\calV}{\ensuremath{\mathcal{V}}}
\newcommand*{\calW}{\ensuremath{\mathcal{W}}}
\newcommand*{\calX}{\ensuremath{\mathcal{X}}}
\newcommand*{\calY}{\ensuremath{\mathcal{Y}}}
\newcommand*{\calZ}{\ensuremath{\mathcal{Z}}}

%% Declarators for functions and relations.
\newcommand*{\reld}[3]{\mathord{#1}\subseteq#2\times#3}
\newcommand*{\fund}[3]{\mathord{#1}\colon#2\to#3}
\newcommand*{\pard}[3]{\mathord{#1}\colon#2\rightarrowtail#3}

%% Logical quantifiers stuff.
\newcommand{\st}{\mathrel{.}}
\newcommand{\itc}{\mathrel{:}}

%% Domain, codomain and range of a function.
\newcommand*{\dom}{\mathop{\mathrm{dom}}\nolimits}
%\newcommand*{\cod}{\mathop{\mathrm{cod}}\nolimits}
%\newcommand*{\range}{\mathop{\mathrm{range}}\nolimits}

%% Restriction of a function.
\newcommand*{\restrict}[1]{\mathop{\mid}\nolimits_{#1}}

%% Type of a constant.
\newcommand*{\type}{\mathop{\mathrm{type}}\nolimits}

%% Lubs, glbs, and fixed points.
\newcommand*{\lub}{\mathop{\mathrm{lub}}\nolimits}
\newcommand*{\glb}{\mathop{\mathrm{glb}}\nolimits}
\newcommand*{\lfp}{\mathop{\mathrm{lfp}}\nolimits}
\newcommand*{\gfp}{\mathop{\mathrm{gfp}}\nolimits}

%% Generic widening.
\newcommand*{\widen}{\mathbin{\nabla}}

%% Set theory.
\renewcommand{\emptyset}{\varnothing}
\newcommand*{\wpf}{\mathop{\wp_\mathrm{f}}\nolimits}

\newcommand*{\sseq}{\subseteq}
\newcommand*{\sseqf}{\mathrel{\subseteq_\mathrm{f}}}
\newcommand*{\sslt}{\subset}
%\newcommand*{\Sseq}{\supseteq}
%\newcommand*{\Ssgt}{\supset}

%\newcommand{\Nsseq}{\nsubseteq}

\newcommand*{\union}{\cup}
\newcommand*{\bigunion}{\bigcup}
%\newcommand*{\biginters}{\bigcap}
\newcommand*{\inters}{\cap}
\newcommand*{\setdiff}{\setminus}

\newcommand{\sset}[2]{{\renewcommand{\arraystretch}{1.2}
                      \left\{\,#1 \,\left|\,
                               \begin{array}{@{}l@{}}#2\end{array}
                      \right.   \,\right\}}}

%% Base sets.
\newcommand*{\true}{\mathrm{true}}
\newcommand*{\false}{\mathrm{false}}
\newcommand*{\ttv}{\mathrm{tt}}
\newcommand*{\ffv}{\mathrm{ff}}
\newcommand*{\divop}{\mathbin{/}}
\newcommand*{\modop}{\mathbin{\%}}
\newcommand*{\andop}{\mathbin{\textbf{\textup{and}}}}
\newcommand*{\orop}{\mathbin{\textbf{\textup{or}}}}
\newcommand*{\notop}{\mathop{\textbf{\textup{not}}}}

\newcommand*{\FI}{\mathop{\mathrm{FI}}\nolimits}
\newcommand*{\DI}{\mathop{\mathrm{DI}}\nolimits}
\newcommand*{\SL}{\mathop{\mathrm{SL}}\nolimits}
%\newcommand*{\match}{\mathop{\mathrm{match}}\nolimits}

\newcommand*{\Env}{\mathord{\mathrm{Env}}}
\newcommand*{\emptystring}{\mathord{\epsilon}}

%% Exceptions.
\newcommand*{\RTSExcept}{\mathord{\mathrm{RTSExcept}}}
\newcommand*{\rtsexcept}{\chi}
\newcommand*{\Except}{\mathord{\mathrm{Except}}}
\newcommand*{\except}{\xi}
\newcommand*{\none}{\mathtt{none}}
\newcommand*{\divbyzero}{\mathtt{divbyzero}}
\newcommand*{\stkovflw}{\mathtt{stkovflw}}
\newcommand*{\datovflw}{\mathtt{datovflw}}
\newcommand*{\memerror}{\mathtt{memerror}}
%\newcommand*{\inerror}{\mathtt{inerror}}
%\newcommand*{\nullptr}{\mathtt{nullptr}}
%\newcommand*{\outofboundsptr}{\mathtt{outofboundsptr}}

%% Flags for terminal configurations of catch clauses.
\newcommand*{\caught}{\mathtt{caught}}
\newcommand*{\uncaught}{\mathtt{uncaught}}

%% Static semantics.
\newcommand*{\TEnv}{\mathord{\mathrm{TEnv}}}
\newcommand*{\tinteger}{\mathrm{integer}}
\newcommand*{\tboolean}{\mathrm{boolean}}
\newcommand*{\trtsexcept}{\mathrm{rts\_exception}}

%% Memory structures.
\newcommand*{\Loc}{\mathord{\mathrm{Loc}}}
\newcommand*{\Ind}{\mathrm{Ind}}
\newcommand*{\Addr}{\mathrm{Addr}}
\newcommand*{\Map}{\mathrm{Map}}
%\newcommand*{\eMap}{\mathrm{eMap}}
\newcommand*{\Stack}{\mathord{\mathrm{Stack}}}
\newcommand*{\Mem}{\mathord{\mathrm{Mem}}}
\newcommand*{\stknew}{\mathop{\mathrm{new}_\mathrm{s}}\nolimits}
\newcommand*{\datnew}{\mathop{\mathrm{new}_\mathrm{d}}\nolimits}
\newcommand*{\txtnew}{\mathop{\mathrm{new}_\mathrm{t}}\nolimits}
\newcommand*{\heapnew}{\mathop{\mathrm{new}_\mathrm{h}}\nolimits}
\newcommand*{\heapdel}{\mathop{\mathrm{delete}_\mathrm{h}}\nolimits}
\newcommand*{\datcleanup}{\mathop{\mathrm{cleanup}_\mathrm{d}}\nolimits}
\newcommand*{\smark}{\mathop{\mathrm{mark}_\mathrm{s}}\nolimits}
\newcommand*{\sunmark}{\mathop{\mathrm{unmark}_\mathrm{s}}\nolimits}
\newcommand*{\slink}{\mathop{\mathrm{link}_\mathrm{s}}\nolimits}
\newcommand*{\sunlink}{\mathop{\mathrm{unlink}_\mathrm{s}}\nolimits}
\newcommand*{\asmark}{\mathop{\mathrm{mark}_\mathrm{s}^\sharp}\nolimits}
\newcommand*{\asunmark}{\mathop{\mathrm{unmark}_\mathrm{s}^\sharp}\nolimits}
\newcommand*{\aslink}{\mathop{\mathrm{link}_\mathrm{s}^\sharp}\nolimits}
\newcommand*{\asunlink}{\mathop{\mathrm{unlink}_\mathrm{s}^\sharp}\nolimits}
\newcommand*{\aswiden}{\mathop{\mathrm{widen}}\nolimits}
\newcommand*{\sm}{\dag}
\newcommand*{\fm}{\ddag}
\newcommand*{\topmost}{\mathop{\mathrm{tf}}\nolimits}
%% Short forms of \datcleanup, \sunmark, \sunlink for table.
\newcommand*{\datcleanupshort}{\mathop{\mathrm{cu}_\mathrm{d}}\nolimits}
\newcommand*{\sunmarkshort}{\mathop{\mathrm{um}_\mathrm{s}}\nolimits}
\newcommand*{\sunlinkshort}{\mathop{\mathrm{ul}_\mathrm{s}}\nolimits}

\newcommand*{\location}[1]{\mathord{#1 \; \mathrm{loc}}}
%\newcommand*{\saeval}{\mathop{\mathrm{aeval}}\nolimits}
%\newcommand*{\saupd}{\mathop{\mathrm{aupd}}\nolimits}
\newcommand*{\asupported}{\mathop{\mathrm{supported}^\sharp}\nolimits}
\newcommand*{\aeval}{\mathop{\mathrm{eval}^\sharp}\nolimits}
\newcommand*{\ceval}[1]{\mathop{\mathrm{eval}_{#1}}\nolimits}

%% Abstracts.
\newcommand*{\Abstract}{\mathord{\mathrm{Abstract}}}
\newcommand*{\abs}{\mathord{\mathrm{abs}}}

%% Integer part function.
\newcommand{\intp}{\mathop{\mathrm{int}}\nolimits}

%% Concrete functions and operations.
% Aritmethic
\newcommand*{\conadd}{\mathbin{\boxplus}}
\newcommand*{\consub}{\mathbin{\boxminus}}
\newcommand*{\conmul}{\mathbin{\boxdot}}
\newcommand*{\condiv}{\mathbin{\boxslash}}
\newcommand*{\conmod}{\mathbin{\boxbar}}
% Boolean
\newcommand*{\coneq}{\mathbin{\circeq}}
\newcommand*{\conineq}{\mathbin{\leqq}}
\newcommand*{\conneg}{\mathbin{\daleth}}
\newcommand*{\conor}{\mathbin{\triangledown}}
\newcommand*{\conand}{\mathbin{\vartriangle}}
\newcommand*{\bneg}{\mathop{\neg}\nolimits}

%% Abstract functions and operations.
% Domains
\newcommand*{\Sign}{\mathrm{Sign}}
\newcommand*{\AbBool}{\mathrm{AbBool}}

% Aritmethic
\newcommand*{\absuminus}{\mathop{\ominus}\nolimits}
\newcommand*{\absadd}{\mathbin{\oplus}}
\newcommand*{\abssub}{\mathbin{\ominus}}
\newcommand*{\absmul}{\mathbin{\odot}}
\newcommand*{\absdiv}{\mathbin{\oslash}}
\newcommand*{\absmod}{\mathbin{\obar}}
% Boolean
\newcommand*{\abseq}{\mathrel{\triangleq}}
\newcommand*{\absneq}{\mathrel{\not\triangleq}}
\newcommand*{\absleq}{\mathrel{\trianglelefteq}}
\newcommand*{\abslt}{\mathrel{\vartriangleleft}}
\newcommand*{\absgeq}{\mathrel{\trianglerighteq}}
\newcommand*{\absgt}{\mathrel{\vartriangleright}}
\newcommand*{\absneg}{\mathrel{\circleddash}}
\newcommand*{\absor}{\mathrel{\ovee}}
\newcommand*{\absand}{\mathrel{\owedge}}
% Figures
\newcommand*{\signtop}{\top}
\newcommand*{\signbot}{\bot}
\newcommand*{\signge}{\mathord{\geq}}
\newcommand*{\signgt}{\mathord{>}}
\newcommand*{\signle}{\mathord{\leq}}
\newcommand*{\signlt}{\mathord{<}}
\newcommand*{\signeq}{\mathord{=}}
\newcommand*{\signne}{\mathord{\neq}}

%% Summaries for theorem-like environments
\newcommand{\summary}[1]{\textrm{\textbf{\textup{#1}}}}

% Annotations in equations
\newcommand{\law}[1]{{\hspace*{\fill}\text{\footnotesize\rm{[#1]}}}}

%% Filter function extracting the relevant and irrelevant parts.
\newcommand*{\sel}{\mathop{\mathrm{sel}}\nolimits}
\newcommand*{\mem}{\mathop{\mathrm{mem}}\nolimits}

%% Modeling definite exceptions.
%\newcommand*{\None}{\mathrm{None}}

%% Strict Cartesian products.
\newcommand*{\stimes}{\otimes}
\newcommand*{\spair}[2]{{#1} \otimes {#2}}
%\newcommand*{\rstimes}{\rtimes}
%\newcommand*{\rspair}[2]{{#1} \rtimes {#2}}
%\newcommand*{\lstimes}{\ltimes}
%\newcommand*{\lspair}[2]{{#1} \ltimes {#2}}

%% chain
\newcommand*{\chain}{\mathop{\mathrm{chain}}\nolimits}


%% Additional macros for the extension for extra numeric types
%% Floating point types.
\newcommand*{\tfloat}{\mathrm{float}}
%% Numeric types
\newcommand*{\nType}{\mathrm{nType}}
\newcommand*{\nT}{\mathrm{nT}}

%% Additional macros for the extension to pointer and arrays:
%% Elementary types.
\newcommand*{\eType}{\mathrm{eType}}
\newcommand*{\eT}{\mathrm{eT}}
%% Elementary values.
%\newcommand*{\eValue}{\mathrm{eVal}}
%% Array types.
\newcommand*{\aType}{\mathrm{aType}}
\newcommand*{\aT}{\mathrm{aT}}
%% Record types.
\newcommand*{\rType}{\mathrm{rType}}
\newcommand*{\rT}{\mathrm{rT}}
%% Object types.
\newcommand*{\oType}{\mathrm{oType}}
\newcommand*{\oT}{\mathrm{oT}}
%% Function types.
\newcommand*{\fType}{\mathrm{fType}}
\newcommand*{\fT}{\mathrm{fT}}
%% Memory types.
\newcommand*{\mType}{\mathrm{mType}}
\newcommand*{\mT}{\mathrm{mT}}
%% Pointer types.
\newcommand*{\pType}{\mathrm{pType}}
\newcommand*{\pT}{\mathrm{pT}}
%% Offsets.
\newcommand*{\Offset}{\mathrm{Offset}}
\newcommand*{\nooffset}{\boxempty}
\newcommand*{\indexoffset}[1]{\mathopen{\boldsymbol{[}}{#1}\mathclose{\boldsymbol{]}}}
\newcommand*{\fieldoffset}[1]{\mathop{\boldsymbol{.}}{#1}}
%% Lvalues.
\newcommand*{\lValue}{\mathrm{LValue}}
\newcommand*{\lvalue}{\mathrm{lval}}
%% Rvalues.
\newcommand*{\rValue}{\mathrm{RValue}}
\newcommand*{\rvalue}{\mathrm{rval}}
%%
\newcommand*{\pointer}[1]{{#1}\boldsymbol{\ast}}
\newcommand*{\maddress}[1]{\mathop{\&}{#1}}
\newcommand*{\indirection}[1]{\mathop{\boldsymbol{\ast}}{#1}}
%%
\newcommand*{\locnull}{\mathord{l_\mathrm{null}}}
\newcommand*{\ptrmove}{{\mathop{\mathrm{ptrmove}}\nolimits}}
\newcommand*{\ptrdiff}{{\mathop{\mathrm{ptrdiff}}\nolimits}}
\newcommand*{\ptrcmp}{{\mathop{\mathrm{ptrcmp}}\nolimits}}
%%
\newcommand*{\arraysyntax}[3]{\kw{#1} {#2} \kw{of}\,{#3}}
\newcommand*{\arraytype}[2]{\arraysyntax{array}{#1}{#2}}
\newcommand*{\firstof}{{\mathop{\mathrm{firstof}}\nolimits}}
\newcommand*{\arrayindex}{\mathop{\mathrm{index}}\nolimits}
\newcommand*{\locindex}{\mathop{\mathrm{locindex}}\nolimits}
%%
\newcommand*{\recordsyntax}[3]{\kw{#1} {#2} \kw{of}\,{#3}}
\newcommand*{\recordtype}[2]{\recordsyntax{record}{#1}{#2}}
\newcommand*{\field}{\mathop{\mathrm{field}}\nolimits}
\newcommand*{\locfield}{\mathop{\mathrm{locfield}}\nolimits}
%%
\newcommand*{\NTo}{\Gamma_\mathrm{o}}
\newcommand*{\To}{T_\mathrm{o}}
\newcommand*{\NTl}{\Gamma_\mathrm{l}}
\newcommand*{\Tl}{T_\mathrm{l}}
%\newcommand*{\NTr}{\Gamma_\mathrm{r}}
%\newcommand*{\Tr}{T_\mathrm{r}}
%%
\newcommand*{\arraydatnew}{\mathop{\mathrm{newarray}_\mathrm{d}}\nolimits}
\newcommand*{\arraystknew}{\mathop{\mathrm{newarray}_\mathrm{s}}\nolimits}

\theoremstyle{definition}
\newtheorem{definizione}{Definizione}
\newtheorem{teorema}{Teorema}
\newtheorem{proposizione}{Proposizione}
\newtheorem{lemma}{Lemma}

\begin{document}

\title{Interpretazione astratta}
\author{
	Università degli studi di Parma \\
	Francesco Trombi \\
	Corso: Semantica dei linguaggi di programmazione \\
	Docente: Roberto Bagnara
}
\date{2014/15}
\maketitle
\tableofcontents

\newpage

\chapter{Astrazione del Linguaggio \tt{while}}

\section{Domini}

\subsection{Dominio concreto}
Prima di definire un dominio astratto per il linguaggio {\tt while}, si ricorda qual è il dominio concreto:
\begin{center}
	$ (\calP{(\Zset), \subseteq)} $
\end{center}
Questo dominio concreto è giustificato dal fatto che le variabili del linguaggio possono contenere solamente numeri interi.

\subsection{Dominio astratto degli interi}
[Inserire figura]\\
I simboli del dominio astratto sono tutti rispetto al numero $\mathrm{0}$.

\section{Funzione di concretizzazione $\gamma$}

\subsection{Definizione di $\gamma$}
\begin{center} $\gamma : A \rightarrow \calP({(\Zset), \subseteq)} $\end{center}

\subsection{Monotonia di $\gamma$}
Argomentiamo la monotonia di $\gamma$
\section{Funzione di astrazione $\alpha$}

\subsection{Definizione di $\alpha$}
\begin{center} $\alpha : A \rightarrow \calP({(\Zset), \subseteq)} $\end{center}

\subsection{Monotonia di $\alpha$}
Dimostriamo la monotonia di $\alpha$.

\section{Operazioni astratte}
Le operazioni astratte vanno definite seguendo le operazioni concrete; esse si applicano tra elementi del dominio astratto.

\subsection{Moltiplicazione astratta}

La moltiplicazione astratta è definita come segue:

\begin{center}
	$\mathrm{\otimes = \lambda a_1,a_2 \in A . \alpha(\gamma(a_1) \boxtimes \gamma(a_2))}$
\end{center}
con $\boxtimes$ moltiplicazione concreta, definita come segue:

\begin{center}
	$\mathrm{S_1 \boxtimes S_2 = \{ n_1 * n_2 \mid n_1 \in S_1, n_2 \in S_2 \}}$
\end{center}

\begin{center}
	\begin{tabular}{| c | c | c | c | c | c | c | c | c | }
		\hline
		$\otimes$ & $\top$ & $\leq$ & $\neq$ & $\geq$ & $<$ & $=$ & $>$ & $\bot$ \\
		\hline
		$\top$ & $\top$ & $\top$ & $\top$ & $\top$ & $\top$ & $=$ & $\top$ & $\bot$  \\
		\hline
		$\leq$ & $\top$ & $\geq$ & $\top$ & $\leq$ & $\geq$ & $=$ & $\leq$ & $\bot$\\
		\hline
		$\neq$ & $\top$ & $\top$ & $\neq$ & $\top$ & $\neq$ & $=$ & $\neq$ & $\bot$ \\
		\hline
		$\geq$ & $\top$ & $\leq$ & $\top$ & $\geq$ & $\leq$ & $=$ & $\geq$ & $\bot$ \\
		\hline
		$<$ & $\top$ & $\geq$ & $\neq$ & $\leq$ & $>$ & $=$ & $<$ & $\bot$ \\
		\hline
		$=$ & $=$ & $=$ & $=$ & $=$ & $=$ & $=$ & $=$ & $\bot$\\
		\hline
		$>$ & $\top$ & $\leq$ & $\neq$ & $\geq$ & $<$ & $=$ & $>$ & $\bot$\\
		\hline
		$\bot$ & $\bot$ & $\bot$ & $\bot$ & $\bot$ & $\bot$ & $\bot$ & $\bot$ & $\bot$ \\
		\hline
	\end{tabular}
\end{center}

Come si può notare, la tabella risultante è simmetrica: questo perché l'operatore di moltiplicazione astratta è commutativo.

\subsection{Somma astratta}

La somma astratta è definita come segue:

\begin{center}
	$\mathrm{\oplus = \lambda a_1,a_2 \in A . \alpha(\gamma(a_1) \boxplus \gamma(a_2))}$
\end{center}
con $\boxplus$ somma concreta, definita come segue:

\begin{center}
	$\mathrm{S_1 \boxplus S_2 = \{ n_1 + n_2 \mid n_1 \in S_1, n_2 \in S_2 \}}$
\end{center}

\begin{center}
	\begin{tabular}{| c | c | c | c | c | c | c | c | c | }
		\hline
		$\oplus$ & $\top$ & $\leq$ & $\neq$ & $\geq$ & $<$ & $=$ & $>$ & $\bot$ \\
		\hline
		$\top$ & $\top$ & $\top$ & $\top$ & $\top$ & $\top$ & $\top$ & $\top$ & $\bot$ \\
		\hline
		$\leq$ & $\top$ & $\leq$ & $\top$ & $\top$ & $<$ & $\leq$ & $\top$ & $\bot$\\
		\hline
		$\neq$ & $\top$ & $\top$ & $\top$ & $\top$ & $\top$ & $\neq$ & $\top$ & $\bot$ \\
		\hline
		$\geq$ & $\top$ & $\top$ & $\top$ & $\geq$ & $\top$ & $\geq$ & $>$ & $\bot$\\
		\hline
		$<$ & $\top$ & $<$ & $\top$ & $\top$ & $<$ & $<$ & $\top$ & $\bot$\\
		\hline
		$=$ & $\top$ & $\leq$ & $\neq$ & $\geq$ & $<$ & $=$ & $>$ & $\bot$\\
		\hline
		$>$ & $\top$ & $\top$ & $\top$ & $>$ & $\top$ & $>$ & $>$ & $\bot$ \\
		\hline
		$\bot$ & $\bot$ & $\bot$ & $\bot$ & $\bot$ & $\bot$ & $\bot$ & $\bot$ & $\bot$ \\
		\hline
	\end{tabular}
\end{center}

Osservando il numero di ricorrenze di $\top$ all'interno della tabella e confrontandolo con quello all'interno della tabella della moltiplicazione astratta, si nota che su questo dominio la moltiplicazione astratta è più precisa della somma astratta.

\subsection{Divisione intera astratta}

La divisione intera astratta viene definita come segue:

\begin{center}
	$\mathrm{\oslash = \lambda a_1,a_2 \in A . \alpha(\gamma(a_1) \boxslash \gamma(a_2))}$
\end{center}

con $\boxslash$ divisione intera concreta, definita come segue:

\begin{center}
	$\mathrm{S_1 \boxslash S_2 = \{ n_1 / n_2 \mid n_1 \in S_1, n_2 \in S_2 \backslash \{0\} \}}$
\end{center}
Nella seguente tabella, gli indici delle colonne indicano il dividendo e gli indici delle righe indicano il divisore.

\begin{center}
	\begin{tabular}{| c | c | c | c | c | c | c | c | c | }
		\hline
		$\oslash$ & $\top$ & $\leq$ & $\neq$ & $\geq$ & $<$ & $=$ & $>$ & $\bot$ \\
		\hline
		$\top$ & $\top$ & $\top$ & $\top$ & $\top$ & $\top$ & $=$ & $\top$ & $\bot$\\
		\hline
		$\leq$ & $\top$ & $\geq$ & $\top$ & $\leq$ & $\geq$ & $=$ & $\leq$ & $\bot$\\
		\hline
		$\neq$ & $\top$ & $\top$ & $\top$ & $\top$ & $\top$ & $=$ & $\top$ & $\bot$\\
		\hline
		$\geq$ & $\top$ & $\leq$ & $\top$ & $\geq$ & $\leq$ & $=$ & $\geq$ & $\bot$\\
		\hline
		$<$ & $\top$ & $\geq$ & $\top$ & $\leq$ & $\geq$ & $=$ & $\leq$ & $\bot$\\
		\hline
		$=$ & $\bot$ & $\bot$ & $\bot$ & $\bot$ & $\bot$ & $\bot$ & $\bot$ & $\bot$\\
		\hline
		$>$ & $\top$ & $\leq$ & $\top$ & $\geq$ & $\leq$ & $=$ & $\geq$ & $\bot$\\
		\hline
		$\bot$ & $\bot$ & $\bot$ & $\bot$ & $\bot$ & $\bot$ & $\bot$ & $\bot$ & $\bot$\\
		\hline
	\end{tabular}
\end{center}

\subsection{Meno unario astratto}

Il meno unario astratto ($\ominus$) può essere definito per semplificare la definizione di meno binario astratto ($\mathrm{\ominus}_b$). Infatti l'operazione astratta
\begin{center}
	$\mathrm{a \mathrm{\ominus}_b b}$
\end{center}
può essere riscritta come:

\begin{center}
	$\mathrm{a \mathrm{\ominus}_b b = a \oplus (\ominus b)}$
\end{center}

\begin{center}
	\begin{tabular}{| c | c | c | c | c | c | c | c | c | }
		\hline
		$\ominus$ & $\top$ & $\leq$ & $\neq$ & $\geq$ & $<$ & $=$ & $>$ & $\bot$ \\
		\hline
		  & $\top$ & $\geq$ & $\neq$ & $\leq$ & $>$ & $=$ & $<$ & $\bot$\\
		\hline
	\end{tabular}
\end{center}

\section{Dominio astratto dei booleani}

Consideriamo ora come dominio astratto quello dei booleani.

[inserire immagine sul dominio astratto dei booleani]

\section{Operazioni astratte su booleani}

Il dominio astratto dei booleani ci permette di ragionare su altre operazioni astratte; poiché si parla di booleani le operazioni interessanti saranno quelle di confronto. 

\subsection{Uguaglianza astratta}

\begin{center}
	\begin{tabular}{| c | c | c | c | c | c | c | c | c | }
		\hline
		$=_a$ & $\top$ & $\leq$ & $\neq$ & $\geq$ & $<$ & $=$ & $>$ & $\bot$ \\
		\hline
		$\top$ & $\top$ & $\top$ & $\top$ & $\top$ & $\top$ & $\top$ & $\top$ & $\bot$ \\
		\hline
		$\leq$ & $\top$ & $\top$ & $\top$ & $\top$ & $\top$ & $\top$ & $0$ & $\bot$\\
		\hline
		$\neq$ & $\top$ & $\top$ & $\top$ & $\top$ & $\top$ & $0$ & $\top$ & $\bot$\\
		\hline
		$\geq$ & $\top$ & $\top$ & $\top$ & $\top$ & $0$ & $\top$ & $\top$ & $\bot$\\
		\hline
		$<$ & $\top$ & $\top$ & $\top$ & $0$ & $\top$ & $\top$ & $0$ & $\bot$\\
		\hline
		$=$ & $\top$ & $\top$ & $0$ & $\top$ & $0$ & $1$ & $0$ & $\bot$\\
		\hline
		$>$ & $\top$ & $0$ & $\top$ & $\top$ & $0$ & $\top$ & $\top$ & $\bot$\\
		\hline
		$\bot$ & $\bot$ & $\bot$ & $\bot$ & $\bot$ & $\bot$ & $\bot$ & $\bot$ & $\bot$\\
		\hline
	\end{tabular}
\end{center}

Per definire l'operazione astratta di diverso basta definire quella di negato astratto:
\begin{center}
	$\mathrm{ c_1 \neq c_2 \iff \neg (c_2 = c_2)}$
\end{center}

\begin{center}
	$\mathrm{ a_1 \neq _a a_2 \iff \neg _a (a_1 =_a a_2)}$
\end{center}

\subsection{Disuguaglianza astratta $\preceq$}

Rappresentiamo la disuguaglianza larga astratta con il seguente simbolo: $\preceq$.

Nella seguente tabella, gli indici delle colonne indicano l'elemento a sinistra della disuguaglianza e gli indici delle righe indicano l'elemento a destra.

\begin{center}
	\begin{tabular}{| c | c | c | c | c | c | c | c | c | }
		\hline
		$\preceq$ & $\top$ & $\leq$ & $\neq$ & $\geq$ & $<$ & $=$ & $>$ & $\bot$ \\
		\hline
		$\top$ & $\top$ & $\top$ & $\top$ & $\top$ & $\top$ & $\top$ & $\top$ & $\bot$\\
		\hline
		$\leq$ & $\top$ & $\top$ & $\top$ & $\top$ & $\top$ & $\top$ & $0$ & $\bot$\\
		\hline
		$\neq$ & $\top$ & $\top$ & $\top$ & $\top$ & $\top$ & $\top$ & $\top$ & $\bot$\\
		\hline
		$\geq$ & $\top$ & $1$ & $\top$ & $\top$ & $1$ & $1$ & $\top$ & $\bot$ \\
		\hline
		$<$ & $\top$ & $\top$ & $\top$ & $0$ & $\top$ & $0$ & $0$ & $\bot$\\
		\hline
		$=$ & $\top$ & $1$ & $\top$ & $\top$ & $1$ & $1$ & $0$ & $\bot$ \\
		\hline
		$>$ & $\top$ & $1$ & $\top$ & $\top$ & $1$ & $0$ & $\top$ & $\bot$ \\
		\hline
		$\bot$ & $\bot$ & $\bot$ & $\bot$ & $\bot$ & $\bot$ & $\bot$ & $\bot$ & $\bot$\\
		\hline
	\end{tabular}
\end{center}

\end{document}

