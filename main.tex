\documentclass[12pt,a4]{book}
\usepackage{lmodern}
\usepackage[T1]{fontenc}
\usepackage[italian]{babel} % Adatta Latex alle convenzioni tipogafiche italiane
\usepackage[utf8]{inputenc} % Consente l'uso dei caratteri accentati italiani
\usepackage{amsmath}
\usepackage{amsthm}
\usepackage{listings}
\usepackage{stmaryrd}
\usepackage{amssymb}
\usepackage{hhline}
%\usepackage{vmargin}
\usepackage{hyperref}
\usepackage{makeidx}
\usepackage{nicefrac}
\usepackage{listings}
\usepackage{marginnote}
\usepackage{xr}
\hypersetup{colorlinks=true, urlcolor=blue, linkcolor=blue}
\message{<Paul Taylor's Proof Trees, 2 August 1996>}
%% Build proof tree for Natural Deduction, Sequent Calculus, etc.
%% WITH SHORTENING OF PROOF RULES!
%% Paul Taylor, begun 10 Oct 1989
%% *** THIS IS ONLY A PRELIMINARY VERSION AND THINGS MAY CHANGE! ***
%%
%% 2 Aug 1996: fixed \mscount and \proofdotnumber
%%
%%      \prooftree
%%              hyp1            produces:
%%              hyp2
%%              hyp3            hyp1    hyp2    hyp3
%%      \justifies              -------------------- rulename
%%              concl                   concl
%%      \thickness=0.08em
%%      \shiftright 2em
%%      \using
%%              rulename
%%      \endprooftree
%%
%% where the hypotheses may be similar structures or just formulae.
%%
%% To get a vertical string of dots instead of the proof rule, do
%%
%%      \prooftree                      which produces:
%%              [hyp]
%%      \using                                  [hyp]
%%              name                              .
%%      \proofdotseparation=1.2ex                 .name
%%      \proofdotnumber=4                         .
%%      \leadsto                                  .
%%              concl                           concl
%%      \endprooftree
%%
%% Within a prooftree, \[ and \] may be used instead of \prooftree and
%% \endprooftree; this is not permitted at the outer level because it
%% conflicts with LaTeX. Also,
%%      \Justifies
%% produces a double line. In LaTeX you can use \begin{prooftree} and
%% \end{prootree} at the outer level (however this will not work for the inner
%% levels, but in any case why would you want to be so verbose?).
%%
%% All of of the keywords except \prooftree and \endprooftree are optional
%% and may appear in any order. They may also be combined in \newcommand's
%% eg "\def\Cut{\using\sf cut\thickness.08em\justifies}" with the abbreviation
%% "\prooftree hyp1 hyp2 \Cut \concl \endprooftree". This is recommended and
%% some standard abbreviations will be found at the end of this file.
%%
%% \thickness specifies the breadth of the rule in any units, although
%% font-relative units such as "ex" or "em" are preferable.
%% It may optionally be followed by "=".
%% \proofrulebreadth=.08em or \setlength\proofrulebreadth{.08em} may also be
%% used either in place of \thickness or globally; the default is 0.04em.
%% \proofdotseparation and \proofdotnumber control the size of the
%% string of dots
%%
%% If proof trees and formulae are mixed, some explicit spacing is needed,
%% but don't put anything to the left of the left-most (or the right of
%% the right-most) hypothesis, or put it in braces, because this will cause
%% the indentation to be lost.
%%
%% By default the conclusion is centered wrt the left-most and right-most
%% immediate hypotheses (not their proofs); \shiftright or \shiftleft moves
%% it relative to this position. (Not sure about this specification or how
%% it should affect spreading of proof tree.)
%
% global assignments to dimensions seem to have the effect of stretching
% diagrams horizontally.
%
%%==========================================================================

\def\introrule{{\cal I}}\def\elimrule{{\cal E}}%%
\def\andintro{\using{\land}\introrule\justifies}%%
\def\impelim{\using{\Rightarrow}\elimrule\justifies}%%
\def\allintro{\using{\forall}\introrule\justifies}%%
\def\allelim{\using{\forall}\elimrule\justifies}%%
\def\falseelim{\using{\bot}\elimrule\justifies}%%
\def\existsintro{\using{\exists}\introrule\justifies}%%

%% #1 is meant to be 1 or 2 for the first or second formula
\def\andelim#1{\using{\land}#1\elimrule\justifies}%%
\def\orintro#1{\using{\lor}#1\introrule\justifies}%%

%% #1 is meant to be a label corresponding to the discharged hypothesis/es
\def\impintro#1{\using{\Rightarrow}\introrule_{#1}\justifies}%%
\def\orelim#1{\using{\lor}\elimrule_{#1}\justifies}%%
\def\existselim#1{\using{\exists}\elimrule_{#1}\justifies}

%%==========================================================================

\newdimen\proofrulebreadth \proofrulebreadth=.05em
\newdimen\proofdotseparation \proofdotseparation=1.25ex
\newdimen\proofrulebaseline \proofrulebaseline=2ex
\newcount\proofdotnumber \proofdotnumber=3
\let\then\relax
\def\hfi{\hskip0pt plus.0001fil}
\mathchardef\squigto="3A3B
%
% flag where we are
\newif\ifinsideprooftree\insideprooftreefalse
\newif\ifonleftofproofrule\onleftofproofrulefalse
\newif\ifproofdots\proofdotsfalse
\newif\ifdoubleproof\doubleprooffalse
\let\wereinproofbit\relax
%
% dimensions and boxes of bits
\newdimen\shortenproofleft
\newdimen\shortenproofright
\newdimen\proofbelowshift
\newbox\proofabove
\newbox\proofbelow
\newbox\proofrulename
%
% miscellaneous commands for setting values
\def\shiftproofbelow{\let\next\relax\afterassignment\setshiftproofbelow\dimen0 }
\def\shiftproofbelowneg{\def\next{\multiply\dimen0 by-1 }%
\afterassignment\setshiftproofbelow\dimen0 }
\def\setshiftproofbelow{\next\proofbelowshift=\dimen0 }
\def\setproofrulebreadth{\proofrulebreadth}

%=============================================================================
\def\prooftree{% NESTED ZERO (\ifonleftofproofrule)
%
% first find out whether we're at the left-hand end of a proof rule
\ifnum  \lastpenalty=1
\then   \unpenalty
\else   \onleftofproofrulefalse
\fi
%
% some space on left (except if we're on left, and no infinity for outermost)
\ifonleftofproofrule
\else   \ifinsideprooftree
        \then   \hskip.5em plus1fil
        \fi
\fi
%
% begin our proof tree environment
\bgroup% NESTED ONE (\proofbelow, \proofrulename, \proofabove,
%               \shortenproofleft, \shortenproofright, \proofrulebreadth)
\setbox\proofbelow=\hbox{}\setbox\proofrulename=\hbox{}%
\let\justifies\proofover\let\leadsto\proofoverdots\let\Justifies\proofoverdbl
\let\using\proofusing\let\[\prooftree
\ifinsideprooftree\let\]\endprooftree\fi
\proofdotsfalse\doubleprooffalse
\let\thickness\setproofrulebreadth
\let\shiftright\shiftproofbelow \let\shift\shiftproofbelow
\let\shiftleft\shiftproofbelowneg
\let\ifwasinsideprooftree\ifinsideprooftree
\insideprooftreetrue
%
% now begin to set the top of the rule (definitions local to it)
\setbox\proofabove=\hbox\bgroup$\displaystyle % NESTED TWO
\let\wereinproofbit\prooftree
%
% these local variables will be copied out:
\shortenproofleft=0pt \shortenproofright=0pt \proofbelowshift=0pt
%
% flags to enable inner proof tree to detect if on left:
\onleftofproofruletrue\penalty1
}

%=============================================================================
% end whatever box and copy crucial values out of it
\def\eproofbit{% NESTED TWO
%
% various hacks applicable to hypothesis list
\ifx    \wereinproofbit\prooftree
\then   \ifcase \lastpenalty
        \then   \shortenproofright=0pt  % 0: some other object, no indentation
        \or     \unpenalty\hfil         % 1: empty hypotheses, just glue
        \or     \unpenalty\unskip       % 2: just had a tree, remove glue
        \else   \shortenproofright=0pt  % eh?
        \fi
\fi
%
% pass out crucial values from scope
\global\dimen0=\shortenproofleft
\global\dimen1=\shortenproofright
\global\dimen2=\proofrulebreadth
\global\dimen3=\proofbelowshift
\global\dimen4=\proofdotseparation
\global\count255=\proofdotnumber
%
% end the box
$\egroup  % NESTED ONE
%
% restore the values
\shortenproofleft=\dimen0
\shortenproofright=\dimen1
\proofrulebreadth=\dimen2
\proofbelowshift=\dimen3
\proofdotseparation=\dimen4
\proofdotnumber=\count255
}

%=============================================================================
\def\proofover{% NESTED TWO
\eproofbit % NESTED ONE
\setbox\proofbelow=\hbox\bgroup % NESTED TWO
\let\wereinproofbit\proofover
$\displaystyle
}%
%
%=============================================================================
\def\proofoverdbl{% NESTED TWO
\eproofbit % NESTED ONE
\doubleprooftrue
\setbox\proofbelow=\hbox\bgroup % NESTED TWO
\let\wereinproofbit\proofoverdbl
$\displaystyle
}%
%
%=============================================================================
\def\proofoverdots{% NESTED TWO
\eproofbit % NESTED ONE
\proofdotstrue
\setbox\proofbelow=\hbox\bgroup % NESTED TWO
\let\wereinproofbit\proofoverdots
$\displaystyle
}%
%
%=============================================================================
\def\proofusing{% NESTED TWO
\eproofbit % NESTED ONE
\setbox\proofrulename=\hbox\bgroup % NESTED TWO
\let\wereinproofbit\proofusing
\kern0.3em$
}

%=============================================================================
\def\endprooftree{% NESTED TWO
\eproofbit % NESTED ONE
% \dimen0 =     length of proof rule
% \dimen1 =     indentation of conclusion wrt rule
% \dimen2 =     new \shortenproofleft, ie indentation of conclusion
% \dimen3 =     new \shortenproofright, ie
%                space on right of conclusion to end of tree
% \dimen4 =     space on right of conclusion below rule
  \dimen5 =0pt% spread of hypotheses
% \dimen6, \dimen7 = height & depth of rule
%
% length of rule needed by proof above
\dimen0=\wd\proofabove \advance\dimen0-\shortenproofleft
\advance\dimen0-\shortenproofright
%
% amount of spare space below
\dimen1=.5\dimen0 \advance\dimen1-.5\wd\proofbelow
\dimen4=\dimen1
\advance\dimen1\proofbelowshift \advance\dimen4-\proofbelowshift
%
% conclusion sticks out to left of immediate hypotheses
\ifdim  \dimen1<0pt
\then   \advance\shortenproofleft\dimen1
        \advance\dimen0-\dimen1
        \dimen1=0pt
%       now it sticks out to left of tree!
        \ifdim  \shortenproofleft<0pt
        \then   \setbox\proofabove=\hbox{%
                        \kern-\shortenproofleft\unhbox\proofabove}%
                \shortenproofleft=0pt
        \fi
\fi
%
% and to the right
\ifdim  \dimen4<0pt
\then   \advance\shortenproofright\dimen4
        \advance\dimen0-\dimen4
        \dimen4=0pt
\fi
%
% make sure enough space for label
\ifdim  \shortenproofright<\wd\proofrulename
\then   \shortenproofright=\wd\proofrulename
\fi
%
% calculate new indentations
\dimen2=\shortenproofleft \advance\dimen2 by\dimen1
\dimen3=\shortenproofright\advance\dimen3 by\dimen4
%
% make the rule or dots, with name attached
\ifproofdots
\then
        \dimen6=\shortenproofleft \advance\dimen6 .5\dimen0
        \setbox1=\vbox to\proofdotseparation{\vss\hbox{$\cdot$}\vss}%
        \setbox0=\hbox{%
                \advance\dimen6-.5\wd1
                \kern\dimen6
                $\vcenter to\proofdotnumber\proofdotseparation
                        {\leaders\box1\vfill}$%
                \unhbox\proofrulename}%
\else   \dimen6=\fontdimen22\the\textfont2 % height of maths axis
        \dimen7=\dimen6
        \advance\dimen6by.5\proofrulebreadth
        \advance\dimen7by-.5\proofrulebreadth
        \setbox0=\hbox{%
                \kern\shortenproofleft
                \ifdoubleproof
                \then   \hbox to\dimen0{%
                        $\mathsurround0pt\mathord=\mkern-6mu%
                        \cleaders\hbox{$\mkern-2mu=\mkern-2mu$}\hfill
                        \mkern-6mu\mathord=$}%
                \else   \vrule height\dimen6 depth-\dimen7 width\dimen0
                \fi
                \unhbox\proofrulename}%
        \ht0=\dimen6 \dp0=-\dimen7
\fi
%
% set up to centre outermost tree only
\let\doll\relax
\ifwasinsideprooftree
\then   \let\VBOX\vbox
\else   \ifmmode\else$\let\doll=$\fi
        \let\VBOX\vcenter
\fi
% this \vbox or \vcenter is the actual output:
\VBOX   {\baselineskip\proofrulebaseline \lineskip.2ex
        \expandafter\lineskiplimit\ifproofdots0ex\else-0.6ex\fi
        \hbox   spread\dimen5   {\hfi\unhbox\proofabove\hfi}%
        \hbox{\box0}%
        \hbox   {\kern\dimen2 \box\proofbelow}}\doll%
%
% pass new indentations out of scope
\global\dimen2=\dimen2
\global\dimen3=\dimen3
\egroup % NESTED ZERO
\ifonleftofproofrule
\then   \shortenproofleft=\dimen2
\fi
\shortenproofright=\dimen3
%
% some space on right and flag we've just made a tree
\onleftofproofrulefalse
\ifinsideprooftree
\then   \hskip.5em plus 1fil \penalty2
\fi
}

%==========================================================================
% IDEAS
% 1.    Specification of \shiftright and how to spread trees.
% 2.    Spacing command \m which causes 1em+1fil spacing, over-riding
%       exisiting space on sides of trees and not affecting the
%       detection of being on the left or right.
% 3.    Hack using \@currenvir to detect LaTeX environment; have to
%       use \aftergroup to pass \shortenproofleft/right out.
% 4.    (Pie in the sky) detect how much trees can be "tucked in"
% 5.    Discharged hypotheses (diagonal lines).

%% alcune macro aggiunte da me (segalini)
%% in attesa di revisione e sistemazione
%% funzione non definita
\newcommand*{\convrg}{\downarrow}
%% funzione definita
\newcommand*{\divrg}{\uparrow}
%% configurazioni della semantica <skip, s>
\newcommand*{\config}[2]{\langle\ #1,\ #2\ \rangle}
%% configurazione della semantica denotazionale C[com](s)
\newcommand*{\denotC}[2]{\calC \llbracket #1 \rrbracket (#2)}
\newcommand*{\denotE}[2]{\calE \llbracket #1 \rrbracket (#2)}
\newcommand*{\denotB}[2]{\calB \llbracket #1 \rrbracket (#2)}
%% assegnamento del while :=
\newcommand*{\weq}{:=}
\newcommand*{\assign}[3]{#1[#2\mapsto#3]} % comodo per s[x -> n]
\newcommand*{\bigassign}[3]{#1\bigl[#2\mapsto#3\bigr]} % comodo per s[x -> n]
%% transizione della small step
\newcommand*{\ssarrow}{\longrightarrow}
\newcommand*{\nssarrow}{\not \longrightarrow}
%% operatore di sostituzione s[n/x]
\newcommand*{\subst}[3]{#1[\nicefrac{#2}{#3}]}
\newcommand*{\substt}[2]{[\nicefrac{#1}{#2}]}
% macro da verificare per la composizione funzionale (trombi)
\newcommand*{\compfun}[2]{\mathrm{#1}\mathop{\circ}\mathrm{#2}}
% e per if then else
\newcommand*{\condif}[3]{\kw{if}\mathrm{#1}\kw{then}\mathrm{#2}\kw{else}\mathrm{#3}}
\newcommand*{\while}[2]{\kw{while}\mathrm{#1}\kw{do}\mathrm{#2}}
% macro per $ecc
\newcommand*{\ecc}{\mathrm{\$ecc}}
% macro per operatore di anullamento var /
\newcommand{\unvar}{\mathrel{/}}

% Comando per scrivere con texttt dentro una formula matematica
\newcommand{\mactext}[1]{\text{\texttt{#1}}}

%% Proof trees.
\input prooftree
\newcommand*{\nohyp}{\phantom{x}}
\newcommand*{\rulename}[1]{\text{\footnotesize (#1)}\;}
\newcommand*{\pts}{\\[12pt]} % (marano) da controllare

%% C++.
\newcommand*{\Cplusplus}{{C\nolinebreak[4]\hspace{-.05em}\raisebox{.4ex}
{\tiny\bf ++}}}

%% BNF rules.
\newcommand*{\vbar}{\mathrel{\mid}}

%% Abstract syntax of the analyzed language.
\newcommand*{\Integer}{\mathrm{Integer}}
\newcommand*{\Bool}{\mathrm{Bool}}
\newcommand*{\Ghost}{\mathrm{Ghost}}
\newcommand*{\LE}{\mathrm{LE}}
\newcommand*{\LExp}{\mathrm{LExp}}
\newcommand*{\FV}{\mathrm{FV}}
\newcommand*{\Var}{\mathrm{Var}}
\newcommand*{\Pred}{\mathrm{Pred}}
\newcommand*{\AExp}{\mathrm{AExp}}
\newcommand*{\BExp}{\mathrm{BExp}}
\newcommand*{\Com}{\mathrm{Com}}
\newcommand*{\Exp}{\mathrm{Exp}}
\newcommand*{\ST}{\mathrm{ST}}

%% Sets of configurations
\newcommand*{\NTe}{\Gamma_\mathrm{e}}
\newcommand*{\NTb}{\Gamma_\mathrm{b}}
\newcommand*{\NTd}{\Gamma_\mathrm{d}}
\newcommand*{\NTg}{\Gamma_\mathrm{g}}
\newcommand*{\NTs}{\Gamma_\mathrm{s}}
\newcommand*{\NTk}{\Gamma_\mathrm{k}}
\newcommand*{\Te}{T_\mathrm{e}}
\newcommand*{\Tb}{T_\mathrm{b}}
\newcommand*{\Td}{T_\mathrm{d}}
\newcommand*{\Tg}{T_\mathrm{g}}
\newcommand*{\Ts}{T_\mathrm{s}}
\newcommand*{\Tk}{T_\mathrm{k}}

%% Lambda notation.
\newcommand*{\lambdaop}{\mathop{\lambda}\nolimits}

%% Sets of (no better specified) configurations.
\newcommand*{\NT}[1]{\Gamma_{#1}}
\newcommand*{\NTq}{\Gamma_q}
\newcommand*{\Tq}{T_q}

%% Denotable values.
\newcommand*{\dVal}{\mathrm{dVal}}
%% Storeable values.
\newcommand*{\sVal}{\mathrm{sVal}}
\newcommand*{\sval}{\mathrm{sval}}

%% Control modes.
\newcommand*{\CtrlMode}{\mathord{\mathrm{CtrlMode}}}
\newcommand*{\cm}{\mathrm{cm}}
%% Branch modes.
%\newcommand*{\BranchMode}{\mathord{\mathrm{BranchMode}}}
\newcommand*{\GotoMode}{\mathord{\mathrm{GotoMode}}}
\newcommand*{\SwitchMode}{\mathord{\mathrm{SwitchMode}}}
\newcommand*{\cmgoto}{\mathop{\mathrm{goto}}\nolimits}
\newcommand*{\cmswitch}{\mathop{\mathrm{switch}}\nolimits}
\newcommand*{\cmbreak}{\mathop{\mathrm{break}}\nolimits}
\newcommand*{\cmcontinue}{\mathop{\mathrm{continue}}\nolimits}
\newcommand*{\cmreturn}{\mathop{\mathrm{return}}\nolimits}
%% Exec mode.
\newcommand*{\cmexec}{\mathrm{exec}}
%% Value mode.
\newcommand*{\ValMode}{\mathord{\mathrm{ValMode}}}
\newcommand*{\cmvalue}{\mathop{\mathrm{value}}\nolimits}
%% Environment mode.
\newcommand*{\EnvMode}{\mathord{\mathrm{EnvMode}}}
\newcommand*{\cmenv}{\mathrm{env}}
%% Exception modes.
\newcommand*{\ExceptMode}{\mathord{\mathrm{ExceptMode}}}
\newcommand*{\cmexcept}{\mathrm{except}}

%% Control states.
\newcommand*{\CtrlState}{\mathord{\mathrm{CtrlState}}}
\newcommand*{\cs}{\mathord{\mathrm{cs}}}
%% Value states.
\newcommand*{\ValState}{\mathord{\mathrm{ValState}}}
\newcommand*{\valstate}{\upsilon}
%% Environment states.
%\newcommand*{\EnvState}{\mathord{\mathrm{EnvState}}}
%% Exception states.
\newcommand*{\ExceptState}{\mathord{\mathrm{ExceptState}}}
\newcommand*{\exceptstate}{\varepsilon}

%% Keywords.
\newcommand*{\kw}[1]{\mathop{\textup{\textbf{#1}}}}

\newcommand*{\bop}{\mathbin{\mathrm{bop}}}
%\newcommand*{\uop}{\mathop{\mathrm{uop}}}

%% Things that hold by definition.
\newcommand{\defrel}[1]{\mathrel{\buildrel \mathrm{def} \over {#1}}}
\newcommand{\defeq}{\defrel{=}}
\newcommand{\defiff}{\defrel{\Longleftrightarrow}}
%\newcommand{\defeq}{=}
%\newcommand{\defiff}{\Longleftrightarrow}

%% Divergence relation
\newcommand{\diverges}{\,\mathord{\buildrel \infty \over \longrightarrow}}

%% Special letters denoting sets and algebras.
\providecommand*{\Nset}{\mathbb{N}}             % Naturals
\providecommand*{\Qset}{\mathbb{Q}}             % Rationals
\providecommand*{\Zset}{\mathbb{Z}}             % Integers
\providecommand*{\Rset}{\mathbb{R}}             % Reals

%% Calligraphic alphabet.
\newcommand*{\calA}{\ensuremath{\mathcal{A}}}
\newcommand*{\calB}{\ensuremath{\mathcal{B}}}
\newcommand*{\calC}{\ensuremath{\mathcal{C}}}
\newcommand*{\calD}{\ensuremath{\mathcal{D}}}
\newcommand*{\calE}{\ensuremath{\mathcal{E}}}
\newcommand*{\calF}{\ensuremath{\mathcal{F}}}
\newcommand*{\calG}{\ensuremath{\mathcal{G}}}
\newcommand*{\calH}{\ensuremath{\mathcal{H}}}
\newcommand*{\calI}{\ensuremath{\mathcal{I}}}
\newcommand*{\calJ}{\ensuremath{\mathcal{J}}}
\newcommand*{\calK}{\ensuremath{\mathcal{K}}}
\newcommand*{\calL}{\ensuremath{\mathcal{L}}}
\newcommand*{\calM}{\ensuremath{\mathcal{M}}}
\newcommand*{\calN}{\ensuremath{\mathcal{N}}}
\newcommand*{\calO}{\ensuremath{\mathcal{O}}}
\newcommand*{\calP}{\ensuremath{\mathcal{P}}}
\newcommand*{\calQ}{\ensuremath{\mathcal{Q}}}
\newcommand*{\calR}{\ensuremath{\mathcal{R}}}
\newcommand*{\calS}{\ensuremath{\mathcal{S}}}
\newcommand*{\calT}{\ensuremath{\mathcal{T}}}
\newcommand*{\calU}{\ensuremath{\mathcal{U}}}
\newcommand*{\calV}{\ensuremath{\mathcal{V}}}
\newcommand*{\calW}{\ensuremath{\mathcal{W}}}
\newcommand*{\calX}{\ensuremath{\mathcal{X}}}
\newcommand*{\calY}{\ensuremath{\mathcal{Y}}}
\newcommand*{\calZ}{\ensuremath{\mathcal{Z}}}

%% Declarators for functions and relations.
\newcommand*{\reld}[3]{\mathord{#1}\subseteq#2\times#3}
\newcommand*{\fund}[3]{\mathord{#1}\colon#2\to#3}
\newcommand*{\pard}[3]{\mathord{#1}\colon#2\rightarrowtail#3}

%% Logical quantifiers stuff.
\newcommand{\st}{\mathrel{.}}
\newcommand{\itc}{\mathrel{:}}

%% Domain, codomain and range of a function.
\newcommand*{\dom}{\mathop{\mathrm{dom}}\nolimits}
%\newcommand*{\cod}{\mathop{\mathrm{cod}}\nolimits}
%\newcommand*{\range}{\mathop{\mathrm{range}}\nolimits}

%% Restriction of a function.
\newcommand*{\restrict}[1]{\mathop{\mid}\nolimits_{#1}}

%% Type of a constant.
\newcommand*{\type}{\mathop{\mathrm{type}}\nolimits}

%% Lubs, glbs, and fixed points.
\newcommand*{\lub}{\mathop{\mathrm{lub}}\nolimits}
\newcommand*{\glb}{\mathop{\mathrm{glb}}\nolimits}
\newcommand*{\lfp}{\mathop{\mathrm{lfp}}\nolimits}
\newcommand*{\gfp}{\mathop{\mathrm{gfp}}\nolimits}

%% Generic widening.
\newcommand*{\widen}{\mathbin{\nabla}}

%% Set theory.
\renewcommand{\emptyset}{\varnothing}
\newcommand*{\wpf}{\mathop{\wp_\mathrm{f}}\nolimits}

\newcommand*{\sseq}{\subseteq}
\newcommand*{\sseqf}{\mathrel{\subseteq_\mathrm{f}}}
\newcommand*{\sslt}{\subset}
%\newcommand*{\Sseq}{\supseteq}
%\newcommand*{\Ssgt}{\supset}

%\newcommand{\Nsseq}{\nsubseteq}

\newcommand*{\union}{\cup}
\newcommand*{\bigunion}{\bigcup}
%\newcommand*{\biginters}{\bigcap}
\newcommand*{\inters}{\cap}
\newcommand*{\setdiff}{\setminus}

\newcommand{\sset}[2]{{\renewcommand{\arraystretch}{1.2}
                      \left\{\,#1 \,\left|\,
                               \begin{array}{@{}l@{}}#2\end{array}
                      \right.   \,\right\}}}

%% Base sets.
\newcommand*{\true}{\mathrm{true}}
\newcommand*{\false}{\mathrm{false}}
\newcommand*{\ttv}{\mathrm{tt}}
\newcommand*{\ffv}{\mathrm{ff}}
\newcommand*{\divop}{\mathbin{/}}
\newcommand*{\modop}{\mathbin{\%}}
\newcommand*{\andop}{\mathbin{\textbf{\textup{and}}}}
\newcommand*{\orop}{\mathbin{\textbf{\textup{or}}}}
\newcommand*{\notop}{\mathop{\textbf{\textup{not}}}}

\newcommand*{\FI}{\mathop{\mathrm{FI}}\nolimits}
\newcommand*{\DI}{\mathop{\mathrm{DI}}\nolimits}
\newcommand*{\SL}{\mathop{\mathrm{SL}}\nolimits}
%\newcommand*{\match}{\mathop{\mathrm{match}}\nolimits}

\newcommand*{\Env}{\mathord{\mathrm{Env}}}
\newcommand*{\emptystring}{\mathord{\epsilon}}

%% Exceptions.
\newcommand*{\RTSExcept}{\mathord{\mathrm{RTSExcept}}}
\newcommand*{\rtsexcept}{\chi}
\newcommand*{\Except}{\mathord{\mathrm{Except}}}
\newcommand*{\except}{\xi}
\newcommand*{\none}{\mathtt{none}}
\newcommand*{\divbyzero}{\mathtt{divbyzero}}
\newcommand*{\stkovflw}{\mathtt{stkovflw}}
\newcommand*{\datovflw}{\mathtt{datovflw}}
\newcommand*{\memerror}{\mathtt{memerror}}
%\newcommand*{\inerror}{\mathtt{inerror}}
%\newcommand*{\nullptr}{\mathtt{nullptr}}
%\newcommand*{\outofboundsptr}{\mathtt{outofboundsptr}}

%% Flags for terminal configurations of catch clauses.
\newcommand*{\caught}{\mathtt{caught}}
\newcommand*{\uncaught}{\mathtt{uncaught}}

%% Static semantics.
\newcommand*{\TEnv}{\mathord{\mathrm{TEnv}}}
\newcommand*{\tinteger}{\mathrm{integer}}
\newcommand*{\tboolean}{\mathrm{boolean}}
\newcommand*{\trtsexcept}{\mathrm{rts\_exception}}

%% Memory structures.
\newcommand*{\Loc}{\mathord{\mathrm{Loc}}}
\newcommand*{\Ind}{\mathrm{Ind}}
\newcommand*{\Addr}{\mathrm{Addr}}
\newcommand*{\Map}{\mathrm{Map}}
%\newcommand*{\eMap}{\mathrm{eMap}}
\newcommand*{\Stack}{\mathord{\mathrm{Stack}}}
\newcommand*{\Mem}{\mathord{\mathrm{Mem}}}
\newcommand*{\stknew}{\mathop{\mathrm{new}_\mathrm{s}}\nolimits}
\newcommand*{\datnew}{\mathop{\mathrm{new}_\mathrm{d}}\nolimits}
\newcommand*{\txtnew}{\mathop{\mathrm{new}_\mathrm{t}}\nolimits}
\newcommand*{\heapnew}{\mathop{\mathrm{new}_\mathrm{h}}\nolimits}
\newcommand*{\heapdel}{\mathop{\mathrm{delete}_\mathrm{h}}\nolimits}
\newcommand*{\datcleanup}{\mathop{\mathrm{cleanup}_\mathrm{d}}\nolimits}
\newcommand*{\smark}{\mathop{\mathrm{mark}_\mathrm{s}}\nolimits}
\newcommand*{\sunmark}{\mathop{\mathrm{unmark}_\mathrm{s}}\nolimits}
\newcommand*{\slink}{\mathop{\mathrm{link}_\mathrm{s}}\nolimits}
\newcommand*{\sunlink}{\mathop{\mathrm{unlink}_\mathrm{s}}\nolimits}
\newcommand*{\asmark}{\mathop{\mathrm{mark}_\mathrm{s}^\sharp}\nolimits}
\newcommand*{\asunmark}{\mathop{\mathrm{unmark}_\mathrm{s}^\sharp}\nolimits}
\newcommand*{\aslink}{\mathop{\mathrm{link}_\mathrm{s}^\sharp}\nolimits}
\newcommand*{\asunlink}{\mathop{\mathrm{unlink}_\mathrm{s}^\sharp}\nolimits}
\newcommand*{\aswiden}{\mathop{\mathrm{widen}}\nolimits}
\newcommand*{\sm}{\dag}
\newcommand*{\fm}{\ddag}
\newcommand*{\topmost}{\mathop{\mathrm{tf}}\nolimits}
%% Short forms of \datcleanup, \sunmark, \sunlink for table.
\newcommand*{\datcleanupshort}{\mathop{\mathrm{cu}_\mathrm{d}}\nolimits}
\newcommand*{\sunmarkshort}{\mathop{\mathrm{um}_\mathrm{s}}\nolimits}
\newcommand*{\sunlinkshort}{\mathop{\mathrm{ul}_\mathrm{s}}\nolimits}

\newcommand*{\location}[1]{\mathord{#1 \; \mathrm{loc}}}
%\newcommand*{\saeval}{\mathop{\mathrm{aeval}}\nolimits}
%\newcommand*{\saupd}{\mathop{\mathrm{aupd}}\nolimits}
\newcommand*{\asupported}{\mathop{\mathrm{supported}^\sharp}\nolimits}
\newcommand*{\aeval}{\mathop{\mathrm{eval}^\sharp}\nolimits}
\newcommand*{\ceval}[1]{\mathop{\mathrm{eval}_{#1}}\nolimits}

%% Abstracts.
\newcommand*{\Abstract}{\mathord{\mathrm{Abstract}}}
\newcommand*{\abs}{\mathord{\mathrm{abs}}}

%% Integer part function.
\newcommand{\intp}{\mathop{\mathrm{int}}\nolimits}

%% Concrete functions and operations.
% Aritmethic
\newcommand*{\conadd}{\mathbin{\boxplus}}
\newcommand*{\consub}{\mathbin{\boxminus}}
\newcommand*{\conmul}{\mathbin{\boxdot}}
\newcommand*{\condiv}{\mathbin{\boxslash}}
\newcommand*{\conmod}{\mathbin{\boxbar}}
% Boolean
\newcommand*{\coneq}{\mathbin{\circeq}}
\newcommand*{\conineq}{\mathbin{\leqq}}
\newcommand*{\conneg}{\mathbin{\daleth}}
\newcommand*{\conor}{\mathbin{\triangledown}}
\newcommand*{\conand}{\mathbin{\vartriangle}}
\newcommand*{\bneg}{\mathop{\neg}\nolimits}

%% Abstract functions and operations.
% Domains
\newcommand*{\Sign}{\mathrm{Sign}}
\newcommand*{\AbBool}{\mathrm{AbBool}}

% Aritmethic
\newcommand*{\absuminus}{\mathop{\ominus}\nolimits}
\newcommand*{\absadd}{\mathbin{\oplus}}
\newcommand*{\abssub}{\mathbin{\ominus}}
\newcommand*{\absmul}{\mathbin{\odot}}
\newcommand*{\absdiv}{\mathbin{\oslash}}
\newcommand*{\absmod}{\mathbin{\obar}}
% Boolean
\newcommand*{\abseq}{\mathrel{\triangleq}}
\newcommand*{\absneq}{\mathrel{\not\triangleq}}
\newcommand*{\absleq}{\mathrel{\trianglelefteq}}
\newcommand*{\abslt}{\mathrel{\vartriangleleft}}
\newcommand*{\absgeq}{\mathrel{\trianglerighteq}}
\newcommand*{\absgt}{\mathrel{\vartriangleright}}
\newcommand*{\absneg}{\mathrel{\circleddash}}
\newcommand*{\absor}{\mathrel{\ovee}}
\newcommand*{\absand}{\mathrel{\owedge}}
% Figures
\newcommand*{\signtop}{\top}
\newcommand*{\signbot}{\bot}
\newcommand*{\signge}{\mathord{\geq}}
\newcommand*{\signgt}{\mathord{>}}
\newcommand*{\signle}{\mathord{\leq}}
\newcommand*{\signlt}{\mathord{<}}
\newcommand*{\signeq}{\mathord{=}}
\newcommand*{\signne}{\mathord{\neq}}

%% Summaries for theorem-like environments
\newcommand{\summary}[1]{\textrm{\textbf{\textup{#1}}}}

% Annotations in equations
\newcommand{\law}[1]{{\hspace*{\fill}\text{\footnotesize\rm{[#1]}}}}

%% Filter function extracting the relevant and irrelevant parts.
\newcommand*{\sel}{\mathop{\mathrm{sel}}\nolimits}
\newcommand*{\mem}{\mathop{\mathrm{mem}}\nolimits}

%% Modeling definite exceptions.
%\newcommand*{\None}{\mathrm{None}}

%% Strict Cartesian products.
\newcommand*{\stimes}{\otimes}
\newcommand*{\spair}[2]{{#1} \otimes {#2}}
%\newcommand*{\rstimes}{\rtimes}
%\newcommand*{\rspair}[2]{{#1} \rtimes {#2}}
%\newcommand*{\lstimes}{\ltimes}
%\newcommand*{\lspair}[2]{{#1} \ltimes {#2}}

%% chain
\newcommand*{\chain}{\mathop{\mathrm{chain}}\nolimits}


%% Additional macros for the extension for extra numeric types
%% Floating point types.
\newcommand*{\tfloat}{\mathrm{float}}
%% Numeric types
\newcommand*{\nType}{\mathrm{nType}}
\newcommand*{\nT}{\mathrm{nT}}

%% Additional macros for the extension to pointer and arrays:
%% Elementary types.
\newcommand*{\eType}{\mathrm{eType}}
\newcommand*{\eT}{\mathrm{eT}}
%% Elementary values.
%\newcommand*{\eValue}{\mathrm{eVal}}
%% Array types.
\newcommand*{\aType}{\mathrm{aType}}
\newcommand*{\aT}{\mathrm{aT}}
%% Record types.
\newcommand*{\rType}{\mathrm{rType}}
\newcommand*{\rT}{\mathrm{rT}}
%% Object types.
\newcommand*{\oType}{\mathrm{oType}}
\newcommand*{\oT}{\mathrm{oT}}
%% Function types.
\newcommand*{\fType}{\mathrm{fType}}
\newcommand*{\fT}{\mathrm{fT}}
%% Memory types.
\newcommand*{\mType}{\mathrm{mType}}
\newcommand*{\mT}{\mathrm{mT}}
%% Pointer types.
\newcommand*{\pType}{\mathrm{pType}}
\newcommand*{\pT}{\mathrm{pT}}
%% Offsets.
\newcommand*{\Offset}{\mathrm{Offset}}
\newcommand*{\nooffset}{\boxempty}
\newcommand*{\indexoffset}[1]{\mathopen{\boldsymbol{[}}{#1}\mathclose{\boldsymbol{]}}}
\newcommand*{\fieldoffset}[1]{\mathop{\boldsymbol{.}}{#1}}
%% Lvalues.
\newcommand*{\lValue}{\mathrm{LValue}}
\newcommand*{\lvalue}{\mathrm{lval}}
%% Rvalues.
\newcommand*{\rValue}{\mathrm{RValue}}
\newcommand*{\rvalue}{\mathrm{rval}}
%%
\newcommand*{\pointer}[1]{{#1}\boldsymbol{\ast}}
\newcommand*{\maddress}[1]{\mathop{\&}{#1}}
\newcommand*{\indirection}[1]{\mathop{\boldsymbol{\ast}}{#1}}
%%
\newcommand*{\locnull}{\mathord{l_\mathrm{null}}}
\newcommand*{\ptrmove}{{\mathop{\mathrm{ptrmove}}\nolimits}}
\newcommand*{\ptrdiff}{{\mathop{\mathrm{ptrdiff}}\nolimits}}
\newcommand*{\ptrcmp}{{\mathop{\mathrm{ptrcmp}}\nolimits}}
%%
\newcommand*{\arraysyntax}[3]{\kw{#1} {#2} \kw{of}\,{#3}}
\newcommand*{\arraytype}[2]{\arraysyntax{array}{#1}{#2}}
\newcommand*{\firstof}{{\mathop{\mathrm{firstof}}\nolimits}}
\newcommand*{\arrayindex}{\mathop{\mathrm{index}}\nolimits}
\newcommand*{\locindex}{\mathop{\mathrm{locindex}}\nolimits}
%%
\newcommand*{\recordsyntax}[3]{\kw{#1} {#2} \kw{of}\,{#3}}
\newcommand*{\recordtype}[2]{\recordsyntax{record}{#1}{#2}}
\newcommand*{\field}{\mathop{\mathrm{field}}\nolimits}
\newcommand*{\locfield}{\mathop{\mathrm{locfield}}\nolimits}
%%
\newcommand*{\NTo}{\Gamma_\mathrm{o}}
\newcommand*{\To}{T_\mathrm{o}}
\newcommand*{\NTl}{\Gamma_\mathrm{l}}
\newcommand*{\Tl}{T_\mathrm{l}}
%\newcommand*{\NTr}{\Gamma_\mathrm{r}}
%\newcommand*{\Tr}{T_\mathrm{r}}
%%
\newcommand*{\arraydatnew}{\mathop{\mathrm{newarray}_\mathrm{d}}\nolimits}
\newcommand*{\arraystknew}{\mathop{\mathrm{newarray}_\mathrm{s}}\nolimits}

\theoremstyle{definition}
\newtheorem{definizione}{Definizione}
\newtheorem{teorema}{Teorema}
\newtheorem{proposizione}{Proposizione}
\newtheorem{lemma}{Lemma}


\begin{document}

\title{Appunti di semantica dei linguaggi di programmazione}
\author{Bartolomeo Lombardi \and Amerigo Mancino \and Valentino Marano
  \and Andrea Segalini \and Francesco Trombi \and Alessandro Benedetti
  \and Riccardo Melioli \and Michele Chiari \and Roberto Bagnara}
%\date{31/10/2014}

\maketitle

\tableofcontents

\chapter{Nozioni preliminari}

[Introduzione al capitolo da scrivere.]


\section{Notazione basilare}

\subsection{Insiemi e sequenze}

Sia $S$ un insieme.
Se $S$ \`e l'insieme vuoto allora si scrive $S = \emptyset$.
Inoltre:
\begin{list}{}{}
\item[$|S|$] denota la cardinalit\`a di S; se $S$ \`e un insieme finito
             allora $|S|$ \`e il numero di elementi di $S$; se $S$
             pu\`o essere messo in corrispondenza biunivoca con
             l'insieme dei numeri naturali $\Nset$ allora si dice che
             \`e infinito e numerabile oppure che ha la
             \emph{potenza del numerabile} oppure che ha cardinalit\`a
             $\aleph_0$ e si scrive $|S| = \aleph_0$; per contro, le
             espressioni ``S \`e finito'' e ``$|S| < \aleph_0$'' sono
             equivalenti;
\item[$\wp(S)$] denota l'insieme di tutti i sottoinsiemi di $S$, anche detto
                insieme delle parti di $S$ o
                insieme potenza di $S$;
\item[$\wpf(S)$] denota l'insieme di tutti i sottoinsiemi \emph{finiti} di
                  $S$;

\item[$S^{\ast}$] denota l'insieme delle sequenze finite, possibilmente vuote,
                 di elementi di $S$; la sequenza vuota si indica
                 con $\varepsilon$;
\item[$S^{+}$] denota l'insieme delle sequenze finite e non vuote
               di elementi di $S$.
\end{list}

Siano $S$ e $T$ due insiemi. La notazione $S \sseqf T$ significa
$S \sseq T$ e $|S| < \aleph_0$, ovvero che $S$ \`e un sottoinsieme finito
di $T$.

\subsection{Funzioni}

Se $\fund{f}{A}{B}$ \`e una funzione e $S \sseq A$, allora si pone
\[
    f(S) = \bigl\{\, f(x) \bigm| x \in S \,\bigr\} \sseq B.
\]


\section{Strutture algebriche}

\begin{definizione} \summary{(Struttura algebrica.)}
Una \emph{struttura algebrica} o \emph{algebra mono-sortale}
\`e una coppia $(S, Q)$, dove
\begin{enumerate}
\item
$S$ \`e un insieme non vuoto detto \emph{insieme base} o \emph{carrier};
\item
$Q$ \`e una funzione definita su un insieme di indici $I$,
possibilmente infinito non numerabile, tale che, per ogni $i \in I$,
$Q(i)$ \`e un'operazione finitaria (ovvero $n$-aria con $n$ finito,
possibilmente nullo) da elementi di $S$ ad elementi di $S$. Ad
esempio, se $Q(i)$ \`e $n$-aria, allora
\[
    \fund{Q(i)}{\overbrace{S \times\cdots\times S}^{n}}{S}.
\]
\end{enumerate}
\end{definizione}

Anzich\'e con la coppia $(S, Q)$, nel seguito le algebre saranno
usualmente denotate con una $(k+1)$-pla
\[
    (S, o_1, \ldots, o_k)
\]
dove $S$ \`e il medesimo insieme non vuoto della definizione, e $o_1$,
$\ldots$, $o_k$ sono alcune operazioni nel codominio di $Q$. L'ariet\`a
delle $o_i$ sar\`a specificata di volta in volta.

\begin{definizione} \summary{(Reticolo.)}
\label{def:reticolo1}
Un'algebra
\[
    A = (L, \otimes, \oplus)
\]
con le due operazioni binarie $\otimes$ (detto \emph{meet}) e $\oplus$ (detto
\emph{join}) \`e un \emph{reticolo} se valgono, per ogni $x, y, z \in L$,
le seguenti identit\`a:
\begin{itemize}
\item[$L_1 (a):$] $x \otimes y = y \otimes x$,
\item[$L_1 (b):$] $x \oplus  y = y \oplus  x$,
    \law{leggi commutative}
\item[$L_2 (a):$] $x \otimes (y \otimes z) = (x \otimes y) \otimes z$,
\item[$L_2 (b):$] $x \oplus  (y \oplus  z) = (x \oplus  y) \oplus  z$,
    \law{leggi associative}
\item[$L_3 (a):$] $x \otimes x = x$,
\item[$L_3 (b):$] $x \oplus  x = x$,
    \law{leggi di idempotenza}
\item[$L_4 (a):$] $x \otimes (x \oplus  y) = x$,
\item[$L_4 (b):$] $x \oplus  (x \otimes y) = x$.
    \law{leggi di assorbimento}
\end{itemize}
\end{definizione}

\begin{definizione}  \summary{(Ordinamento parziale e totale.)}
Una relazione binaria $\preceq$ definita su un insieme
$S$ \`e una \emph{relazione di ordinamento parziale} su $S$
se le seguenti condizioni valgono in $S$,
per ogni $a, b, c \in S$:
\begin{itemize}
\item[$O_1:$] $a \preceq a$,
    \law{riflessivit\`a}
\item[$O_2:$] $(a \preceq b) \land (b \preceq a) \implies a = b$,
    \law{antisimmetria}
\item[$O_3:$] $(a \preceq b) \land (b \preceq c) \implies a \preceq c$.
    \law{transitivit\`a}
\end{itemize}
Se inoltre, per ogni $a, b \in S$, vale
\begin{itemize}
\item[$O_4:$] $(a \preceq b) \lor (b \preceq a)$,
\end{itemize}
allora $\preceq$ \`e detta \emph{relazione di ordinamento totale} su $S$.
\end{definizione}

\begin{definizione} \summary{(Insieme parzialmente/totalmente ordinato.)}
Un insieme $P$ equipaggiato con una relazione di ordinamento
parziale $\preceq$ su $P$ \`e chiamato
\emph{insieme parzialmente ordinato} o \emph{poset}.
Se $\preceq$ \`e una relazione di
ordinamento totale su $P$ allora $P$ viene detto
\emph{insieme totalmente ordinato}.
Se $(P, \preceq)$ \`e un insieme parzialmente ordinato,
l'espressione $a \prec b$ denota
la condizione $a \preceq b \land a \neq b$.
\end{definizione}

\begin{definizione} \summary{(Catena.)}
Se $(P, \preceq)$ \`e un insieme totalmente ordinato
tale che $|P| \leq \aleph _0$, allora $(P, \preceq)$ si dice essere una
\emph{catena}.
\end{definizione}

Quando il poset $(P, \preceq)$ \`e implicito dal contesto
e $A \sseq P$ scriveremo $\chain(A)$ intendendo con questo
che $(A, \preceq)$ \`e una catena.

\begin{proposizione}
\label{prop:catena-finita-ha-elemento-massimo}
Ogni catena finita ha un elemento massimo.
\end{proposizione}
\begin{proof}
Sia $(C, \preceq)$ una catena tale che $|C| < \aleph_0$.
Esiste quindi $k \in \Nset$ tale che $C = \{ a_1, \dots, a_k \}$
e $a_1 \prec \cdots \prec a_k$.
Dalla transitivit\`a di $\preceq$ discende la transitivit\`a di $\prec$,
la quale implica $a_i \prec a_k$, per ogni $i = 1$, \dots,~$k-1$.
Poich\'e $a_i \prec a_k$ implica $a_i \preceq a_k$, abbiamo
$a_i \preceq a_k$ per ogni $i = 1$, \dots,~$k$, ovvero $a_k = \max_\preceq C$.
\end{proof}


\begin{proposizione} \summary{(Ogni sottoinsieme di una catena è una catena.)}
Sia $(C, \preceq)$ una catena, allora ogni suo sottoinsieme è una catena, ovvero $\forall X \subseteq C \chain(X)$.
\end{proposizione}
\begin{proof}
È ovvio che essendo $X$ un sottoinsieme di $C$ esso avrà cardinalità minore o uguale a $C$, dunque $|X| \leq |C| \leq \aleph_0 \Rightarrow |X| \leq \aleph_0$. Abbiamo quindi dimostrato che $X$ è sicuramente numerabile, dobbiamo solo dimostrare che è totalmente ordinato:
$\forall x,y \in X \Rightarrow x,y \in C \Rightarrow x \preceq y \lor y \preceq x$
\end{proof}

\begin{definizione} \summary{(Limite superiore/inferiore, sup/inf, lub/glb.)}
Sia $(P, \preceq)$ un poset e sia $A \sseq P$.
Un elemento $p \in P$ \`e un \emph{limite superiore} per $A$
se $a \preceq p$ per ogni $a\in A$.
Un elemento $p \in P$ \`e un \emph{sup} (o \emph{lub}) di $A$,
scritto $p \in \sup A$ (o $p \in \lub A$),
se $p$ \`e un limite superiore per $A$ e,
per ogni altro limite superiore $p'$ per $A$,
risulta $p \preceq p'$.
In modo del tutto analogo un elemento $p \in P$ \`e un \emph{limite inferiore}
per $A$ se $p \preceq a$ per ogni $a \in A$.
Un elemento $p \in P$ \`e un \emph{inf} (o \emph{glb}) di $A$,
scritto $p \in \inf A$ (o $p \in \glb A$),
se $p$ \`e un limite inferiore per $A$ e,
per ogni altro limite inferiore $p'$ per $A$, risulta $p' \preceq p$.
\end{definizione}

%%%%%%%%%%%%%%
\begin{proposizione} \summary{(Unicità di $\lub$ e $\glb$.)}
Sia $(S, \preceq)$ un poset e sia $T \subseteq S$.
Se $u, v \in \lub T$ allora $u = v$.
Analogamente, se $u, v \in \glb T$ allora $u = v$.
\end{proposizione}
\begin{proof}
Siano $u, v \in \lub T$.  Per definizione di $\lub$ abbiamo
\begin{align*}
  (\forall x \in T \itc x \preceq u)
    \land
      \forall u' \in S
        &\itc (\forall x \in T \itc x \preceq u') \itc u \preceq u', \\
  (\forall x \in T \itc x \preceq v)
    \land
      \forall v' \in S
        &\itc (\forall x \in T \itc x \preceq v') \itc v \preceq v'.
\end{align*}
Da queste discende $u \preceq v$ e $v \preceq u$ il che,
per l'antisimmetria di $\preceq$ implica $u = v$.

La dimostrazione inerente l'unicità di $\glb T$ è analoga.
\end{proof}

Il risultato appena dimostrato ci consente di scrivere $p = \lub T$ e
$p = \glb T$, nel caso tali $\lub$ e $\glb$ esistano,
invece di $p \in \lub T$ e $p \in \glb T$.

%%%%%%%%%%%%%%%
\begin{proposizione} \summary{($\lub T \in T \Rightarrow \lub T = \max T$.)}
Sia $(S, \preceq)$ un ordinamento parziale, sia $T \subseteq S$ e sia $u \in S \lub\ T$, allora il $\lub T$ è anche massimo dell'insieme.
\end{proposizione}
\begin{proof}
Applicando la definizione di $\lub$ otteniamo che:
$\forall x \in T \itc x \preceq u \land u \in T \Rightarrow \nexists y \in T \itc y \succeq u$
Non esistendo dunque alcun elemento in $T$ maggiore di $u$ è ovvio che $u = \max T$.
\end{proof}

\begin{proposizione} \summary{($\exists \max S \Rightarrow \lub S = \max S$.)}
Sia $(S, \preceq)$ un ordinamento parziale, allora se esiste $u = \max S$ $u$ è anche il $\lub$ di tale insieme.
\end{proposizione}
\begin{proof}
Applichiamo dapprima la definizione di $\max$ ipotizzando che $u = \max S$:
$\forall x \in S \itc x \preceq u$. Dunque $u = ub\ S$, $u$ è un upper bound per l'insieme. Ora dobbiamo solo dimostrare che è proprio il minimo dei maggioranti: sappiamo che $\not \exists u' \in S \itc \forall y \in S \itc y \preceq u'$, non esiste dunque un altro maggiorante $u'$ (a meno che non coincida con $u$). $u$ è perciò l'unico maggiorante, dunque $u = \lub S$.
\end{proof}


\begin{definizione} \summary{(CPO.)}
\label{def:cpo}
Un poset $(D, \preceq)$ \`e un \emph{CPO} se e solo se
ha un elemento minimo e, per ogni catena $K \sseq D$, esiste $\lub K \in D$.
\end{definizione}

\begin{definizione}  \summary{(Reticolo.)}
\label{def:reticolo2}
Un insieme parzialmente ordinato $L$ \`e un \emph{reticolo} se e solo se,
per ogni $a, b \in L$, esistono in $L$ sia $\sup\{a,b\}$ che $\inf\{a,b\}$.
\end{definizione}

\begin{proposizione}
\textup{\cite{BurrisS81}} Le definizioni di reticolo
delle definizioni~\textup{\ref{def:reticolo1}} e \textup{\ref{def:reticolo2}}
sono equivalenti.
Inoltre, se $L$ \`e
un reticolo per la prima definizione, la seconda definizione si ottiene
ponendo, per ogni $a,b \in L$,
\[
    a \preceq b \quad\iff\quad a \otimes b = a \quad\iff\quad a \oplus b = b.
\]
Viceversa, se $L$ \`e un reticolo per la seconda definizione,
la prima definizione si ottiene ponendo, per ogni $a,b \in L$,
\[
    a \otimes b = \inf\{a,b\} \qquad\mbox{e}\qquad a \oplus b = \sup\{a,b\}.
\]
Le due trasformazioni sono inverse l'una dell'altra.
\end{proposizione}

\begin{definizione}
Un reticolo $L$ \`e un \emph{reticolo distributivo} se e solo se soddisfa,
per ogni $x, y \in L$, le seguenti identit\`a:
\begin{itemize}
\item[$D_1:$] $x \otimes (y \oplus   z) = (x \otimes y) \oplus  (x \otimes z)$,
\item[$D_2:$] $x \oplus  (y \otimes  z) = (x \oplus  y) \otimes (x \oplus  z)$.
    \law{leggi distributive}
\end{itemize}
\end{definizione}

\begin{proposizione}
{\rm \cite{BurrisS81}} Un reticolo soddisfa $D_1$ se e solo se soddisfa $D_2$.
\end{proposizione}

\begin{definizione}
Un poset $P$ \`e \emph{completo} se, per ogni $A \sseq P$ esistono in $P$ sia
$\lub A$ che $\glb A$. Tutti i poset completi sono reticoli e un reticolo
$L$ che \`e completo come poset \`e detto \emph{reticolo completo}.
\end{definizione}

\begin{proposizione}
\label{prop:complete-lattice0}
\textup{\cite{BurrisS81}}
Sia $P$ un poset tale che esista $\lub A$ per ogni $A \sseq P$,
oppure tale che esista $\glb A$ per ogni $A \sseq P$.
Allora $P$ \`e un reticolo completo.
\end{proposizione}

Nella precedente proposizione l'esistenza di $\lub \emptyset$ garantisce
l'esistenza di un elemento minimo in $P$ e, similmente, l'esistenza di
$\glb \emptyset$ garantisce l'esistenza di un elemento massimo in $P$.
Si pu\`o perci\`o riformulare la proposizione \ref{prop:complete-lattice0}
in un modo equivalente.

\begin{proposizione}
\label{prop:complete-lattice}
\textup{\cite{BurrisS81}} Un poset $P$ \`e completo se possiede un elemento minimo
ed esiste $\lub A$ per ogni $A$ sottoinsieme non vuoto di $P$, oppure se
possiede un elemento massimo ed esiste $\glb A$ per ogni $A$ sottoinsieme non
vuoto di~$P$.
\end{proposizione}

\begin{definizione}  \summary{(Reticolo limitato.)}
{\rm \cite{BurrisS81}} Un'algebra
\[
    (L, \otimes, \oplus, \top, \bot)
\]
con due operatori binari e due operatori nullari
\`e un \emph{reticolo limitato} se soddisfa, per ogni $x \in L$:
\begin{itemize}
\item[$B_1 \phantom{(a)}:$] $(L, \otimes, \oplus)$ \`e un reticolo,
\item[$B_2 (a):$] $x \otimes \bot = \bot$,
\item[$B_2 (b):$] $x \oplus  \top = \top$.
    \law{leggi di annichilazione}
\end{itemize}
\end{definizione}
Si osservi che se $(L, \otimes, \oplus)$ \`e un reticolo completo, allora
\[
    \bigl(L, \otimes, \oplus, \lub L, \glb L\bigr)
\]
\`e un reticolo limitato.
Si utilizzer\`a talvolta la nozione di reticolo limitato e
completo, ma al solo scopo di esibire, assieme al \emph{carrier} e agli
operatori, anche gli elementi estremi (massimo e minimo) dell'algebra.

\begin{definizione}  \summary{(Monoide.)}
Un \emph{monoide} \`e un'algebra
\[
    (M, \cdot, 1)
\]
con un operatore binario ``$\,\cdot$'' e un operatore nullario ``$\,1$'' che
soddisfa, per ogni $x,y,z\in M$ le seguenti identit\`a:
\begin{itemize}
\item[$M_1:$] $x \cdot (y \cdot z) = (x \cdot y) \cdot z$,
    \law{legge associativa}
\item[$M_2:$] $x \cdot 1 = 1 \cdot x = 1$.
    \law{legge dell'unit\`a}
\end{itemize}
Un \emph{monoide commutativo} \`e un monoide che soddisfa, per ogni $x,y \in M$,
\begin{itemize}
\item[$M_3:$] $x \cdot y = y \cdot x$.
\end{itemize}
Un \emph{monoide idempotente} \`e un monoide che soddisfa, per ogni $x \in M$,
\begin{itemize}
\item[$M_4:$] $x \cdot x = x$.
\end{itemize}
\end{definizione}

\section{Funzioni su poset}

\begin{definizione} \summary{(Funzione monotona/continua/additiva.)}
Consideriamo due poset $(S, \preceq)$ e $(T, \sqsubseteq)$,
e una funzione $\fund{f}{S}{T}$.
Allora
$f$ è \emph{monotona} se, per ogni $x, y \in S$,
\[
  x \preceq y \implies f(x) \sqsubseteq f(y).
\]
Se $(S, \preceq)$ \`e un CPO,
$f$ si dice \emph{continua} se, per ogni $X \subseteq S$ tale che $\chain(X)$,
abbiamo
\[
  \exists \lub f(X) \land f(\lub X) = \lub f(X).
\]
Se $(S, \preceq)$ \`e un reticolo completo,
$f$ è \emph{additiva} se, per $X \subseteq S$ qualsiasi,
\[
  \exists \lub f(X) \land f(\lub X) = \lub f(X).
\]
\end{definizione}


\begin{proposizione} \summary{(Se $f$ è continua allora è anche monotona.)}
Siano $(S, \preceq)$ e $(T, \sqsubseteq)$ due CPO e sia $\fund{f}{S}{T}$, se $f$ è continua allora è anche monotona.
\end{proposizione}
\begin{proof}
Siano $x,y \in S \itc x \preceq y$ allora definiamo $X = \left\{x,y\right\}$. Siamo sicuri che vale $\chain(X)$ per come è stato definito l'insieme, dunque risulta banale che $\lub X = y \land \exists \lub f(X) \land f(y) = \lub f(X)$. Perciò $f$ è monotona.
\end{proof}

\begin{proposizione} \summary{(f monotona $\Arrownot \Rightarrow$ f continua.)}
Siano $(S, \preceq)$ e $(T, \sqsubseteq)$ due CPO e sia $\fund{f}{S}{T}$, se $f$ è monotona non è detto che sia anche continua.
\end{proposizione}
\begin{proof}
Scegliamo un controesempio con $(\Nset \cup \infty, \leq)$ e $(\left\{0,1\right\},\leq)$ e
\[
\fund{f}{\Nset \cup \infty}{\left\{0,1\right\}} \lambda x.
        \begin{cases}
        0 & \text{ se } x \in \Nset; \\
        1 & \text{ se } x = \infty;
    \end{cases}
\]
Scegliamo come catena proprio l'insieme $\Nset$, dunque:
$\Nset \subseteq \Nset \cup \left\{\infty\right\} \land \chain(\Nset)$ perciò $\lub f(\Nset) = \lub \left\{0\right\}$ ma $f(\lub \Nset) = f(\infty) = 1$.
\end{proof}

\begin{proposizione} \summary{(L'immagine di una catena tramite funzione monotona è una catena.)}
Siano $(X, \sqsubseteq_X)$ e $(Y, \sqsubseteq_Y)$ due poset e sia $\fund{f}{X}{Y}$ monotona e $C \subset X . \chain(C)$, allora vale anche $\chain(f(C))$
\end{proposizione}
\begin{proof}
Sia $I \subseteq \Nset \itc C = \left\{a_i\right\}_{i \in I}$, allora $f(C) = \left\{f(a_i) | i \in I\right\}$ e
$|f(C)| \leq |I| \leq |\Nset|$, perciò $f(C)$ è numerabile. Siano $i,j \in I$ allora abbiamo due casi:
\begin{itemize}
        \setlength{\itemindent}{20mm}
        \item $i \leq j \Rightarrow a_i \sqsubseteq_X a_j \Rightarrow f(a_i) \sqsubseteq_Y f(a_j)$
        \item $j \leq i \Rightarrow a_j \sqsubseteq_X a_i \Rightarrow f(a_j) \sqsubseteq_Y f(a_i)$
\end{itemize}
In entrambi i casi abbiamo dimostrato che $f(C)$ è totalmente ordinato, ovvero che $\forall i,j \in I \itc f(a_i) \sqsubseteq_Y f(a_j) \lor f(a_j) \sqsubseteq_Y f(a_j)$. $f(C)$ è quindi una catena.
\end{proof}

\begin{proposizione} \summary{(Composizione di funzioni continue.)}
Siano $(A, \sqsubseteq_A), \; (B, \sqsubseteq_B) \; e \; (D, \sqsubseteq_D) $ dei CPO e siano $\fund{f}{A}{B}$ e $\fund{g}{B}{D}$ funzioni continue, allora $\fund{\compfun{g}{f}}{A}{D}$ è continua.
\end{proposizione}
\begin{proof}
Sia $C \subseteq A \itc \chain(C)$ allora sappiamo che esiste $\lub f(C)$ e che $\lub f(C) = f(\lub C)$.
Dall'esercizio precedente sappiamo che $f(C)$ è una catena, perciò esiste $\lub g(f(C))$ e $g(\lub f(C)) = \lub g(f(C))$.
Ma allora $\compfun{g}{f}(\lub C) = g(f(\lub C)) = g(\lub f(C)) = \lub g(f(C)) = \lub \compfun{g}{f}(C)$.
$\compfun{g}{f}$ è perciò continua.
\end{proof}

\section{Punti fissi}

\begin{definizione} \summary{(Punto fisso.)}
Sia $S$ un insieme e sia $\fund{f}{S}{S}$ una funzione.
Allora un \emph{punto fisso} di $f$ è un qualunque $x \in S$ tale
che $f(x) = x$.
\end{definizione}

Una funzione può avere zero punti fissi (esempio: $f(x) = x+1$),
un solo punto fisso
(esempio: $g(x) = 7$ ha un punto fisso in $x=7$),
o addirittura infiniti punti fissi (esempio: $h(x) = x$).
Quando un punto fisso di $f$ gode di una certa proprietà rispetto
all'insieme di tutti i punti fissi di $f$, ad esempio è il minimo
oppure il massimo di questi, diremo che è il
\emph{minimo punto fisso di $f$}, denotato con $\lfp f$,
o il \emph{massimo punto fisso di $f$}, denotato con $\gfp f$.

\subsection{Teorema di Tarski}
\label{sec:Tarski}

\begin{teorema} \summary{(Teorema di Tarski.)}
Sia $(S, \preceq)$ un CPO, sia $\bot = \min S$ e sia
$\fund{f}{S}{S}$ continua.
Allora $f$ ha un minimo punto fisso e questo coincide con il limite
delle iterate di $f$ a partire da $\bot$.
Formalmente:
\[
  \exists \lfp f
    \land
      \lfp f = \lub \bigl\{\, f^n(\bot) \bigm| n \in \Nset \,\bigr\}.
\]
\begin{proof}
Come prima cosa dimostriamo per induzione su $n \in \Nset$
che l'insieme
\[
  C \defeq \bigl\{\, f^n(\bot) \bigm| n \in \Nset \,\bigr\}
\]
è una catena, ovvero che, per ogni $i \in \Nset$,
$f^i(\bot) \preceq f^{i+1}(\bot)$.

Per $i = 0$, abbiamo $f^0(\bot) = \bot \preceq f^1(\bot)$
dal momento che $\bot = \min S$.

Per $i > 0$, dalla monotonia di $f$ e dall'ipotesi induttiva discende che
\[
  f^i(\bot) = f\bigl(f^{i-1}(\bot)\bigr)
            \preceq f\bigl(f^i(\bot)\bigr)
            = f^{i+1}(\bot).
\]

Da $\chain(C)$ e dal fatto che $(S, \preceq)$ è un CPO deriva l'esistenza
di $\lub C$.
Dalla continuità di $f$ discende $f(\lub C) = \lub f(C)$ ma,
dalla definizione di $C$ sappiamo che
\begin{align*}
  \lub f(C)
    &= \lub f\Bigl(\bigl\{\, f^n(\bot) \bigm| n \in \Nset \,\bigr\}\Bigr) \\
    &= \lub \Bigl(\bigl\{\, f(f^n(\bot)) \bigm| n \in \Nset \,\bigr\}\Bigr) \\
    &= \lub \Bigl(\bigl\{\, f^{n+1}(\bot) \bigm| n \in \Nset \,\bigr\}\Bigr) \\
    &= \lub \Bigl(\bigl\{\, f^n(\bot) \bigm| n \in \Nset \,\bigr\}\Bigr) \\
    &= \lub C.
\end{align*}
Mettendo insieme questi due fatti abbiamo che $f(\lub C) = \lub C$,
ovvero che $\lub C$ è un punto fisso di $f$.
Dimostriamo per assurdo che $\lub C$ è il minimo dei punti fissi di $f$:
sia $m$ un punto fisso di $f$ con $m \prec \lub C$;
dunque esisterebbe $i \in \Nset$ tale che $m = f^j(\bot)$ per ogni $j \geq i$,
ma questo implicherebbe che $\lub C$ non sia il least upper bound di C,
il che è assurdo.
  \end{proof}
\end{teorema}

\chapter{Il linguaggio WHILE}

In questo capitolo si introduce un semplice linguaggio di programmazione,
chiamato ``WHILE'',
che sarà utilizzato per esemplificare varie tecniche di definizione della
semantica dei linguaggi di programmazione.
Il linguaggio per quanto molto semplice, soprattutto nella sua versione di
base, ammette qualche estensione interessante che sarà illustrata nel
seguito.

\section{Sintassi}

\begin{description}
\item[Variabili]
$x \in \Var = \{ x_0, x_1, \ldots \}$,%
\item[Interi]
$n \in \Integer \defeq \Zset$;
\item[Booleani]
$t \in \Bool \defeq \{ \ttv, \ffv \}$;
\item[Espressioni aritmetiche]%
\begin{align*}
  \AExp \ni
  E &::= n \vbar x \vbar E_0 + E_1 \vbar E_0 - E_1 \vbar E_0 * E_1 \vbar \ldots
\end{align*}
\item[Espressioni booleane]%
\begin{align*}
  \BExp \ni
  B &::= \ttv \vbar \ffv \vbar B_0 \kw{and} B_1 \vbar E_0 \kw{and} E_1
              \vbar \kw{not} E \vbar \ldots
\end{align*}
\item[Comandi]%
\begin{align*}
  \Com \ni
  C &::= \kw{skip} \vbar x := E \vbar C_0 ; C_1
         \vbar \kw{if} B \kw{then} C_0 \kw{else} C_1 \\
    &\vbar \kw{while} B \kw{do} C \vbar \ldots
\end{align*}
\end{description}

\section{Semantica informale} \marginpar{Mancino}
La semantica di questo linguaggio, ora definito solo sintatticamente, verrà espressa in
maniera operazionale, denotazionale e assiomatica nei prossimi capitoli. Di seguito ne
diamo solo una sua semantica intuitiva.

La variabili sono celle di memoria che possono contenere interi o booleani. Ad ogni istante
viene tenuta traccia dello stato (o store) delle variabili per mezzo di una
funzione così definita
$ \forall x, \forall y, \forall n$:

$$ s[x \rightarrow n](y) =
	\begin{cases} n, & \mbox{ se } y=x \\
				  s(y) & \mbox{ altrimenti}
	\end{cases}
$$

Le espressioni aritmetiche sono insiemi di numeri e/o variabili legati da segni di operazione.
Allo stesso modo le espressioni booleane sono insiemi di variabili e/o valori booleani legati da
operatori logici.
I punti di sospensione, sia nel caso di espressioni aritmetiche che booleane,
garantiscono la possibilità di introdurre nuove espressioni.
I comandi sono istruzioni che modificano lo stato. In particolare, ad esempio,
$\kw{skip}$ è il comando che lascia lo store invariato, mentre $x:=E$ è il comando che 
assegna ad x il risultato dell'espressione E.
\chapter{Semantica operazionale ``big step''}
\marginpar{Mancino}
In questo capitolo introduciamo la semantica operazionale "big step" (letteralmente 'a grandi passi'),
un particolare tipo di semantica operazionale che descrive formalmente
come il risultato complessivo dell'esecuzione di un'operazione è ottenuto. 
Per fare questo, sfrutta schemi di assiomi e regole che hanno la seguente forma:

$$
\prooftree
	premessa_1
	\cdots
	premessa_k
   \justifies
   		conclusioni
	\using
		(nome \; regola)
\endprooftree
$$

\section{Un primo esempio di applicazione}\marginpar{Mancino}
Consideriamo un linguaggio con la seguente sintassi:
$$ Exp \ni E ::= n \bigm| (E+E) \bigm| \cdots $$
dove $n$ si espande nei naturali $1, 2, \dots$ 
e dove i punti di sospensione garantiscono la possibilità di inserire nuove espressioni.
Allora una sua possibile semantica big step potrebbe essere, ad esempio:

\begin{align*}
\prooftree
   \justifies
   		n \Downarrow n
	\using
		(B-NUM)
\endprooftree
&&
\prooftree
	E_1 \Downarrow n_1 \; \; E_2 \Downarrow n_2
   \justifies
   		(E_1 + E_2) \Downarrow n_3
	\using
		(B-ADD)
\endprooftree
\end{align*}

L'operatore $\Downarrow$ può essere letto come 'valuta a' e permette
in un solo colpo di compiere tutti i passi affinché un comando o un'espressione
giunga al termine della propria computazione. Operando in questo modo,
ci è possibile provare, tramite dimostrazioni ad albero, asserzioni come la seguente:

$$
\prooftree
	\prooftree
   		\justifies
   			3 \Downarrow 3
   		\using
   			(B-NUM)
	\endprooftree
	\;
	\;
	\prooftree
		\prooftree
   			\justifies
   				2 \Downarrow 2
   			\using
   				(B-NUM)
		\endprooftree		
		\;
		\;
		\prooftree
   			\justifies
   				3 \Downarrow 3
   			\using
   				(B-NUM)
		\endprooftree	
   		\justifies
   			(2+1) \Downarrow 3
   		\using
   			(B-ADD)
		\endprooftree
   \justifies
   		(3+(2+1)) \Downarrow 6
   	\using
   		(B-ADD)
\endprooftree
$$

La dimostrazione, come si vede, è un albero che va dal basso verso l'alto, dove
in fondo troviamo ciò che vogliamo dimostrare e, applicando le due regole date precedentemente,
giungiamo a degli assiomi (le foglie dell'albero) che quindi provano quanto volevamo.
In sostanza, per dimostrare la conclusione dobbiamo far valere tutte le premesse che ricaviamo
applicando le regole.

\section{Determinatezza}\marginpar{Mancino}
La proprietà di determinatezza asserisce che un qualunque oggetto sintattico non può
avere due semantiche differenti. In particolare, possiamo dimostrare il seguente

\begin{teorema}[Determinatezza]
La semantica big step di Exp gode della proprietà di determinatezza, ossia
$\forall E \in Exp, \; \forall \; n, n' \in Num \text{ tali che }
 E\Downarrow n \; \land \; E \Downarrow n' \; \Rightarrow n = n'$
\end{teorema}

\begin{proof}
Per induzione strutturale sulla struttura di E:
\begin{itemize}
	\item Base $(n \in Num)$ - 
		  In questo caso abbiamo che $n \Downarrow n_1$ e che $n \Downarrow n_2$.
		  Dalla regola semantica del caso base allora sarebbe $n=n_1$ e $n=n_2$.
		  Quindi, per la transitività dell'uguaglianza otteniamo $n_1 = n_2$.
	\item Passo induttivo $(E \in Exp$ con $E=E_1 + E_2$) - 
		  In questo caso abbiamo che $(E_1 + E_2) \Downarrow m_3$ e che $(E_1+E_2) \Downarrow p_3$.
		  Questo implica che $ \exists \; m_1,m_2,p_1,p_2 \in Num$ tali che:
		  \begin{align*}
		   & m_1 + m_2 = m_3 \\
		     \land \; & p_1+p_2 = p_3 \\
		     \land \; & E_1 \Downarrow m_1 \\
		     \land \; & E_2 \Downarrow m_2 \\ 
		     \land \; & E_1 \Downarrow p_1 \\ 
		     \land \; & E_2 \Downarrow p_2
		    \end{align*}
		  Ma allora, per ipotesi induttiva:
		  \begin{align*}
		  & m_1 = p_1 \\
		    \land & \; m_2 = p_2 \\
		    \land & \; m_1+m_2=m_3 \\
		    \land & \; p_1 + p_2 = p_3
		  \end{align*}
		  Da cui deduciamo che:
		  $ m_1 + m_2 = p_1 + p_2 \Rightarrow m_3 = p_3$ come volevamo dimostrare.
\end{itemize}
\end{proof}

\section{Normalizzazione}\marginpar{Mancino}
La proprietà di normalizzazione asserisce che per ogni espressione $E$ esiste almeno un risultato
$n$ tale che $E \Downarrow n$.

Osserviamo che, in generale, ci sono esempi in cui non è possibile normalizzare:
$$ while \; true \; do \; skip \; \Downarrow \; ? $$
A destra dell'operatore di 'valuta a' non posso scriverci nulla, il ciclo non termina.
Un altro esempio è il seguente:
$$ \nicefrac{0}{0} \; \Downarrow \; ? $$
A destra dell'operatore di 'valuta a' posso scrivere la codifica del disastro (se esiste)
altrimenti, come nel caso precedente, non possiamo scriverci nulla.

\begin{teorema}[Normalizzazione]
La semantica big step di Exp gode della proprietà di normalizzazione, ossia
$\forall E \in Exp, \; \exists \; n \in Num \text{ tale che } E \Downarrow n$.
\end{teorema}

\section{Semantica big step per While}\marginpar{Mancino}
Se invece di usare un linguaggio didattico semplice come quello descritto nel capitolo
precedente volessimo definire tramite la big step la semantica del linguaggio while
avremo, logicamente, che una coppia \emph{(comando, stato)} viene valutata ad un
nuovo stato, ossia, dato un comando C e uno stato s, in generale:

$$
\langle C,s \rangle \Downarrow s'
$$

Con questa premessa, definendo con E la generica espressione, con C
il generico comando e con s il generico stato,
è facile immaginare il comportamento di tutti i comandi per while:

\begin{align*}
\prooftree
   \justifies
   		\langle skip,s \rangle \Downarrow s
\endprooftree
&&
\prooftree
	E \Downarrow n
   \justifies
   		\langle x:=E,s \rangle \Downarrow s[n \rightarrow x]
\endprooftree
\end{align*}

\begin{align*}
\prooftree
	B \Downarrow true \; \; \langle C_1,s \rangle \Downarrow s'
   \justifies
   		\langle if \; B \; then \; C_1 \; else \; C_2, s \rangle \Downarrow s'
\endprooftree
&&
\prooftree
	B \Downarrow false \; \; \langle C_2,s \rangle \Downarrow s'
   \justifies
   		\langle if \; B \; then \; C_1 \; else \; C_2, s \rangle \Downarrow s'
\endprooftree
\end{align*}

\begin{align*}
\prooftree
	\langle C_1,s \rangle \Downarrow s' \; \; \langle C_2,s' \rangle \Downarrow s''
   \justifies
   		\langle C_1; \; C_2,s \rangle \Downarrow s''
\endprooftree
\end{align*}

Infine rimane la regola del while per la quale, ovviamente, abbiamo due casi distinti:

\begin{align*}
\prooftree
	B \Downarrow true \; \; \langle C;s \rangle \Downarrow s' \; \;
	\langle while \; B \; do \; C, s' \rangle \Downarrow s''
   \justifies
   		\langle while \; B \; do \; C,s \rangle \Downarrow s''
\endprooftree
&&
\prooftree
	B \Downarrow false
   \justifies
   		\langle while \; B \; do \; C, s \rangle \Downarrow s
\endprooftree
\end{align*}

\section{Effetti collaterali}\marginpar{Mancino}
In questo paragrafo ridefiniamo la semantica data precedentemente
per il linguaggio While introducendo degli effetti collaterali.

\begin{definizione} \summary{(Effetti collaterali.)}
Si dice che si verifica un \emph{effetto collaterale} quando
una funzione modifica un valore o uno stato al di fuori del proprio
scoping locale.
\end{definizione}

In questo caso, dobbiamo ricordarci che, rispetto alle definizioni date
nel paragrafo precedente, detto C il generico comando, E la generica espressione,
s il generico stato, si ha:

\begin{align*}
& \langle C,s \rangle \Downarrow s' \\
& \langle E,s \rangle \Downarrow \langle n,s \rangle
\end{align*}

Partendo da questo assunto non è quindi difficile ricostruire la semantica:

\begin{align*}
\prooftree
   \justifies
   		\langle skip,s \rangle \Downarrow s
\endprooftree
&&
\prooftree
	\langle E,s \rangle \Downarrow \langle n,s' \rangle
   \justifies
   		\langle x:=E,s \rangle \Downarrow s'[n \rightarrow x]
\endprooftree
\end{align*}

\begin{align*}
\prooftree
	\langle B,s \rangle \Downarrow \langle true,s' \rangle \; \; \langle C_1,s' \rangle \Downarrow s''
   \justifies
   		\langle if \; B \; then \; C_1 \; else \; C_2, s \rangle \Downarrow s''
\endprooftree
&&
\prooftree
	\langle B,s \rangle \Downarrow \langle true,s' \rangle \; \; \langle C;s' \rangle \Downarrow s'' \; \;
	\langle while \; B \; do \; C, s' \rangle \Downarrow s'''
   \justifies
   		\langle while \; B \; do \; C,s \rangle \Downarrow s'''
\endprooftree
\end{align*}

E così via per le restanti.
\chapter{Semantica operazionale ``small step''}
\marginpar{Marano}
In questo capitolo parleremo della semantica operazionale "small step", una semantica che descrive formalmente dopo ogni piccolo passo di calcolo che comando
viene eseguito e come viene modificato lo store. Per fare ciò sfrutta assiomi e regole che hanno la stessa forma di quelle viste nel capitolo precedente

$$
\prooftree
premessa_1
\cdots
premessa_k
\justifies
conclusioni
\using
(nome \; regola)
\endprooftree
$$

\section{Introduzione}\marginpar{Marano}
Se quindi abbiamo un linguaggio con la seguente sintassi
$$ Exp \ni E ::= n \bigm| (E+E) \bigm| \cdots $$
dove $n$ si espande nei naturali $1, 2, \dots$ (nei puntini di sospensione si può inserire qualsiasi espressione).
Allora una possibile semantica small step è:

\begin{align*}
  \prooftree
  \justifies
  n \rightarrow n
  \using
  (S-NUM)
  \endprooftree
  &&
  \prooftree
  E_1 \rightarrow E_1'
  \justifies
  (E_1 + E_2) \rightarrow (E_1' + E_2)
  \using
  (S-LEFT)
  \endprooftree
\\ \\
  \prooftree
  E_2 \rightarrow E_2'
  \justifies
  (E_1 + E_2) \rightarrow (E_1 + E_2')
  \using
  (S-RIGHT)
  \endprooftree
  &&
  \prooftree
  \justifies
  (1 + 2) \rightarrow 3
  \using
  (S-ADD)
  \endprooftree
\end{align*}

L'operatore $\rightarrow$ quindi rappresenta a tutti gli effetti un passo di computazione.
Con questa semantica abbiamo lo stesso potere espressivo della semantica Big step ma,
essendo più verbosa, da più informazione sulla sequenzialità e sui "risultati intermedi".
Vediamo un esempio di dimostrazione:

\begin{align*}
\prooftree
	\prooftree
   		\justifies
   			(1+2) \rightarrow 3
   		\using
   			(S-ADD)
	\endprooftree
\justifies
       	((1+2)+(3+4)) \rightarrow (3+(3+4))
\using
	(S-ADD)
\endprooftree
\\ \\
\prooftree
	\prooftree
   		\justifies
   			(3+4) \rightarrow 7
   		\using
   			(S-ADD)
	\endprooftree
\justifies
       	(3+(3+4)) \rightarrow (3+7)
\using
	(S-ADD)
\endprooftree
\\ \\
\prooftree
	\justifies
		(3+7) \rightarrow 10
	\using
		(S-ADD)
\endprooftree
\end{align*}

Possiamo subito osservare che differentemente dalla semantica Big Step
nella semantica Small Step si usano tanti alberi di dimostrazione più piccoli per dimostrare
ogni asserzione. 

\section{Determinatezza della semantica Small Step} \marginpar{Marano}
Dimostriamo ora che la semantica Small Step gode della proprietà
della determinatezza, ogni oggetto sintattico può avere perciò
al massimo una semantica. Prima dimostriamo però il seguente

\begin{lemma}
$\forall E_1, E_2, E \in Exp . E \rightarrow _1 E_1 \land E \rightarrow _1 E_2 \Rightarrow E_1 = E_2 $
\end{lemma}

\begin{proof}
  Dimostriamo per induzione strutturale
  \begin{itemize}
  \item Caso Base ($E \in Num$) - $E = n \Rightarrow E \nrightarrow$, ovvero E non transita.
  \item Passo induttivo ($E \in Exp$) - $E = E_1 + E_2$ \\
    perciò $E_1 + E_2 \rightarrow \begin{cases} E_1 \oplus E_2, & \mbox{se }E_1, E_2 \in \mbox{ Num} \\
    E_1 + E_2', & \mbox{se }E_1 \in \mbox{ Num} \land E_2 \rightarrow E_2' \\
    E_1' + E_2, & \mbox{se }E_2 \in \mbox{ Num} \land E_1 \rightarrow E_1' \end{cases}$\\
    È ovvio che questi tre casi sono mutuamente esclusivi.
  \end{itemize}
\end{proof}

Ora possiamo dimostrare il seguente

\begin{teorema}[Determinatezza]
La semantica Small Step di Exp gode della proprietà di determinatezza, ovvero
$\forall E \in Exp, \; \forall \; n, n' \in Num \text{ tali che }\\
 E\rightarrow ^* n \; \land \; E \rightarrow ^* n' \; \Rightarrow n = n'$
\end{teorema}

\begin{proof}
  Sappiamo che $E \rightarrow E_1^{(1)} \rightarrow E_1^{(2)} \rightarrow \dots \rightarrow E_1^{(k)} = n_1$ e che 
  $E \rightarrow E_2^{(1)} \rightarrow E_2^{(2)} \rightarrow \dots \rightarrow E_2^{(h)} = n_2$.
  Ma per il lemma appena dimostrato sappiamo che $E_1^{(k)} = E_2^{(h)} \text{ con } h = k \Rightarrow n_1 = n_2$
\end{proof}

\section{Normalizzazione della semantica Small Step} \marginpar{Marano}
Altra proprietà molto importante per una semantica è la normalizzazione:
ogni oggetto semantico ha almeno una semantica. Perciò per ogni espressione $E$
esiste almeno un risultato $n$ tale che $E \rightarrow ^* n$.
Dimostriamo prima però il seguente

\begin{lemma}
  $\forall E \in Exp : E \in Num \land \exists E' \in Num \Rightarrow E \rightarrow E'$
  Questa proprietà ci aiuterà nella dimostrazione del successivo teorema ma dobbiamo anche dimostrare che le espressioni non possono transitare per sempre,
  ovvero che $\forall E \in Exp : \exists k \in \mathbb{N} . E \rightarrow ^k F \land F \nrightarrow$
\end{lemma}

\begin{proof}
  Definiamo perciò una funzione $f : Exp \rightarrow \mathbb{N}$ che calcola il numero di "+", ovvero il numero di passi da fare per arrivare fino a una
  espressione che non possa più transitare, f restituisce ovvero il \emph{k} definito nella tesi del teorema. Definiamo per induzione strutturale:
  \begin{itemize}
    \item Caso Base ($E \in Num$) - $E = n \in Num \Rightarrow f(n) = 0$
    \item Passo Induttivo ($E = \not \in Num$) - $\exists E_1, E_2 \in Exp\ .\ E = E_1 + E_2 \Rightarrow f(E) = f(E_1 + E_2) = f(E_1) + f(E_2) + 1$
  \end{itemize}
  Dimostriamo ora che questa funzione calcola proprio ciò che volevamo, ovviamente lo dimostriamo per induzione strutturale:
  \begin{itemize}
  \item Caso Base ($E \in Num$) - $E = n \Rightarrow E \nrightarrow \Rightarrow f(E) = k = 0$
  \item Passo Induttivo ($E = \not \in Num$) - $E = E_1 + E_2 \Rightarrow \\ E_1 + E_2 \rightarrow ^i n_1 + E_2 \rightarrow ^j n_1 + n_2 \rightarrow n \\
    \Rightarrow k = i + j + 1 = f(E_1) + f(E_2) + 1 = f(E_1 + E_2) = f(E)$
  \end{itemize}
\end{proof}

\begin{teorema}[Normalizzazione]
  $\forall E \in Exp\ \exists n \in Num\ .\ E \rightarrow ^* n$
\end{teorema}

\begin{proof}
  La proprietà deriva direttamente dai due lemmi dimostrati precedentemente
\end{proof}

\section{Semantica Small Step per While} \marginpar{Marano}
Diamo ora una semantica al linguaggio While tramite la small step, dovremo
perciò valutare non più una singola espressione ma una coppia formata da
comando e stato. Per ora tralasciamo la possibilità di effetti collaterali.
$$
\langle C,s \rangle \rightarrow \langle C',s' \rangle
$$
Indichiamo perciò con E la generica espressione, con C il generico comando 
e con s il generico stato:

\begin{align*}
(S-SKIP)
&&
\prooftree
\justifies
\langle skip,\ s \rangle \rightarrow s
\endprooftree
\\ \\
(S-EXP.NUM)
&&
\prooftree
   \justifies
   \langle x,\ s \rangle \rightarrow \langle n,\ s \rangle
   \using
   s(x) = n 
\endprooftree
\\ \\
(S-EXP.LEFT)
&&
\prooftree
	\langle E_1,\ s \rangle \rightarrow \langle E_1',\ s \rangle
   \justifies
	\langle (E_1 + E_2),\ s \rangle \rightarrow \langle (E_1' + E_2),\ s \rangle
\endprooftree
\\ \\
(S-EXP.RIGHT)
&&
\prooftree
	\langle E_2,\ s \rangle \rightarrow \langle E_2',\ s \rangle
   \justifies
	\langle (E_1 + E_2),\ s \rangle \rightarrow \langle (E_1 + E_2'),\ s \rangle
\endprooftree
\\ \\
(S-ADD)
&&
\prooftree
   \justifies
   \langle (n_1 + n_2),\ s \rangle \rightarrow \langle n_3,\ s \rangle
   \using
   n_3 = n_1 + n_2
\endprooftree
\\ \\
(S-ASS.EXP)
&&
\prooftree
	\langle E,\ s \rangle \rightarrow \langle E',\ s \rangle
   \justifies
   	\langle x:= E,\ s \rangle \rightarrow \langle x:= E',\ s \rangle
\endprooftree
\\ \\
(S-ASS.NUM)
&&
\prooftree
   \justifies
   	\langle x:= n,\ s \rangle \rightarrow \langle skip,\ s[x \rightarrow n] \rangle
\endprooftree
\\ \\
(S-SEQ.LEFT)
&&
\prooftree
	\langle C_1,\ s \rangle \rightarrow \langle C_1',\ s' \rangle
   \justifies
   	\langle C_1 ; C_2,\ s \rangle \rightarrow \langle C_1' ; C_2,\ s' \rangle
\endprooftree
\\ \\
(S-SEQ.SKIP)
&&
\prooftree
   \justifies
   	\langle skip ; C_2,\ s \rangle \rightarrow \langle C_2,\ s' \rangle
\endprooftree
\\ \\
(S-COND.BOOL)
&&
\prooftree
	\langle B,\ s \rangle \rightarrow \langle B',\ s \rangle
   \justifies
   	\langle if\ B\ then\ C_1\ else\ C_2,\ s \rangle \rightarrow \langle if\ B'\ then\ C_1\ else\ C_2,\ s \rangle
\endprooftree
\\ \\
(S-COND.TRUE)
&&
\prooftree
   \justifies
   	\langle if\ true\ then\ C_1\ else\ C_2,\ s \rangle \rightarrow \langle C_1,\ s \rangle
\endprooftree
\\ \\
(S-COND.FALSE)
&&
\prooftree
   \justifies
   	\langle if\ false\ then\ C_1\ else\ C_2,\ s \rangle \rightarrow \langle C_2,\ s \rangle
\endprooftree
\\ \\
(S-WHILE.BOOL)
&&
\prooftree
	\langle B,\ s \rangle \rightarrow \langle B',\ s \rangle
   \justifies
   	\langle while\ B\ do\ C,\ s \rangle \rightarrow \langle while\ B'\ do\ C,\ s \rangle
\endprooftree
\\ \\
(S-WHILE.FALSE)
&&
\prooftree
   \justifies
   	\langle while\ false\ do\ C,\ s \rangle \rightarrow \langle skip,\ s \rangle
\endprooftree
\\ \\
(S-WHILE)
&&
\prooftree
   \justifies
   	\langle while\ B\ do\ C,\ s \rangle \rightarrow \langle if\ B\ then\ (C ; while\ B\ do\ C)\ else\ skip,\ s \rangle
\endprooftree
\\ \\
(S-LOG.AND.BOOl)
&&
\prooftree
	\langle B_1,\ s \rangle \rightarrow \langle B_1',\ s \rangle
   \justifies
   	\langle B_1\ \land\ B_2,\ s \rangle \rightarrow \langle B_1'\ \land\ B_2,\ s \rangle
\endprooftree
\\ \\
(S-LOG.AND.TRUE)
&&
\prooftree
	\langle B_2,\ s \rangle \rightarrow \langle B_2',\ s \rangle
   \justifies
   	\langle true\ \land\ B_2,\ s \rangle \rightarrow \langle true\ \land\ B_2',\ s \rangle
\endprooftree
\\ \\
(S-LOG.AND.FALSE)
&&
\prooftree
	\langle B_2,\ s \rangle \rightarrow \langle B_2',\ s \rangle
   \justifies
   	\langle false\ \land\ B_2,\ s \rangle \rightarrow \langle false\ \land\ B_2',\ s \rangle
\endprooftree
\\ \\ 
(S-LOG.OR.BOOl)
&&
\prooftree
	\langle B_1,\ s \rangle \rightarrow \langle B_1',\ s \rangle
   \justifies
   	\langle B_1\ \lor\ B_2,\ s \rangle \rightarrow \langle B_1'\ \lor\ B_2,\ s \rangle
\endprooftree
\\ \\
(S-LOG.OR.TRUE)
&&
\prooftree
	\langle B_2,\ s \rangle \rightarrow \langle B_2',\ s \rangle
   \justifies
   	\langle true\ \lor\ B_2,\ s \rangle \rightarrow \langle true\ \lor\ B_2',\ s \rangle
\endprooftree
\\ \\
(S-LOG.OR.FALSE)
&&
\prooftree
	\langle B_2,\ s \rangle \rightarrow \langle B_2',\ s \rangle
   \justifies
   	\langle false\ \lor\ B_2,\ s \rangle \rightarrow \langle false\ \lor\ B_2',\ s \rangle
\endprooftree
\\ \\
(S-CORT.AND.BOOl)
&&
\prooftree
	\langle B_1,\ s \rangle \rightarrow \langle B_1',\ s \rangle
   \justifies
   	\langle B_1\ \& \&\ B_2,\ s \rangle \rightarrow \langle B_1'\ \& \&\ B_2,\ s \rangle
\endprooftree
\\ \\
(S-CORT.AND.TRUE)
&&
\prooftree
	\langle B_2,\ s \rangle \rightarrow \langle B_2',\ s \rangle
   \justifies
   	\langle true\ \& \&\ B_2,\ s \rangle \rightarrow \langle B_2',\ s \rangle
\endprooftree
\\ \\
(S-CORT.AND.FALSE)
&&
\prooftree
	\langle B_2,\ s \rangle \rightarrow \langle B_2',\ s \rangle
   \justifies
   	\langle false\ \& \&\ B_2,\ s \rangle \rightarrow \langle false,\ s \rangle
\endprooftree
\\ \\ 
(S-CORT.OR.BOOl)
&&
\prooftree
	\langle B_1,\ s \rangle \rightarrow \langle B_1',\ s \rangle
   \justifies
   	\langle B_1\ ||\ B_2,\ s \rangle \rightarrow \langle B_1'\ ||\ B_2,\ s \rangle
\endprooftree
\\ \\
(S-CORT.OR.TRUE)
&&
\prooftree
	\langle B_2,\ s \rangle \rightarrow \langle B_2',\ s \rangle
   \justifies
   	\langle true\ ||\ B_2,\ s \rangle \rightarrow \langle true,\ s \rangle
\endprooftree
\\ \\
(S-CORT.OR.FALSE)
&&
\prooftree
	\langle B_2,\ s \rangle \rightarrow \langle B_2',\ s \rangle
   \justifies
   	\langle false\ ||\ B_2,\ s \rangle \rightarrow \langle B_2',\ s \rangle
\endprooftree
\end{align*}

\section{Effetti collaterali}\marginpar{Marano}
Cerchiamo ora di considerare gli effetti collaterali e
osservare come viene modificata la semantica del linguaggio
While. Fino ad ora solo i comandi potevano modificare lo stato
ma adesso anche la valutazione di espressioni e di booleani ma
non l'accesso in memoria.
Abbiamo già visto nel paragrafo precedente la definizione di
\emph{effetto collaterale} e cosa comporta.
Ecco quindi la nuova semantica del linguaggio While:

\begin{align*}
(S-SKIP)
&&
\prooftree
\justifies
\langle skip,\ s \rangle \rightarrow s
\endprooftree
\\ \\
(S-EXP.NUM)
&&
\prooftree
   \justifies
   \langle x,\ s \rangle \rightarrow \langle n,\ s \rangle
   \using
   s(x) = n 
\endprooftree
\\ \\
(S-EXP.LEFT)
&&
\prooftree
	\langle E_1,\ s \rangle \rightarrow \langle E_1',\ s' \rangle
   \justifies
	\langle (E_1 + E_2),\ s \rangle \rightarrow \langle (E_1' + E_2),\ s' \rangle
\endprooftree
\\ \\
(S-EXP.RIGHT)
&&
\prooftree
	\langle E_2,\ s \rangle \rightarrow \langle E_2',\ s' \rangle
   \justifies
	\langle (E_1 + E_2),\ s \rangle \rightarrow \langle (E_1 + E_2'),\ s' \rangle
\endprooftree
\\ \\
(S-ADD)
&&
\prooftree
   \justifies
   \langle (n_1 + n_2),\ s \rangle \rightarrow \langle n_3,\ s' \rangle
   \using
   n_3 = n_1 + n_2
\endprooftree
\\ \\
(S-ASS.EXP)
&&
\prooftree
	\langle E,\ s \rangle \rightarrow \langle E',\ s' \rangle
   \justifies
   	\langle x:= E,\ s \rangle \rightarrow \langle x:= E',\ s' \rangle
\endprooftree
\\ \\
(S-ASS.NUM)
&&
\prooftree
   \justifies
   	\langle x:= n,\ s \rangle \rightarrow \langle skip,\ s[x \rightarrow n] \rangle
\endprooftree
\\ \\
(S-SEQ.LEFT)
&&
\prooftree
	\langle C_1,\ s \rangle \rightarrow \langle C_1',\ s' \rangle
   \justifies
   	\langle C_1 ; C_2,\ s \rangle \rightarrow \langle C_1' ; C_2,\ s' \rangle
\endprooftree
\\ \\
(S-SEQ.SKIP)
&&
\prooftree
   \justifies
   	\langle skip ; C_2,\ s \rangle \rightarrow \langle C_2,\ s' \rangle
\endprooftree
\\ \\
(S-COND.BOOL)
&&
\prooftree
	\langle B,\ s \rangle \rightarrow \langle B',\ s' \rangle
   \justifies
   	\langle if\ B\ then\ C_1\ else\ C_2,\ s \rangle \rightarrow \langle if\ B'\ then\ C_1\ else\ C_2,\ s' \rangle
\endprooftree
\\ \\
(S-COND.TRUE)
&&
\prooftree
   \justifies
   	\langle if\ true\ then\ C_1\ else\ C_2,\ s \rangle \rightarrow \langle C_1,\ s \rangle
\endprooftree
\\ \\
(S-COND.FALSE)
&&
\prooftree
   \justifies
   	\langle if\ false\ then\ C_1\ else\ C_2,\ s \rangle \rightarrow \langle C_2,\ s \rangle
\endprooftree
\\ \\
(S-WHILE.BOOL)
&&
\prooftree
	\langle B,\ s \rangle \rightarrow \langle B',\ s' \rangle
   \justifies
   	\langle while\ B\ do\ C,\ s \rangle \rightarrow \langle while\ B'\ do\ C,\ s' \rangle
\endprooftree
\\ \\
(S-WHILE.FALSE)
&&
\prooftree
   \justifies
   	\langle while\ false\ do\ C,\ s \rangle \rightarrow \langle skip,\ s \rangle
\endprooftree
\\ \\
(S-LOG.AND.BOOL)
&&
\prooftree
	\langle B_1,\ s \rangle \rightarrow \langle B_1',\ s' \rangle
   \justifies
   	\langle B_1\ \land\ B_2,\ s \rangle \rightarrow \langle B_1'\ \land\ B_2,\ s' \rangle
\endprooftree
\\ \\
(S-LOG.AND.TRUE)
&&
\prooftree
	\langle B_2,\ s \rangle \rightarrow \langle B_2',\ s' \rangle
   \justifies
   	\langle true\ \land\ B_2,\ s \rangle \rightarrow \langle true\ \land\ B_2',\ s' \rangle
\endprooftree
\\ \\
(S-LOG.AND.FALSE)
&&
\prooftree
	\langle B_2,\ s \rangle \rightarrow \langle B_2',\ s' \rangle
   \justifies
   	\langle false\ \land\ B_2,\ s \rangle \rightarrow \langle false\ \land\ B_2',\ s' \rangle
\endprooftree
\\ \\ 
(S-LOG.OR.BOOl)
&&
\prooftree
	\langle B_1,\ s \rangle \rightarrow \langle B_1',\ s' \rangle
   \justifies
   	\langle B_1\ \lor\ B_2,\ s \rangle \rightarrow \langle B_1'\ \lor\ B_2,\ s' \rangle
\endprooftree
\\ \\
(S-LOG.OR.TRUE)
&&
\prooftree
	\langle B_2,\ s \rangle \rightarrow \langle B_2',\ s' \rangle
   \justifies
   	\langle true\ \lor\ B_2,\ s \rangle \rightarrow \langle true\ \lor\ B_2',\ s' \rangle
\endprooftree
\\ \\
(S-LOG.OR.FALSE)
&&
\prooftree
	\langle B_2,\ s \rangle \rightarrow \langle B_2',\ s' \rangle
   \justifies
   	\langle false\ \lor\ B_2,\ s \rangle \rightarrow \langle false\ \lor\ B_2',\ s' \rangle
\endprooftree
\\ \\
(S-CORT.AND.BOOl)
&&
\prooftree
	\langle B_1,\ s \rangle \rightarrow \langle B_1',\ s' \rangle
   \justifies
   	\langle B_1\ \& \&\ B_2,\ s \rangle \rightarrow \langle B_1'\ \& \&\ B_2,\ s' \rangle
\endprooftree
\\ \\
(S-CORT.AND.TRUE)
&&
\prooftree
	\langle B_2,\ s \rangle \rightarrow \langle B_2',\ s' \rangle
   \justifies
   	\langle true\ \& \&\ B_2,\ s \rangle \rightarrow \langle B_2',\ s' \rangle
\endprooftree
\\ \\
(S-CORT.AND.FALSE)
&&
\prooftree
	\langle B_2,\ s \rangle \rightarrow \langle B_2',\ s' \rangle
   \justifies
   	\langle false\ \& \&\ B_2,\ s \rangle \rightarrow \langle false,\ s' \rangle
\endprooftree
\\ \\ 
(S-CORT.OR.BOOL)
&&
\prooftree
	\langle B_1,\ s \rangle \rightarrow \langle B_1',\ s' \rangle
   \justifies
   	\langle B_1\ ||\ B_2,\ s \rangle \rightarrow \langle B_1'\ ||\ B_2,\ s' \rangle
\endprooftree
\\ \\
(S-CORT.OR.TRUE)
&&
\prooftree
	\langle B_2,\ s \rangle \rightarrow \langle B_2',\ s' \rangle
   \justifies
   	\langle true\ ||\ B_2,\ s \rangle \rightarrow \langle true,\ s' \rangle
\endprooftree
\\ \\
(S-CORT.OR.FALSE)
&&
\prooftree
	\langle B_2,\ s \rangle \rightarrow \langle B_2',\ s' \rangle
   \justifies
   	\langle false\ ||\ B_2,\ s \rangle \rightarrow \langle B_2',\ s' \rangle
\endprooftree
\end{align*}

\chapter{Semantica denotazionale}

[Da scrivere.]

\chapter{Semantica assiomatica}

In questo capitolo introdurremo la semantica assiomatica, un particolare
tipo di semantica che permette di dimostrare proprietà di programmi.

\section{Correttezza parziale}\marginpar{Mancino}
\begin{definizione}\summary{(Correttezza parziale)}
Un \emph{giudizio di correttezza parziale} è una tripla
della forma
\[
  \{P\} C \{Q\},
\]
dove $P$ e $Q$ sono espressioni logiche dette,
rispettivamente, \emph{precondizione} e \emph{postcondizione},
e $C$ è un comando. Il significato di questa scrittura è
il seguente: se viene eseguito il comando $C$ in uno stato che
soddisfa $P$ e $C$ termina, allora lo stato risultante dall'esecuzione di $C$
soddisferà $Q$.
\end{definizione}

Esempi di giudizi di correttezza parziale sono i seguenti:
\begin{itemize}
\item
$\{ x = 3 \} x := 4 \{ x = 4 \}$, vera in ogni plausibile linguaggio imperativo;
\item
$\{ x = 3 \} x := 4 \{ y = 3 \}$, vera in WHILE ma, in generale,
falsa in altri linguaggi (si consideri il caso, in C++, in cui $y$ sia
un riferimento a $x$);
\item
$\{ P \} \kw{skip} \{ P \}$, vera in WHILE, il che caratterizza
$\kw{skip}$ come un comando che preserva la verità di ogni asserzione.
\end{itemize}

Come si può notare dai precedenti esempi, ciò che asserisce un
giudizio di correttezza parziale può essere vero o falso
in dipendenza dalla semantica del linguaggio di programmazione
considerato.
Definire la verità/falsità di tali giudizi è quindi il modo
per definire tale semantica che va sotto il nome di
\emph{semantica assiomatica}.

\section{Correttezza totale}\marginpar{Lombardi}
\begin{definizione}\summary{(Correttezza totale)}
Un \emph{giudizio di correttezza totale} è una tripla della forma
\[
  [P] C [Q].
\]
A differenza della correttezza parziale la verità di tale giudizio
implica che $C$, se eseguito in uno stato che soddisfa $P$,
termina sempre e risulta in uno store che soddisfa $Q$.
\end{definizione}

I giudizi di correttezza parziale e quelli di correttezza totale
sono anche detti \emph{triple di Hoare}, dal nome del matematico,
C.A.R.\ Hoare che le ha introdotte.

Nel seguito ci occuperemo principalmente di correttezza parziale.
Come detto, ciò che manca per arrivare alla correttezza totale
è una prova di terminazione.  I due aspetti sono relativamente
ortogonali e possono essere affrontati separatamente l'uno
dall'altro.  Informalmente si può scrivere l'equazione:
\[
  \textrm{correttezza totale}
    = \textrm{correttezza parziale} + \textrm{terminazione}.
\]

\section{Logica di specifica}\marginpar{Lombardi}
\subsection{Definizione sintattica}

Definiamo precisamente un linguaggio logico per le
\emph{precondizioni} e \emph{postcondizioni} di una logica di Floyd-Hoare.
Sia $x \in \Var \union \Ghost$, dove $\Var$ è l'insieme delle variabili del
programma e $\Ghost$ è un insieme di cosiddette ``variabili fantasma''
(variabili della logica) tale che $\Var \inters \Ghost = \emptyset$.
Sia $c \in C \sslt \Rset$ una generica costante numerica presa dall'insieme
$C$ delle costanti rappresentabili nel linguaggio.
L'insieme delle espressioni aritmetiche della logica, denotato con $\LExp$,
è ora definibile per mezzo di una grammatica in forma BNF del tipo
\[
  \LE \in \LExp ::= x
                \vbar c
                \vbar \LE_1 + \LE_2
                \vbar \sqrt{\LE}
                \vbar \LE^{\LE_2}
                \vbar \LE!
                \vbar \lfloor \LE \rfloor
                \vbar \dots
\]
L'insieme dei predicati della logica, denotato con $\Pred$, è definito anch'esso con una grammatica in forma BNF del tipo
\begin{align*}
P \in \Pred &::= \true
            \vbar \false
            \vbar\LE_1 = \LE_2
            \vbar \dots
            \vbar P_1 \land P_2
            \vbar P_1 \lor P_2
            \vbar \neg P \\
            &\vbar \forall x \itc P
            \vbar \exists x \st P
            \vbar P_1 \Rightarrow P_2
            \vbar P_1 \Leftrightarrow P_2 \vbar P_1 \Leftarrow P_2
\end{align*}
dove $x \in\Ghost$.

\subsection{Variabili libere}
In logica matematica e in particolare in un linguaggio del primo ordine si dice che una variabile occorre libera in una formula ben formata se nella formula tale variabile appare al di fuori del dominio di un quantificatore sulla variabile stessa. Una variabile $\Ghost$ può essere libera per questo ce ne interessiamo.
Definiamo due funzioni con lo stesso nome (sovraccaricate) per calcolare le variabili libere in una formula.

\begin{definizione} \summary{($\FV$ per $\LExp$.)}
Definiamo la funzione per catturare le variabili libere in una espressione della logica,
\[
  \fund{\FV}{\LExp}{\wp(\Ghost)}
\] 
definita per induzione strutturale su $\LExp$:
\begin{align*}
   \FV(x)
      &=
        \begin{cases}
            \emptyset,&\text{se $x  \in \Var$};\\
            \{x\},    &\text{se $x  \in \Ghost$};
        \end{cases} \\
   \FV(c)
      &= \emptyset;\\
   \FV(\LE_1 + \LE_2)
      &= \FV(\LE_1) \union \FV(\LE_2);\\
   \FV(\sqrt{\LE})
      &= \FV(\LE);\\
   \FV(\LE_1^{\LE_2})
      &= \FV(\LE_1) \union \FV(\LE_2);\\
   \FV(\LE!)
      &= \FV(\LE);\\
   \FV(\lfloor \LE \rfloor)
      &= \FV(\LE).
\end{align*}
\end{definizione}

\begin{definizione} \summary{($\FV$ per $\Pred$.)}
Definiamo la funzione per catturare
le variabili libere nei predicati della logica,
\[
  \fund{\FV}{\Pred}{\wp(\Ghost)} 
\]
definita per induzione strutturale su $\Pred$:
\begin{align*}
   \FV(\true)
      &= \emptyset;\\
   \FV(\false)
      &= \emptyset;\\
   \FV(\LE_1 = \LE_2)
      &= \FV(\LE_1) \union \FV(\LE_2);\\
   \FV(P_1 \land P_2)
      &= \FV(P_1) \union \FV(P_2);\\
   \FV(P_1 \lor P_2)
      &= \FV(P_1) \union \FV(P_2);\\
   \FV(\neg P)
      &= \FV(P);\\
   \FV(\forall x \itc P)
      &= \FV(P)\setminus \{x\};\\
   \FV(\exists x \st P)
      &= \FV(P)\setminus \{x\};\\
   \FV(P_1 \Rightarrow P_2)
      &= \FV(P_1) \union \FV(P_2);\\
   \FV(P_1 \Leftrightarrow P_2)
      &= \FV(P_1) \union \FV(P_2);\\
   \FV(P_1 \Leftarrow P_2)
      &= \FV(P_1) \union \FV(P_2).
\end{align*}
\end{definizione}

\subsection{Sostituzione sintattica}
Per definire la semantica di un tale linguaggio logico abbiamo bisogno
di formalizzare l'operazione di sostituzione sintattica di termini
al posto di variabili.

\begin{definizione} \summary{(Sostituzione per $\LExp$.)}
Sia $x \in \Var \union \Ghost$ una variabile e siano $\LE_0, \LE \in \LExp$
due espressioni.
L'espressione $\LE_0\substt{\LE}{x}$ è un elemento di $\LExp$ definito
per induzione strutturale su $\LE_0$ come segue:
\begin{align*}
   x_0\substt{\LE}{x}
    &=
      \begin{cases}
        \LE, &\text{se $x_0 = x$;} \\
        x_0, &\text{se $x_0 \neq x$;}
      \end{cases} \\
   c\substt{\LE}{x}
    &= c;\\
  (\LE_1 + \LE_2)\substt{\LE}{x}
     &= \bigl(\LE_1\substt{\LE}{x}\bigr) + \bigl(\LE_2\substt{\LE}{x}\bigr);\\
  \sqrt{\LE_1}\substt{\LE}{x}
     &= \sqrt{{\LE_1}\substt{\LE}{x}};\\
  \LE_1^{\LE_2}\substt{\LE}{x}
     &= \LE_1\substt{\LE}{x}^{\LE_2\substt{\LE}{x}};\\
  {\LE_1}!\substt{\LE}{x}
     &= {{\LE_1}\substt{\LE}{x}}!;\\
  \lfloor \LE_1 \rfloor\substt{\LE}{x}
     &= \bigl\lfloor \LE_1\substt{\LE}{x} \bigr\rfloor.\\
\end{align*}
La definizione si estende in modo ovvio ad ogni altro operatore presente
nella sintassi delle espressioni.
\end{definizione}

\begin{definizione} \summary{(Sostituzione per $\Pred$.)}
Sia $x \in \Var \union \Ghost$ una variabile, $\LE \in \LExp$ un'espressione
e $P \in \Pred$ un predicato.
L'espressione $P\substt{\LE}{x}$ è un elemento di $\Pred$ definito
per induzione strutturale su $P$ come segue:
\begin{align*}
  \true\substt{\LE}{x}
    &= \true;\\
  \false\substt{\LE}{x}
    &= \false;\\
  (\LE_1 = \LE_2)\substt{\LE}{x}
    &= \bigl(\LE_1\substt{\LE}{x}\bigr) = \bigl(\LE_2\substt{\LE}{x}\bigr);\\
  (P_1 \land P_2)\substt{\LE}{x}
    &= \bigl(\LE_1\substt{\LE}{x}\bigr) \land \bigl(\LE_2\substt{\LE}{x}\bigr);\\
  (P_1 \lor P_2)\substt{\LE}{x}
    &= \bigl(\LE_1\substt{\LE}{x}\bigr) \lor \bigl(\LE_2\substt{\LE}{x}\bigr);\\
  (\neg P)\substt{\LE}{x}
    &= \neg \bigl(P \substt{\LE}{x}\bigr);\\
  (\forall y\itc P)\substt{\LE}{x}
    &= \forall z\itc \bigl(P \substt{z}{y}\bigr)\substt{\LE}{x} \text{ con: $z\notin\FV(\LE), z\notin\FV(P), z\neq x$};\\
  (\exists  y\st P)\substt{\LE}{x}
    &= \exists z\st \bigl(P \substt{z}{y}\bigr)\substt{\LE}{x} \text{ con: $z\notin\FV(\LE), z\notin\FV(P), z\neq x$};\\
  (P_1 \Rightarrow P_2)\substt{\LE}{x}
    &= \bigl(P_1\substt{\LE}{x}\bigr) \Rightarrow \bigl(P_2\substt{\LE}{x}\bigr);\\
  (P_1 \Leftrightarrow P_2)\substt{\LE}{x}
    &= \bigl(P_1\substt{\LE}{x}\bigr) \Leftrightarrow \bigl(P_2\substt{\LE}{x}\bigr);\\
  (P_1 \Leftarrow P_2)\substt{\LE}{x}
    &= \bigl(P_1\substt{\LE}{x}\bigr) \Leftarrow \bigl(P_2\substt{\LE}{x}\bigr).
\end{align*}
\end{definizione}

\subsection{Valutazione semantica}
Per definire la semantica di ogni oggetto sintattico del linguaggio,
abbiamo bisogno di: un insieme degli stati denotato con $\Sigma$ utile in quanto nelle espressioni possono comparire variabili del programma e un ambiente delle variabili denotato con $\Gamma$
definita come $\fund{\Gamma}{\Ghost}{\Zset}$.

\begin{definizione} \summary{(Valutazione semantica per $\LExp$.)}
Definiamo la funzione di valutazione semantica delle espressioni 
\[
   \fund{\llbracket \cdot \rrbracket ^\LExp}{\LExp}{((\Sigma \times \Gamma) \rightarrow \Zset)}
\]
definita per induzione strutturale su $\LExp$:
\begin{align*}
   \llbracket x \rrbracket(s,t)
      &=
        \begin{cases}
                 s(x), &\text{se $x \in \Var$};\\
                 t(x), &\text{se $x \in \Ghost$};
        \end{cases}\\
   \llbracket  c \rrbracket(s,t)
     &=  c;\\
   \llbracket \LE_1 + \LE_2 \rrbracket(s,t)
     &= \llbracket \LE_1 \rrbracket(s,t) + \llbracket \LE_2 \rrbracket(s,t);\\
   \llbracket \sqrt{\LE} \rrbracket(s,t)
     &= \sqrt{\llbracket \LE \rrbracket}(s,t);\\
   \llbracket \LE^{\LE_2} \rrbracket(s,t)
     &= \llbracket \LE \rrbracket^{\llbracket \LE_2 \rrbracket}(s,t);\\
   \llbracket \LE! \rrbracket(s,t)
     &= \llbracket \LE \rrbracket!(s,t);\\
   \llbracket \lfloor \LE \rfloor \rrbracket(s,t)
     &= \bigl \lfloor \llbracket \LE \rrbracket \bigr \rfloor(s,t).
\end{align*}
\end{definizione}

\begin{definizione} \summary{(Valutazione semantica per $\Pred$.)}
La funzione di valutazione semantica dei predicati è definita come segue:
\[
   \fund{\llbracket \cdot \rrbracket ^{\Pred}}{\Pred}{((\Sigma \times \Gamma) \rightarrow \{\ttv,\ffv\})}
\]
definita per induzione strutturale su $\Pred$:
\begin{align*}
   \llbracket \true \rrbracket(s,t)
      &= \ttv;\\
   \llbracket \false \rrbracket(s,t)
      &= \ffv;\\
   \llbracket \LE_1 = \LE_2\rrbracket(s,t)
      &=
        \begin{cases}
                \ttv, &\text{se $\llbracket\LE_1\rrbracket(s,t) = \llbracket \LE_2 \rrbracket(s,t)$};\\
                \ffv, &\text{altrimenti};
        \end{cases}\\
    \llbracket P_1 \land P_2\rrbracket(s,t)
      &=
        \begin{cases}
                \ttv, &\text{se $\llbracket P_1\rrbracket(s,t) = \llbracket P_2\rrbracket(s,t) = \true$};\\
                \ffv, &\text{altrimenti};
        \end{cases}\\
    \llbracket P_1 \lor P_2\rrbracket(s,t)
      &=
        \begin{cases}
                \ffv, &\text{se $\llbracket P_1\rrbracket(s,t) = \llbracket P_2\rrbracket(s,t) = \false$};\\
                \ttv, &\text{altrimenti};
        \end{cases} \\
    \llbracket \neg P\rrbracket(s,t)
       &=
        \begin{cases}
                \ttv,&\text{se $\llbracket P\rrbracket(s,t) = \ffv$};\\
                \ffv,&\text{se $\llbracket P\rrbracket(s,t) = \ttv$};
        \end{cases}
\end{align*}
Poniamo attenzione ai due casi particolari:
\begin{align*}
   \llbracket\forall x \itc P\rrbracket(s,t)
      &=
        \begin{cases}
                \ttv, &\text{se $\forall k\in\Zset\itc(\llbracket P\rrbracket(s,t\substt{k}{x}) = \ttv$});\\
                \ffv, &\text{se $\exists k\in\Zset\st(\llbracket P\rrbracket(s,t\substt{k}{x}) = \ffv$});
        \end{cases}\\
   \llbracket \exists x \st P\rrbracket(s,t)
      &=
        \begin{cases}
                \ttv,  &\text{se $\exists k\in \Zset\st(\llbracket P\rrbracket(s,t\substt{k}{x}) = \ttv$});\\
                \ffv,  &\text{se $\forall k\in \Zset\itc(\llbracket P\rrbracket(s,t\substt{k}{x}) = \ffv$});
        \end{cases}
\end{align*}
Dove:
\begin{align*}
   \forall x,y\in\Ghost\itc\forall t \in\Gamma\itc\forall k\in\Zset \itc t\substt{k}{x}(y)
      &=
        \begin{cases}
                 k,   &\text{se $x = y$};\\
                 t(y),&\text{se $x  \neq  y$};
        \end{cases}
\end{align*}
\begin{align*}
    \llbracket P_1 \Rightarrow P_2 \rrbracket(s,t)
       &=
        \begin{cases}
            \ffv, &\text{se $\llbracket P_1\rrbracket(s,t) = \true \land \llbracket P_2\rrbracket(s,t) = \false$};\\
            \ttv, &\text{altrimenti};
        \end{cases}\\
    \llbracket P_1 \Leftrightarrow P_2\rrbracket(s,t)
       &=
        \begin{cases}
            \ttv,&\text{se $\llbracket P_1\rrbracket(s,t) = \llbracket P_2\rrbracket(s,t) = \true$}\\
                 &\text{$\lor\llbracket P_1\rrbracket(s,t) = \llbracket P_2 \rrbracket(s,t) = \false$};\\
            \ffv, &\text{altrimenti};
        \end{cases}\\
   \llbracket P_1 \Leftarrow P_2\rrbracket(s,t)
      &=
        \begin{cases}
            \ffv,&\text{se $\llbracket P_1\rrbracket(s,t) = \false \land \llbracket P_2\rrbracket(s,t) = \true$};\\
            \ttv,&\text{altrimenti}.
   \end{cases}
\end{align*}
\end{definizione}

\subsection{Correttezza della logica di Floyd-Hoare}

\begin{definizione} \summary{(Verità di una tripla di Floyd-Hoare.)}
Ora che abbiamo tutti gli elementi, possiamo definire quando una tripla di Floyd-Hoare è vera.
\begin{align*}
    \llbracket \{P\} C\{Q\} \rrbracket 
       = \ttv \Leftrightarrow \forall t \in \Gamma_{\FV(P)\union \FV(Q)}\itc
       &(\forall  s,s'\in \Sigma\itc(\llbracket P \rrbracket(s,t) = \ttv\\
       &\land \config{C}{s} \ssarrow ^*\config{skip}{s'}\\
       &\Rightarrow\llbracket Q \rrbracket(s',t)=\ttv);
\end{align*}
Dove:
\[
   \forall v \subset \Ghost,\Gamma_{v} =\{t \in \Gamma \text{ t.c; $\dom(t) = v$\}}.
\]
\end{definizione}

\begin{teorema} \summary{(Correttezza della logica di Floyd-Hoare.)}
Se la tripla di Floyd-Hoare ha dimostrazione, allora essa è vera.
\[
   \forall P,Q \in \Pred \itc \forall C \in \Com : \vdash \{P\}C\{Q\} \Rightarrow \llbracket \{P\}C\{Q\}\rrbracket;
\]
\begin{proof}
Si dimostra per induzione strutturale sull'albero che dimostra \{P\}C\{Q\}:
\textbf{Skip}
\[
 \prooftree
 \justifies
    \{P\}\kw{skip}\{P\}
 \thickness=0.08em
 \shiftright 2em
 \using
 \endprooftree
\]
\begin{align*}
   \llbracket \{P\}\kw{skip}\{P\} \rrbracket \Leftrightarrow \forall t \in \Gamma_{\FV(P)}\itc
      &\forall s,s'\in \Sigma \itc(\llbracket P \rrbracket(s,t) = \ttv\\
      &\land \config{skip}{s} \ssarrow ^*\config{skip}{s'}\\
      &\Rightarrow\llbracket P \rrbracket(s',t)=\ttv);
\end{align*}

Dato che la premessa è vera abbiamo che $s=s'$, dunque:
\[
   \llbracket P \rrbracket(s,t)= \ttv = \llbracket P \rrbracket(s',t).
\]

\textbf{Assegnamento}
\[
 \prooftree
 \justifies
    \{Q\substt{\LE}{x}\}x:=\LE\{Q\}
 \thickness=0.08em
 \shiftright 2em
 \using
 \endprooftree
\]
\begin{align*}
   \llbracket \{Q\substt{\LE}{x}\}x:=\LE\{Q\} \rrbracket \Leftrightarrow
      &\forall t \in \Gamma_{\FV(\LE)\union \FV(Q)}\itc\forall s,s'\in \Sigma \itc\\
      &(\llbracket Q \rrbracket\substt{\LE}{x}(s,t) = \ttv\\
      &\land \config{x:=\LE}{s} \ssarrow ^*\config{skip}{s'}\\
      &\Rightarrow\llbracket Q \rrbracket(s',t)=\ttv);
\end{align*}
Questo grazie al \textit{Lemma 1}.
\begin{lemma}
\begin{align*}
   \forall k \in \Nset \itc
      &( \config{x:=\LE}{s} \ssarrow ^k\config{skip}{s'}\\
      &\Rightarrow \exists n \in \Zset\st \config{\LE}{s} \ssarrow ^{k-1}\config{n}{s}
      \land s'=s \substt{n}{x});
\end{align*}
\end{lemma}

\begin{proof}
Caso base: con $k = 0$ è banalmente vera poiché l'antecedente è falso;\\
Con: $k = 1$ abbiamo:
\[
   \config{x:=\LE}{s} \ssarrow ^{1}\config{skip}{s'};
\]
ma allora $\LE \in \Zset$ e $s'=s\substt{n}{x},$ quindi $\exists n=\LE$ tali che:
\[
   \config{\LE}{s} \ssarrow ^{0}\config{\LE}{s} \equiv \config{n}{s}.
\]
Questo per definizione di $\rightarrow ^0$ chiusura riflessiva.\\
Passo induttivo: $k = h+1$
\[
   \config{x:=\LE}{s} \ssarrow ^{h+1}\config{skip}{s'};
\]
Allora si può ricondurre ad $h+1$ passi per poi applicare l'ipotesi induttiva:
\[
   \config{x:=\LE}{s} \ssarrow ^{1}\config{x:=\LE'}{s};
\]
Ora:
\[
   \config{x:=\LE'}{s} \ssarrow ^{h}\config{skip}{s'};
\]
per ipotesi induttiva si ha che:
\[
   \exists n \in\Zset\itc \config{\LE'}{s} \ssarrow ^{h-1}\config{n}{s};
\]
e inoltre:
\[
   s'=s \substt{n}{x};
\]
Allora aggiungendo l'ultimo passo:
\[
   \config{x:=\LE}{s} \ssarrow ^{1}\config{x:=\LE'}{s} \ssarrow ^{h} \config{skip}{s'};
\]
\[
   \config{\LE}{s} \ssarrow ^{1}\config{\LE'}{s} \ssarrow ^{h-1} \config{n}{s'};
\]
Allora:
\[
   \config{\LE}{s} \ssarrow ^{h} \config{skip}{s'}.
\]
\end{proof}

\textbf{Composizione Sequenziale}
\[
  \prooftree
     \{P\} C_1\{Q\}
   \hspace{1mm}
     \{Q\} C_2\{R\}
   \justifies
     \{P\}C_1;C_2\{R\}
   \thickness=0.08em
   \shiftright 0em
   \using
  \endprooftree
\]
\begin{align*}
   \llbracket \{P\} C_1;C_2\{R\} \rrbracket \Leftrightarrow \forall t \in \Gamma_{\FV(P)\union\FV(R)}\itc
      &(\forall s,s'\in \Sigma \itc(\llbracket P \rrbracket(s,t) = \ttv\\
      &\land    \config{C_1;C_2}{s} \ssarrow ^*\config{skip}{s'}\\
      &\Rightarrow\llbracket R \rrbracket(s',t)=\ttv);
\end{align*}
\begin{align*}
   \llbracket \{P\}C_1\{Q\} \rrbracket \land \llbracket \{Q\}C_2\{R\} \rrbracket \Leftrightarrow
      &\forall t \in \Gamma_{\FV(P)\union\FV(R)\union\FV(Q)}\itc\\
      &\forall s,s'\in \Sigma \itc(\llbracket P \rrbracket(s,t) = \ttv\\
      &\land \config{C_1}{s} \ssarrow ^*\config{skip}{s'}\\
      &\land \llbracket Q\rrbracket(s',t)=\ttv \\
      &\land \config{C_2}{s'} \ssarrow ^*\config{skip}{s''})\\
      &\Rightarrow \llbracket R \rrbracket(s'',t)=\ttv;
\end{align*}
Questo grazie al \textit{Lemma 2}.
\begin{lemma}
\begin{align*}
   \forall C_1,C_2 \in \Com \itc 
   &\forall s,s' \in \Sigma \itc \forall n \in \Nset \config{C_1;C_2}{s} \ssarrow ^n\config{skip}{s'} \\
   &\Rightarrow \exists n_1,n_2 \in \Nset \st \exists s'' \in \Sigma \st \\
   &\config{C_1}{s} \ssarrow ^{n_1}\config{skip}{s'} \land \config{C_2}{s'} \ssarrow ^{n_2}\config{skip}{s''}
   \land n = n_1 + n_2.
\end{align*}
\end{lemma}
[Segalini]
\end{proof}
\end{teorema}

\subsection{Programma propriamente annotato}\marginpar{Mancino}
\begin{definizione}
Un programma si dice \emph{propriamente annotato} se ci sono annotazioni nei seguenti punti:
\begin{itemize}
        \item Prima ogni comando $C_i$ in una composizione sequenziale $C_1;C_2; \dots ;C_n$ in cui $C_i$ non è un comando di assegnamento.
        \item Dopo la parola \textbf{do} in un comando \textbf{while}.
\end{itemize}
\end{definizione}


\section{Esercizi}\marginpar{Mancino}

\subsection{Esercizio 1}
Dare una dimostrazione per la seguente tripla di Hoare:
\[ \{X=a  \land Y=b\}  X:=X+Y; \; Y:=X-Y; \; X:=X-Y  \{Y=a  \land  X=b\} \]

\begin{proof}[Svolgimento]
Dimostriamo la tripla partendo dal fondo. Applicando l'assioma dell'assegnamento si ha:
\[ \{S[\nicefrac{X-Y}{X}] \}  X:=X-Y  \{Y=a  \land  X=b\} \]
\[ \Downarrow \]
\[ \{Y=a  \land X-Y=b \}  X:=X-Y  \{Y=a  \land  X=b\} \]
Risaliamo ora di un "livello" applicando ancora l'assioma dell'assegnamento:
\[ \{S[\nicefrac{X-Y}{Y}] \}  Y:=X-Y  \{Y=a  \land  X-Y=b\} \]
\[ \Downarrow \]
\[ \{X-Y=a  \land X-(X-Y)=b \}  Y:=X-Y  \{Y=a  \land  X-Y=b\} \]
Semplificando la precondizione si ottiene:
\[ \{X-Y=a  \land Y=b \}  Y:=X-Y  \{Y=a  \land  X-Y=b\} \]
Risaliamo ulteriormente di un livello, applicando l'assioma dell'assegnamento. Si ha:
\[ \{S[\nicefrac{X+Y}{Y}] \}  X:=X+Y  \{X-Y=a  \land  Y=b\} \]
\[ \Downarrow \]
\[ \{(X+Y)-Y=a  \land Y=b \}  X:=X+Y  \{X-Y=a  \land  Y=b\} \]
Semplificando ancora una volta la precondizione si ottiene:
\[ \{X=a  \land Y=b \}  X:=X+Y  \{X-Y=a  \land  Y=b\} \]
Che è quanto volevamo dimostrare.
\end{proof}

\subsection{Esercizio 2}
Dare una dimostrazione per la seguente tripla di Hoare:
\[ \{X=R+(Y \times Q) \} \; BEGIN \; R:=R-Y; \; Q:=Q+1 \; END \; \{X=R+(Y \times Q)\} \]

\begin{proof}[Svolgimento]
Come nell'esercizio precedente, procediamo dal fondo.\\
Applichiamo quindi l'assioma dell'assegnamento:
\[ \{S[\nicefrac{Q+1}{Q}] \} \; Q:=Q+1 \; \{X=R+(Y \times Q)\} \]
\[ \Downarrow \]
\[ \{X=R+(Y \times (Q+1)) \} \; Q:=Q+1 \; \{X=R+(Y \times Q)\} \]
Semplificando la precondizione si ottiene:
\[ \{X=R+Y \times Q + Y \} \; Q:=Q+1 \; \{X=R+(Y \times Q)\} \]
Risalendo di un livello e applicando ancora l'assioma dell'assegnamento si trova:
\[ \{S[\nicefrac{R-Y}{R}] \} \; R:=R-Y \; \{X=R+Y \times Q + Y \} \]
\[ \Downarrow \]
\[ \{X= R-Y+Y \times Q+Y \} \; R:=R-Y \; \{X=R+Y \times Q + Y \} \]
Semplificando la precondizione si ottiene:
\[ \{X= R+(Y \times Q) \} \; R:=R-Y \; \{X=R+Y \times Q + Y \} \]
Che è quanto volevamo dimostrare.
\end{proof}

\subsection{Esercizio 3 \emph{(Esonero del 28/11/2014)}}
Lo xor bit a bit, $ \oplus: \mathbb{N} \times \mathbb{N} \rightarrow \mathbb{N} $, gode delle
seguenti proprietà:
\begin{itemize}
        \item $ x \oplus (y \oplus z) = (x \oplus y) \oplus z $ \emph{(associatività)}
        \item $ x \oplus y = y \oplus x $ \emph{(commutatività)}
        \item $ x \oplus 0 = x $ \emph{(elemento neutro)}
        \item $ x \oplus x = 0 $ \emph{(nipotenza/ogni elemento è l'inverso di se stesso)}
\end{itemize}
Si dimostri, usando la logica di Floyd-Hoare, che la composizione sequenziale:
\begin{center}
\texttt{X := X XOR Y; Y := X XOR Y; X := X XOR Y}
\end{center}
realizza lo scambio dei valori tra le variabili \texttt{X} e \texttt{Y} se queste, all'inizio della computazione, contengono valori naturali.

\begin{proof}[Svolgimento]
Dobbiamo dimostrare che la seguente tripla di Hoare è corretta:

\begin{center}
$ \{ \mactext{X=a} \; \land \; \mactext{Y=b} \} $
\texttt{X := X XOR Y; Y := X XOR Y; X := X XOR Y}
$\{\mactext{X=b} \; \land \; \mactext{Y=a} \} $
\end{center}
Procediamo, al solito, partendo dal fondo. Applichiamo dunque l'assioma dell'assegnamento:

\begin{center}
$ \{ S[\nicefrac{\mactext{X XOR Y}}{\mactext{X}}] \} $
\texttt{X := X XOR Y}
$\{\mactext{X=b} \; \land \; \mactext{Y=a} \} $
\end{center}
\[ \Downarrow \]
\begin{center}
$ \{\mactext{X XOR Y = b} \; \land \mactext{Y=a}  \} $
\texttt{X := X XOR Y}
$\{\mactext{X=b} \; \land \; \mactext{Y=a} \} $
\end{center}

Passiamo al successivo assegnamento:
\begin{center}
$ \{ S[\nicefrac{\mactext{X XOR Y}}{\mactext{Y}} ] \} $
\texttt{Y := X XOR Y}
$ \{\mactext{X XOR Y = b} \; \land \mactext{Y=a} \} $
\end{center}
\[ \Downarrow \]
\begin{center}
$ \{ \mactext{X XOR (X XOR Y) = b} \; \land \; \mactext{X XOR Y = a} \} $
\texttt{Y := X XOR Y}
$ \{ \mactext{X XOR Y = b} \; \land \; \mactext{Y=a} \} $
\end{center}
Applichiamo la proprietà di associatività dello XOR in precondizione:
\begin{center}
$ \{ \mactext{(X XOR X) XOR Y = b} \; \land \; \mactext{X XOR Y = a} \} $
\texttt{Y := X XOR Y}
$ \{ \mactext{X XOR Y = b} \; \land \; \mactext{Y=a} \} $
\end{center}
Applichiamo la proprietà di nipotenza dello XOR in precondizione:
\begin{center}
$ \{ \mactext{0 XOR Y = b} \; \land \; \mactext{X XOR Y = a} \} $
\texttt{Y := X XOR Y}
$ \{ \mactext{X XOR Y = b} \; \land \; \mactext{Y=a} \} $
\end{center}
Applichiamo la proprietà di elemento neutro dello XOR in precondizione:
\begin{center}
$ \{ \mactext{Y=b} \; \land \; \mactext{X XOR Y = a} \} $
\texttt{Y := X XOR Y}
$ \{ \mactext{X XOR Y = b} \; \land \; \mactext{Y=a} \} $
\end{center}

Passiamo ora al successivo, e ultimo, assegnamento della tripla:
\begin{center}
$ \{ S[\nicefrac{\mactext{X XOR Y}}{\mactext{X}} ] \} $
\texttt{X := X XOR Y}
$ \{ \mactext{Y=b} \; \land \; \mactext{X XOR Y = a} \} $
\end{center}
\[ \Downarrow \]
\begin{center}
$ \{ \mactext{Y=b} \; \land \; \mactext{(X XOR Y) XOR Y = a} \} $
\texttt{X := X XOR Y}
$ \{ \mactext{Y=b} \; \land \; \mactext{X XOR Y = a} \} $
\end{center}
Applichiamo la proprietà di associatività dello XOR in precondizione:
\begin{center}
$ \{ \mactext{Y=b} \; \land \; \mactext{X XOR (Y XOR Y) = a} \} $
\texttt{X := X XOR Y}
$ \{ \mactext{Y=b} \; \land \; \mactext{X XOR Y = a} \} $
\end{center}
Applichiamo la proprietà di nipotenza dello XOR in precondizione:
\begin{center}
$ \{ \mactext{Y=b} \; \land \; \mactext{X XOR 0 = a} \} $
\texttt{X := X XOR Y}
$ \{ \mactext{Y=b} \; \land \; \mactext{X XOR Y = a} \} $
\end{center}
Applichiamo la proprietà di elemento neutro dello XOR in precondizione:
\begin{center}
$ \{ \mactext{Y=b} \; \land \; \mactext{X=a} \} $
\texttt{X := X XOR Y}
$ \{ \mactext{Y=b} \; \land \; \mactext{X XOR Y = a} \} $
\end{center}
Otteniamo dunque quello che volevamo dimostrare.
\end{proof}


\subsection{Esercizio 4}

Dimostrare la correttezza della seguente tripla di Hoare:
\begin{lstlisting}[mathescape, numberfirstline=false, frame=single]
$ [x = n] $
if x < 0 then
        y := 0;
else
        y := 1;
        while x > 0 do $ \{y = 2^{n-x} \land x \geq 0 \} $
                y = y*2;
                x = x-1;
$ [y = \lfloor 2^n \rfloor ] $
\end{lstlisting}

\begin{proof}[Svolgimento]
Il codice è completamente annotato. Dimostriamo in primo luogo la correttezza parziale della tripla. Applicando l'assioma dell'if allora otteniamo le due triple:
\begin{align}
& \{x=n \; \land \; x<0 \} \; \mactext{y:=0} \; \{y= \lfloor 2^n \rfloor \} \\
& \{x=n \; \land \; x \geq 0 \} \; \mactext{y:=1; while} \texttt{\dots } \; \{ y= \lfloor 2^n \rfloor \}
\end{align}
Dalla (1) ricaviamo la prima VC:
\[ x = n \; \land \; x<0 \; \Rightarrow \; 0 = \lfloor 2^n \rfloor \]
Ed è banale argomentarne la veridicità.\
Dalla (2), dal momento che la postcondizione dopo l'assegnamento \texttt{y:=1} è l'invariante del While, applicando l'assioma derivato dell'assegnamento si ottiene la seconda VC:
\[ x=n \; \land \; x \geq 0 \; \Rightarrow \; 1 = 2^{n-x} \; \land \; x \geq 0 \]
Ed anche in questo caso è banale argomentare la veridicità di questa implicazione.\
Ci rimane da dimostrare la tripla di Hoare:
\[ \{ y=2^{n-x} \; \land \; x>0 \} \; \mactext{while} \texttt{\dots } \; \{y= \lfloor 2^n \rfloor \} \]
Applicando la regola derivata del While si ottengono le seguenti VCs:
\begin{itemize}
        \item $ y=2^{n-x} \; \land \; x \geq 0 \; \Rightarrow \; y=2^{n-x} \; \land \; x \geq 0 $ \\
                  che è banalmente vera.
        \item $ y=2^{n-x} \; \land \; x \geq 0 \; \land \; x \leq 0 \; \Rightarrow \; y = \lfloor 2^n \rfloor $ \\
                  ed è vera in quanto $ x \geq 0 \; \land \; x \leq 0 \; \Rightarrow \; x = 0 $ e quindi
                  dall'antecedente si ottiene il conseguente.
        \item Tutte le VCs generate dalla tripla:
                  \[ \{y=2^{n-x} \; \land \; x \geq 0 \; \land \; x>0 \} \;
                     \mactext{y:=y*2; x=x-1}
                     \; \{y=2^{n-x} \; \land \; x \geq 0 \} \]
                  Applicando la regola dell'assegnamento in sequenza e la regola derivata dell'assegnamento ricaviamo
                  l'ultima VC:
                  \[ y=2^{n-x} \; \land \;  x \geq 0 \; \land \; x>0 \;
                         \Rightarrow
                         \; y*2 = 2^{n-x+1} \; \land x-1 \geq 0 \]
                  che si può semplificare in:
                  \[ y=2^{n-x} \; \land \; x>0 \;
                         \Rightarrow
                         \; y = 2^{n-x} \; \land x-1 \geq 0 \]
                  che ancora una volta è vera.
\end{itemize}
Quindi la correttezza parziale della tripla è dimostrata. Ci rimane da dimostrare la totalità e per farlo dobbiamo provare che valgono le seguenti:
\begin{align*}
& P \; \land \; S \; \Rightarrow \; V \geq 0 \\
& \{P \; \land \; S \; \land \; V=k \} \; C \; \{ P \; \land \; V <k \}
\end{align*}
Si ha:
\begin{itemize}
\item $ y=2^{n-x} \; \land \; x \geq 0 \; \land \; x>0 \; \Rightarrow \; x \geq 0 $ \\
          che è vera, dato che $ x>0 \; \Rightarrow \; x \geq 0 $.
\item Dobbiamo dimostrare la veridicità della tripla:
                \[ \{y=2^{n-x} \; \land \; x \geq 0 \; \land \; x>0 \; \land x=k \} \;
                \mactext{y:=y*2; x=x-1}
                \; \{y=2^{n-x} \; \land \; x < k \} \]
          Applicando ancora la regola dell'assegnamento in sequenza e l'assioma derivato dell'assegnamento si ha:
          \[ y = 2^{n-x} \; \land \; x \geq 0 \; \land \; x=k \;
             \Rightarrow
             \; y*2 = 2^{n-x+1} \; \land \; x-1<k \]
          che si può semplificare in:
          \[ y = 2^{n-x} \; \land \; x \geq 0 \; \land \; x=k \;
             \Rightarrow
             \; y = 2^{n-x} \; \land \; x-1<k \]
          che è vera, dal momento che si otterrebbe $ k-1<k $.
\end{itemize}

Questo prova la correttezza totale.
\end{proof}

\subsection{Esercizio 5}
Estendere la regola del blocco vista a lezione, svincolandola dalla condizione secondo la quale nessuna variabile che compare all'interno del blocco possa comparire in precondizione o in postcondizione.

\begin{proof}[Svolgimento]
Possiamo, a livello di preprocessore, rinominare tutte le variabili di blocco con un nome univoco $ y_1,\dots ,y_n $, facendo attenzione al fatto che abbiano un identificatore univoco. La regola allora è riscrivibile come:
\[
\prooftree
        \vdash \; \{ P \} \; C[\nicefrac{y_1}{v_1}]\dots [\nicefrac{y_n}{v_n}] \; \{ Q \}
        \justifies
                \{ P \} \; \mactext{BEGIN Var } v_1; \dots v_n; \; C \; \mactext{END} \; \{ Q \}
        \thickness=0.08em
        \using
                (Vars(P) \union Vars(Q)) \inters \{ y_1,\dots ,y_n \} = \emptyset
\endprooftree
\]
\end{proof}

\chapter{Relazioni tra le semantiche}
In questo capitolo verranno introdotti alcuni esercizi e alcune osservazioni
che non riguardano una specifica semantica vista nei capitoli precedenti,
ma le affrontano un po' tutte.

\subsection{Conservazione passi small step rispetto a denotazionale}

\begin{proposizione}[Conservazione passi small step per espressioni]
\label{conservazione-passi-small-step-expr}
\[
  \forall E \in \AExp \itc \config{E}{s} \ssarrow \config{E'}{s}
  \implies \calE\llbracket E \rrbracket(s) = \calE\llbracket E' \rrbracket(s)
\]

\begin{proof}
Per induzione strutturale sulla costruzione di $\AExp$ (consideriamo
espressioni senza ``side effects'' nella semantica small step poichè sono
definite in talo modo anche nella denotazionale):

Caso base: $E = x$.

Per ipotesi abbiamo che:
\[
  \config{x}{s} \ssarrow \config{n}{s} \land s(x) = n,
\]
dunque:
\begin{align*}
  \calE\llbracket x \rrbracket(s) &= s(x),
    &\law{per def. della denotazionale} \\
  &= n,
    &\law{per ipotesi} \\
  &= \calE\llbracket n \rrbracket(s).
\end{align*}

Caso base: $E = n_1 + n_2$.

Per ipotesi abbiamo che:
\[
  \config{n_1 + n_2}{s} \ssarrow \config{n_3}{s} \land n_1 + n_2 = n_3,
\]
dunque:
\begin{align*}
  \calE\llbracket n_1 + n_2 \rrbracket(s)
  &= \calE\llbracket n_1 \rrbracket(s)
  + \calE\llbracket n_2 \rrbracket(s),
    &\law{per def. della denotazionale} \\
  &= n_1 + n_2,
    &\law{per def. della denotazionale} \\
  &= n_3,
    &\law{per ipotesi} \\
  &= \calE\llbracket n_3 \rrbracket(s).
\end{align*}

Passo induttivo: $E = E_1 + E_2$.

Per ipotesi abbiamo che:
\[
  \config{E_1 + E_2}{s} \ssarrow \config{E_1' + E_2}{s} \land
  \config{E_1}{s} \ssarrow \config{E_1'}{s}
\]
dunque:
\begin{align*}
  \calE\llbracket E_1 + E_2 \rrbracket(s)
  &= \calE\llbracket E_1 \rrbracket(s) +  \calE\llbracket E_2
    \rrbracket(s),
    &\law{per def. della denotazionale} \\
  &= \calE\llbracket E_1' \rrbracket(s) +  \calE\llbracket E_2
    \rrbracket(s),
    &\law{per ipotesi induttiva} \\
  &= \calE\llbracket E_1' + E_2 \rrbracket(s).
\end{align*}
\end{proof}
\end{proposizione}

\begin{proposizione}[Conservazione passi small step per comandi]
\[
  \forall C \in \Com \itc \config{C}{s} \ssarrow \config{C'}{s'}
  \implies \calC\llbracket C \rrbracket(s) = \calC\llbracket C' \rrbracket(s')
\]
\begin{proof}
Per induzione sulla struttura di $\Com$ (come per il lemma precedente
la semantica small step delle espressioni data è senza ``side effects'':

Caso base: $C = (x \weq n)$.

Per definizione della semantica small step:
\[
  \config{x \weq n}{s} \ssarrow \config{\kw{skip}}{\assign{s}{n}{x}},
\]
dunque:
\begin{align*}
  \calC\llbracket x \weq n \rrbracket(s)
  &= \assign{s}{n}{x},
    &\law{per def. della denotazionale} \\
  &= \id(\assign{s}{n}{x}),
    &\law{per def. di $\id$} \\
  &= \calC\llbracket \kw{skip} \rrbracket(\assign{s}{n}{x}).
     &\law{per def. della denotazionale} \\
\end{align*}

Caso base: $C = (x \weq E)$.

Per definizione della semantica small step:
\begin{align*}
  &\config{x \weq E}{s} \ssarrow \config{x \weq E'}{s}, \\
  &\config{E}{s} \ssarrow \config{E'}{s},
\end{align*}
dunque:
\begin{align*}
  \calC\llbracket x \weq E \rrbracket(s)
  &= \assign{s}{\calE\llbracket E \rrbracket(s)}{x},
    &\law{per def. della denotazionale} \\
  &= \assign{s}{\calE\llbracket E' \rrbracket(s)}{x},
    &\law{per il lemma
      \ref{conservazione-passi-small-step-expr}:
      $\calE\llbracket E \rrbracket(s)
      = \calE\llbracket E' \rrbracket(s)$} \\
  &= \calC\llbracket x \weq E' \rrbracket(s).
    &\law{per def. della denotazionale}
\end{align*}

Caso base: $C = (\kw{if} \ttv \kw{then} C_1 \kw{else} C_2)$.

Per definizione della semantica small step:
\[
  \config{\kw{if} \ttv \kw{then} C_1 \kw{else} C_2}{s} \ssarrow \config{C_1}{s}
\]
dunque:
\begin{align*}
  \calC\llbracket \kw{if} \ttv \kw{then} C_1 \kw{else} C_2 \rrbracket(s)
  &= \calC\llbracket C_1 \rrbracket(s).
    &\law{per def. della denotazionale} \\
\end{align*}
Analogo il caso per $\false$.

Caso base: $C = (\kw{if} B \kw{then} C_1 \kw{else} C_2)$.

Per definizione della semantica small step:
\begin{align*}
  &\config{\kw{if} B \kw{then} C_1 \kw{else} C_2}{s} \ssarrow
  \config{\kw{if} B' \kw{then} C_1 \kw{else} C_2}{s}, \\
  &\config{B}{s} \ssarrow \config{B'}{s},
\end{align*}
dunque:
\begin{align*}
  \calC\llbracket \kw{if} B \kw{then} C_1 \kw{else} C_2 \rrbracket(s)
  &= \cond(\calB\llbracket B \rrbracket,
    \calC\llbracket C_1 \rrbracket,
    \calC\llbracket C_2 \rrbracket),
    &\law{def. denotazionale} \\
  &= \cond(\calB\llbracket B' \rrbracket,
    \calC\llbracket C_1 \rrbracket,
    \calC\llbracket C_2 \rrbracket),
    &\law{lemma
      \ref{conservazione-passi-small-step-expr}:
      $\calE\llbracket E \rrbracket(s)
      = \calE\llbracket E' \rrbracket(s)$} \\
  &= \calC\llbracket \kw{if} B' \kw{then} C_1 \kw{else} C_2 \rrbracket(s).
     &\law{def. denotazionale}
\end{align*}

Caso base:  $C = (\kw{while} B \kw{do} C)$.

Per definizione della semantica small step:
\[
  \config{\kw{while} B \kw{do} C}{s} \ssarrow
    \config{\kw{if} B \kw{then} (C ; \kw{while} B \kw{do} C) \kw{else} C_2}{s},
\]
dunque per definizione di $\kw{while} B \kw{do} C$ nella denotazionale:
\[
  \calC\llbracket \kw{while} B \kw{do} C \rrbracket(s)
    =  \calC\llbracket \kw{if} B \kw{then} (C ; \kw{while} B \kw{do} C)
    \kw{else} C_2 \rrbracket(s).
\]


Passo induttivo: $C = (C_1; C_2)$.

Per definizione della semantica small step:
\begin{align*}
  &\config{C_1; C_2}{s} \ssarrow \config{C_1'; C_2}{s'}, \\
  &\config{C_1}{s} \ssarrow \config{C_1'}{s'},
\end{align*}
dunque:
\begin{align*}
  \calC\llbracket C_1; C_2 \rrbracket(s)
  &= \compfun{\calC\llbracket C_2 \rrbracket(s)}{\calC\llbracket C_1
    \rrbracket(s)}, \\
  &= \compfun{\calC\llbracket C_2 \rrbracket(s)}{\calC\llbracket C_1'
    \rrbracket(s')},
    &\law{per ipotesi induttiva} \\
  &= \calC\llbracket C_1'; C_2 \rrbracket(s').
\end{align*}
\end{proof}
\end{proposizione}

\subsection{Esercizio 1}
Aggiungere al linguaggio WHILE visto a lezione l'espressione:
\[ Exp \ni E ::= \dots \vbar \textrm{$B$ ? $E_2$ : $E_3$} \]
definendone la semantica operazionale Big Step, Small Step e denotazionale.

\begin{proof}[Svolgimento]
Di seguito illustriamo le regole e gli assiomi in semantica Small Step:
\[
\prooftree
  \langle B,s \rangle \textrm{ $\rightarrow$ } \langle B', s' \rangle
  \justifies
    \langle \textrm{$B$ ? $E_2$ : $E_3$, $s$ $\rangle$
    $\rightarrow$ $\langle$ $B'$ ? $E_2$ : $E_3$, $s'$ $\rangle$}
  \thickness=0.08em
\endprooftree
\]

\[
\prooftree
  \justifies
    \textrm{ $\langle$ tt ? $E_2$ : $E_3$, $s$
    $\rangle$ $\rightarrow$ $\langle$ $E_2$, $s'$ $\rangle$}
  \thickness=0.08em
\endprooftree
\]

\[
\prooftree
  \justifies
    \textrm{$\langle$ ff ? $E_2$ : $E_3$, $s$
    $\rangle$ $\rightarrow$ $\langle$ $E_3$, $s'$ $\rangle$}
  \thickness=0.08em
\endprooftree
\]

Di seguito illustriamo le regole e gli assiomi in semantica Big Step:
\[
\prooftree
  \langle B,s \rangle \Downarrow \langle \textrm{tt}, s' \rangle \quad
  \langle E_2,s' \rangle \Downarrow \langle n, s'' \rangle
  \justifies
     \langle \textrm{$B$ ? $E_2$ : $E_3$, $s$ $\rangle$
     $\Downarrow \langle n$, $s'' \rangle$}
  \thickness=0.08em
\endprooftree
\]

\[
\prooftree
  \langle B,s \rangle \Downarrow \langle \textrm{ff}, s' \rangle \quad
  \langle E_3,s' \rangle \Downarrow \langle n, s'' \rangle
  \justifies
     \langle \textrm{$B$ ? $E_2$ : $E_3$, $s \rangle \Downarrow \langle n, s'' \rangle$}
  \thickness=0.08em
\endprooftree
\]

Di seguito illustriamo la definizione dell'espressione in semantica denotazionale:
\[
\llbracket \textrm{$B$ ? $E_2$ : $E_3 \rrbracket (s) $} =
\begin{cases}
    \varepsilon \llbracket E_2 \rrbracket (s)
    & \textrm{se } \emph{B} \llbracket B \rrbracket (s) = \textrm{tt} \\
    \varepsilon \llbracket E_3 \rrbracket (s)
    & \textrm{se } \emph{B} \llbracket B \rrbracket (s) = \textrm{ff} \\
    \bot & \textrm{altrimenti}
\end{cases}
\]
\end{proof}

\chapter{Estensioni}

\section{Array} \marginpar{Segalini}

[Breve introduzione\dots Aggiungere nel linguaggio while gli array]

\subsection{Sintassi}

Aggiungiamo alla sintassi del WHILE l'espressione che accede
all'$i$-esimo elemento dell'array:
\begin{align*}
  \AExp \ni E &::= \dots \vbar x(E). \\
\intertext{%
Aggiungiamo inoltre il comando d'assegnamento all'$i$-esimo elemento
dell'array:
}
  \Com \ni C &::= \dots \vbar x(E_1) := E_2.
\end{align*}

\subsection{Semantica informale}

Gli array in questo semplice linguaggio di programmazione (per scelta
di progettazione) sono dinamici: la loro dimensione è illimitata, ma
solo un numero finito di elementi è definito in ogni momento della
computazione.  Gli elementi non precedentemente assegnati dell'array
hanno il valore $0$.  Verranno mostrate due semantiche diverse: in una
l'accesso ad una variabile scalare $x$ con un'espressione della forma
$x(E)$ restituisce il valore contenuto in $x$, ignorando quindi $E$.
Nell'altra semantica, l'accesso $x(E)$ trasforma la variabile scalare
$x$ in un array con tutti gli elementi non definiti eccetto quello
nella posizione indicata da $E$, la quale assumerà il valore contenuto
in $x$ prima di tale accesso.

\subsection{Semantica formale}

\begin{definizione} \summary{(Array semantici.)}
Un \emph{array} (semantico) è una funzione da interi (indice
dell'array) a interi (contenuto della cella) in cui solo ad un insieme
finito di elementi del dominio corrisponde un numero diverso da $0$.
Formalmente:

\[
  \Lambda
    \defeq
      \Bigl\{\,
        \fund{a}{\Zset}{\Zset}
      \Bigm|
        \bigl\{\, x \in \Zset \bigm| a(x) \neq 0 \,\bigr\} \sseqf \Zset
      \,\Bigr\}.
\]
\end{definizione}

\begin{definizione} \summary{(Insieme degli stati con array.)}
Modifichiamo l'insieme degli store $\Sigma$ per includere gli array tra
le variabili del programma:

\[
  \pard{\Sigma}{\Var}{\Zset \union \Lambda}.
\]
\end{definizione}

\subsubsection{Denotazionale}
Semantica di $x(E)$ denotazionale senza conversione scalare-array:

\[
  \calE\llbracket x(E) \rrbracket(s) =
  \begin{cases}
    s(x), &\text{se $s(x) \in \Zset$; (x è uno scalare)} \\
    s(x)\bigl(\calE\llbracket E \rrbracket(s)\bigr), &\text{se $s(x)
      \in \Lambda \land \calE\llbracket E \rrbracket(s)\convrg$;} \\
    \divrg, &\text{se $s(x) \in \Lambda \land \calE\llbracket E
      \rrbracket(s)\divrg$;} \\
    \divrg, &\text{se $s(x)\divrg$;}
  \end{cases}
\]
\subsubsection{Operazionale small-step}
Semantica di $x(E)$ small step su di un array:

\begin{align*}
  &\prooftree
    \config{E}{s} \ssarrow \config{E'}{s'}
  \justifies
    \config{x(E)}{s} \ssarrow \config{x(E')}{s'}
    \thickness=0.08em
  \endprooftree
\intertext{%
se $x$ è un array:
}
  &\prooftree
  \justifies
    \config{x(n)}{s} \ssarrow \config{s(x)(n)}{s}
  \thickness=0.08em
  \using
    s(x) \in \Lambda
  \endprooftree
\intertext{%
se $x$ è uno scalare:
}
  &\prooftree
  \justifies
    \config{x(n)}{s} \ssarrow \config{0}{s'}
  \thickness=0.08em
  \using
    s(x) \in \Zset \land s' = \substt{a}{x}
  \endprooftree
\end{align*}
\[
\text{dove } a = \lambdaop i.0
\]
Semantica di $x(E)$ alternativa che trasforma lo scalare $x$ in un
array vuoto (composto interamente da 0), eccezione fatta per la
posizione $n$-ma la quale conterrà il valore di $x$:

\[
  \prooftree
  \justifies
    \config{x(n)}{s} \ssarrow \config{s(x)}{s'}
  \thickness=0.08em
  \using
    s(x) \in \Zset \land s' = \substt{a}{x}
  \endprooftree
\]
\begin{align*}
  \text{dove } a = \lambdaop i.
    \begin{cases}
      s(x), &\text{se $i = n$} \\
      0,    &\text{altrimenti}
    \end{cases}
\end{align*}
Semantica di $x(E_0) \weq E_1$ small step che permetta la conversione di uno
scalare in un array:

\[
  \prooftree
    \config{E_0}{s} \ssarrow \config{E_0'}{s'}
  \justifies
    \config{x(E_0) \weq E_1}{s} \ssarrow \config{x(E_0') \weq E_1}{s'}
    \thickness=0.08em
  \endprooftree
\]
\[
  \prooftree
    \config{E_1}{s} \ssarrow \config{E_1'}{s'}
  \justifies
    \config{x(n) \weq E_1}{s} \ssarrow \config{x(n) \weq E_1'}{s'}
    \thickness=0.08em
  \endprooftree
\]
se $x$ è un array:
\[
  \prooftree
  \justifies
    \config{x(n)\ \weq\ k}{s} \ssarrow \config{\kw{skip}}{s'}
  \thickness=0.08em
  \using
    s(x) \in \Lambda \land s' = \substt{a}{x}
  \endprooftree
\]
\begin{align*}
  \text{dove } a = \lambdaop i.
  \begin{cases}
    k, &\text{se $i = n$} \\
    s(x)(i), &\text{se $i \neq n$}
  \end{cases}
\end{align*}
se $x$ è uno scalare (assegnamento distruttivo):
\[
  \prooftree
  \justifies
    \config{x(n)\ \weq\ k}{s} \ssarrow \config{\kw{skip}}{s'}
  \thickness=0.08em
  \using
    s(x) \in \Zset \land s' = \substt{a}{x}
  \endprooftree
\]
\begin{align*}
  \text{dove } a = \lambdaop i.
  \begin{cases}
    k, &\text{se $i = n$} \\
    0, &\text{se $i \neq n$}
  \end{cases}
\end{align*}
oppure conserviamo il valore contenuto in $x$ e lo salviamo
alla posizione~0 del nuovo array:
\[
  \prooftree
  \justifies
    \config{x(n)\ \weq\ k}{s} \ssarrow \config{\kw{skip}}{s'}
  \thickness=0.08em
  \using
    s(x) \in \Zset \land s' = \substt{a}{x}
  \endprooftree
\]
\begin{align*}
  \text{dove } a = \lambdaop i.
  \begin{cases}
    k, &\text{se $i = n$} \\
    0, &\text{se $i \neq n$} \\
    s(x), &\text{se $i = 0 \land h \neq 0$}
  \end{cases}
\end{align*}

\section{Eccezioni} \marginpar{Segalini}

[Breve introduzione\dots Aggiungere nel linguaggio while le eccezioni...]


Creiamo un nuovo insieme di possibili variabili utilizzabili in WHILE
chiamato $\Var'$. Questo insieme contiene tutte le variabili di $\Var$
ed una variabile speciale, non utilizzabile nei programmi, chiamata
$\$ecc$.

\begin{align*}
  &\$ecc \not \in \Var \\
  &\Var' \defeq \Var \union \{\$ecc\}
\end{align*}
l'insieme degli store è quindi definito come:

\[
  \pard{\Sigma}{\Var'}{\Zset}
\]
\emph{N.B.:} nella sintassi delle espressioni ($E \in \AExp ::= \dots
\vbar x$) la $x$ appartiene a $\Var$ e non $\Var'$ (questo
impedisce l'uso della varibile nel programma).

\subsection{Sintassi}
Aggiungiamo alla sintassi del WHILE i comandi try-catch e throw:

Siano $x \in \Var$ e $E \in \AExp$
\begin{align*}
\Com \ni C ::= \dots \vbar \kw{try} C_1 \kw{catch}(x) C_2\vbar \kw{throw} E
\end{align*}

\subsection{Semantica formale}

\begin{align*}
  &\prooftree \config{E}{s} \ssarrow \config{E'}{s'}
  \justifies \config{\kw{throw} E}{s} \ssarrow 
    \config{\kw{throw} E'}{s'}
  \thickness=0.08em
  \using s(\$ecc) \divrg
  \endprooftree
  \\ \\
  &\prooftree
  \justifies \config{\kw{throw} n}{s} \ssarrow 
    \config{\kw{skip}}{s \substt{n}{\$ecc}}
  \thickness=0.08em
  \using s(\$ecc) \divrg
  \endprooftree
  \\ \\
  &\prooftree
  \justifies \config{x \weq E}{s} \ssarrow \config{\kw{skip}}{s}
  \thickness=0.08em
  \using s(\$ecc) \convrg
  \endprooftree 
  \\ \\
  &\prooftree
  \justifies \config{C_1 \weq C_2}{s} \ssarrow \config{\kw{skip}}{s}
  \thickness=0.08em
  \using s(\$ecc) \convrg
  \endprooftree
  \\ \\
  &\prooftree
  \justifies \config{\kw{if} B \kw{then} C_1 \kw{else} C_2}{s} 
    \ssarrow \config{\kw{skip}}{s}
  \thickness=0.08em
  \using s(\$ecc) \convrg
  \endprooftree
  \\ \\
  &\prooftree
  \justifies \config{\kw{while} B \kw{do} C}{s}
     \ssarrow \config{\kw{skip}}{s}
  \thickness=0.08em
  \using s(\$ecc) \convrg
  \endprooftree
  \\ \\
  &\prooftree
  \justifies \config{\kw{try} C_1 \kw{catch}(x) C_2}{s}
     \ssarrow \config{\kw{skip}}{s}
  \thickness=0.08em
  \using s(\$ecc) \convrg
  \endprooftree
  \\ \\
  &\prooftree
  \justifies \config{\kw{throw} E}{s} \ssarrow \config{\kw{skip}}{s}
  \thickness=0.08em
  \using s(\$ecc) \convrg
  \endprooftree
\end{align*}
Quando viene eseguito il blocco catch, per uscire dalla modalità
eccezionale, dobbiamo riportare la variabile $\$ecc$ ad uno stato
indefinito. \'E necessario definire un operatore di
uccisione di variabile:
\[
  \fund{\backslash}{\Sigma \times \Var'}{\Sigma} 
\]
\[
  \forall s \in \Sigma \itc \forall x \in \Var' \itc 
    \forall y \in \Var' \itc (s \backslash x)(y) = \begin{cases}
        s(y),   &\text{se $x \neq y$} \\
        \divrg, &\text{se $x = y$}   
      \end{cases}
\]
\begin{align*}
  &\prooftree \config{C_1}{s} \ssarrow \config{C_1'}{s'}
  \justifies \config{\kw{try} C_1 \kw{catch}(x) C_2}{s} \ssarrow
    \config{\kw{try} C_1' \kw{catch}(x) C_2}{s'}
  \thickness=0.08em
  \using s(\$ecc) \divrg \land\ s'(\$ecc) \divrg
  \endprooftree
  \\ \\
  &\prooftree \config{C_1}{s} \ssarrow \config{C_1'}{s'}
  \justifies \config{\kw{try} C_1 \kw{catch}(x) C_2}{s} \ssarrow
    \config{C_2}{\bigl(s'\substt{s(\$ecc)}{x}\bigr) \backslash \$ecc}
  \thickness=0.08em
  \using s(\$ecc) \divrg \land\ s'(\$ecc) \convrg
  \endprooftree
\end{align*}


\chapter{Interpretazione astratta}

[Introduzione al capitolo da scrivere.]

\section{Analisi dei segni per WHILE}

\subsection{Dominio concreto}

Le variabili del linguaggio WHILE possono contenere solamente numeri interi,
il che giustifica il fatto che il dominio concreto dell'interpretazione
astratta che andiamo a definire è
\[
  \bigl(\wp(\Zset), \subseteq\bigr).
\]

\subsection{Dominio astratto}

Il dominio astratto cattura alcune caratteristiche del dominio concreto,
tralasciando quelle restanti; in questo caso
$\Sign$ cattura semplici relazioni rispetto al
numero $0$, ad esempio $\cdot < 0$, $\cdot = 0$, \dots
\[
  (\Sign, \sqsubseteq).
\]
L'ordinamento del dominio è indicato nel diagramma di Hasse di
figura~\ref{fig:ordering-rels-lattice}.

\begin{figure}
\begin{center}
\setlength{\unitlength}{1.8mm}
\begin{picture}(28, 40)
{\thicklines
\put(14, 2){\circle{4}}
\put(14, 2){\makebox(0, 0){$\signbot$}}

\put( 2, 14){\circle{4}}
\put( 2, 14){\makebox(0, 0){$\signlt$}}
\put(14, 14){\circle{4}}
\put(14, 14){\makebox(0, 0){$\signeq$}}
\put(26, 14){\circle{4}}
\put(26, 14){\makebox(0, 0){$\signgt$}}

\put( 2, 26){\circle{4}}
\put( 2, 26){\makebox(0, 0){$\signle$}}
\put(14, 26){\circle{4}}
\put(14, 26){\makebox(0, 0){$\signne$}}
\put(26, 26){\circle{4}}
\put(26, 26){\makebox(0, 0){$\signge$}}

\put(14, 38){\circle{4}}
\put(14, 38){\makebox(0, 0){$\signtop$}}

\put( 2, 24){\line(0, -1){8}}
\put(26, 24){\line(0, -1){8}}

\put(14, 36){\line(0, -1){8}}
\put(14, 12){\line(0, -1){8}}

\put(15.42, 36.58){\line(1, -1){9.16}}
\put( 3.42, 24.58){\line(1, -1){9.16}}
\put(15.42, 24.58){\line(1, -1){9.16}}
\put( 3.42, 12.58){\line(1, -1){9.16}}

\put( 3.42, 27.42){\line(1, 1){9.16}}
\put( 3.42, 15.42){\line(1, 1){9.16}}
\put(15.42, 15.42){\line(1, 1){9.16}}
\put(15.42,  3.42){\line(1, 1){9.16}}
}
\end{picture}
\end{center}
\caption{Diagramma di Hasse del dominio $\Sign$.}
\label{fig:ordering-rels-lattice}
\end{figure}


\subsection{Funzione di concretizzazione}

La funzione di concretizzazione definisce la semantica di ogni
elemento di $\Sign$. Più in generale, la funzione $\gamma$ mappa ogni
valore astratto nell'insieme di valori concreti che tale valore astratto
rappresenta.
\begin{definizione} \summary{($\fund{\gamma}{\Sign}{\wp(\Zset)}$.)}
La funzione $\fund{\gamma}{\Sign}{\wp(\Zset)}$ è definita
da:
\begin{align*}
  \gamma(\top) &= \Zset, \\
  \gamma(\bot) &= \emptyset, \\
  \gamma(\bowtie) &= \{\, n \in \Zset \mid n \bowtie 0 \,\},
    \quad\text{per $\bowtie \in \{ \leq, \neq, \geq, <, =, > \}$.}
\end{align*}
\end{definizione}

\begin{proposizione}
La funzione $\gamma$ è monotona.
\end{proposizione}
\begin{proof}
  Bisogna mostrare che, se $x, y \in \Sign$ e $x$ precede $y$ nel diagramma
  di figura~\ref{fig:ordering-rels-lattice}, allora
  $\gamma(x) \subseteq \gamma(y)$.
  Questo può essere facilmente stabilito per casi, tutti banali.
  Eccone alcuni:
  \begin{itemize}
    \item
      se $x = \bot$ allora, per ogni $y \in \Sign$,
      $\gamma(\bot) = \emptyset \subseteq \gamma(y)$
      per definizione di~$\emptyset$;
    \item
      se $y = \top$ allora, per ogni $x \in \Sign$,
      $\gamma(x) \subseteq \Zset = \gamma(\top)$
      per definizione di contenimento insiemistico;
    \item
      se $x = (=)$ e $y = (\geq)$, allora
      $\gamma(=) = \{0\} \subseteq \gamma(\geq)$.
  \end{itemize}
\end{proof}

\subsection{Funzione di astrazione}

\begin{definizione} \summary{($\fund{\alpha}{\wp(\Zset)}{\Sign}$.)}
La funzione $\fund{\alpha}{\wp(\Zset)}{\Sign}$ mappa un insieme
di elementi concreti nel più preciso valore astratto che rappresenta
tale insieme; per poterla definire usiamo una funzione ausiliaria
$\beta$:
\[
  \fund{\beta}{\Zset}{\Sign}
\]
che si comporta nel seguente modo, per ogni $n \in \Zset$:
\begin{align*}
  \beta(n) &=
    \begin{cases}
      >,      &\text{se $n  >  0$;} \\
      =,      &\text{se $n = 0$;} \\
      <,      &\text{se $n < 0$.}
    \end{cases} 
\end{align*}
Quindi, la funzione $\alpha$ rappresenta il $\lub$ della
funzione $\beta$ applicata ad ogni elemento di un insieme
$S \in \wp(\Zset)$ arbitrario:
\[
  \alpha(S) = \sqcup \{\, \beta(n) \mid n \in S \,\}
\]
Questo fatto giustifica le approssimazioni compiute nel processo
di astrazione.
\end{definizione}

\begin{teorema}
Siano $S_1, S_2 \in \wp(\Zset)$ arbitrari; allora la funzione $\alpha$
è monotona:
\[
  S_1 \subseteq S_2 \implies \alpha(S_1) \sqsubseteq \alpha(S_2).
\]
\end{teorema}
\begin{proof}
  \begin{align*}
    S_1 \subseteq S_2
      &\implies \bigl\{\beta(n_1) \bigm| n_1 \in S_1\bigr\}
        \sqsubseteq \bigl\{\beta(n_2) \bigm| n_2 \in S_2\bigr\}
      &\law{per definizione di $\beta$} \\
      &\implies \sqcup \bigl\{\beta(n_1) \bigm| n_1 \in S_1\bigr\}
        \sqsubseteq \sqcup \bigl\{\beta(n_2) \bigm| n_2 \in S_2\bigr\}
      &\law{per proprietà del $\lub$} \\
      &\implies \alpha(S_1) \sqsubseteq \alpha(S_2)
      &\law{per definizione di $\alpha$.}
  \end{align*}
\end{proof}

\subsection{Inserzione di Galois}

Oltre alla monotonia delle funzioni di astrazione e concretizzazione,
è interessante sapere se stiamo lavorando in una
``inserzione di Galois''.
Mostrare questo fatto vuol dire anche verificare che
$\compfun{\gamma}{\alpha}$ è estensiva. 
Per dimostrare l'estensività di $\compfun{\gamma}{\alpha}$, conviene
prima considerare $\gamma$ additiva.
\begin{proposizione}
$\gamma$ è additiva.
\end{proposizione}
\begin{proof}
Per verificarlo, si mostrano nel seguito alcuni esempi, tutti banali:
\begin{itemize}
  \item $\gamma(=) \union \gamma(<) = \gamma(\leq)$;
  \item $\gamma(=) \union \gamma(=) = \gamma(=)$.
\end{itemize}
\end{proof}
Avendo l'additività di $\gamma$, possiamo procedere alla dimostrazione
dell'estensività di $\compfun{\gamma}{\alpha}$, dunque che,
per un qualunque $S \in \wp(\Zset)$:
\[
  \compfun{\gamma}{\alpha}(S) = \union \Bigl\{\, \gamma\bigl(\beta(n)\bigr) \bigm| n \in \Zset \,\Bigl\} \supseteq S
\]
Per dimostrare quest'ultima affermazione ci serve il seguente lemma:
\begin{lemma}
  Sia $n \in \Zset$ arbitrario; allora
  $n \in \gamma\bigl(\beta(n)\bigr)$.
\end{lemma}

\begin{proposizione}
$\compfun{\gamma}{\alpha}$ è estensiva.
\end{proposizione}
\begin{proof}
Sia $S \in \wp(\Zset)$ arbitrario.
Abbiamo:
\begin{align*}
  \compfun{\gamma}{\alpha}(S)
    &= \gamma\bigl(\alpha(S)\bigr)
    &\law{definizione di $\circ$} \\
    &= \gamma\bigl(\sqcup \{\, \beta(n) \mid n \in S \,\} \bigr)
    &\law{definizione di $\alpha$} \\
    &= \union \Bigl\{\, \gamma\bigl(\beta(n)\bigr) \Bigm| n \in S \,\Bigr\}
    &\law{additività di $\gamma$} \\
    &\supseteq S
    &\law{poiché $n \in \gamma\bigl(\beta(n)\bigr)$.}
\end{align*}
\end{proof}

Infine, mostriamo banalmente che $\compfun{\alpha}{\gamma}$ corrisponde
all'identità:
\begin{proposizione}
  $\compfun{\alpha}{\gamma}$ è l'identità.
\end{proposizione}
\begin{proof}
Bisogna mostrare che per un qualunque $\bowtie \in \Sign$,
$\compfun{\alpha}{\gamma}(\bowtie)$ corrisponde proprio a $\bowtie$.
Questo può facilmente essere stabilito per casi, tutti banali.
Eccone alcuni:
\begin{itemize}
	\item $\compfun{\alpha}{\gamma}(\bot) = \bot$;
    \item $\compfun{\alpha}{\gamma}(\top) = \top$;
    \item $\compfun{\alpha}{\gamma}(\geq) = \geq$.
\end{itemize}
\end{proof}

\subsection{Operazioni astratte}
Le operazioni astratte vanno definite seguendo le operazioni concrete;
esse si applicano tra elementi del dominio astratto.
Descriviamo in seguito le operazioni più interessanti applicabili
al nostro dominio astratto.

\begin{definizione}\summary{Moltiplicazione astratta.}
La moltiplicazione astratta è definita come segue:
\[
  \absmul \defeq \lambda a_1,a_2 \in \Sign \st \alpha\bigl(\gamma(a_1) \conmul \gamma(a_2)\bigr)
\]
con $\conmul$ moltiplicazione concreta, definita come segue:
\[
  S_1 \conmul S_2 \defeq \{\, n_1 * n_2 \mid n_1 \in S_1, n_2 \in S_2 \,\}
\]
\end{definizione}

\begin{center}
  \begin{tabular}{ c | c c c c c c c c }
    $\absmul$ & $\top$ & $\leq$ & $\neq$ & $\geq$ & $<$ & $=$ & $>$ & $\bot$ \\
    \hline
    $\top$ & $\top$ & $\top$ & $\top$ & $\top$ & $\top$ & $=$ & $\top$ & $\bot$  \\
    $\leq$ & $\top$ & $\geq$ & $\top$ & $\leq$ & $\geq$ & $=$ & $\leq$ & $\bot$\\
    $\neq$ & $\top$ & $\top$ & $\neq$ & $\top$ & $\neq$ & $=$ & $\neq$ & $\bot$ \\
    $\geq$ & $\top$ & $\leq$ & $\top$ & $\geq$ & $\leq$ & $=$ & $\geq$ & $\bot$ \\
    $<$ & $\top$ & $\geq$ & $\neq$ & $\leq$ & $>$ & $=$ & $<$ & $\bot$ \\
    $=$ & $=$ & $=$ & $=$ & $=$ & $=$ & $=$ & $=$ & $\bot$\\
    $>$ & $\top$ & $\leq$ & $\neq$ & $\geq$ & $<$ & $=$ & $>$ & $\bot$\\
    $\bot$ & $\bot$ & $\bot$ & $\bot$ & $\bot$ & $\bot$ & $\bot$ & $\bot$ & $\bot$ \\
  \end{tabular}
\end{center}

Come si può notare, la tabella risultante è simmetrica:
questo perché l'operatore di moltiplicazione astratta è commutativo.

\begin{definizione}\summary{Somma astratta.}
La somma astratta è definita come segue:
\[
  \absadd \defeq \lambda a_1,a_2 \in \Sign \st \alpha\bigl(\gamma(a_1) \conadd \gamma(a_2)\bigr)
\]
con $\conadd$ somma concreta, definita come segue:

\[
  S_1 \conadd S_2 \defeq \{\, n_1 + n_2 \mid n_1 \in S_1, n_2 \in S_2 \,\}
\]
\end{definizione}

\begin{center}
  \begin{tabular}{c | c c c c c c c c}
    $\absadd$ & $\top$ & $\leq$ & $\neq$ & $\geq$ & $<$ & $=$ & $>$ & $\bot$ \\
    \hline
    $\top$ & $\top$ & $\top$ & $\top$ & $\top$ & $\top$ & $\top$ & $\top$ & $\bot$ \\
    $\leq$ & $\top$ & $\leq$ & $\top$ & $\top$ & $<$ & $\leq$ & $\top$ & $\bot$\\
    $\neq$ & $\top$ & $\top$ & $\top$ & $\top$ & $\top$ & $\neq$ & $\top$ & $\bot$ \\
    $\geq$ & $\top$ & $\top$ & $\top$ & $\geq$ & $\top$ & $\geq$ & $>$ & $\bot$\\
    $<$ & $\top$ & $<$ & $\top$ & $\top$ & $<$ & $<$ & $\top$ & $\bot$\\
    $=$ & $\top$ & $\leq$ & $\neq$ & $\geq$ & $<$ & $=$ & $>$ & $\bot$\\
    $>$ & $\top$ & $\top$ & $\top$ & $>$ & $\top$ & $>$ & $>$ & $\bot$ \\
    $\bot$ & $\bot$ & $\bot$ & $\bot$ & $\bot$ & $\bot$ & $\bot$ & $\bot$ & $\bot$
  \end{tabular}
\end{center}

Osservando il numero di ricorrenze di $\top$ all'interno della tabella
e confrontandolo con quello all'interno della tabella
della moltiplicazione astratta, si nota che su questo dominio
la moltiplicazione astratta è più precisa della somma astratta.

\begin{definizione}\summary{Divisione intera astratta.}
La divisione intera astratta viene definita come segue:
\[
  \absdiv \defeq \lambda a_1,a_2 \in \Sign \st \alpha\bigl(\gamma(a_1) \condiv \gamma(a_2)\bigr)
\]
con $\condiv$ divisione intera concreta, definita come segue:
\[
  S_1 \condiv S_2 \defeq \bigl\{\, n_1 / n_2 \mid n_1 \in S_1, n_2 \in S_2 \backslash \{0\} \,\bigr\}
\]
\end{definizione}
Nella seguente tabella, gli indici delle colonne indicano il dividendo e gli indici delle righe indicano il divisore.
\begin{center}
  \begin{tabular}{c | c c c c c c c c }
    $\absdiv$ & $\top$ & $\leq$ & $\neq$ & $\geq$ & $<$ & $=$ & $>$ & $\bot$ \\
    \hline
    $\top$ & $\top$ & $\top$ & $\top$ & $\top$ & $\top$ & $=$ & $\top$ & $\bot$\\
    $\leq$ & $\top$ & $\geq$ & $\top$ & $\leq$ & $\geq$ & $=$ & $\leq$ & $\bot$\\
    $\neq$ & $\top$ & $\top$ & $\top$ & $\top$ & $\top$ & $=$ & $\top$ & $\bot$\\
    $\geq$ & $\top$ & $\leq$ & $\top$ & $\geq$ & $\leq$ & $=$ & $\geq$ & $\bot$\\
    $<$ & $\top$ & $\geq$ & $\top$ & $\leq$ & $\geq$ & $=$ & $\leq$ & $\bot$\\
    $=$ & $\bot$ & $\bot$ & $\bot$ & $\bot$ & $\bot$ & $\bot$ & $\bot$ & $\bot$\\
    $>$ & $\top$ & $\leq$ & $\top$ & $\geq$ & $\leq$ & $=$ & $\geq$ & $\bot$\\
    $\bot$ & $\bot$ & $\bot$ & $\bot$ & $\bot$ & $\bot$ & $\bot$ & $\bot$ & $\bot$
  \end{tabular}
\end{center}

\begin{definizione}\summary{Meno unario astratto.}
Il meno unario astratto ($\absuminus$) può essere definito
per semplificare la definizione di meno binario astratto ($\abssub$).
Infatti l'operazione astratta:
\[
  a \abssub b
\]
può essere riscritta come:
\[
  a  \abssub b = a \absadd (\absuminus b)
\]
\end{definizione}

\begin{center}
  \begin{tabular}{ c | c c c c c c c c }
    $\absuminus$ & $\top$ & $\leq$ & $\neq$ & $\geq$ & $<$ & $=$ & $>$ & $\bot$ \\
    \hline
    & $\top$ & $\geq$ & $\neq$ & $\leq$ & $>$ & $=$ & $<$ & $\bot$
  \end{tabular}
\end{center}

\section{Dominio astratto dei booleani}

Consideriamo ora come dominio astratto quello dei booleani.
Il dominio astratto dei booleani $\AbBool$ viene definito per poter
estendere l'insieme delle operazioni su $\Sign$; alcune operazioni,
in particolare quelle di confronto tra elementi di $\Sign$,
restituiscono un risultato appartenente a $\AbBool$.
L'ordinamento del dominio è indicato nel diagramma di Hasse di
figura~\ref{fig:ordering-bool-lattice}.

\begin{figure}
\begin{center}
\setlength{\unitlength}{1.8mm}
\begin{picture}(28, 28)
{\thicklines
\put(14, 2){\circle{4}}
\put(14, 2){\makebox(0, 0){$\signbot$}}

\put( 2, 14){\circle{4}}
\put( 2, 14){\makebox(0, 0){$1$}}
\put(26, 14){\circle{4}}
\put(26, 14){\makebox(0, 0){$0$}}

\put(14, 26){\circle{4}}
\put(14, 26){\makebox(0, 0){$\signtop$}}

\put(15.42, 24.58){\line(1, -1){9.16}}
\put( 3.42, 12.58){\line(1, -1){9.16}}

\put( 3.42, 15.42){\line(1, 1){9.16}}
\put(15.42,  3.42){\line(1, 1){9.16}}
}
\end{picture}
\end{center}
\caption{Diagramma di Hasse del dominio $\AbBool$.}
\label{fig:ordering-bool-lattice}
\end{figure}

\section{Operazioni astratte su booleani}

Come accennato, il dominio astratto dei booleani ci permette di
ragionare su altre operazioni astratte;
poiché si parla di booleani le operazioni interessanti
saranno quelle di confronto.
Le operazioni astratte sui booleani sono definite come segue:
\begin{definizione}\summary{Operazioni astratte su booleani.}
\[
  \fund{\eqcirc}{\Sign \times \Sign}{\AbBool}
\]
\end{definizione}

\begin{definizione}\summary{Uguaglianza astratta.}
L'uguaglianza astratta viene definita come segue:
\[
  \abseq \defeq \lambda a_1,a_2 \in \Sign \st \alpha\bigl(\gamma(a_1) \coneq \gamma(a_2)\bigr)
\]
con $\coneq$ uguaglianza concreta, definita come segue:
\[
  S_1 \coneq S_2 \defeq \{\, n_1 = n_2 \mid n_1 \in S_1, n_2 \in S_2 \,\}
\]
\end{definizione}

\begin{center}
  \begin{tabular}{ c | c c c c c c c c }
    $\abseq$ & $\top$ & $\leq$ & $\neq$ & $\geq$ & $<$ & $=$ & $>$ & $\bot$ \\
    \hline
    $\top$ & $\top$ & $\top$ & $\top$ & $\top$ & $\top$ & $\top$ & $\top$ & $\bot$ \\
    $\leq$ & $\top$ & $\top$ & $\top$ & $\top$ & $\top$ & $\top$ & $0$ & $\bot$\\
    $\neq$ & $\top$ & $\top$ & $\top$ & $\top$ & $\top$ & $0$ & $\top$ & $\bot$\\
    $\geq$ & $\top$ & $\top$ & $\top$ & $\top$ & $0$ & $\top$ & $\top$ & $\bot$\\
    $<$ & $\top$ & $\top$ & $\top$ & $0$ & $\top$ & $\top$ & $0$ & $\bot$\\
    $=$ & $\top$ & $\top$ & $0$ & $\top$ & $0$ & $1$ & $0$ & $\bot$\\
    $>$ & $\top$ & $0$ & $\top$ & $\top$ & $0$ & $\top$ & $\top$ & $\bot$\\
    $\bot$ & $\bot$ & $\bot$ & $\bot$ & $\bot$ & $\bot$ & $\bot$ & $\bot$ & $\bot$
  \end{tabular}
\end{center}


\begin{definizione}\summary{Diverso astratto.}
Per definire l'operazione astratta di diverso basta definire quella di negato astratto:
\begin{align*}
  c_1 \neq c_2 &\iff \neg (c_2 = c_2) \\
  a_1 \absneq a_2 &\iff \absneg (a_1 =_a a_2)
\end{align*}
\end{definizione}

\begin{definizione}\summary{Disuguaglianza astratta.}
\[
  \absleq \defeq \lambda a_1,a_2 \in \Sign \st \alpha\bigl(\gamma(a_1) \conineq \gamma(a_2)\bigr)
\]
con $\conineq$ disuguaglianza concreta, definita come segue:
\[
  S_1 \conineq S_2 \defeq \{\, n_1 \leq n_2 \mid n_1 \in S_1, n_2 \in S_2 \,\}
\]
\end{definizione}
Nella seguente tabella, gli indici delle righe indicano
l'elemento a sinistra della disuguaglianza e gli indici delle colonne
indicano l'elemento a destra.

\begin{center}
  \begin{tabular}{ c | c c c c c c c c }
    $\absleq$ & $\top$ & $\leq$ & $\neq$ & $\geq$ & $<$ & $=$ & $>$ & $\bot$ \\
    \hline
    $\top$ & $\top$ & $\top$ & $\top$ & $\top$ & $\top$ & $\top$ & $\top$ & $\bot$ \\
    $\leq$ & $\top$ & $\top$ & $\top$ & $\top$ & $\top$ & $\top$ & $0$ & $\bot$ \\
    $\neq$ & $\top$ & $\top$ & $\top$ & $\top$ & $\top$ & $\top$ & $\top$ & $\bot$\\
    $\geq$ & $\top$ & $1$ & $\top$ & $\top$ & $1$ & $1$ & $\top$ & $\bot$ \\
    $<$ & $\top$ & $\top$ & $\top$ & $0$ & $\top$ & $0$ & $0$ & $\bot$\\
    $=$ & $\top$ & $1$ & $\top$ & $\top$ & $1$ & $1$ & $0$ & $\bot$ \\
    $>$ & $\top$ & $1$ & $\top$ & $\top$ & $1$ & $0$ & $\top$ & $\bot$ \\
    $\bot$ & $\bot$ & $\bot$ & $\bot$ & $\bot$ & $\bot$ & $\bot$ & $\bot$ & $\bot$
  \end{tabular}
\end{center}

\subsection{Introduzione alla semantica astratta small-step}
Dopo aver definito le operazioni astratte,
definiamo una semantica astratta per $\AExp$, $\BExp$ e $\Com$; cioè
le componenti sintattiche del linguaggio WHILE.
L'interpretazione della semantica di un programma con i valori
astratti è un'interpretazione astratta; in altre parole,
definire una semantica astratta vuol dire capire come
``eseguire astrattamente'' il programma sul dominio astratto,
per osservare le proprietà che il dominio astratto modella.

\begin{definizione}\summary{Store astratto.}
Uno \emph{store astratto} $s \in \Sigma^\#$ è un'approssimazione
delle celle di memoria usate dal linguaggio WHILE per memorizzare
le variabili di tipo intero.
\[
  \fund{\Sigma^\#}{\Var}{\Sign}
\]
\end{definizione}

\begin{definizione}\summary{Semantica concreta.}
La semantica concreta di un programma in linguaggio WHILE
consisteva in una serie di transizioni per ridurre il programma $P$
alla configurazione terminale $\kw{skip}$:
\[      
  \llbracket P \rrbracket (s) = s' \text{, se } \config{P}{s} \ssarrow^* \config{\kw{skip}}{s'}
\]
\end{definizione}
La semantica concreta del programma prevede un solo insieme di passi
per poter raggiungere la configurazione terminale;
la semantica astratta introduce invece il non determinismo, in quanto
approssimazione della semantica concreta.
\begin{definizione}\summary{Semantica astratta.}
È una funzione totale che va da un insieme di comandi astratti $\Com^\#$
ad una funzione parziale tra l'insieme degli stati astratti $\Sigma^\#:$
\[
  \llbracket \cdot \rrbracket^\# : \Com^\# \mapsto (\Sigma^\# \rightarrowtail \Sigma^\#)
\]
La semantica astratta di un programma diventa dunque la miglior
approssimazione (grazie alle proprietà del $\lub$) del programma:
\[
  \llbracket P \rrbracket^\# (s) =  \sqcup \bigl\{ s^\#_1 \in \Sigma^\# \mid \config{\alpha(P)}{s^\#} \ssarrow \config{\kw{skip}}{s^\#_1} \bigr\}
\]
\end{definizione}

\begin{definizione}\summary{Ordinamento su $\Sigma^\#$.}
Si definisce ora l'ordinamento sull'insieme degli store astratti:
\[
  \sqsubseteq_{\Sigma^\#}\, \subseteq \Sigma^\# \times \Sigma^\#
\]
\end{definizione}
Nel dettaglio, per ogni $s_1^\#,\ s_2^\# \in \Sigma^\#$
e per ogni $x \in \dom(s_1^\#)$:
\[
	s_1^\# \sqsubseteq_{\Sigma^\#} s_2^\# 
\]
se e solo se:
\[
	\dom(s_1^\#) = \dom(s_2^\#) \land s_1^\#(x) \sqsubseteq_\Sign s_2^\#(x)
\]

Completate tali definizioni, è possibile ipotizzare un insieme di regole
che costruiscano la semantica astratta del linguaggio WHILE. La forma
scelta per presentare tale semantica è quella operazionale small-step.

\subsection{Semantica di $\AExp$}

\begin{gather*}
  \rulename{ABS-EXP.LEFT}
  \prooftree
    \config{E_1}{s} \ssarrow \config{E_1'}{s'}
  \justifies
    \config{E_1 + E_2}{s} \ssarrow \config{E_1' + E_2}{s'}
  \endprooftree
\\
  \rulename{ABS-EXP.RIGHT}
  \prooftree
    \config{E_2}{s} \ssarrow \config{E_2'}{s'}
  \justifies
    \config{a + E_2}{s} \ssarrow \config{a + E_2'}{s'}
  \endprooftree
\\
  \rulename{ABS-ADD}
  \prooftree
    \nohyp
  \justifies
    \config{a_1 + a_2}{s} \ssarrow \config{a_3}{s} 
    \using a_3 = a_1 \absadd a_2
  \endprooftree
\end{gather*}

\subsection{Semantica di $\BExp$}

\begin{definizione}\summary{And cortocircuitato astratto.}
Prima di definire la semantica per le espressioni booleane,
definiamo un operatore di congiunzione cortocircuitato.
\end{definizione}

\begin{center}
  \begin{tabular}{ c | c c c c }
    $\absand$ & $\top$ & 1 & 0 & $\bot$ \\
    \hline
    $\top$ & $\top$ & $\top$ & 0 & $\bot$ \\
    1 & $\top$ & 1 & 0 & $\bot$ \\
    0 & 0 & 0 & 0 & $\bot$ \\
    $\bot$ & $\bot$ & $\bot$ & $\bot$ & $\bot$
  \end{tabular}
\end{center} 

\subsubsection{Semantica}
\begin{gather*}
  \rulename{AND-SHORT.LEFT}
  \prooftree
    \config{B_1}{s} \ssarrow \config{B_1'}{s'}
  \justifies
    \config{B_1 \land B_2}{s} \ssarrow \config{B_1' \land B_2}{s'}
  \endprooftree
  \\
  \rulename{AND-SHORT.TOP}
  \prooftree
    \config{B_2}{s} \ssarrow \config{B_2'}{s'}
  \justifies
    \config{\top \land B_2}{s} \ssarrow \config{\top \land B_2'}{s'}
  \endprooftree
  \\
  \rulename{AND-SHORT.TRUE}
  \prooftree
    \nohyp
  \justifies
    \config{1 \land B_2}{s} \ssarrow \config{B_2}{s}
  \endprooftree
  \\
  \rulename{AND-SHORT.FALSE}
  \prooftree
    \nohyp
  \justifies
    \config{0 \land B_2}{s} \ssarrow \config{0}{s}
  \endprooftree
  \\
  \rulename{ABS-AND-SHORT.TOP}
  \prooftree
    \nohyp
  \justifies
    \config{\top \land b_2}{s} \ssarrow \config{\top \absand b_2}{s}
  \endprooftree
\end{gather*}

\subsection{Semantica di $\Com$}

\begin{gather*}
  \rulename{ASSIGN}
  \prooftree
    \config{E}{s} \ssarrow \config{E'}{s'}
  \justifies
    \config{x \weq E}{s} \ssarrow \config{x \weq E'}{s'}
  \endprooftree
  \\
  \rulename{ASSIGN-SKIP}
  \prooftree
    \nohyp
  \justifies
    \config{x \weq a}{s} \ssarrow \config{\kw{skip}}{\assign{s}{x}{a}}
  \endprooftree
  \\
  \rulename{COND-BOOL}
  \prooftree
    \config{B}{s} \ssarrow \config{B'}{s'}
  \justifies
    \config{\condif{B}{C_1}{C_2}}{s} \ssarrow \config{\condif{B'}{C_1}{C_2}}{s'}
  \endprooftree
  \\
  \rulename{IF-TRUE}
  \prooftree
    \nohyp
  \justifies
    \config{\condif{b}{C_1}{C_2}}{s} \ssarrow \config{C_1}{s}
    \using b \sqsubseteq_B 1
  \endprooftree
  \\
  \rulename{IF-FALSE}
  \prooftree
    \nohyp
  \justifies
    \langle \condif{b}{C_1}{C_2}, s \rangle \ssarrow \langle C_2, s \rangle
    \using b \sqsubseteq_B 0
  \endprooftree
\end{gather*}


\bibliographystyle{plain}
\bibliography{biblio}

\end{document}
