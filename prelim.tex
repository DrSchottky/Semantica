\chapter{Nozioni preliminari}

[Introduzione al capitolo da scrivere.]


\section{Notazione basilare}

\subsection{Insiemi e sequenze}

Sia $S$ un insieme.
Se $S$ \`e l'insieme vuoto allora si scrive $S = \emptyset$.
Inoltre:
\begin{list}{}{}
\item[$|S|$] denota la cardinalit\`a di S; se $S$ \`e un insieme finito
             allora $|S|$ \`e il numero di elementi di $S$; se $S$
             pu\`o essere messo in corrispondenza biunivoca con
             l'insieme dei numeri naturali $\Nset$ allora si dice che
             \`e infinito e numerabile oppure che ha la
             \emph{potenza del numerabile} oppure che ha cardinalit\`a
             $\aleph_0$ e si scrive $|S| = \aleph_0$; per contro, le
             espressioni ``S \`e finito'' e ``$|S| < \aleph_0$'' sono
             equivalenti;
\item[$\wp(S)$] denota l'insieme di tutti i sottoinsiemi di $S$, anche detto
                insieme delle parti di $S$ o
                insieme potenza di $S$;
\item[$\wpf(S)$] denota l'insieme di tutti i sottoinsiemi \emph{finiti} di
                  $S$;

\item[$S^{\ast}$] denota l'insieme delle sequenze finite, possibilmente vuote,
                 di elementi di $S$; la sequenza vuota si indica
                 con $\varepsilon$;
\item[$S^{+}$] denota l'insieme delle sequenze finite e non vuote
               di elementi di $S$.
\end{list}

Siano $S$ e $T$ due insiemi. La notazione $S \sseqf T$ significa
$S \sseq T$ e $|S| < \aleph_0$, ovvero che $S$ \`e un sottoinsieme finito
di $T$.

\subsection{Funzioni}

Se $\fund{f}{A}{B}$ \`e una funzione e $S \sseq A$, allora si pone
\[
    f(S) = \bigl\{\, f(x) \bigm| x \in S \,\bigr\} \sseq B.
\]


\section{Strutture algebriche}

\begin{definizione} \summary{(Struttura algebrica.)}
Una \emph{struttura algebrica} o \emph{algebra mono-sortale}
\`e una coppia $(S, Q)$, dove
\begin{enumerate}
\item
$S$ \`e un insieme non vuoto detto \emph{insieme base} o \emph{carrier};
\item
$Q$ \`e una funzione definita su un insieme di indici $I$,
possibilmente infinito non numerabile, tale che, per ogni $i \in I$,
$Q(i)$ \`e un'operazione finitaria (ovvero $n$-aria con $n$ finito,
possibilmente nullo) da elementi di $S$ ad elementi di $S$. Ad
esempio, se $Q(i)$ \`e $n$-aria, allora
\[
    \fund{Q(i)}{\overbrace{S \times\cdots\times S}^{n}}{S}.
\]
\end{enumerate}
\end{definizione}

Anzich\'e con la coppia $(S, Q)$, nel seguito le algebre saranno
usualmente denotate con una $(k+1)$-pla
\[
    (S, o_1, \ldots, o_k)
\]
dove $S$ \`e il medesimo insieme non vuoto della definizione, e $o_1$,
$\ldots$, $o_k$ sono alcune operazioni nel codominio di $Q$. L'ariet\`a
delle $o_i$ sar\`a specificata di volta in volta.

\begin{definizione} \summary{(Reticolo.)}
\label{def:reticolo1}
Un'algebra
\[
    A = (L, \otimes, \oplus)
\]
con le due operazioni binarie $\otimes$ (detto \emph{meet}) e $\oplus$ (detto
\emph{join}) \`e un \emph{reticolo} se valgono, per ogni $x, y, z \in L$,
le seguenti identit\`a:
\begin{itemize}
\item[$L_1 (a):$] $x \otimes y = y \otimes x$,
\item[$L_1 (b):$] $x \oplus  y = y \oplus  x$,
    \law{leggi commutative}
\item[$L_2 (a):$] $x \otimes (y \otimes z) = (x \otimes y) \otimes z$,
\item[$L_2 (b):$] $x \oplus  (y \oplus  z) = (x \oplus  y) \oplus  z$,
    \law{leggi associative}
\item[$L_3 (a):$] $x \otimes x = x$,
\item[$L_3 (b):$] $x \oplus  x = x$,
    \law{leggi di idempotenza}
\item[$L_4 (a):$] $x \otimes (x \oplus  y) = x$,
\item[$L_4 (b):$] $x \oplus  (x \otimes y) = x$.
    \law{leggi di assorbimento}
\end{itemize}
\end{definizione}

\begin{definizione}  \summary{(Ordinamento parziale e totale.)}
Una relazione binaria $\preceq$ definita su un insieme
$S$ \`e una \emph{relazione di ordinamento parziale} su $S$
se le seguenti condizioni valgono in $S$,
per ogni $a, b, c \in S$:
\begin{itemize}
\item[$O_1:$] $a \preceq a$,
    \law{riflessivit\`a}
\item[$O_2:$] $(a \preceq b) \land (b \preceq a) \implies a = b$,
    \law{antisimmetria}
\item[$O_3:$] $(a \preceq b) \land (b \preceq c) \implies a \preceq c$.
    \law{transitivit\`a}
\end{itemize}
Se inoltre, per ogni $a, b \in S$, vale
\begin{itemize}
\item[$O_4:$] $(a \preceq b) \lor (b \preceq a)$,
\end{itemize}
allora $\preceq$ \`e detta \emph{relazione di ordinamento totale} su $S$.
\end{definizione}

\begin{definizione} \summary{(Poset, catena.)}
Un insieme non vuoto $P$ equipaggiato con una relazione di ordinamento
parziale $\preceq$ su $P$ \`e chiamato
\emph{insieme parzialmente ordinato} o \emph{poset}.
Se $\preceq$ \`e una relazione di
ordinamento totale su $P$, allora $P$ viene detto
\emph{insieme totalmente ordinato} o \emph{catena}.
In un insieme parzialmente
ordinato si usa l'espressione $a \prec b$ con il significato $a\preceq b$ ma
$a \neq b$.
\end{definizione}

Quando il poset $(P, \preceq)$ \`e implicito dal contesto
e $A \sseq P$ scriveremo $\chain(A)$ intendendo con questo
che $(A, \preceq)$ \`e una catena.

\begin{proposizione} \summary{(Ogni catena finita ha un massimo.)}
Sia $(C, \preceq)$ una catena finita, allora sicuramente tale catena ammette massimo.
\end{proposizione}
\begin{proof}
Essendo $C$ finita siamo sicuri che $|C| < \aleph_0$, definiamo dunque un insieme I $\subseteq \Nset \itc C = \left\{a_i\right\}_{i \in I}$. Dunque esiste un $i \in I \itc \forall j \in I \itc a_j \preceq a_i$. Ma allora $a_i = max\ C$.
\end{proof}


\begin{proposizione} \summary{(Ogni sottoinsieme di una catena è una catena.)}
Sia $(C, \preceq)$ una catena, allora ogni suo sottoinsieme è una catena, ovvero $\forall X \subseteq C \chain(X)$.
\end{proposizione}
\begin{proof}
È ovvio che essendo $X$ un sottoinsieme di $C$ esso avrà cardinalità minore o uguale a $C$, dunque $|X| \leq |C| \leq \aleph_0 \Rightarrow |X| \leq \aleph_0$. Abbiamo quindi dimostrato che $X$ è sicuramente numerabile, dobbiamo solo dimostrare che è totalmente ordinato:
$\forall x,y \in X \Rightarrow x,y \in C \Rightarrow x \preceq y \lor y \preceq x$
\end{proof}

\begin{definizione} \summary{(Limite superiore/inferiore, sup/inf, lub/glb.)}
Sia $(P, \preceq)$ un poset e sia $A \sseq P$.
Un elemento $p \in P$ \`e un \emph{limite superiore} per $A$
se $a \preceq p$ per ogni $a\in A$.
Un elemento $p \in P$ \`e un \emph{sup} (o \emph{lub}) di $A$,
scritto $p \in \sup A$ (o $p \in \lub A$),
se $p$ \`e un limite superiore per $A$ e,
per ogni altro limite superiore $p'$ per $A$,
risulta $p \preceq p'$.
In modo del tutto analogo un elemento $p \in P$ \`e un \emph{limite inferiore}
per $A$ se $p \preceq a$ per ogni $a \in A$.
Un elemento $p \in P$ \`e un \emph{inf} (o \emph{glb}) di $A$,
scritto $p \in \inf A$ (o $p \in \glb A$),
se $p$ \`e un limite inferiore per $A$ e,
per ogni altro limite inferiore $p'$ per $A$, risulta $p' \preceq p$.
\end{definizione}

%%%%%%%%%%%%%%
\begin{proposizione} \summary{(Unicità di $\lub$ e $\glb$.)}
Sia $(S, \preceq)$ un poset e sia $T \subseteq S$.
Se $u, v \in \lub T$ allora $u = v$.
Analogamente, se $u, v \in \glb T$ allora $u = v$.
\end{proposizione}
\begin{proof}
Siano $u, v \in \lub T$.  Per definizione di $\lub$ abbiamo
\begin{align*}
  (\forall x \in T \itc x \preceq u)
    \land
      \forall u' \in S
        &\itc (\forall x \in T \itc x \preceq u') \itc u \preceq u', \\
  (\forall x \in T \itc x \preceq v)
    \land
      \forall v' \in S
        &\itc (\forall x \in T \itc x \preceq v') \itc v \preceq v'.
\end{align*}
Da queste discende $u \preceq v$ e $v \preceq u$ il che,
per l'antisimmetria di $\preceq$ implica $u = v$.

La dimostrazione inerente l'unicità di $\glb T$ è analoga.
\end{proof}

Il risultato appena dimostrato ci consente di scrivere $p = \lub T$ e
$p = \glb T$, nel caso tali $\lub$ e $\glb$ esistano,
invece di $p \in \lub T$ e $p \in \glb T$.

%%%%%%%%%%%%%%%
\begin{proposizione} \summary{($\lub T \in T \Rightarrow \lub T = \max T$.)}
Sia $(S, \preceq)$ un ordinamento parziale, sia $T \subseteq S$ e sia $u \in S \lub\ T$, allora il $\lub T$ è anche massimo dell'insieme.
\end{proposizione}
\begin{proof}
Applicando la definizione di $\lub$ otteniamo che:
$\forall x \in T \itc x \preceq u \land u \in T \Rightarrow \nexists y \in T \itc y \succeq u$
Non esistendo dunque alcun elemento in $T$ maggiore di $u$ è ovvio che $u = \max T$.
\end{proof}

\begin{proposizione} \summary{($\exists \max S \Rightarrow \lub S = \max S$.)}
Sia $(S, \preceq)$ un ordinamento parziale, allora se esiste $u = \max S$ $u$ è anche il $\lub$ di tale insieme.
\end{proposizione}
\begin{proof}
Applichiamo dapprima la definizione di $\max$ ipotizzando che $u = \max S$:
$\forall x \in S \itc x \preceq u$. Dunque $u = ub\ S$, $u$ è un upper bound per l'insieme. Ora dobbiamo solo dimostrare che è proprio il minimo dei maggioranti: sappiamo che $\not \exists u' \in S \itc \forall y \in S \itc y \preceq u'$, non esiste dunque un altro maggiorante $u'$ (a meno che non coincida con $u$). $u$ è perciò l'unico maggiorante, dunque $u = \lub S$.
\end{proof}


\begin{definizione} \summary{(CPO.)}
\label{def:cpo}
Un poset $(D, \preceq)$ \`e un \emph{CPO} se e solo se
ha un elemento minimo e, per ogni catena $K \sseq D$, esiste $\lub K \in D$.
\end{definizione}

\begin{definizione}  \summary{(Reticolo.)}
\label{def:reticolo2}
Un insieme parzialmente ordinato $L$ \`e un \emph{reticolo} se e solo se,
per ogni $a, b \in L$, esistono in $L$ sia $\sup\{a,b\}$ che $\inf\{a,b\}$.
\end{definizione}

\begin{proposizione}
\textup{\cite{BurrisS81}} Le definizioni di reticolo
delle definizioni~\textup{\ref{def:reticolo1}} e \textup{\ref{def:reticolo2}}
sono equivalenti.
Inoltre, se $L$ \`e
un reticolo per la prima definizione, la seconda definizione si ottiene
ponendo, per ogni $a,b \in L$,
\[
    a \preceq b \quad\iff\quad a \otimes b = a \quad\iff\quad a \oplus b = b.
\]
Viceversa, se $L$ \`e un reticolo per la seconda definizione,
la prima definizione si ottiene ponendo, per ogni $a,b \in L$,
\[
    a \otimes b = \inf\{a,b\} \qquad\mbox{e}\qquad a \oplus b = \sup\{a,b\}.
\]
Le due trasformazioni sono inverse l'una dell'altra.
\end{proposizione}

\begin{definizione}
Un reticolo $L$ \`e un \emph{reticolo distributivo} se e solo se soddisfa,
per ogni $x, y \in L$, le seguenti identit\`a:
\begin{itemize}
\item[$D_1:$] $x \otimes (y \oplus   z) = (x \otimes y) \oplus  (x \otimes z)$,
\item[$D_2:$] $x \oplus  (y \otimes  z) = (x \oplus  y) \otimes (x \oplus  z)$.
    \law{leggi distributive}
\end{itemize}
\end{definizione}

\begin{proposizione}
{\rm \cite{BurrisS81}} Un reticolo soddisfa $D_1$ se e solo se soddisfa $D_2$.
\end{proposizione}

\begin{definizione}
Un poset $P$ \`e \emph{completo} se, per ogni $A \sseq P$ esistono in $P$ sia
$\lub A$ che $\glb A$. Tutti i poset completi sono reticoli e un reticolo
$L$ che \`e completo come poset \`e detto \emph{reticolo completo}.
\end{definizione}

\begin{proposizione}
\label{prop:complete-lattice0}
\textup{\cite{BurrisS81}}
Sia $P$ un poset tale che esista $\lub A$ per ogni $A \sseq P$,
oppure tale che esista $\glb A$ per ogni $A \sseq P$.
Allora $P$ \`e un reticolo completo.
\end{proposizione}

Nella precedente proposizione l'esistenza di $\lub \emptyset$ garantisce
l'esistenza di un elemento minimo in $P$ e, similmente, l'esistenza di
$\glb \emptyset$ garantisce l'esistenza di un elemento massimo in $P$.
Si pu\`o perci\`o riformulare la proposizione \ref{prop:complete-lattice0}
in un modo equivalente.

\begin{proposizione}
\label{prop:complete-lattice}
\textup{\cite{BurrisS81}} Un poset $P$ \`e completo se possiede un elemento minimo
ed esiste $\lub A$ per ogni $A$ sottoinsieme non vuoto di $P$, oppure se
possiede un elemento massimo ed esiste $\glb A$ per ogni $A$ sottoinsieme non
vuoto di~$P$.
\end{proposizione}

\begin{definizione}  \summary{(Reticolo limitato.)}
{\rm \cite{BurrisS81}} Un'algebra
\[
    (L, \otimes, \oplus, \top, \bot)
\]
con due operatori binari e due operatori nullari
\`e un \emph{reticolo limitato} se soddisfa, per ogni $x \in L$:
\begin{itemize}
\item[$B_1 \phantom{(a)}:$] $(L, \otimes, \oplus)$ \`e un reticolo,
\item[$B_2 (a):$] $x \otimes \bot = \bot$,
\item[$B_2 (b):$] $x \oplus  \top = \top$.
    \law{leggi di annichilazione}
\end{itemize}
\end{definizione}
Si osservi che se $(L, \otimes, \oplus)$ \`e un reticolo completo, allora
\[
    \bigl(L, \otimes, \oplus, \lub L, \glb L\bigr)
\]
\`e un reticolo limitato.
Si utilizzer\`a talvolta la nozione di reticolo limitato e
completo, ma al solo scopo di esibire, assieme al \emph{carrier} e agli
operatori, anche gli elementi estremi (massimo e minimo) dell'algebra.

\begin{definizione}  \summary{(Monoide.)}
Un \emph{monoide} \`e un'algebra
\[
    (M, \cdot, 1)
\]
con un operatore binario ``$\,\cdot$'' e un operatore nullario ``$\,1$'' che
soddisfa, per ogni $x,y,z\in M$ le seguenti identit\`a:
\begin{itemize}
\item[$M_1:$] $x \cdot (y \cdot z) = (x \cdot y) \cdot z$,
    \law{legge associativa}
\item[$M_2:$] $x \cdot 1 = 1 \cdot x = 1$.
    \law{legge dell'unit\`a}
\end{itemize}
Un \emph{monoide commutativo} \`e un monoide che soddisfa, per ogni $x,y \in M$,
\begin{itemize}
\item[$M_3:$] $x \cdot y = y \cdot x$.
\end{itemize}
Un \emph{monoide idempotente} \`e un monoide che soddisfa, per ogni $x \in M$,
\begin{itemize}
\item[$M_4:$] $x \cdot x = x$.
\end{itemize}
\end{definizione}

\section{Funzioni su poset}\marginpar{Mancino}

\begin{definizione} \summary{(Funzione monotona/continua/additiva.)}
Consideriamo due poset $(S, \preceq)$ e $(T, \sqsubseteq)$,
e una funzione $\fund{f}{S}{T}$.
Allora
$f$ è \emph{monotona} se, per ogni $x, y \in S$,
\[
  x \preceq y \implies f(x) \sqsubseteq f(y).
\]
Se $(S, \preceq)$ \`e un CPO,
$f$ si dice \emph{continua} se, per ogni $X \subseteq S$ tale che $\chain(X)$,
abbiamo
\[
  \exists \lub f(X) \land f(\lub X) = \lub f(X).
\]
Se $(S, \preceq)$ \`e un reticolo completo,
$f$ è \emph{additiva} se, per $X \subseteq S$ qualsiasi,
\[
  \exists \lub f(X) \land f(\lub X) = \lub f(X).
\]
\end{definizione}


\begin{proposizione} \summary{(Se $f$ è continua allora è anche monotona.)}
Siano $(S, \preceq)$ e $(T, \sqsubseteq)$ due CPO e sia $\fund{f}{S}{T}$, se $f$ è continua allora è anche monotona.
\end{proposizione}
\begin{proof}
Siano $x,y \in S \itc x \preceq y$ allora definiamo $X = \left\{x,y\right\}$. Siamo sicuri che vale $\chain(X)$ per come è stato definito l'insieme, dunque risulta banale che $\lub X = y \land \exists \lub f(X) \land f(y) = \lub f(X)$. Perciò $f$ è monotona.
\end{proof}

\begin{proposizione} \summary{(f monotona $\Arrownot \Rightarrow$ f continua.)}
Siano $(S, \preceq)$ e $(T, \sqsubseteq)$ due CPO e sia $\fund{f}{S}{T}$, se $f$ è monotona non è detto che sia anche continua.
\end{proposizione}
\begin{proof}
Scegliamo un controesempio con $(\Nset \cup \infty, \leq)$ e $(\left\{0,1\right\},\leq)$ e
\[
\fund{f}{\Nset \cup \infty}{\left\{0,1\right\}} \lambda x.
        \begin{cases}
        0 & \text{ se } x \in \Nset; \\
        1 & \text{ se } x = \infty;
    \end{cases}
\]
Scegliamo come catena proprio l'insieme $\Nset$, dunque:
$\Nset \subseteq \Nset \cup \left\{\infty\right\} \land \chain(\Nset)$ perciò $\lub f(\Nset) = \lub \left\{0\right\}$ ma $f(\lub \Nset) = f(\infty) = 1$.
\end{proof}

\begin{proposizione} \summary{(L'immagine di una catena tramite funzione monotona è una catena.)}
Siano $(X, \sqsubseteq_X)$ e $(Y, \sqsubseteq_Y)$ due poset e sia $\fund{f}{X}{Y}$ monotona e $C \subset X . \chain(C)$, allora vale anche $\chain(f(C))$
\end{proposizione}
\begin{proof}
Sia $I \subseteq \Nset \itc C = \left\{a_i\right\}_{i \in I}$, allora $f(C) = \left\{f(a_i) | i \in I\right\}$ e
$|f(C)| \leq |I| \leq |\Nset|$, perciò $f(C)$ è numerabile. Siano $i,j \in I$ allora abbiamo due casi:
\begin{itemize}
        \setlength{\itemindent}{20mm}
        \item $i \leq j \Rightarrow a_i \sqsubseteq_X a_j \Rightarrow f(a_i) \sqsubseteq_Y f(a_j)$
        \item $j \leq i \Rightarrow a_j \sqsubseteq_X a_i \Rightarrow f(a_j) \sqsubseteq_Y f(a_i)$
\end{itemize}
In entrambi i casi abbiamo dimostrato che $f(C)$ è totalmente ordinato, ovvero che $\forall i,j \in I \itc f(a_i) \sqsubseteq_Y f(a_j) \lor f(a_j) \sqsubseteq_Y f(a_j)$. $f(C)$ è quindi una catena.
\end{proof}

\begin{proposizione} \summary{(Composizione di funzioni continue.)}
Siano $(A, \sqsubseteq_A), \; (B, \sqsubseteq_B) \; e \; (D, \sqsubseteq_D) $ dei CPO e siano $\fund{f}{A}{B}$ e $\fund{g}{B}{D}$ funzioni continue, allora $\fund{\compfun{g}{f}}{A}{D}$ è continua.
\end{proposizione}
\begin{proof}
Sia $C \subseteq A \itc \chain(C)$ allora sappiamo che esiste $\lub f(C)$ e che $\lub f(C) = f(\lub C)$.
Dall'esercizio precedente sappiamo che $f(C)$ è una catena, perciò esiste $\lub g(f(C))$ e $g(\lub f(C)) = \lub g(f(C))$.
Ma allora $\compfun{g}{f}(\lub C) = g(f(\lub C)) = g(\lub f(C)) = \lub g(f(C)) = \lub \compfun{g}{f}(C)$.
$\compfun{g}{f}$ è perciò continua.
\end{proof}

\section{Punti fissi}

\begin{definizione} \summary{(Punto fisso.)}\marginpar{Mancino}
Sia $S$ un insieme e sia $\fund{f}{S}{S}$ una funzione.
Allora un \emph{punto fisso} di $f$ è un qualunque $x \in S$ tale
che $f(x) = x$.
\end{definizione}

Una funzione può avere zero punti fissi (esempio: $f(x) = x+1$),
un solo punto fisso
(esempio: $g(x) = 7$ ha un punto fisso in $x=7$),
o addirittura infiniti punti fissi (esempio: $h(x) = x$).
Quando un punto fisso di $f$ gode di una certa proprietà rispetto
all'insieme di tutti i punti fissi di $f$, ad esempio è il minimo
oppure il massimo di questi, diremo che è il
\emph{minimo punto fisso di $f$}, denotato con $\lfp f$,
o il \emph{massimo punto fisso di $f$}, denotato con $\gfp f$.

\subsection{Teorema di Tarski}\marginpar{Marano}

\begin{teorema} \summary{(Teorema di Tarski.)}
Sia $(S, \preceq)$ un CPO, sia $\bot = \min S$ e sia
$\fund{f}{S}{S}$ continua.
Allora $f$ ha un minimo punto fisso e questo coincide con il limite
delle iterate di $f$ a partire da $\bot$.
Formalmente:
\[
  \exists \lfp f
    \land
      \lfp f = \lub \bigl\{\, f^n(\bot) \bigm| n \in \Nset \,\bigr\}.
\]
\begin{proof}
Come prima cosa dimostriamo per induzione su $n \in \Nset$
che l'insieme
\[
  C \defeq \bigl\{\, f^n(\bot) \bigm| n \in \Nset \,\bigr\}
\]
è una catena, ovvero che, per ogni $i \in \Nset$,
$f^i(\bot) \preceq f^{i+1}(\bot)$.

Per $i = 0$, abbiamo $f^0(\bot) = \bot \preceq f^1(\bot)$
dal momento che $\bot = \min S$.

Per $i > 0$, dalla monotonia di $f$ e dall'ipotesi induttiva discende che
\[
  f^i(\bot) = f\bigl(f^{i-1}(\bot)\bigr)
            \preceq f\bigl(f^i(\bot)\bigr)
            = f^{i+1}(\bot).
\]

Da $\chain(C)$ e dal fatto che $(S, \preceq)$ è un CPO deriva l'esistenza
di $\lub C$.
Dalla continuità di $f$ discende $f(\lub C) = \lub f(C)$ ma,
dalla definizione di $C$ sappiamo che
\begin{align*}
  \lub f(C)
    &= \lub f\Bigl(\bigl\{\, f^n(\bot) \bigm| n \in \Nset \,\bigr\}\Bigr) \\
    &= \lub \Bigl(\bigl\{\, f(f^n(\bot)) \bigm| n \in \Nset \,\bigr\}\Bigr) \\
    &= \lub \Bigl(\bigl\{\, f^{n+1}(\bot) \bigm| n \in \Nset \,\bigr\}\Bigr) \\
    &= \lub \Bigl(\bigl\{\, f^n(\bot) \bigm| n \in \Nset \,\bigr\}\Bigr) \\
    &= \lub C.
\end{align*}
Mettendo insieme questi due fatti abbiamo che $f(\lub C) = \lub C$,
ovvero che $\lub C$ è un punto fisso di $f$.
Dimostriamo per assurdo che $\lub C$ è il minimo dei punti fissi di $f$:
sia $m$ un punto fisso di $f$ con $m \prec \lub C$;
dunque esisterebbe $i \in \Nset$ tale che $m = f^j(\bot)$ per ogni $j \geq i$,
ma questo implicherebbe che $\lub C$ non sia il least upper bound di C,
il che è assurdo.
  \end{proof}
\end{teorema}
