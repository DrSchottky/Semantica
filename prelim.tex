
\chapter{Nozioni preliminari}

[Introduzione al capitolo da scrivere.]


\section{Notazione basilare}

\subsection{Insiemi e sequenze}

Sia $S$ un insieme.
Se $S$ \`e l'insieme vuoto allora si scrive $S = \emptyset$.
Inoltre:
\begin{list}{}{}
\item[$|S|$] denota la cardinalit\`a di S; se $S$ \`e un insieme finito
             allora $|S|$ \`e il numero di elementi di $S$; se $S$
             pu\`o essere messo in corrispondenza biunivoca con
             l'insieme dei numeri naturali $\Nset$ allora si dice che
             \`e infinito e numerabile oppure che ha la
             \emph{potenza del numerabile} oppure che ha cardinalit\`a
             $\aleph_0$ e si scrive $|S| = \aleph_0$; per contro, le
             espressioni ``S \`e finito'' e ``$|S| < \aleph_0$'' sono
             equivalenti;
\item[$\wp(S)$] denota l'insieme di tutti i sottoinsiemi di $S$, anche detto
                insieme delle parti di $S$ o
                insieme potenza di $S$;
\item[$\wp_f(S)$] denota l'insieme di tutti i sottoinsiemi \emph{finiti} di
                  $S$;

\item[$S^{\ast}$] denota l'insieme delle sequenze finite, possibilmente vuote,
                 di elementi di $S$; la sequenza vuota si indica
                 con $\varepsilon$;
\item[$S^{+}$] denota l'insieme delle sequenze finite e non vuote
               di elementi di $S$.
\end{list}

Siano $S$ e $T$ due insiemi. La notazione $S \sseqf T$ significa
$S \sseq T$ e $|S| < \aleph_0$, ovvero che $S$ \`e un sottoinsieme finito
di $T$.

\subsection{Funzioni}

Se $\fund{f}{A}{B}$ \`e una funzione e $S \sseq A$, allora si pone
\[
    f(S) = \bigl\{\, f(x) \bigm| x \in S \,\bigr\} \sseq B.
\]


\section{Strutture algebriche}

\begin{definizione} \summary{(Struttura algebrica.)}
Una \emph{struttura algebrica} o \emph{algebra mono-sortale}
\`e una coppia $(S, Q)$, dove
\begin{enumerate}
\item
$S$ \`e un insieme non vuoto detto \emph{insieme base} o \emph{carrier};
\item
$Q$ \`e una funzione definita su un insieme di indici $I$,
possibilmente infinito non numerabile, tale che, per ogni $i \in I$,
$Q(i)$ \`e un'operazione finitaria (ovvero $n$-aria con $n$ finito,
possibilmente nullo) da elementi di $S$ ad elementi di $S$. Ad
esempio, se $Q(i)$ \`e $n$-aria, allora
\[
    \fund{Q(i)}{\overbrace{S \times\cdots\times S}^{n}}{S}.
\]
\end{enumerate}
\end{definizione}

Anzich\'e con la coppia $(S, Q)$, nel seguito le algebre saranno
usualmente denotate con una $(k+1)$-pla
\[
    (S, o_1, \ldots, o_k)
\]
dove $S$ \`e il medesimo insieme non vuoto della definizione, e $o_1$,
$\ldots$, $o_k$ sono alcune operazioni nel codominio di $Q$. L'ariet\`a
delle $o_i$ sar\`a specificata di volta in volta.

\begin{definizione} \summary{(Reticolo.)}
\label{def:reticolo1}
Un'algebra
\[
    A = (L, \otimes, \oplus)
\]
con le due operazioni binarie $\otimes$ (detto \emph{meet}) e $\oplus$ (detto
\emph{join}) \`e un \emph{reticolo} se valgono, per ogni $x, y, z \in L$,
le seguenti identit\`a:
\begin{itemize}
\item[$L_1 (a):$] $x \otimes y = y \otimes x$,
\item[$L_1 (b):$] $x \oplus  y = y \oplus  x$,
    \law{leggi commutative}
\item[$L_2 (a):$] $x \otimes (y \otimes z) = (x \otimes y) \otimes z$,
\item[$L_2 (b):$] $x \oplus  (y \oplus  z) = (x \oplus  y) \oplus  z$,
    \law{leggi associative}
\item[$L_3 (a):$] $x \otimes x = x$,
\item[$L_3 (b):$] $x \oplus  x = x$,
    \law{leggi di idempotenza}
\item[$L_4 (a):$] $x \otimes (x \oplus  y) = x$,
\item[$L_4 (b):$] $x \oplus  (x \otimes y) = x$.
    \law{leggi di assorbimento}
\end{itemize}
\end{definizione}

\begin{definizione}  \summary{(Ordinamento parziale e totale.)}
Una relazione binaria $\preceq$ definita su un insieme
$S$ \`e una \emph{relazione di ordinamento parziale} su $S$
se le seguenti condizioni valgono identicamente in $S$,
per ogni $a, b, c \in S$,
\begin{itemize}
\item[$O_1:$] $a \preceq a$,
    \law{riflessivit\`a}
\item[$O_2:$] $(a \preceq b) \land (b \preceq a) \implies a = b$,
    \law{antisimmetria}
\item[$O_3:$] $(a \preceq b) \land (b \preceq c) \implies a \preceq c$.
    \law{transitivit\`a}
\end{itemize}
Se inoltre, $\forall a, b \in S$, vale
\begin{itemize}
\item[$O_4:$] $(a \preceq b) \lor (b \preceq a)$,
\end{itemize}
allora $\preceq$ \`e detta \emph{relazione di ordinamento totale} su $S$.
\end{definizione}

\begin{definizione} \summary{(Poset, catena.)}
Un insieme non vuoto $P$ equipaggiato con una relazione di ordinamento
parziale $\preceq$ su $P$ \`e chiamato
\emph{insieme parzialmente ordinato} o \emph{poset}.
Se $\preceq$ \`e una relazione di
ordinamento totale su $P$, allora $P$ viene detto
\emph{insieme totalmente ordinato} o \emph{catena}.
In un insieme parzialmente
ordinato si usa l'espressione $a<b$ con il significato $a\preceq b$ ma
$a \neq b$.
\end{definizione}

Quando il poset $(P, \preceq)$ \`e implicito dal contesto
e $A \sseq P$ scriveremo $\chain(A)$ intendendo con questo
che $(A, \preceq)$ \`e una catena.

\begin{definizione} \summary{(Limite superiore/inferiore, sup/inf.)}
Sia $(P, \preceq)$ un poset e sia $A \sseq P$.
Un elemento $p \in P$ \`e un \emph{limite superiore} per $A$
se $a \preceq p$ per ogni $a\in A$.
Un elemento $p \in P$ \`e il \emph{sup} di $A$, scritto $p = \sup(A)$,
se $p$ \`e un limite superiore per $A$ e,
per ogni altro limite superiore $p'$ per $A$,
risulta $p \preceq p'$.
In modo del tutto analogo un elemento $p \in P$ \`e un \emph{limite inferiore}
per $A$ se $p \preceq a$ per ogni $a \in A$.
Un elemento $p \in P$ \`e detto \emph{inf} di $A$,
scritto $p = \inf(A)$, se $p$ \`e un limite inferiore per $A$ e,
per ogni altro limite inferiore $p'$ per $A$, risulta $p' \preceq p$.
\end{definizione}

\begin{definizione} \summary{(CPO.)}
\label{def:cpo}
Un poset $(D, \preceq)$ \`e un \emph{CPO}, o \emph{dominio}, se e solo se
ha un elemento minimo e, per ogni catena $K \sseq D$, esiste $\sup(K) \in D$.
\end{definizione}

\begin{definizione}  \summary{(Reticolo.)}
\label{def:reticolo2}
Un insieme parzialmente ordinato $L$ \`e un ``reticolo'' se e solo se,
per ogni $a, b \in L$, esistono in $L$ sia $\sup\{a,b\}$ che $\inf\{a,b\}$.
\end{definizione}

\begin{proposizione}
\textup{\cite{BurrisS81}} Le definizioni di reticolo
delle definizioni~\textup{\ref{def:reticolo1}} e \textup{\ref{def:reticolo2}}
sono equivalenti.
Inoltre, se $L$ \`e
un reticolo per la prima definizione, la seconda definizione si ottiene
ponendo, per ogni $a,b \in L$,
\[
    a \preceq b \quad\iff\quad a \otimes b = a \quad\iff\quad a \oplus b = b.
\]
Viceversa, se $L$ \`e un reticolo per la seconda definizione,
la prima definizione si ottiene ponendo, per ogni $a,b \in L$,
\[
    a \otimes b = \inf\{a,b\} \qquad\mbox{e}\qquad a \oplus b = \sup\{a,b\}.
\]
Le due trasformazioni sono inverse l'una dell'altra.
\end{proposizione}

\begin{definizione}
Un reticolo $L$ \`e un ``reticolo distributivo'' se e solo se soddisfa,
$\forall x, y \in L$ le seguenti identit\`a:
\begin{itemize}
\item[$D_1:$] $x \otimes (y \oplus   z) = (x \otimes y) \oplus  (x \otimes z)$,
\item[$D_2:$] $x \oplus  (y \otimes  z) = (x \oplus  y) \otimes (x \oplus  z)$.
    \law{leggi distributive}
\end{itemize}
\end{definizione}

\begin{proposizione}
{\rm \cite{BurrisS81}} Un reticolo soddisfa $D_1$ se e solo se soddisfa $D_2$.
\end{proposizione}

\begin{definizione}
Un poset $P$ \`e ``completo'' se, per ogni $A \sseq P$ esistono in $P$ sia
$\sup(A)$ che $\inf(A)$. Tutti i poset completi sono reticoli e un reticolo
$L$ che \`e completo come poset \`e detto ``reticolo completo''.
\end{definizione}

\begin{proposizione}
\label{prop:complete-lattice0}
\textup{\cite{BurrisS81}}
Sia $P$ un poset tale che esista $\sup(A)$ per ogni $A \sseq P$,
oppure tale che esista $\inf(A)$ per ogni $A \sseq P$.
Allora $P$ \`e un reticolo completo.
\end{proposizione}

Nella precedente proposizione l'esistenza di $\sup(\emptyset)$ garantisce
l'esistenza di un elemento minimo in $P$ e, similmente, l'esistenza di
$\inf(\emptyset)$ garantisce l'esistenza di un elemento massimo in $P$.
Si pu\`o perci\`o riformulare la proposizione \ref{prop:complete-lattice} in un
modo equivalente.

\begin{proposizione}
\label{prop:complete-lattice}
{\rm \cite{BurrisS81}} Un poset $P$ \`e completo se possiede un elemento minimo
ed esiste $\sup(A)$ per ogni $A$ sottoinsieme non vuoto di $P$, oppure se
possiede un elemento massimo ed esiste $\inf(A)$ per ogni $A$ sottoinsieme non
vuoto di $P$.
\end{proposizione}

\begin{definizione}  \summary{(Reticolo limitato.)}
{\rm \cite{BurrisS81}} Un'algebra
\[
    (L, \otimes, \oplus, \top, \bot)
\]
con due operatori binari e due operatori nullari
\`e un \emph{reticolo limitato} se soddisfa, per ogni $x \in L$:
\begin{itemize}
\item[$B_1 \phantom{(a)}:$] $(L, \otimes, \oplus)$ \`e un reticolo,
\item[$B_2 (a):$] $x \otimes \bot = \bot$,
\item[$B_2 (b):$] $x \oplus  \top = \top$.
    \law{leggi di annichilazione}
\end{itemize}
\end{definizione}
Si osservi che se $(L, \otimes, \oplus)$ \`e un reticolo completo, allora
\[
    \bigl(L, \otimes, \oplus, \sup(L), \inf(L)\bigr)
\]
\`e un reticolo limitato.
Si utilizzer\`a talvolta la nozione di reticolo limitato e
completo, ma al solo scopo di esibire, assieme al \emph{carrier} e agli
operatori, anche gli elementi estremi (massimo e minimo) dell'algebra.

\begin{definizione}  \summary{(Monoide.)}
Un \emph{monoide} \`e un'algebra
\[
    (M, \cdot, 1)
\]
con un operatore binario ``$\,\cdot$'' e un operatore nullario ``$\,1$'' che
soddisfa, per ogni $x,y,z\in M$ le seguenti identit\`a:
\begin{itemize}
\item[$M_1:$] $x \cdot (y \cdot z) = (x \cdot y) \cdot z$,
    \law{legge associativa}
\item[$M_2:$] $x \cdot 1 = 1 \cdot x = 1$.
    \law{legge dell'unit\`a}
\end{itemize}
Un \emph{monoide commutativo} \`e un monoide che soddisfa, per ogni $x,y \in M$,
\begin{itemize}
\item[$M_3:$] $x \cdot y = y \cdot x$.
\end{itemize}
Un \emph{monoide idempotente} \`e un monoide che soddisfa, per ogni $x \in M$,
\begin{itemize}
\item[$M_4:$] $x \cdot x = x$.
\end{itemize}
\end{definizione}

\section{Funzioni su poset}

\begin{definizione} \summary{(Funzione monotona/continua/additiva.)}
Siano $(S, \preceq)$ e $(T, \sqsubseteq)$ due poset
e sia $\fund{f}{S}{T}$ una funzione.
Allora
$f$ è \emph{monotona} se, per ogni $x, y \in S$,
$x \preceq y \implies f(x) \sqsubseteq f(y)$.
Se $(S, \preceq)$ \`e un CPO,
$f$ si dice \emph{continua} se, per ogni $X \subseteq S$,
$\chain(X) \implies \bigl(\exists \lub f(X) \land f(\lub X) = \lub f(X)\bigr)$.
Se $(S, \preceq)$ \`e un reticolo completo,
$f$ è \emph{additiva} se, per ogni $X \subseteq S$,
$\exists \lub f(X) \land f(\lub X) = \lub f(X)$
\end{definizione}


\section{Punti fissi}\marginpar{Marano}

\begin{definizione} \summary{(Punto fisso.)}
Sia $S$ un insieme e sia $\fund{f}{S}{S}$ una funzione.
Allora un \emph{punto fisso} di $f$ è un qualunque $x \in S$ tale
che $f(x) = x$.
\end{definizione}

Una funzione può avere zero punti fissi (esempio: ...), un solo punto fisso
(esempio: ...), o addirittura infiniti punti fissi (esempio: ...).
Quando un punto fisso di $f$ gode di una certa proprietà rispetto
all'insieme di tutti i punti fissi di $f$, ad esempio è il minimo
oppure il massimo di questi, diremo che è il
\emph{minimo punto fisso di $f$}, denotato con $\lfp f$,
o il \emph{massimo punto fisso di $f$}, denotato con $\gfp f$,

\subsection{Teorema di Tarski}
\begin{teorema}[Teorema di Tarski]
Sia $(S, \preceq)$ un CPO, sia $\bot = \min S$ e sia
$\fund{f}{S}{S}$ continua.
Allora $f$ ha un minimo punto fisso e questo coincide con il limite
delle iterate di $f$ a partire da $\bot$.
Formalmente:
\[
  \exists \lfp f
    \land
      \lfp f = \lub \bigl\{\, f^n(\bot) \bigm| n \in \Nset \,\bigr\}.
\]
  \begin{proof}
Come prima cosa dimostriamo per induzione su $n \in \Nset$
che l'insieme
\[
  C \defeq \bigl\{\, f^n(\bot) \bigm| n \in \Nset \,\bigr\}
\]
è una catena, ovvero che, per ogni $i \in \Nset$,
$f^i(\bot) \preceq f^{i+1}(\bot)$.

Per $i = 0$, abbiamo $f^0(\bot) = \bot \preceq f^1(\bot)$
dal momento che $\bot = \min S$.

Per $i > 0$, dalla monotonia di $f$ e dall'ipotesi induttiva discende che
\[
  f^i(\bot) = f\bigl(f^{i-1}(\bot)\bigr)
            \preceq f\bigl(f^i(\bot)\bigr)
            = f^{i+1}(\bot).
\]

$Chain(C) \land (S, \preceq)\ \emph{CPO} \Rightarrow \exists lub\ C$. $f$ è continua $\Rightarrow f(lub\ C) = lub\ f(C)$ ma per definizione di C sappiamo
    che $lub\ f(C) = lub\ C \Rightarrow lub\ C$ è un \emph{FP} di $f$. Dimostriamo per assurdo che è proprio il \emph{LFP}:
    Sia $m$ un \emph{FP} di f con $m \prec lub\ C \Rightarrow \exists i \in \mathbb{N} : m = f^j(\bot) \forall j \leq i$ ma così il $lub\ C$ non sarebbe più
    un \emph{LUB} che è assurdo.
  \end{proof}
\end{teorema}

\section{Esercizi sui CPO}

\theoremstyle{definizione}
\newtheorem{esercizio}{Esercizio}

\subsection{Unicità lub}
\begin{esercizio}[Unicità lub]
  Sia $(S, \preceq)$ un \emph{PO}, sia $T \subseteq S \Rightarrow (\exists lub\ T \Rightarrow lub\ T$ è unico)
\end{esercizio}
\begin{proof}
  \begin{center}
    Siano u,u' $\in$ S due lub di T\\
    $\Downarrow$\\
    $(\forall x \in T : x \preceq u) \land \forall u' \in S :
    (\forall x' \in T : x' \preceq u') : u \preceq u'$
  \end{center}
  Ma anche:
  \begin{center}
    $(\forall x \in T : x \preceq u') \land \forall u \in S :
    (\forall x' \in T : x' \preceq u) : u' \preceq u$\\
    $\Downarrow$\\
    $u \preceq u' \land u' \preceq u \Rightarrow u = u'$
  \end{center}
\end{proof}

\subsection{Se il lub appartiene a un insieme è il massimo}
\begin{esercizio}[$lub\ T \in T \Rightarrow lub\ T = max\ T$]
  Sia $(S, \preceq)$ un ordinamento parziale, sia $T \subseteq S$ e\\
  sia $u \in S lub\ T \Rightarrow (u \in T \Rightarrow u = max\ T)$
  \begin{proof}
    $\forall x \in T : x \preceq u \land u \in T \Rightarrow \not \exists y \in T : y \succeq u \Rightarrow u = max\ T$
  \end{proof}
\end{esercizio}

\subsection{Il massimo di un insieme è anche il lub}
\begin{esercizio}[$\exists max\ S \Rightarrow lub\ S = max\ S$]
  Sia $(S, \preceq)$ un ordinamento parziale
  \begin{center}
    $\Downarrow$\\
    $\exists max\ S \Rightarrow lub\ S = max\ S$
  \end{center}
  \begin{proof}
    È banale guardando la definizione di massimo e di lub:\\
    Sia $u \in S : u = max\ S \Rightarrow \forall x \in S : x \preceq u \Rightarrow u = ub\ S$.\\
    È anche lub visto che $\not \exists u' \in S : \forall y \in S : y \preceq u'$ con $ u' \neq u$
  \end{proof}
\end{esercizio}

\subsection{Ogni catena finita ha massimo}
\begin{esercizio}
  Sia $(C, \preceq)$ un \emph{PO} $: Chain(C) \land |C| < \aleph_0 \Rightarrow \exists max\ C$
  \begin{proof}
    Sia I $\subseteq \mathbb{N} : C = \left\{a_i\right\}_{i \in I} \Rightarrow \exists i \in I : \forall j \in I : a_j \preceq a_i \Rightarrow a_i = max\ C$
    \end{proof}
\end{esercizio}

\subsection{Ogni sottoinsieme di una catena è una catena}
\begin{esercizio}
  Sia $(C, \preceq)$ un \emph{PO} $: Chain(C) \Rightarrow (X \subseteq C \Rightarrow Chain(X))$
  \begin{proof}
    $X \subseteq C \Rightarrow |X| \leq |C| \leq \aleph_0 \Rightarrow |X| \leq \aleph_0$. $\forall x,y \in X \Rightarrow x,y \in C \Rightarrow x \preceq y \lor y \preceq x$\\
    X è dunque numerabile e totalmente ordinato
  \end{proof}
\end{esercizio}

\subsection{$(ST, \sqsubseteq)$ è un CPO}
\begin{esercizio}[$(ST, \sqsubseteq)$ è un CPO]
  Sia ST l'insieme degli \emph{State Transformer} e
  \begin{center}
    sia $\sqsubseteq\ \subseteq ST \times ST :
    \forall f,g \in ST : f \sqsubseteq g \Longleftrightarrow \forall s \in \sum :
    f(s)\downarrow \Rightarrow f(s) = g(s)$\\
    $\Downarrow$\\
    $(ST, \sqsubseteq)$ è un \emph{CPO}
  \end{center}
  \begin{proof}
    Sia $\bot \in ST : \forall s \in \sum : \bot(s)\uparrow \Rightarrow \forall f \in ST : \bot \sqsubseteq f \Rightarrow \bot = min\ ST $\\
    Sia I $\subseteq \mathbb{N} : C = \left\{a_i\right\}_{i \in I} \Rightarrow$ se $|I| < \aleph_0 \Rightarrow$ il lub è l'elemento massimo nella catena.\\
    Se |I| = $\aleph_0$ definisco $\forall x \in \sum : f_l(x) =
    \begin{cases}
      f_h(x) & \text{se } \exists h \in I : f_h(x)\downarrow
      \\
      \uparrow & \text{altrimenti}
    \end{cases} $
    \begin{center}
      $\Downarrow$\\
      $u = lub\ C \Longleftrightarrow (\forall i \in I : f_i \sqsubseteq u) \land (\forall u' \in ST : (\forall j \in I : f_j \sqsubseteq u') \Rightarrow u \sqsubseteq u')$\\
      Sia $i \in I : \forall s \in \sum : f_i(s)\downarrow \Rightarrow (u(s)\downarrow \land u(s)=f_i(s))$\\
      Sia $s \in \sum : u(s)\downarrow \Rightarrow (\exists k \in I : f_k(s)\downarrow \land f_k(s)=u(s))$\\
      $\forall j \in I : f_j \sqsubseteq f_l \Rightarrow f_l = lub\ C$\\
      $(ST, \sqsubseteq)$ ha minimo e ogni sua catena ha lub $\Rightarrow (ST, \sqsubseteq)$ è un \emph{CPO}
    \end{center}
  \end{proof}
\end{esercizio}


\subsection{$f$  continua $\Rightarrow f$  monotona}
\begin{esercizio}[$f$  continua $\Rightarrow f$  monotona]
  Siano $(S, \preceq)$ e $(T, \sqsubseteq)$ due \emph{CPO} e sia $f: S \rightarrow T $
  \begin{center}
    $\Downarrow$\\
    $(f\ continua \Rightarrow f\ monotona)$
  \end{center}
  \begin{proof}
    Siano $x,y \in S : x \preceq y \Rightarrow $ definiamo $X = \left\{x,y\right\}$. Siamo sicuri che vale $Chain(X)$ per come è stato definito l'insieme $\Rightarrow lub\ X = y \land \exists lub\ f(X) \land f(y) = lub\ f(x) \Rightarrow f(x) \sqsubseteq f(y) \Rightarrow f$ è monotona.
  \end{proof}
\end{esercizio}


\subsection{f monotona $\Arrownot \Rightarrow$ f continua}
\begin{esercizio}[f monotona $\Arrownot \Rightarrow$ f continua]
  Siano $(S, \preceq)$ e $(T, \sqsubseteq)$ due \emph{CPO} e sia $f: S \rightarrow T$
  \begin{center}
    $\Downarrow$\\
    $(f\ monotona \Arrownot \Rightarrow f\ continua)$
  \end{center}
  \begin{proof}
    Scegliamo un controesempio con $(\mathbb{N} \cup \infty, \leq)$ e $(\left\{0,1\right\},\leq)$ e
    \begin{center}
      $f: \mathbb{N} \cup \infty \rightarrow \left\{0,1\right\} \lambda x.$
      $\begin{cases}
        0 & \text{ se } x \in \mathbb{N}
        \\
        1 & \text{ se } x = \infty
      \end{cases}
      $
    \end{center}
    $\mathbb{N} \subseteq \mathbb{N} \cup \left\{\infty\right\} \land Chain(\mathbb{N}) \Rightarrow lub\ f(\mathbb{N}) = lub\ \left\{0\right\}$ ma $f(lub\ \mathbb{N}) = f(\infty) = 1$
  \end{proof}
\end{esercizio}

\
\subsection{L'immagine di una catena è una catena}
\begin{esercizio}
  Siano $(X, \sqsubseteq_X)$ e $(Y, \sqsubseteq_Y)$ due \emph{PO} e sia $f: X \rightarrow Y$ monotona
  \begin{center}
    $\Downarrow$\\
    $(\forall C \subseteq X : Chain(C) \Rightarrow Chain(f(C))$
  \end{center}
  \begin{proof}
    Sia $I \subseteq \mathbb{N} : C = \left\{a_i\right\}_{i \in I} \subseteq X \Rightarrow f(C) = \left\{f(a_i) | i \in I\right\} \Rightarrow |f(C)| \leq |I| \leq |\mathbb{N}| \Rightarrow f(C)$ è numerabile.\\
    Siano $i,j \in I \Rightarrow $
    \begin{itemize}
      \setlength{\itemindent}{20mm}
    \item $i \leq j \Rightarrow a_i \sqsubseteq_X a_j \Rightarrow f(a_i) \sqsubseteq_Y f(a_j)$
    \item $j \leq i \Rightarrow a_j \sqsubseteq_X a_i \Rightarrow f(a_j) \sqsubseteq_Y f(a_i)$
    \end{itemize}
    $\Rightarrow \forall i,j \in I : f(a_i) \sqsubseteq_Y f(a_j) \lor f(a_j) \sqsubseteq_Y f(a_j) \Rightarrow Chain(f(C))$
  \end{proof}
\end{esercizio}

\subsection{Composizione di funzioni continue}
\begin{esercizio}[Composizione di funzioni continue]
  Siano $ (A, \sqsubseteq_A), \; (B, \sqsubseteq_B) \; e \; (D, \sqsubseteq_D) $ dei \emph{CPO} e siano $ f: A \rightarrow B $ e $ g: B \rightarrow D $ funzioni continue $ \Rightarrow g \circ f : A \rightarrow D $ è continua.
  \begin{proof}
    Sia $C \subseteq A : Chain(C) \Rightarrow \exists lub \; f(C) \land lub \; f(C) = f(lub \; C)$.
    Dall'esercizio precedente sappiamo che:
    \begin{center}
      $ Chain(f(C)) \Rightarrow (\exists lub \; g(f(C)) \land g(lub \; f(C)) = lub \; g(f(C))) $\\
      $ \Rightarrow g \circ f(lub \; C) = g(f(lub \; C)) = g(lub \; f(C)) = lub \; g(f(C)) = lub \; g \circ f(C) $\\
      $ \Rightarrow g \circ f$ è continua.
    \end{center}
  \end{proof}
\end{esercizio}

\section{Continuità del comando While}
Il comando $ while\; B\; do\; C $ è continuo.

\begin{proof}

  Il comando $$\emph{C}\llbracket while\; B\; do\; C \rrbracket $$ è riconducibile a:
  $$ \emph{C}\llbracket if\; B\; then\; (C;\; while\; B\; do\; C)\; else\; skip \rrbracket = $$
  $$ = cond( \emph{B}\llbracket B \rrbracket\; ,\; seq(\emph{C}\llbracket C \rrbracket , C \llbracket \\ while\; B\; do\; C \rrbracket , id ) $$
  $$ = cond( \emph{B}\llbracket B \rrbracket\;, \emph{C}\llbracket while\; B\; do\; C \rrbracket \circ \emph{C}\llbracket C \rrbracket , id) $$

  Ora sia:
  $$ f = cond(B \llbracket B \rrbracket , f \circ C \llbracket C \rrbracket , id) $$
  E sia:
  $ f = F(f) $, con:
  $$ F = \lambda f . cond(\emph{B} \llbracket B \rrbracket , f \circ C \llbracket C \rrbracket , id) $$
  dove $ F: ST \rightarrow ST $. Dobbiamo dimostrare che F è continua. Dimostriamo in primo luogo che il \emph{lub} esiste. Per farlo, presi \emph{b} e \emph{g} generici e $ \emph{X} \subseteq ST \land chain(X) $, facciamo vedere che $$ cond(b,X,g) $$ è monotono sul secondo argomento, tenendo fissi il primo e il terzo argomento:
  $$ (f_1 \sqsubseteq f_2) \rightarrow cond(b,f_{1},g) \sqsubseteq cond(b,f_{2},g) $$
  Preso $s \in \Sigma $ generico, ipotizzo che:
  $$ cond(b,f_1,g)(s) \downarrow $$
  sia definito. Allora ci sono due possibilità:
  \begin{enumerate}
  \item $ b(s) = false $ \\
    $ cond(b,f_1,g)(s)=g(s)=cond(b,f_2,g)(s) $
  \item $ b(s) = true $ \\
    $ cond(b,f_1,g)(s) = f_{1}(s) \sqsubseteq f_{2}(s) = cond(b,f_2,g)(s) $
  \end{enumerate}
  Concludiamo allora che la monotonia vale. \\
  Sia ora $ X \subseteq ST \; . \; chain(X) $. Allora dimostriamo che:
  $$ chain( \{cond(b,f,g) | f \in X\}) $$
  La cardinalità di questo insieme è minore o uguale a quella di \emph{X}, e quindi di $ \aleph_0 $. In particolare se \emph{X} è infinito c'è almeno un elemento che è g (quando b è falso). Quindi l'insieme è numerabile. \\ Ora devo dimostrare che è totalmente ordinato. Siano $f_1,f_2 \in \emph{X}$ generiche tali che \emph{chain(X)}. Allora:
  \begin{enumerate}
  \item $cond(b,f_1,g) \sqsubseteq cond(b,f_2,g)$, oppure
  \item $cond(b,f_2,g) \sqsubseteq cond(b,f_1,g)$
  \end{enumerate}
  Dato che \emph{X} è una catena, $f_1 \sqsubseteq f_2 \lor f_2 \sqsubseteq f_1 $. E otteniamo nel primo caso la (1) mentre la (2) nel secondo caso. Siaccome siamo in un \emph{CPO} concludiamo che il lub esiste. \\
  Ora dobbiamo far vedere chi è il lub. Intuitivamente è:
  $$ lub( \{ cond(b,f,g) \; | \; f \in X \})= \begin{cases} \uparrow , & \mbox{se } b(s)=true \land \forall f \in X f(s) \uparrow \; \\ g(s), & \mbox{se } b(s)=false \; \\ f(s) & \mbox{se } b(s)=true \land f \in X \mbox{ tale che } f(s) \downarrow
  \end{cases} $$
  Dobbiamo dimostrare che il \emph{lub} è proprio lui e che:
  $$ lub(cond(b,X,g)) = cond(b, lub\; X, g) $$
  In questo modo avremo dimostrato che \emph{cond} è continuo sul secondo argomento. Chiamiamo allora \emph{h(s)} l'espressione in parentesi graffe e dimostriamo che h(s) è proprio il \emph{lub} di:
  $$ \{cond(b,f,g) | f \in X \} $$
  Si ha $ \forall b \in BT, \forall g \in ST, \forall f \in X : cond(b,f,g) \sqsubseteq h $, cioè che:
  $$ \forall f_1, f_2 \in ST : f_1 \sqsubseteq f_2 \Leftrightarrow $$
  $$ \Leftrightarrow \forall s \in \Sigma \mbox{ tale che } f_{1}(s) \downarrow \Rightarrow f_{2}(s) \downarrow \land f_{2}(s) = f_{1}(s) $$
  Applicando la suddetta definizione:
  \begin{itemize}
  \item Se \emph{h(s)} diverge non c'è nulla da dimostrare;
  \item Supponiamo che $cond(b,f,g)$ converga. Dobbiamo dimostrare che:
    $$ \forall s \in \Sigma \mbox{ tale che } cond(b,f,g)(s) \downarrow \Rightarrow h(s) \downarrow \land cond(b,f,g)(s) = h(s) $$
    Ci sono due ulteriori casi:
    \begin{enumerate}
    \item Se $b(s) = false$ per definizione di \emph{cond}:
      $$ cond(b,f,g)(s) = g(s) \land h(s) = g(s) $$
    \item Se $b(s)=true$ per definizione di \emph{cond}:
      $$ cond(b,f,g)(s) = f(s) \Rightarrow f(s) \downarrow $$
      Poichè $f \in X$ e sappiamo che termina, siamo nel terzo caso. Ma allora:
      $$ h(s) = f(s) $$
    \end{enumerate}
  \end{itemize}

  Quindi sappiamo che è un upper bound, dobbiamo solo dimostrare che è il lub. Dobbiamo cioè dimostrare che:
  $$ lub( \{ (cond(b,f,g)) | f \in X \} ) = cond(b, lub \; X, g) $$
  Sia $s \in \Sigma$ arbitrario. Sia:
  $$ h(s) = cond(b,lub \; X, g)(s) $$
  Scartiamo la possibilità di non terminazione. Allora abbiamo due casi:
  \begin{enumerate}
  \item $b(s) = false \Rightarrow h(s) = g(s) $ \\
    $g(s) = cond(b,lub \; X, g)(s) $, per definizione di cond.
  \item $b(s)=true$ \\
    Abbiamo due ulteriori casi:
    \begin{itemize}
    \item $b(s) = true \land \forall f \in X \; : \; f(s)
      \uparrow \; \Rightarrow lub \; X \uparrow$ \\
      Cioè se non termina da un lato, non termina neanche dall'altro.
    \item $b(s)=true \land f \in X : f(s) \downarrow$
      Allora:
      $$ lub \; X(s) = f(s) $$
      Come ci si aspettava.
    \end{itemize}
  \end{enumerate}
  Quindi \emph{cond} è effettivamente continuo sul secondo argomento.\\
  Ci rimane da far vedere che se \emph{X} è una catena e compongo tutti gli elementi della catena con uno \emph{ST} allora ottengo sempre una catena, cioè:
  $$ \forall g \in ST : \forall X \subseteq ST : chain(X) \; \Rightarrow chain(\{ f \circ g | f \in X \} ) $$
  In particolare se $f = \perp$ l'insieme è un singoletto. \\
  Dimostriamo quanto scritto ante:
  \begin{itemize}
  \item L'insieme è enumerabile (banalmente vero).
  \item Per dimostrare che è totalmente ordinato devo far vedere che:
    $$ \forall f_1 , f_2 \in X : f \circ g \sqsubseteq f_2 \lor f_2 \circ g \sqsubseteq f_1 $$
    Facciamo vedere che vale una delle due, per esempio la prima:
    $$ f_1 \sqsubseteq s_2 \Rightarrow f_1 \circ g \sqsubseteq f_2 \circ g $$
    Applicando la definizione di $ \sqsubseteq $ si ha:
    $$ \forall s \in \Sigma : f_{1}(s) \downarrow \; \Rightarrow f_{2}(s) \downarrow
    \land f{1}(s)=f_{2}(s) \; \; (*) $$
    Questo è dato. Voglio dimostrare che:
    $$ \forall s \in \Sigma : f_1 \circ g(s) \downarrow \; \Rightarrow f_2 \circ g(s)
    \downarrow \land f_2 \circ g(s) = f_1 \circ g(s) $$
    Che sarebbe:
    $$ \forall s \in \Sigma : f_{1}(g(s)) \downarrow \; \Rightarrow f_{2}(g(s))
    \downarrow \land f_{1}(g(s)) = f_{2}(g(s)) $$
    Che è vera per la (*) ed è proprio quello che ci serve. \\
    L'altro caso è assolutamente duale.
  \end{itemize}
  Abbiamo dimostrato allora la continuità della funzione:
  $$ \forall b \in PT \forall g,h \in ST \; \lambda f \; . \; cond(b,f \circ h,g) $$
  Questa è la \emph{F} che risulta quindi continua. Per il \emph{Teorema di Tarski} possiede un minimo punto fisso che si trova facendo le iterate successive a partire da $\perp$ . Allora possiamo concludere che:
  $$ \emph{C} \llbracket while \; B \; do \; C \rrbracket = lfp \; F $$
  E:
  $$ lfp \; F = lub \{ F^n(\perp ) | n \in \mathbb{N} \} $$
\end{proof}

\section{Logica di Floyd-Hoare}
\newcommand{\mactext}[1]{\text{\texttt{#1}}}
% Comando per scrivere con texttt dentro una formula matematica

\subsection{Esercizio 1}
Dare una dimostrazione per la seguente tripla di Hoare:
$$ \{X=a \; \land Y=b\} \; X=X+Y; \; Y=X-Y; \; X=X-Y \; \{Y=a \; \land \; X=b\} $$

\begin{proof}[Svolgimento]
Dimostriamo la tripla partendo dal fondo. Applicando l'assioma dell'assegnamento si ha:
$$ \{S[\nicefrac{X-Y}{X}] \} \; X=X-Y \; \{Y=a \; \land \; X=b\} $$
$$ \Downarrow $$
$$ \{Y=a \; \land X-Y=b \} \; X=X-Y \; \{Y=a \; \land \; X=b\} $$
Risaliamo ora di un "livello" applicando ancora l'assioma dell'assegnamento:
$$ \{S[\nicefrac{X-Y}{Y}] \} \; Y=X-Y \; \{Y=a \; \land \; X-Y=b\} $$
$$ \Downarrow $$
$$ \{X-Y=a \; \land X-(X-Y)=b \} \; Y=X-Y \; \{Y=a \; \land \; X-Y=b\} $$
Semplificando la precondizione si ottiene:
$$ \{X-Y=a \; \land Y=b \} \; Y=X-Y \; \{Y=a \; \land \; X-Y=b\} $$
Risaliamo ulteriormente di un livello, applicando l'assioma dell'assegnamento. Si ha:
$$ \{S[\nicefrac{X+Y}{Y}] \} \; X=X+Y \; \{X-Y=a \; \land \; Y=b\} $$
$$ \Downarrow $$
$$ \{(X+Y)-Y=a \; \land Y=b \} \; X=X+Y \; \{X-Y=a \; \land \; Y=b\} $$
Semplificando ancora una volta la precondizione si ottiene:
$$ \{X=a \; \land Y=b \} \; X=X+Y \; \{X-Y=a \; \land \; Y=b\} $$
Che è quanto volevamo dimostrare.
\end{proof}

\subsection{Esercizio 2}
Dare una dimostrazione per la seguente tripla di Hoare:
$$ \{X=R+(Y \times Q) \} \; BEGIN \; R=R-Y; \; Q=Q+1 \; END \; \{X=R+(Y \times Q)\} $$

\begin{proof}[Svolgimento]
Come nell'esercizio precedente, procediamo dal fondo.\\
Applichiamo quindi l'assioma dell'assegnamento:
$$ \{S[\nicefrac{Q+1}{Q}] \} \; Q=Q+1 \; \{X=R+(Y \times Q)\} $$
$$ \Downarrow $$
$$ \{X=R+(Y \times (Q+1)) \} \; Q=Q+1 \; \{X=R+(Y \times Q)\} $$
Semplificando la precondizione si ottiene:
$$ \{X=R+Y \times Q + Y \} \; Q=Q+1 \; \{X=R+(Y \times Q)\} $$
Risalendo di un livello e applicando ancora l'assioma dell'assegnamento si trova:
$$ \{S[\nicefrac{R-Y}{R}] \} \; R=R-Y \; \{X=R+Y \times Q + Y \} $$
$$ \Downarrow $$
$$ \{X= R-Y+Y \times Q+Y \} \; R=R-Y \; \{X=R+Y \times Q + Y \} $$
Semplificando la precondizione si ottiene:
$$ \{X= R+(Y \times Q) \} \; R=R-Y \; \{X=R+Y \times Q + Y \} $$
Che è quanto volevamo dimostrare.
\end{proof}

\subsection{Esercizio 3 \emph{(Esonero del 28/11/2014)}}
Lo xor bit a bit, $ \oplus: \mathbb{N} \times \mathbb{N} \rightarrow \mathbb{N} $, gode delle
seguenti proprietà:
\begin{itemize}
	\item $ x \oplus (y \oplus z) = (x \oplus y) \oplus z $ \emph{(associatività)}
	\item $ x \oplus y = y \oplus x $ \emph{(commutatività)}
	\item $ x \oplus 0 = x $ \emph{(elemento neutro)}
	\item $ x \oplus x = 0 $ \emph{(nipotenza/ogni elemento è l'inverso di se stesso)}
\end{itemize}
Si dimostri, usando la logica di Floyd-Hoare, che la composizione sequenziale:
\begin{center}
\texttt{X := X XOR Y; Y := X XOR Y; X := X XOR Y}
\end{center}
realizza lo scambio dei valori tra le variabili \texttt{X} e \texttt{Y} se queste, all'inizio della computazione, contengono valori naturali.

\begin{proof}[Svolgimento]
Dobbiamo dimostrare che la seguente tripla di Hoare è corretta:

\begin{center}
$ \{ \mactext{X=a} \; \land \; \mactext{Y=b} \} $
\texttt{X := X XOR Y; Y := X XOR Y; X := X XOR Y}
$\{\mactext{X=b} \; \land \; \mactext{Y=a} \} $
\end{center}
Procediamo, al solito, partendo dal fondo. Applichiamo dunque l'assioma dell'assegnamento:

\begin{center}
$ \{ S[\nicefrac{\mactext{X XOR Y}}{\mactext{X}}] \} $
\texttt{X := X XOR Y}
$\{\mactext{X=b} \; \land \; \mactext{Y=a} \} $
\end{center}
$$ \Downarrow $$
\begin{center}
$ \{\mactext{X XOR Y = b} \; \land \mactext{Y=a}  \} $
\texttt{X := X XOR Y}
$\{\mactext{X=b} \; \land \; \mactext{Y=a} \} $
\end{center}

Passiamo al successivo assegnamento:
\begin{center}
$ \{ S[\nicefrac{\mactext{X XOR Y}}{\mactext{Y}} ] \} $
\texttt{Y := X XOR Y}
$ \{\mactext{X XOR Y = b} \; \land \mactext{Y=a} \} $
\end{center}
$$ \Downarrow $$
\begin{center}
$ \{ \mactext{X XOR (X XOR Y) = b} \; \land \; \mactext{X XOR Y = a} \} $
\texttt{Y := X XOR Y}
$ \{ \mactext{X XOR Y = b} \; \land \; \mactext{Y=a} \} $
\end{center}
Applichiamo la proprietà di associatività dello XOR in precondizione:
\begin{center}
$ \{ \mactext{(X XOR X) XOR Y = b} \; \land \; \mactext{X XOR Y = a} \} $
\texttt{Y := X XOR Y}
$ \{ \mactext{X XOR Y = b} \; \land \; \mactext{Y=a} \} $
\end{center}
Applichiamo la proprietà di nipotenza dello XOR in precondizione:
\begin{center}
$ \{ \mactext{0 XOR Y = b} \; \land \; \mactext{X XOR Y = a} \} $
\texttt{Y := X XOR Y}
$ \{ \mactext{X XOR Y = b} \; \land \; \mactext{Y=a} \} $
\end{center}
Applichiamo la proprietà di elemento neutro dello XOR in precondizione:
\begin{center}
$ \{ \mactext{Y=b} \; \land \; \mactext{X XOR Y = a} \} $
\texttt{Y := X XOR Y}
$ \{ \mactext{X XOR Y = b} \; \land \; \mactext{Y=a} \} $
\end{center}

Passiamo ora al successivo, e ultimo, assegnamento della tripla:
\begin{center}
$ \{ S[\nicefrac{\mactext{X XOR Y}}{\mactext{X}} ] \} $
\texttt{X := X XOR Y}
$ \{ \mactext{Y=b} \; \land \; \mactext{X XOR Y = a} \} $
\end{center}
$$ \Downarrow $$
\begin{center}
$ \{ \mactext{Y=b} \; \land \; \mactext{(X XOR Y) XOR Y = a} \} $
\texttt{X := X XOR Y}
$ \{ \mactext{Y=b} \; \land \; \mactext{X XOR Y = a} \} $
\end{center}
Applichiamo la proprietà di associatività dello XOR in precondizione:
\begin{center}
$ \{ \mactext{Y=b} \; \land \; \mactext{X XOR (Y XOR Y) = a} \} $
\texttt{X := X XOR Y}
$ \{ \mactext{Y=b} \; \land \; \mactext{X XOR Y = a} \} $
\end{center}
Applichiamo la proprietà di nipotenza dello XOR in precondizione:
\begin{center}
$ \{ \mactext{Y=b} \; \land \; \mactext{X XOR 0 = a} \} $
\texttt{X := X XOR Y}
$ \{ \mactext{Y=b} \; \land \; \mactext{X XOR Y = a} \} $
\end{center}
Applichiamo la proprietà di elemento neutro dello XOR in precondizione:
\begin{center}
$ \{ \mactext{Y=b} \; \land \; \mactext{X=a} \} $
\texttt{X := X XOR Y}
$ \{ \mactext{Y=b} \; \land \; \mactext{X XOR Y = a} \} $
\end{center}
Otteniamo dunque quello che volevamo dimostrare.
\end{proof}


\subsection{Esercizio 4}

Dimostrare la correttezza della seguente tripla di Hoare:
\begin{lstlisting}[mathescape, numberfirstline=false, frame=single]
$ [x = n] $
if x < 0 then
	y := 0;
else
	y := 1;
	while x > 0 do $ \{y = 2^{n-x} \land x \geq 0 \} $
		y = y*2;
		x = x-1;
$ [y = \lfloor 2^n \rfloor ] $
\end{lstlisting}

\begin{proof}[Svolgimento]
Il codice è completamente annotato. Dimostriamo in primo luogo la correttezza parziale della tripla. Applicando l'assioma dell'if allora otteniamo le due triple:
\begin{align}
& \{x=n \; \land \; x<0 \} \; \mactext{y:=0} \; \{y= \lfloor 2^n \rfloor \} \\
& \{x=n \; \land \; x \geq 0 \} \; \mactext{y:=1; while} \texttt{\dots } \; \{ y= \lfloor 2^n \rfloor \}
\end{align}
Dalla (1) ricaviamo la prima VC:
$$ x = n \; \land \; x<0 \; \Rightarrow \; 0 = \lfloor 2^n \rfloor $$
Ed è banale argomentarne la veridicità.\
Dalla (2), dal momento che la postcondizione dopo l'assegnamento \texttt{y:=1} è l'invariante del While, applicando l'assioma derivato dell'assegnamento si ottiene la seconda VC:
$$ x=n \; \land \; x \geq 0 \; \Rightarrow \; 1 = 2^{n-x} \; \land \; x \geq 0 $$
Ed anche in questo caso è banale argomentare la veridicità di questa implicazione.\
Ci rimane da dimostrare la tripla di Hoare:
$$ \{ y=2^{n-x} \; \land \; x>0 \} \; \mactext{while} \texttt{\dots } \; \{y= \lfloor 2^n \rfloor \} $$
Applicando la regola derivata del While si ottengono le seguenti VCs:
\begin{itemize}
	\item $ y=2^{n-x} \; \land \; x \geq 0 \; \Rightarrow \; y=2^{n-x} \; \land \; x \geq 0 $ \\
		  che è banalmente vera.
	\item $ y=2^{n-x} \; \land \; x \geq 0 \; \land \; x \leq 0 \; \Rightarrow \; y = \lfloor 2^n \rfloor $ \\
		  ed è vera in quanto $ x \geq 0 \; \land \; x \leq 0 \; \Rightarrow \; x = 0 $ e quindi
		  dall'antecedente si ottiene il conseguente.
	\item Tutte le VCs generate dalla tripla:
		  $$ \{y=2^{n-x} \; \land \; x \geq 0 \; \land \; x>0 \} \;
		     \mactext{y:=y*2; x=x-1}
		     \; \{y=2^{n-x} \; \land \; x \geq 0 \} $$
		  Applicando la regola dell'assegnamento in sequenza e la regola derivata dell'assegnamento ricaviamo
		  l'ultima VC:
		  $$ y=2^{n-x} \; \land \;  x \geq 0 \; \land \; x>0 \;
		  	 \Rightarrow
		  	 \; y*2 = 2^{n-x+1} \; \land x-1 \geq 0 $$
		  che si può semplificare in:
		  $$ y=2^{n-x} \; \land \; x>0 \;
		  	 \Rightarrow
		  	 \; y = 2^{n-x} \; \land x-1 \geq 0 $$
		  che ancora una volta è vera.
\end{itemize}
Quindi la correttezza parziale della tripla è dimostrata. Ci rimane da dimostrare la totalità e per farlo dobbiamo provare che valgono le seguenti:
\begin{align*}
& P \; \land \; S \; \Rightarrow \; V \geq 0 \\
& \{P \; \land \; S \; \land \; V=k \} \; C \; \{ P \; \land \; V <k \}
\end{align*}
Si ha:
\begin{itemize}
\item $ y=2^{n-x} \; \land \; x \geq 0 \; \land \; x>0 \; \Rightarrow \; x \geq 0 $ \\
	  che è vera, dato che $ x>0 \; \Rightarrow \; x \geq 0 $.
\item Dobbiamo dimostrare la veridicità della tripla:
		$$ \{y=2^{n-x} \; \land \; x \geq 0 \; \land \; x>0 \; \land x=k \} \;
		\mactext{y:=y*2; x=x-1}
		\; \{y=2^{n-x} \; \land \; x < k \} $$
	  Applicando ancora la regola dell'assegnamento in sequenza e l'assioma derivato dell'assegnamento si ha:
	  $$ y = 2^{n-x} \; \land \; x \geq 0 \; \land \; x=k \;
	     \Rightarrow
	     \; y*2 = 2^{n-x+1} \; \land \; x-1<k $$
	  che si può semplificare in:
	  $$ y = 2^{n-x} \; \land \; x \geq 0 \; \land \; x=k \;
	     \Rightarrow
	     \; y = 2^{n-x} \; \land \; x-1<k $$
	  che è vera, dal momento che si otterrebbe $ k-1<k $.
\end{itemize}

Questo prova la correttezza totale.
\end{proof}

\subsection{Esercizio 5}
Estendere la regola del blocco vista a lezione, svincolandola dalla condizione secondo la quale nessuna variabile che compare all'interno del blocco possa comparire in precondizione o in postcondizione.

\begin{proof}[Svolgimento]
Possiamo, a livello di preprocessore, rinominare tutte le variabili di blocco con un nome univoco $ y_1,\dots ,y_n $, facendo attenzione al fatto che abbiano un identificatore univoco. La regola allora è riscrivibile come:
$$
\prooftree
	\vdash \; \{ P \} \; C[\nicefrac{y_1}{v_1}]\dots [\nicefrac{y_n}{v_n}] \; \{ Q \}
	\justifies
		\{ P \} \; \mactext{BEGIN Var } v_1; \dots v_n; \; C \; \mactext{END} \; \{ Q \}
	\thickness=0.08em
	\using
		(Vars(P) \cup Vars(Q)) \cap \{ y_1,\dots ,y_n \} = \emptyset
\endprooftree
$$
\end{proof}


\section{Semantica operazionale, denotazionale, assiomatica}

\subsection{Esercizio 1}
Aggiungere al linguaggio \emph{While} visto a lezione l'espressione:
$$ Exp \ni E ::= \dots \; | \; B \; ? \; E_2 \; : \; E_3 $$
definendone la semantica operazionale Big Step, Small Step e denotazionale.

\begin{proof}[Svolgimento]
Di seguito illustriamo le regole e gli assiomi in semantica Small Step:
$$ \prooftree
		\langle B,s \rangle \; \rightarrow \; \langle B', s' \rangle
      	\justifies
      		\langle B \; ? \; E_2 \; : E_3, s \rangle \; \rightarrow \; \langle B' \; ? \; E_2 \; : E_3, s' \rangle
	\thickness=0.08em
	\endprooftree
$$

$$ \prooftree
      	\justifies
      		\langle true \; ? \; E_2 \; : E_3, s \rangle \; \rightarrow \; \langle E_2, s' \rangle
	\thickness=0.08em
	\endprooftree
$$

$$ \prooftree
      	\justifies
      		\langle false \; ? \; E_2 \; : E_3, s \rangle \; \rightarrow \; \langle E_3, s' \rangle
	\thickness=0.08em
	\endprooftree
$$

Di seguito illustriamo le regole e gli assiomi in semantica Big Step:
$$ \prooftree
		\langle B,s \rangle \Downarrow \langle true, s' \rangle \; \;
		\langle E_2,s' \rangle \Downarrow \langle n, s'' \rangle
      	\justifies
      		\langle B \; ? \; E_2 \; : E_3, s \rangle \Downarrow \langle n, s'' \rangle
	\thickness=0.08em
	\endprooftree
$$

$$ \prooftree
		\langle B,s \rangle \Downarrow \langle false, s' \rangle \; \;
		\langle E_3,s' \rangle \Downarrow \langle n, s'' \rangle
      	\justifies
      		\langle B \; ? \; E_2 \; : E_3, s \rangle \Downarrow \langle n, s'' \rangle
	\thickness=0.08em
	\endprooftree
$$

Di seguito illustriamo la definizione dell'espressione in semantica denotazionale:
$$
\llbracket B \; ? \; E_2 \; : \; E_3 \rrbracket (s) \; =
\begin{cases}
	\varepsilon \llbracket E_2 \rrbracket (s) & se \; \emph{B} \llbracket B \rrbracket (s) = tt \\
	\varepsilon \llbracket E_3 \rrbracket (s) & se \; \emph{B} \llbracket B \rrbracket (s) = ff \\
	\bot & altrimenti
\end{cases}
$$
\end{proof}
