\chapter{Relazioni tra le semantiche}
In questo capitolo verranno introdotti alcuni esercizi e alcune osservazioni
che non riguardano una specifica semantica vista nei capitoli precedenti,
ma le affrontano un po' tutte.

\subsection{Conservazione passi small step rispetto a denotazionale}

\begin{proposizione}[Conservazione passi small step per espressioni]
\label{conservazione-passi-small-step-expr}
\[
  \forall E \in \AExp \itc \config{E}{s} \ssarrow \config{E'}{s}
  \implies \calE\llbracket E \rrbracket(s) = \calE\llbracket E' \rrbracket(s)
\]

\begin{proof}
Per induzione strutturale sulla costruzione di $\AExp$ (consideriamo
espressioni senza ``side effects'' nella semantica small step poichè sono
definite in talo modo anche nella denotazionale):

Caso base: $E = x$.

Per ipotesi abbiamo che:
\[
  \config{x}{s} \ssarrow \config{n}{s} \land s(x) = n,
\]
dunque:
\begin{align*}
  \calE\llbracket x \rrbracket(s) &= s(x),
    &\law{per def. della denotazionale} \\
  &= n,
    &\law{per ipotesi} \\
  &= \calE\llbracket n \rrbracket(s).
\end{align*}

Caso base: $E = n_1 + n_2$.

Per ipotesi abbiamo che:
\[
  \config{n_1 + n_2}{s} \ssarrow \config{n_3}{s} \land n_1 + n_2 = n_3,
\]
dunque:
\begin{align*}
  \calE\llbracket n_1 + n_2 \rrbracket(s)
  &= \calE\llbracket n_1 \rrbracket(s)
  + \calE\llbracket n_2 \rrbracket(s),
    &\law{per def. della denotazionale} \\
  &= n_1 + n_2,
    &\law{per def. della denotazionale} \\
  &= n_3,
    &\law{per ipotesi} \\
  &= \calE\llbracket n_3 \rrbracket(s).
\end{align*}

Passo induttivo: $E = E_1 + E_2$.

Per ipotesi abbiamo che:
\[
  \config{E_1 + E_2}{s} \ssarrow \config{E_1' + E_2}{s} \land
  \config{E_1}{s} \ssarrow \config{E_1'}{s}
\]
dunque:
\begin{align*}
  \calE\llbracket E_1 + E_2 \rrbracket(s)
  &= \calE\llbracket E_1 \rrbracket(s) +  \calE\llbracket E_2
    \rrbracket(s),
    &\law{per def. della denotazionale} \\
  &= \calE\llbracket E_1' \rrbracket(s) +  \calE\llbracket E_2
    \rrbracket(s),
    &\law{per ipotesi induttiva} \\
  &= \calE\llbracket E_1' + E_2 \rrbracket(s).
\end{align*}
\end{proof}
\end{proposizione}

\begin{proposizione}[Conservazione passi small step per comandi]
\[
  \forall C \in \Com \itc \config{C}{s} \ssarrow \config{C'}{s'}
  \implies \calC\llbracket C \rrbracket(s) = \calC\llbracket C' \rrbracket(s')
\]
\begin{proof}
Per induzione sulla struttura di $\Com$ (come per il lemma precedente
la semantica small step delle espressioni data è senza ``side effects'':

Caso base: $C = (x \weq n)$.

Per definizione della semantica small step:
\[
  \config{x \weq n}{s} \ssarrow \config{\kw{skip}}{\assign{s}{n}{x}},
\]
dunque:
\begin{align*}
  \calC\llbracket x \weq n \rrbracket(s)
  &= \assign{s}{n}{x},
    &\law{per def. della denotazionale} \\
  &= \id(\assign{s}{n}{x}),
    &\law{per def. di $\id$} \\
  &= \calC\llbracket \kw{skip} \rrbracket(\assign{s}{n}{x}).
     &\law{per def. della denotazionale} \\
\end{align*}

Caso base: $C = (x \weq E)$.

Per definizione della semantica small step:
\begin{align*}
  &\config{x \weq E}{s} \ssarrow \config{x \weq E'}{s}, \\
  &\config{E}{s} \ssarrow \config{E'}{s},
\end{align*}
dunque:
\begin{align*}
  \calC\llbracket x \weq E \rrbracket(s)
  &= \assign{s}{\calE\llbracket E \rrbracket(s)}{x},
    &\law{per def. della denotazionale} \\
  &= \assign{s}{\calE\llbracket E' \rrbracket(s)}{x},
    &\law{per il lemma
      \ref{conservazione-passi-small-step-expr}:
      $\calE\llbracket E \rrbracket(s)
      = \calE\llbracket E' \rrbracket(s)$} \\
  &= \calC\llbracket x \weq E' \rrbracket(s).
    &\law{per def. della denotazionale}
\end{align*}

Caso base: $C = (\kw{if} \ttv \kw{then} C_1 \kw{else} C_2)$.

Per definizione della semantica small step:
\[
  \config{\kw{if} \ttv \kw{then} C_1 \kw{else} C_2}{s} \ssarrow \config{C_1}{s}
\]
dunque:
\begin{align*}
  \calC\llbracket \kw{if} \ttv \kw{then} C_1 \kw{else} C_2 \rrbracket(s)
  &= \calC\llbracket C_1 \rrbracket(s).
    &\law{per def. della denotazionale} \\
\end{align*}
Analogo il caso per $\false$.

Caso base: $C = (\kw{if} B \kw{then} C_1 \kw{else} C_2)$.

Per definizione della semantica small step:
\begin{align*}
  &\config{\kw{if} B \kw{then} C_1 \kw{else} C_2}{s} \ssarrow
  \config{\kw{if} B' \kw{then} C_1 \kw{else} C_2}{s}, \\
  &\config{B}{s} \ssarrow \config{B'}{s},
\end{align*}
dunque:
\begin{align*}
  \calC\llbracket \kw{if} B \kw{then} C_1 \kw{else} C_2 \rrbracket(s)
  &= \cond(\calB\llbracket B \rrbracket,
    \calC\llbracket C_1 \rrbracket,
    \calC\llbracket C_2 \rrbracket),
    &\law{def. denotazionale} \\
  &= \cond(\calB\llbracket B' \rrbracket,
    \calC\llbracket C_1 \rrbracket,
    \calC\llbracket C_2 \rrbracket),
    &\law{lemma
      \ref{conservazione-passi-small-step-expr}:
      $\calE\llbracket E \rrbracket(s)
      = \calE\llbracket E' \rrbracket(s)$} \\
  &= \calC\llbracket \kw{if} B' \kw{then} C_1 \kw{else} C_2 \rrbracket(s).
     &\law{def. denotazionale}
\end{align*}

Caso base:  $C = (\kw{while} B \kw{do} C)$.

Per definizione della semantica small step:
\[
  \config{\kw{while} B \kw{do} C}{s} \ssarrow
    \config{\kw{if} B \kw{then} (C ; \kw{while} B \kw{do} C) \kw{else} C_2}{s},
\]
dunque per definizione di $\kw{while} B \kw{do} C$ nella denotazionale:
\[
  \calC\llbracket \kw{while} B \kw{do} C \rrbracket(s)
    =  \calC\llbracket \kw{if} B \kw{then} (C ; \kw{while} B \kw{do} C)
    \kw{else} C_2 \rrbracket(s).
\]


Passo induttivo: $C = (C_1; C_2)$.

Per definizione della semantica small step:
\begin{align*}
  &\config{C_1; C_2}{s} \ssarrow \config{C_1'; C_2}{s'}, \\
  &\config{C_1}{s} \ssarrow \config{C_1'}{s'},
\end{align*}
dunque:
\begin{align*}
  \calC\llbracket C_1; C_2 \rrbracket(s)
  &= \compfun{\calC\llbracket C_2 \rrbracket(s)}{\calC\llbracket C_1
    \rrbracket(s)}, \\
  &= \compfun{\calC\llbracket C_2 \rrbracket(s)}{\calC\llbracket C_1'
    \rrbracket(s')},
    &\law{per ipotesi induttiva} \\
  &= \calC\llbracket C_1'; C_2 \rrbracket(s').
\end{align*}
\end{proof}
\end{proposizione}

\subsection{Esercizio 1}
Aggiungere al linguaggio WHILE visto a lezione l'espressione:
\[ Exp \ni E ::= \dots \vbar \textrm{$B$ ? $E_2$ : $E_3$} \]
definendone la semantica operazionale Big Step, Small Step e denotazionale.

\begin{proof}[Svolgimento]
Di seguito illustriamo le regole e gli assiomi in semantica Small Step:
\[
\prooftree
  \langle B,s \rangle \textrm{ $\rightarrow$ } \langle B', s' \rangle
  \justifies
    \langle \textrm{$B$ ? $E_2$ : $E_3$, $s$ $\rangle$
    $\rightarrow$ $\langle$ $B'$ ? $E_2$ : $E_3$, $s'$ $\rangle$}
  \thickness=0.08em
\endprooftree
\]

\[
\prooftree
  \justifies
    \textrm{ $\langle$ tt ? $E_2$ : $E_3$, $s$
    $\rangle$ $\rightarrow$ $\langle$ $E_2$, $s'$ $\rangle$}
  \thickness=0.08em
\endprooftree
\]

\[
\prooftree
  \justifies
    \textrm{$\langle$ ff ? $E_2$ : $E_3$, $s$
    $\rangle$ $\rightarrow$ $\langle$ $E_3$, $s'$ $\rangle$}
  \thickness=0.08em
\endprooftree
\]

Di seguito illustriamo le regole e gli assiomi in semantica Big Step:
\[
\prooftree
  \langle B,s \rangle \Downarrow \langle \textrm{tt}, s' \rangle \quad
  \langle E_2,s' \rangle \Downarrow \langle n, s'' \rangle
  \justifies
     \langle \textrm{$B$ ? $E_2$ : $E_3$, $s$ $\rangle$
     $\Downarrow \langle n$, $s'' \rangle$}
  \thickness=0.08em
\endprooftree
\]

\[
\prooftree
  \langle B,s \rangle \Downarrow \langle \textrm{ff}, s' \rangle \quad
  \langle E_3,s' \rangle \Downarrow \langle n, s'' \rangle
  \justifies
     \langle \textrm{$B$ ? $E_2$ : $E_3$, $s \rangle \Downarrow \langle n, s'' \rangle$}
  \thickness=0.08em
\endprooftree
\]

Di seguito illustriamo la definizione dell'espressione in semantica denotazionale:
\[
\llbracket \textrm{$B$ ? $E_2$ : $E_3 \rrbracket (s) $} =
\begin{cases}
    \varepsilon \llbracket E_2 \rrbracket (s)
    & \textrm{se } \emph{B} \llbracket B \rrbracket (s) = \textrm{tt} \\
    \varepsilon \llbracket E_3 \rrbracket (s)
    & \textrm{se } \emph{B} \llbracket B \rrbracket (s) = \textrm{ff} \\
    \bot & \textrm{altrimenti}
\end{cases}
\]
\end{proof}
