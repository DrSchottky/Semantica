\chapter{Relazioni tra le semantiche}

\subsection{Esercizio 1} \marginpar{Mancino}
Aggiungere al linguaggio \emph{While} visto a lezione l'espressione:
$$ Exp \ni E ::= \dots \; | \; B \; ? \; E_2 \; : \; E_3 $$
definendone la semantica operazionale Big Step, Small Step e denotazionale.

\begin{proof}[Svolgimento]
Di seguito illustriamo le regole e gli assiomi in semantica Small Step:
$$ \prooftree
		\langle B,s \rangle \; \rightarrow \; \langle B', s' \rangle
      	\justifies
      		\langle B \; ? \; E_2 \; : E_3, s \rangle \; \rightarrow \; \langle B' \; ? \; E_2 \; : E_3, s' \rangle
	\thickness=0.08em
	\endprooftree
$$

$$ \prooftree
      	\justifies
      		\langle true \; ? \; E_2 \; : E_3, s \rangle \; \rightarrow \; \langle E_2, s' \rangle
	\thickness=0.08em
	\endprooftree
$$

$$ \prooftree
      	\justifies
      		\langle false \; ? \; E_2 \; : E_3, s \rangle \; \rightarrow \; \langle E_3, s' \rangle
	\thickness=0.08em
	\endprooftree
$$

Di seguito illustriamo le regole e gli assiomi in semantica Big Step:
$$ \prooftree
		\langle B,s \rangle \Downarrow \langle true, s' \rangle \; \;
		\langle E_2,s' \rangle \Downarrow \langle n, s'' \rangle
      	\justifies
      		\langle B \; ? \; E_2 \; : E_3, s \rangle \Downarrow \langle n, s'' \rangle
	\thickness=0.08em
	\endprooftree
$$

$$ \prooftree
		\langle B,s \rangle \Downarrow \langle false, s' \rangle \; \;
		\langle E_3,s' \rangle \Downarrow \langle n, s'' \rangle
      	\justifies
      		\langle B \; ? \; E_2 \; : E_3, s \rangle \Downarrow \langle n, s'' \rangle
	\thickness=0.08em
	\endprooftree
$$

Di seguito illustriamo la definizione dell'espressione in semantica denotazionale:
$$
\llbracket B \; ? \; E_2 \; : \; E_3 \rrbracket (s) \; =
\begin{cases}
	\varepsilon \llbracket E_2 \rrbracket (s) & se \; \emph{B} \llbracket B \rrbracket (s) = tt \\
	\varepsilon \llbracket E_3 \rrbracket (s) & se \; \emph{B} \llbracket B \rrbracket (s) = ff \\
	\bot & altrimenti
\end{cases}
$$
\end{proof}
