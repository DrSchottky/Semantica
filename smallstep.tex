\chapter{Semantica operazionale ``small step''}
\marginpar{Marano}
In questo capitolo parleremo della semantica operazionale "small step", una semantica che descrive formalmente dopo ogni piccolo passo di calcolo che comando
viene eseguito e come viene modificato lo store. Per fare ciò sfrutta assiomi e regole che hanno la stessa forma di quelle viste nel capitolo precedente

$$
\prooftree
premessa_1
\cdots
premessa_k
\justifies
conclusioni
\using
(nome \; regola)
\endprooftree
$$

\section{Introduzione}\marginpar{Marano}
Se quindi abbiamo un linguaggio con la seguente sintassi
$$ Exp \ni E ::= n \bigm| (E+E) \bigm| \cdots $$
dove $n$ si espande nei naturali $1, 2, \dots$ (nei puntini di sospensione si può inserire qualsiasi espressione).
Allora una possibile semantica small step è:

\begin{align*}
  \prooftree
  \justifies
  n \rightarrow n
  \using
  (S-NUM)
  \endprooftree
  &&
  \prooftree
  E_1 \rightarrow E_1'
  \justifies
  (E_1 + E_2) \rightarrow (E_1' + E_2)
  \using
  (S-LEFT)
  \endprooftree
\\ \\
  \prooftree
  E_2 \rightarrow E_2'
  \justifies
  (E_1 + E_2) \rightarrow (E_1 + E_2')
  \using
  (S-RIGHT)
  \endprooftree
  &&
  \prooftree
  \justifies
  (1 + 2) \rightarrow 3
  \using
  (S-ADD)
  \endprooftree
\end{align*}

L'operatore $\rightarrow$ quindi rappresenta a tutti gli effetti un passo di computazione.
Con questa semantica abbiamo lo stesso potere espressivo della semantica Big step ma,
essendo più verbosa, da più informazione sulla sequenzialità e sui "risultati intermedi".
Vediamo un esempio di dimostrazione:

\begin{align*}
\prooftree
	\prooftree
   		\justifies
   			(1+2) \rightarrow 3
   		\using
   			(S-ADD)
	\endprooftree
\justifies
       	((1+2)+(3+4)) \rightarrow (3+(3+4))
\using
	(S-ADD)
\endprooftree
\\ \\
\prooftree
	\prooftree
   		\justifies
   			(3+4) \rightarrow 7
   		\using
   			(S-ADD)
	\endprooftree
\justifies
       	(3+(3+4)) \rightarrow (3+7)
\using
	(S-ADD)
\endprooftree
\\ \\
\prooftree
	\justifies
		(3+7) \rightarrow 10
	\using
		(S-ADD)
\endprooftree
\end{align*}

Possiamo subito osservare che differentemente dalla semantica Big Step
nella semantica Small Step si usano tanti alberi di dimostrazione più piccoli per dimostrare
ogni asserzione. 

\section{Determinatezza della semantica Small Step} \marginpar{Marano}
Dimostriamo ora che la semantica Small Step gode della proprietà
della determinatezza, ogni oggetto sintattico può avere perciò
al massimo una semantica. Prima dimostriamo però il seguente

\begin{teorema}
$\forall E_1, E_2, E \in Exp . E \rightarrow _1 E_1 \land E \rightarrow _1 E_2 \Rightarrow E_1 = E_2 $
\end{teorema}

\begin{proof}
  Dimostriamo per induzione strutturale
  \begin{itemize}
  \item Caso Base ($E \in Num$) - $E = n \Rightarrow E \nrightarrow$, ovvero E non transisce.
  \item Passo induttivo ($E \in Exp$) - $E = E_1 + E_2$ \\
    perciò $E_1 + E_2 \rightarrow \begin{cases} E_1 \oplus E_2, & \mbox{se }E_1, E_2 \in \mbox{ Num} \\
    E_1 + E_2', & \mbox{se }E_1 \in \mbox{ Num} \land E_2 \rightarrow E_2' \\
    E_1' + E_2, & \mbox{se }E_2 \in \mbox{ Num} \land E_1 \rightarrow E_1' \end{cases}$\\
    È ovvio che questi tre casi sono mutuamente esclusivi.
  \end{itemize}
\end{proof}

Ora possiamo dimostrare il seguente

\begin{teorema}[Determinatezza]
La semantica Small Step di Exp gode della proprietà di determinatezza, ovvero
$\forall E \in Exp, \; \forall \; n, n' \in Num \text{ tali che }\\
 E\rightarrow ^* n \; \land \; E \rightarrow ^* n' \; \Rightarrow n = n'$
\end{teorema}

\begin{proof}
  Sappiamo che $E \rightarrow E_1^{(1)} \rightarrow E_1^{(2)} \rightarrow \dots \rightarrow E_1^{(k)} = n_1$ e che 
  $E \rightarrow E_2^{(1)} \rightarrow E_2^{(2)} \rightarrow \dots \rightarrow E_2^{(h)} = n_2$.
  Ma per il lemma appena dimostrato sappiamo che $E_1^{(k)} = E_2^{(h)} \text{ con } h = k \Rightarrow n_1 = n_2$
\end{proof}

\section{Normalizzazione della semantica Small Step} \marginpar{Marano}
Altra proprietà molto importante per una semantica è la normalizzazione:
ogni oggetto semantico ha almeno una semantica. Perciò per ogni espressione $E$
esiste almeno un risultato $n$ tale che $E \rightarrow ^* n$.
Dimostriamo prima però il seguente

\begin{teorema}
  $\forall E \in Exp : E \in Num \land \exists E' \in Num \Rightarrow E \rightarrow E'$
  Questa proprietà ci aiuterà nella dimostrazione del successivo teorema ma dobbiamo anche dimostrare che le espressioni non possono transire per sempre,
  ovvero che $\forall E \in Exp : \exists k \in \mathbb{N} . E \rightarrow ^k F \land F \nrightarrow$
\end{teorema}

\begin{proof}
  Definiamo perciò una funzione $f : Exp \rightarrow \mathbb{N}$ che calcola il numero di "+", ovvero il numero di passi da fare per arrivare fino a una
  espressione che non possa più transire, f restituisce ovvero il \emph{k} definito nella tesi del teorema. Definiamo per induzione strutturale:
  \begin{itemize}
    \item Caso Base ($E \in Num$) - $E = n \in Num \Rightarrow f(n) = 0$
    \item Passo Induttivo ($E = \not \in Num$) - $\exists E_1, E_2 \in Exp\ .\ E = E_1 + E_2 \Rightarrow f(E) = f(E_1 + E_2) = f(E_1) + f(E_2) + 1$
  \end{itemize}
  Dimostriamo ora che questa funzione calcola proprio ciò che volevamo, ovviamente lo dimostriamo per induzione strutturale:
  \begin{itemize}
  \item Caso Base ($E \in Num$) - $E = n \Rightarrow E \nrightarrow \Rightarrow f(E) = k = 0$
  \item Passo Induttivo ($E = \not \in Num$) - $E = E_1 + E_2 \Rightarrow \\ E_1 + E_2 \rightarrow ^i n_1 + E_2 \rightarrow ^j n_1 + n_2 \rightarrow n \\
    \Rightarrow k = i + j + 1 = f(E_1) + f(E_2) + 1 = f(E_1 + E_2) = f(E)$
  \end{itemize}
\end{proof}

\begin{teorema}[Normalizzazione]
  $\forall E \in Exp\ \exists n \in Num\ .\ E \rightarrow ^* n$
\end{teorema}

\begin{proof}
  La proprietà deriva direttamente dai due lemmi dimostrati precedentemente
\end{proof}

\section{Semantica Small Step per While} \marginpar{Marano}
Diamo ora una semantica al linguaggio While tramite la small step, dovremo
perciò valutare non più una singola espressione ma una coppia formata da
comando e stato. Per ora tralasciamo la possibilità di effetti collaterali.
$$
\langle C,s \rangle \rightarrow \langle C',s' \rangle
$$
Indichiamo perciò con E la generica espressione, con C il generico comando 
e con s il generico stato:

\begin{align*}
(S-SKIP)
&&
\prooftree
\justifies
\langle skip,\ s \rangle \rightarrow s
\endprooftree
\\ \\
(S-EXP.NUM)
&&
\prooftree
   \justifies
   \langle x,\ s \rangle \rightarrow \langle n,\ s \rangle
   \using
   s(x) = n 
\endprooftree
\\ \\
(S-EXP.LEFT)
&&
\prooftree
	\langle E_1,\ s \rangle \rightarrow \langle E_1',\ s \rangle
   \justifies
	\langle (E_1 + E_2),\ s \rangle \rightarrow \langle (E_1' + E_2),\ s \rangle
\endprooftree
\\ \\
(S-EXP.RIGHT)
&&
\prooftree
	\langle E_2,\ s \rangle \rightarrow \langle E_2',\ s \rangle
   \justifies
	\langle (E_1 + E_2),\ s \rangle \rightarrow \langle (E_1 + E_2'),\ s \rangle
\endprooftree
\\ \\
(S-ADD)
&&
\prooftree
   \justifies
   \langle (n_1 + n_2),\ s \rangle \rightarrow \langle n_3,\ s \rangle
   \using
   n_3 = n_1 + n_2
\endprooftree
\\ \\
(S-ASS.EXP)
&&
\prooftree
	\langle E,\ s \rangle \rightarrow \langle E',\ s \rangle
   \justifies
   	\langle x:= E,\ s \rangle \rightarrow \langle x:= E',\ s \rangle
\endprooftree
\\ \\
(S-ASS.NUM)
&&
\prooftree
   \justifies
   	\langle x:= n,\ s \rangle \rightarrow \langle skip,\ s[x \rightarrow n] \rangle
\endprooftree
\\ \\
(S-SEQ.LEFT)
&&
\prooftree
	\langle C_1,\ s \rangle \rightarrow \langle C_1',\ s' \rangle
   \justifies
   	\langle C_1 ; C_2,\ s \rangle \rightarrow \langle C_1' ; C_2,\ s' \rangle
\endprooftree
\\ \\
(S-SEQ.SKIP)
&&
\prooftree
   \justifies
   	\langle skip ; C_2,\ s \rangle \rightarrow \langle C_2,\ s' \rangle
\endprooftree
\\ \\
(S-COND.BOOL)
&&
\prooftree
	\langle B,\ s \rangle \rightarrow \langle B',\ s \rangle
   \justifies
   	\langle if\ B\ then\ C_1\ else\ C_2,\ s \rangle \rightarrow \langle if\ B'\ then\ C_1\ else\ C_2,\ s \rangle
\endprooftree
\\ \\
(S-COND.TRUE)
&&
\prooftree
   \justifies
   	\langle if\ true\ then\ C_1\ else\ C_2,\ s \rangle \rightarrow \langle C_1,\ s \rangle
\endprooftree
\\ \\
(S-COND.FALSE)
&&
\prooftree
   \justifies
   	\langle if\ false\ then\ C_1\ else\ C_2,\ s \rangle \rightarrow \langle C_2,\ s \rangle
\endprooftree
\\ \\
(S-WHILE.BOOL)
&&
\prooftree
	\langle B,\ s \rangle \rightarrow \langle B',\ s \rangle
   \justifies
   	\langle while\ B\ do\ C,\ s \rangle \rightarrow \langle while\ B'\ do\ C,\ s \rangle
\endprooftree
\\ \\
(S-WHILE.FALSE)
&&
\prooftree
   \justifies
   	\langle while\ false\ do\ C,\ s \rangle \rightarrow \langle skip,\ s \rangle
\endprooftree
\\ \\
(S-WHILE)
&&
\prooftree
   \justifies
   	\langle while\ B\ do\ C,\ s \rangle \rightarrow \langle if\ B\ then\ (C ; while\ B\ do\ C)\ else\ skip,\ s \rangle
\endprooftree
\\ \\
(S-LOG.AND.BOOl)
&&
\prooftree
	\langle B_1,\ s \rangle \rightarrow \langle B_1',\ s \rangle
   \justifies
   	\langle B_1\ \land\ B_2,\ s \rangle \rightarrow \langle B_1'\ \land\ B_2,\ s \rangle
\endprooftree
\\ \\
(S-LOG.AND.TRUE)
&&
\prooftree
	\langle B_2,\ s \rangle \rightarrow \langle B_2',\ s \rangle
   \justifies
   	\langle true\ \land\ B_2,\ s \rangle \rightarrow \langle true\ \land\ B_2',\ s \rangle
\endprooftree
\\ \\
(S-LOG.AND.FALSE)
&&
\prooftree
	\langle B_2,\ s \rangle \rightarrow \langle B_2',\ s \rangle
   \justifies
   	\langle false\ \land\ B_2,\ s \rangle \rightarrow \langle false\ \land\ B_2',\ s \rangle
\endprooftree
\\ \\ 
(S-LOG.OR.BOOl)
&&
\prooftree
	\langle B_1,\ s \rangle \rightarrow \langle B_1',\ s \rangle
   \justifies
   	\langle B_1\ \lor\ B_2,\ s \rangle \rightarrow \langle B_1'\ \lor\ B_2,\ s \rangle
\endprooftree
\\ \\
(S-LOG.OR.TRUE)
&&
\prooftree
	\langle B_2,\ s \rangle \rightarrow \langle B_2',\ s \rangle
   \justifies
   	\langle true\ \lor\ B_2,\ s \rangle \rightarrow \langle true\ \lor\ B_2',\ s \rangle
\endprooftree
\\ \\
(S-LOG.OR.FALSE)
&&
\prooftree
	\langle B_2,\ s \rangle \rightarrow \langle B_2',\ s \rangle
   \justifies
   	\langle false\ \lor\ B_2,\ s \rangle \rightarrow \langle false\ \lor\ B_2',\ s \rangle
\endprooftree
\\ \\
(S-CORT.AND.BOOl)
&&
\prooftree
	\langle B_1,\ s \rangle \rightarrow \langle B_1',\ s \rangle
   \justifies
   	\langle B_1\ \& \&\ B_2,\ s \rangle \rightarrow \langle B_1'\ \& \&\ B_2,\ s \rangle
\endprooftree
\\ \\
(S-CORT.AND.TRUE)
&&
\prooftree
	\langle B_2,\ s \rangle \rightarrow \langle B_2',\ s \rangle
   \justifies
   	\langle true\ \& \&\ B_2,\ s \rangle \rightarrow \langle B_2',\ s \rangle
\endprooftree
\\ \\
(S-CORT.AND.FALSE)
&&
\prooftree
	\langle B_2,\ s \rangle \rightarrow \langle B_2',\ s \rangle
   \justifies
   	\langle false\ \& \&\ B_2,\ s \rangle \rightarrow \langle false,\ s \rangle
\endprooftree
\\ \\ 
(S-CORT.OR.BOOl)
&&
\prooftree
	\langle B_1,\ s \rangle \rightarrow \langle B_1',\ s \rangle
   \justifies
   	\langle B_1\ ||\ B_2,\ s \rangle \rightarrow \langle B_1'\ ||\ B_2,\ s \rangle
\endprooftree
\\ \\
(S-CORT.OR.TRUE)
&&
\prooftree
	\langle B_2,\ s \rangle \rightarrow \langle B_2',\ s \rangle
   \justifies
   	\langle true\ ||\ B_2,\ s \rangle \rightarrow \langle true,\ s \rangle
\endprooftree
\\ \\
(S-CORT.OR.FALSE)
&&
\prooftree
	\langle B_2,\ s \rangle \rightarrow \langle B_2',\ s \rangle
   \justifies
   	\langle false\ ||\ B_2,\ s \rangle \rightarrow \langle B_2',\ s \rangle
\endprooftree
\end{align*}

\section{Effetti collaterali}\marginpar{Marano}
Cerchiamo ora di considerare gli effetti collaterali e
osservare come viene modificata la semantica del linguaggio
While. Fino ad ora solo i comandi potevano modificare lo stato
ma adesso anche la valutazione di espressioni e di booleani ma
non l'accesso in memoria.
Abbiamo già visto nel paragrafo precedente la definizione di
\emph{effetto collaterale} e cosa comporta.
Ecco quindi la nuova semantica del linguaggio While:

\begin{align*}
(S-SKIP)
&&
\prooftree
\justifies
\langle skip,\ s \rangle \rightarrow s
\endprooftree
\\ \\
(S-EXP.NUM)
&&
\prooftree
   \justifies
   \langle x,\ s \rangle \rightarrow \langle n,\ s \rangle
   \using
   s(x) = n 
\endprooftree
\\ \\
(S-EXP.LEFT)
&&
\prooftree
	\langle E_1,\ s \rangle \rightarrow \langle E_1',\ s' \rangle
   \justifies
	\langle (E_1 + E_2),\ s \rangle \rightarrow \langle (E_1' + E_2),\ s' \rangle
\endprooftree
\\ \\
(S-EXP.RIGHT)
&&
\prooftree
	\langle E_2,\ s \rangle \rightarrow \langle E_2',\ s' \rangle
   \justifies
	\langle (E_1 + E_2),\ s \rangle \rightarrow \langle (E_1 + E_2'),\ s' \rangle
\endprooftree
\\ \\
(S-ADD)
&&
\prooftree
   \justifies
   \langle (n_1 + n_2),\ s \rangle \rightarrow \langle n_3,\ s' \rangle
   \using
   n_3 = n_1 + n_2
\endprooftree
\\ \\
(S-ASS.EXP)
&&
\prooftree
	\langle E,\ s \rangle \rightarrow \langle E',\ s' \rangle
   \justifies
   	\langle x:= E,\ s \rangle \rightarrow \langle x:= E',\ s' \rangle
\endprooftree
\\ \\
(S-ASS.NUM)
&&
\prooftree
   \justifies
   	\langle x:= n,\ s \rangle \rightarrow \langle skip,\ s[x \rightarrow n] \rangle
\endprooftree
\\ \\
(S-SEQ.LEFT)
&&
\prooftree
	\langle C_1,\ s \rangle \rightarrow \langle C_1',\ s' \rangle
   \justifies
   	\langle C_1 ; C_2,\ s \rangle \rightarrow \langle C_1' ; C_2,\ s' \rangle
\endprooftree
\\ \\
(S-SEQ.SKIP)
&&
\prooftree
   \justifies
   	\langle skip ; C_2,\ s \rangle \rightarrow \langle C_2,\ s' \rangle
\endprooftree
\\ \\
(S-COND.BOOL)
&&
\prooftree
	\langle B,\ s \rangle \rightarrow \langle B',\ s' \rangle
   \justifies
   	\langle if\ B\ then\ C_1\ else\ C_2,\ s \rangle \rightarrow \langle if\ B'\ then\ C_1\ else\ C_2,\ s' \rangle
\endprooftree
\\ \\
(S-COND.TRUE)
&&
\prooftree
   \justifies
   	\langle if\ true\ then\ C_1\ else\ C_2,\ s \rangle \rightarrow \langle C_1,\ s \rangle
\endprooftree
\\ \\
(S-COND.FALSE)
&&
\prooftree
   \justifies
   	\langle if\ false\ then\ C_1\ else\ C_2,\ s \rangle \rightarrow \langle C_2,\ s \rangle
\endprooftree
\\ \\
(S-WHILE.BOOL)
&&
\prooftree
	\langle B,\ s \rangle \rightarrow \langle B',\ s' \rangle
   \justifies
   	\langle while\ B\ do\ C,\ s \rangle \rightarrow \langle while\ B'\ do\ C,\ s' \rangle
\endprooftree
\\ \\
(S-WHILE.FALSE)
&&
\prooftree
   \justifies
   	\langle while\ false\ do\ C,\ s \rangle \rightarrow \langle skip,\ s \rangle
\endprooftree
\\ \\
(S-LOG.AND.BOOL)
&&
\prooftree
	\langle B_1,\ s \rangle \rightarrow \langle B_1',\ s' \rangle
   \justifies
   	\langle B_1\ \land\ B_2,\ s \rangle \rightarrow \langle B_1'\ \land\ B_2,\ s' \rangle
\endprooftree
\\ \\
(S-LOG.AND.TRUE)
&&
\prooftree
	\langle B_2,\ s \rangle \rightarrow \langle B_2',\ s' \rangle
   \justifies
   	\langle true\ \land\ B_2,\ s \rangle \rightarrow \langle true\ \land\ B_2',\ s' \rangle
\endprooftree
\\ \\
(S-LOG.AND.FALSE)
&&
\prooftree
	\langle B_2,\ s \rangle \rightarrow \langle B_2',\ s' \rangle
   \justifies
   	\langle false\ \land\ B_2,\ s \rangle \rightarrow \langle false\ \land\ B_2',\ s' \rangle
\endprooftree
\\ \\ 
(S-LOG.OR.BOOl)
&&
\prooftree
	\langle B_1,\ s \rangle \rightarrow \langle B_1',\ s' \rangle
   \justifies
   	\langle B_1\ \lor\ B_2,\ s \rangle \rightarrow \langle B_1'\ \lor\ B_2,\ s' \rangle
\endprooftree
\\ \\
(S-LOG.OR.TRUE)
&&
\prooftree
	\langle B_2,\ s \rangle \rightarrow \langle B_2',\ s' \rangle
   \justifies
   	\langle true\ \lor\ B_2,\ s \rangle \rightarrow \langle true\ \lor\ B_2',\ s' \rangle
\endprooftree
\\ \\
(S-LOG.OR.FALSE)
&&
\prooftree
	\langle B_2,\ s \rangle \rightarrow \langle B_2',\ s' \rangle
   \justifies
   	\langle false\ \lor\ B_2,\ s \rangle \rightarrow \langle false\ \lor\ B_2',\ s' \rangle
\endprooftree
\\ \\
(S-CORT.AND.BOOl)
&&
\prooftree
	\langle B_1,\ s \rangle \rightarrow \langle B_1',\ s' \rangle
   \justifies
   	\langle B_1\ \& \&\ B_2,\ s \rangle \rightarrow \langle B_1'\ \& \&\ B_2,\ s' \rangle
\endprooftree
\\ \\
(S-CORT.AND.TRUE)
&&
\prooftree
	\langle B_2,\ s \rangle \rightarrow \langle B_2',\ s' \rangle
   \justifies
   	\langle true\ \& \&\ B_2,\ s \rangle \rightarrow \langle B_2',\ s' \rangle
\endprooftree
\\ \\
(S-CORT.AND.FALSE)
&&
\prooftree
	\langle B_2,\ s \rangle \rightarrow \langle B_2',\ s' \rangle
   \justifies
   	\langle false\ \& \&\ B_2,\ s \rangle \rightarrow \langle false,\ s' \rangle
\endprooftree
\\ \\ 
(S-CORT.OR.BOOL)
&&
\prooftree
	\langle B_1,\ s \rangle \rightarrow \langle B_1',\ s' \rangle
   \justifies
   	\langle B_1\ ||\ B_2,\ s \rangle \rightarrow \langle B_1'\ ||\ B_2,\ s' \rangle
\endprooftree
\\ \\
(S-CORT.OR.TRUE)
&&
\prooftree
	\langle B_2,\ s \rangle \rightarrow \langle B_2',\ s' \rangle
   \justifies
   	\langle true\ ||\ B_2,\ s \rangle \rightarrow \langle true,\ s' \rangle
\endprooftree
\\ \\
(S-CORT.OR.FALSE)
&&
\prooftree
	\langle B_2,\ s \rangle \rightarrow \langle B_2',\ s' \rangle
   \justifies
   	\langle false\ ||\ B_2,\ s \rangle \rightarrow \langle B_2',\ s' \rangle
\endprooftree
\end{align*}
