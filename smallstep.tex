\chapter{Semantica operazionale ``small-step''}

In questo capitolo parleremo della semantica operazionale ``small-step'', una semantica che descrive formalmente dopo ogni piccolo passo di calcolo
che comando viene eseguito e come viene modificato lo store. Per fare ciò sfrutta assiomi e regole che hanno la stessa forma di quelle viste nel
capitolo precedente.

\section{Introduzione}
Se quindi abbiamo un linguaggio con la seguente sintassi:
$$ \AExp \ni E ::= n \vbar (E+E)$$
Considerando che $n \in \Zset$, una possibile semantica small step è:

\begin{gather*}
\rulename{S-LEFT}
\prooftree
        E_1 \ssarrow E_1'
\justifies
        (E_1 + E_2) \ssarrow (E_1' + E_2)
\endprooftree
\pts
\rulename{S-RIGHT}
\prooftree
        E_2 \ssarrow E_2'
\justifies
        (n_1 + E_2) \ssarrow (n_1 + E_2')
\endprooftree
\qquad
\rulename{S-ADD}
\prooftree
        \nohyp
\justifies
        (n_1 + n_2) \ssarrow n_3
\using
        n_3 = n_1 + n_2
\endprooftree
\end{gather*}

La relazione $\ssarrow$ quindi rappresenta a tutti gli effetti un passo di computazione.
Con questa semantica abbiamo lo stesso potere espressivo della semantica big-step ma, forzando la valutazione di tutte le espressioni da destra a sinistra un passo alla volta, da più informazione sulla sequenzialità e sui ``risultati intermedi''.
Vediamo un esempio di dimostrazione:

\begin{gather*}
\prooftree
        \prooftree
                \nohyp
        \justifies
                (1+2) \ssarrow 3
        \endprooftree
\justifies
        ((1+2)+(3+4)) \ssarrow (3+(3+4))
\endprooftree
\pts
\prooftree
        \prooftree
                \nohyp
        \justifies
                (3+4) \ssarrow 7
        \endprooftree
\justifies
        (3+(3+4)) \ssarrow (3+7)
\endprooftree
\pts
\prooftree
        \justifies
                (3+7) \ssarrow 10
\endprooftree
\end{gather*}

Possiamo subito osservare che differentemente dalla semantica big-step
nella semantica small-step si usano tanti alberi di dimostrazione più piccoli per dimostrare
ogni asserzione. Perciò per definire compiutamente la semantica di $\AExp$ dobbiamo considerare la chiusura transitiva e riflessiva.

\section{Determinatezza della semantica small-step}
Dimostriamo ora che la semantica small-step gode della proprietà della
determinatezza: ogni oggetto sintattico può avere perciò al massimo
una semantica. Prima dimostriamo però il seguente

\begin{teorema}
Per ogni $E, E_1, E_2 \in \AExp$, se $E \ssarrow E_1$ e $E \ssarrow E_2$
allora $E_1 = E_2$.
\end{teorema}

\begin{proof}
La dimostrazione è per induzione strutturale su $E \in \AExp$.

Caso base: $E = n \in \Zset$.
In tal caso abbiamo $E \nssarrow$, dunque l'antecedente dell'implicazione
è falso e l'implicazione è vera.

Passo induttivo: $E = E_1 + E_2$.
Ci sono dunque tre casi, a seconda della natura di $E_1$ ed $E_2$:
\[
  E_1 + E_2
    \rightarrow
      \begin{cases}
        n,          &\text{se $E_1 = n_1$, $E_2 = n_2$,
                           $\Zset \ni n = n_1 + n_2$;} \\
        E_1' + E_2, &\text{se $E_1 \ssarrow E_1'$;} \\
        E_1 + E_2', &\text{se $E_1 \in \Zset$, $E_2 \ssarrow E_2'$.}
      \end{cases}
\]
È ovvio che questi tre casi sono mutuamente esclusivi.
\end{proof}

Ora possiamo dimostrare il seguente

\begin{teorema}[Determinatezza]
La semantica small-step di $\AExp$ gode della proprietà di determinatezza, ovvero
$\forall E \in \AExp, \; \forall \; n, n' \in \Zset \text{ tali che } E \ssarrow ^* n \; \land \; E \ssarrow ^* n' \; \Rightarrow n = n'$
\end{teorema}

\begin{proof}
Sappiamo che $E \ssarrow E_1^{(1)} \ssarrow E_1^{(2)} \ssarrow \dots \ssarrow E_1^{(k)} = n_1$ e che $E \ssarrow E_2^{(1)} \ssarrow E_2^{(2)} \ssarrow \dots \ssarrow E_2^{(h)} = n_2$.
Ma per il lemma appena dimostrato sappiamo dunque che $E_1^{(k)} = E_2^{(h)} \text{ con } h = k$, perciò possiamo affermare che $n_1 = n_2$
\end{proof}

\section{Normalizzazione della semantica small-step}
Altra proprietà molto importante per una semantica è la normalizzazione: tale proprietà afferma che
ogni oggetto semantico ha almeno una semantica. Perciò per ogni espressione $E$
esiste almeno un risultato $n$ tale che $E \ssarrow ^* n$.
Dimostriamo prima però il seguente

\begin{teorema}
$\forall E \in Exp \itc E \not \in \Zset \Rightarrow \exists E' \in Exp . E \ssarrow E'$
Questa proprietà ci aiuterà nella dimostrazione del successivo teorema ma dobbiamo anche dimostrare che le espressioni non possono transire per sempre, ovvero che $\forall E \in Exp \itc \exists k \in \Nset . E \ssarrow ^k F \land F \nssarrow$
\end{teorema}

\begin{proof}
Definiamo perciò una funzione $\fund{f}{\AExp}{\Nset}$ che calcola il numero di ``+'', ovvero il numero di passi da fare per arrivare fino a una espressione che non possa più transire, $f$ restituisce ovvero il $k$ definito nella tesi del teorema. Definiamo per induzione strutturale:

Caso Base: $E = n \in \Zset$.
Visto che $n$ non può transire è ovvio che $f(n) = 0$, infatti in $n$ non è presente alcun ``+''.

Passo Induttivo: $E = \not \in \Zset$.
$E \in \AExp$ perciò esistono $E_1, E_2 \in \AExp . E = E_1 + E_2$, in tal caso $f(E) = f(E_1 + E_2) = f(E_1) + f(E_2) + 1$

Dimostriamo ora che questa funzione calcola proprio ciò che volevamo, ovviamente lo dimostriamo per induzione strutturale:

Caso Base: $E = n \in \Zset$.
Come detto sopra $n$ non può transire perciò $f(n) = 0$.

Passo Induttivo: $E \not \in \Zset$.
$E \in \AExp$ perciò esistono $E_1, E_2 \in AExp . E = E_1 + E_2$, quindi $E_1 + E_2 \ssarrow ^i n_1 + E_2 \ssarrow ^j n_1 + n_2 \ssarrow n$. Ma allora è ovvio che $k = i + j + 1 = f(E_1) + f(E_2) + 1 = f(E_1 + E_2) = f(E)$ e che $k$ è proprio il numero di ``+'' presenti in $E$.
\end{proof}

\begin{teorema}[Normalizzazione]
  $\forall E \in \AExp\ \exists n \in \Zset . E \ssarrow ^* n$
\end{teorema}

\begin{proof}
  La proprietà deriva direttamente dai due lemmi dimostrati precedentemente.
\end{proof}

\section{Semantica small-step per WHILE}
Diamo ora una semantica al linguaggio WHILE tramite la small-step, dovremo perciò valutare non più una singola espressione ma una coppia formata da comando e store. Consideriamo subito la possibile presenza di effetti collaterali visto che è semplicemente una generalizzazione del caso senza effetti collaterali. In questo caso quindi la valutazione di espressioni e di booleani può modificare lo store. Indichiamo in questa maniera la transizione tra una configurazione e la successiva:
$
\config{C}{s} \ssarrow \config{C'}{s'}
$

Indichiamo perciò con $E$ la generica espressione appartenente a $\AExp$, con $C$ il generico comando appartenente a $\Com$, con $B$ il generico booleano appartenente a $\BExp$ e con $s$ il generico store appartenente a $\Sigma$:

\subsection*{}
\[
\rulename{S-EXP.NUM}
\prooftree
        \nohyp
\justifies
        \config{x}{s} \ssarrow \config{n}{s}
\using
   s(x) = n
\endprooftree
\]
\subsection*{}
\[
\rulename{S-EXP.LEFT}
\prooftree
        \config{E_1}{s} \ssarrow \config{E_1'}{s'}
\justifies
        \config{(E_1 + E_2)}{s} \ssarrow \config{(E_1' + E_2)}{s'}
\endprooftree
\]
\subsection*{}
\[
\rulename{S-EXP.RIGHT}
\prooftree
        \config{E_2}{s} \ssarrow \config{E_2'}{s'}
\justifies
        \config{(E_1 + E_2)}{s} \ssarrow \config{(E_1 + E_2')}{s'}
\endprooftree
\]
\subsection*{}
\[
\rulename{S-ADD}
\prooftree
        \nohyp
\justifies
        \config{(n_1 + n_2)}{s} \ssarrow \config{n_3}{s}
\using
        n_3 = n_1 + n_2
\endprooftree
\]
\subsection*{}
\[
\rulename{S-SKIP}
\prooftree
        \nohyp
\justifies
        \config{\kw{skip}}{s} \ssarrow s
\endprooftree
\]
Nota: La transizione $\config{\kw{skip}}{s} \ssarrow s$ non viola 
la definizione di $\ssarrow$ perchè $s$ è una abbreviazione di
$\config{\epsilon}{s}$
\subsection*{}
\[
\rulename{S-ASS.EXP}
\prooftree
        \config{E}{s} \ssarrow \config{E'}{s'}
\justifies
        \config{x \weq E}{s} \ssarrow \config{x \weq E'}{s'}
\endprooftree
\]
\subsection*{}
\[
\rulename{S-ASS.NUM}
\prooftree
        \nohyp
\justifies
        \config{x \weq n}{s} \ssarrow \config{\kw{skip}}{\assign{s}{x}{n}}
\endprooftree
\]
\subsection*{}
\[
\rulename{S-SEQ.LEFT}
\prooftree
        \config{C_1}{s} \ssarrow \config{C_1'}{s'}
\justifies
        \config{C_1 ; C_2}{s} \ssarrow \config{C_1' ; C_2}{s'}
\endprooftree
\]
\subsection*{}
\[
\rulename{S-SEQ.SKIP}
\prooftree
        \nohyp
\justifies
        \config{\kw{skip} ; C_2}{s} \ssarrow \config{C_2}{s'}
\endprooftree
\]
\subsection*{}
\[
\rulename{S-COND.BOOL}
\prooftree
        \config{B}{s} \ssarrow \config{B'}{s'}
\justifies
        \config{\condif{B}{C_1}{C_2}}{s} \ssarrow \config{\condif{B'}{C_1}{C_2}}{s'}
\endprooftree
\]
\subsection*{}
\[
\rulename{S-COND.TRUE}
\prooftree
        \nohyp
\justifies
        \config{\condif{\true}{C_1}{C_2}}{s} \ssarrow \config{C_1}{s}
\endprooftree
\]
\subsection*{}
\[
\rulename{S-COND.FALSE}
\prooftree
        \nohyp
\justifies
        \config{\condif{\false}{C_1}{C_2}}{s} \ssarrow \config{C_2}{s}
\endprooftree
\]
\subsection*{}
\[
\rulename{S-WHILE.BOOL}
\prooftree
        \config{B}{s} \ssarrow \config{B'}{s'}
\justifies
        \config{\while{B}{C}}{s} \ssarrow \config{\while{B'}{C}}{s'}
\endprooftree
\]
\subsection*{}
\[
\rulename{S-WHILE.FALSE}
\prooftree
        \nohyp
\justifies
        \config{\while{\false}{C}}{s} \ssarrow \config{\kw{skip}}{s}
\endprooftree
\]
\subsection*{}
\[
\rulename{S-WHILE}
\prooftree
        \nohyp
\justifies
        \config{\while{B}{C}}{s} \ssarrow \config{\condif{B}{(C ; \while{B}{C})}{\kw{skip}}}{s}
\endprooftree
\]
\subsection*{}
\[
\rulename{S-LOG.AND.BOOL}
\prooftree
        \config{B_1}{s} \ssarrow \config{B_1'}{s'}
\justifies
        \config{B_1\ \land\ B_2}{s} \ssarrow \config{B_1'\ \land\ B_2}{s'}
\endprooftree
\]
\subsection*{}
\[
\rulename{S-LOG.AND.TRUE}
\prooftree
        \config{B_2}{s} \ssarrow \config{B_2'}{s'}
\justifies
        \config{\true\ \land\ B_2}{s} \ssarrow \config{\true\ \land\ B_2'}{s'}
\endprooftree
\]
\subsection*{}
\[
\rulename{S-LOG.AND.FALSE}
\prooftree
        \config{B_2}{s} \ssarrow \config{B_2'}{s'}
\justifies
        \config{\false\ \land\ B_2}{s} \ssarrow \config{\false\ \land\ B_2'}{s'}
\endprooftree
\]
\subsection*{}
\[
\rulename{S-LOG.OR.BOOL}
\prooftree
        \config{B_1}{s} \ssarrow \config{B_1'}{s'}
\justifies
        \config{B_1\ \lor\ B_2}{s} \ssarrow \config{B_1'\ \lor\ B_2}{s'}
\endprooftree
\]
\subsection*{}
\[
\rulename{S-LOG.OR.TRUE}
\prooftree
        \config{B_2}{s} \ssarrow \config{B_2'}{s'}
\justifies
        \config{\true\ \lor\ B_2}{s} \ssarrow \config{\true\ \lor\ B_2'}{s'}
\endprooftree
\]
\subsection*{}
\[
\rulename{S-LOG.OR.FALSE}
\prooftree
        \config{B_2}{s} \ssarrow \config{B_2'}{s'}
\justifies
        \config{\false\ \lor\ B_2}{s} \ssarrow \config{\false\ \lor\ B_2'}{s'}
\endprooftree
\]
\subsection*{}
\[
\rulename{S-CORT.AND.BOOL}
\prooftree
        \config{B_1}{s} \ssarrow \config{B_1'}{s'}
\justifies
        \config{B_1\ \& \&\ B_2}{s} \ssarrow \config{B_1'\ \& \&\ B_2}{s'}
\endprooftree
\]
\subsection*{}
\[
\rulename{S-CORT.AND.TRUE}
\prooftree
	\nohyp
\justifies
        \config{\true\ \& \&\ B_2}{s} \ssarrow \config{B_2}{s}
\endprooftree
\]
\subsection*{}
\[
\rulename{S-CORT.AND.FALSE}
\prooftree
	\nohyp
\justifies
        \config{\false\ \& \&\ B_2}{s} \ssarrow \config{\false}{s}
\endprooftree
\]
\subsection*{}
\[
\rulename{S-CORT.OR.BOOL}
\prooftree
        \config{B_1}{s} \ssarrow \config{B_1'}{s'}
\justifies
        \config{B_1\ ||\ B_2}{s} \ssarrow \config{B_1'\ ||\ B_2}{s'}
\endprooftree
\]
\subsection*{}
\[
\rulename{S-CORT.OR.TRUE}
\prooftree
	\nohyp
\justifies
        \config{\true\ ||\ B_2}{s} \ssarrow \config{\true}{s}
\endprooftree
\]
\subsection*{}
\[
\rulename{S-CORT.OR.FALSE}
\prooftree
	\nohyp
\justifies
        \config{\false\ ||\ B_2}{s} \ssarrow \config{B_2}{s}
\endprooftree
\]
