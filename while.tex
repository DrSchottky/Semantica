\chapter{Il linguaggio WHILE}

In questo capitolo si introduce un semplice linguaggio di programmazione,
chiamato ``WHILE'',
che sarà utilizzato per esemplificare varie tecniche di definizione della
semantica dei linguaggi di programmazione.
Il linguaggio per quanto molto semplice, soprattutto nella sua versione di
base, ammette qualche estensione interessante che sarà illustrata nel
seguito.

\section{Sintassi}

\begin{description}
\item[Variabili]
$x \in \Var = \{ x_0, x_1, \ldots \}$,%
\item[Interi]
$n \in \Integer \defeq \Zset$;
\item[Booleani]
$t \in \Bool \defeq \{ \ttv, \ffv \}$;
\item[Espressioni aritmetiche]%
\begin{align*}
  \AExp \ni
  E &::= n \vbar x \vbar E_0 + E_1 \vbar E_0 - E_1 \vbar E_0 * E_1 \vbar \ldots
\end{align*}
\item[Espressioni booleane]%
\begin{align*}
  \BExp \ni
  B &::= \ttv \vbar \ffv \vbar B_0 \kw{and} B_1 \vbar B_0 \kw{or} B_1
              \vbar \kw{not} B \vbar E_0 = E_1 \vbar E_0 < E_1 \vbar \ldots
\end{align*}
\item[Comandi]%
\begin{align*}
  \Com \ni
  C &::= \kw{skip} \vbar x := E \vbar C_0 ; C_1
         \vbar \kw{if} B \kw{then} C_0 \kw{else} C_1 \\
    &\vbar \kw{while} B \kw{do} C \vbar \ldots
\end{align*}
\end{description}

Si noti che aver posto $\Integer \defeq \Zset$ ci permette di superare
la complicazione inerente la distinzione tra \emph{numerali} (gli elementi
di $\Integer$) e i \emph{numeri} (gli elementi di $\Zset$).
Tale distinzione, seppure importante, appesantirebbe la presentazione
senza nessun reale vantaggio, essendo ben note le rappresentazione
dei numeri usate nei comuni linguaggi di programmazione e gli algoritmi
per convertire tali rappresentazioni nei rispettivi rappresentati.

I punti di sospensione nelle produzioni indicano la possibilità di
arricchire il linguaggio con nuovi costrutti.  Alcune estensioni
significative, come le variabili di tipo array e la gestione
delle eccezioni, saranno considerate nel seguito.


\section{Semantica informale}

La semantica di questo linguaggio, ora definito solo sintatticamente,
verrà espressa in maniera operazionale, denotazionale e assiomatica
nei prossimi capitoli. Di seguito ne diamo una semantica
intuitiva.

La variabili sono celle di memoria che possono contenere solo interi
(i booleani non sono memorizzabili in WHILE).
Le celle di memoria sono organizzate in uno \emph{store},
modellato da una funzione parziale dall'insieme delle variabili
agli interi.  Uno store $s$ è dunque un elemento di $\Var \rightarrow \Zset$
indefinito per al più un numero finito di variabili.
Formalmente, l'insieme di tutti gli store è dato da
\[
  \Sigma \defeq \bigl\{\,
                  s \in (\Var \rightarrow \Zset)
                \bigm|
                  |\dom(s)| < \aleph_0
                \,\bigr\}.
\]
Lo \emph{stato} del programma è univocamente identificato da due componenti:
quella inerente il controllo, che rappresenta il punto del programma
dalla quale l'esecuzione procede, e quella inerente i dati, modellata
dallo store.  Data la semplicità del linguaggio WHILE, la parte controllo
dello stato può essere fatta coincidere con la porzione di programma
che deve ancora essere eseguita.

Come accade per ogni linguaggio imperativo, l'operazione fondamentale
del linguaggio WHILE è la modifica dello store.  Questa viene modellata
come un modificatore puntuale di funzione: dato uno store iniziale $s$,
una variabile alla quale assegnare $x$ ed un valore da assegnare $n$,
$s[x \mapsto n]$ denota lo store risultante dall'assegnamento.
Formalmente, per ogni $s \in \Sigma$, $x, y \in \Var$ e $n \in \Zset$,
\[
  s[x \mapsto n](y)
    \defeq
      \begin{cases}
        n,    &\text{se $y = x$;} \\
        s(y), &\text{altrimenti.}
      \end{cases}
\]

Si osservi come l'omissione dell'operazione di divisione non sia casuale:
si è voluto evitare di affrontare il problema della eventuale divisione
per zero, cosa senz'altro fattibile ma che avrebbe complicato notevolmente
la presentazione.

L'ordine di valutazione delle sottoespressioni può o meno essere specificato
e, per quanto riguarda gli operatori logici binari, la semantica può essere
quella classica oppure quella cortocircuitata: le varie possibilità
saranno accennate nel seguito.

Il comando $\kw{skip}$ è un comando che lascia lo store invariato.
La sua utilità è legata alla possibilità di rappresentare il condizionale
a un braccio mediante il condizionale a due braccia, usando $\kw{skip}$ per
il ramo else, e alla possibilità di avere un corpo del $\kw{while}$ che
non fa nulla.  A rigore, $\kw{skip}$ è ridondante: ogni programma
che usa $\kw{skip}$ può essere riscritto sostituendo $\kw{skip}$
con $x_u := 777$, dove $x_u$ è una variabile che non compare nel
programma originario.

Il comando $x := E$ è l'assegnamento del valore dell'espressione $E$
alla variabile $x$.  Il costrutto $C_0; C_1$ rappresenta la composizione
sequenziale di $C_0$ e $C_1$ vale a dire che viene eseguito $C_0$ e,
se e quando questo termina, viene eseguito $C_1$ a partire dallo store
risultante dalla esecuzione di $C_0$.  Inoltre è \emph{come se} ogni
effetto dell'esecuzione di $C_0$ sia terminato prima che inizi la
computazione di $C_1$.

Il comando $\kw{while} B \kw{do} C$, che dà il nome al linguaggio,
causa la valutazione dell'espressione booleana $B$, detta \emph{guardia};
se questa dà \emph{vero} viene eseguito il comando $C$, detto \emph{corpo},
e poi si torna a valutare la guardia; se questa dà \emph{vero}\dots
Nel linguaggio WHILE in ``versione base'', quindi, l'unico modo per
uscire da un $\kw{while}$ è che la valutazione della guardia dia \emph{falso}.
Come vedremo, l'estensione con la gestione delle eccezioni cambierà le cose
a questo riguardo.
