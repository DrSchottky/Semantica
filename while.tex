\chapter{Il linguaggio WHILE}

In questo capitolo si introduce un semplice linguaggio di programmazione,
chiamato ``WHILE'',
che sarà utilizzato per esemplificare varie tecniche di definizione della
semantica dei linguaggi di programmazione.
Il linguaggio per quanto molto semplice, soprattutto nella sua versione di
base, ammette qualche estensione interessante che sarà illustrata nel
seguito.

\section{Sintassi}

\begin{description}
\item[Variabili]
$x \in \Var = \{ x_0, x_1, \ldots \}$,%
\item[Interi]
$n \in \Integer \defeq \Zset$;
\item[Booleani]
$t \in \Bool \defeq \{ \ttv, \ffv \}$;
\item[Espressioni aritmetiche]%
\begin{align*}
  \AExp \ni
  E &::= n \vbar x \vbar E_0 + E_1 \vbar E_0 - E_1 \vbar E_0 * E_1 \vbar \ldots
\end{align*}
\item[Espressioni booleane]%
\begin{align*}
  \BExp \ni
  B &::= \ttv \vbar \ffv \vbar B_0 \kw{and} B_1 \vbar E_0 \kw{and} E_1
              \vbar \kw{not} E \vbar \ldots
\end{align*}
\item[Comandi]%
\begin{align*}
  \Com \ni
  C &::= \kw{skip} \vbar x := E \vbar C_0 ; C_1
         \vbar \kw{if} B \kw{then} C_0 \kw{else} C_1 \\
    &\vbar \kw{while} B \kw{do} C \vbar \ldots
\end{align*}
\end{description}

\section{Semantica informale}

[Da scrivere.]
